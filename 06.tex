\chapter{Completezza e duali di qualche spazio}

\section{Elenco di spazi completi}

\begin{proposition}
Sia $S$ insieme e $E$ Banach, allora lo spazio normato $(\Bs(S,E),\normd_{\infty,S})$ \`e completo
\end{proposition}
\begin{proof}
~[PERSO, RIGUARDA POI]

tale che $\norm{f(s)}=\norm{\sum_kf_k(s)}\leq \sum_k\norm{f_k(s)}\leq \sum \norm f_{\infty,S}$

quindi $\norm f_{\infty,S}$
\end{proof}

\filosofia{Uno degli strumenti dell'analista: aggiungere e togliere, cio\`e
\[\pi\rho o\sigma\tau\al\vp\al\acute{\iota}\rho\e\sigma\iota\varsigma\]}

\begin{lemma}
Se $(f_k)_{k\in\N}\subseteq \Bs(S,E)$ con $f_k$ continua in $s_0$ per ogni $k$ e $f_k\to f$ uniformemente allora anche $f$ \`e continua in $s_0$
\end{lemma}
\begin{proof}
Consideriamo
\begin{align*}
\norm{f(s)-f(s_0)}\leq&\norm{f(s)-f_k(s)}+\norm{f_k(s)-f_k(s_0)}+\norm{f_k(s_0)-f(s_0)}\leq\\
\leq&2\norm{f-f_k}_{\infty,S}+\norm{f_k(s)-f_k(s_0)}
\end{align*}
Per la convergenza uniforme di $f_k\to f$ si ha che per ogni $\e>0$ esiste $n\in\N$ tale che $\norm{f-f_n}_{\infty,S}\leq \e/3$.

Per la continuit\`a in $s_0$ di $f_n$ esiste un intorno $U$ di $s_0$ rale che $\norm{f_n(s)-f_n(s_0)}\leq \e/3$ per ogni $s\in U$. Allora per ogni $s\in U$ si ha
\[\norm{f(s)-f(s_0)}\leq2\e/3+\e/3=\e.\]
\end{proof}

\begin{proposition}
Sia $S$ spazio topologico, $E$ banach, allora $\Bs C(S,E)$ \`e completo.
\end{proposition}
\begin{proof}
Basta mostrare che $\Bs C(S,E)$ \`e chiuso in $\Bs(S,E)$. Questo segue dal fatto che la continuit\`a in un punto $s_0\in S$ si conserva per convergenza uniforme, che \`e il lemma precedente.
\end{proof}

\begin{example}
Sia $S=\N\cup\cpa{\infty}$ la compattificazione di Alexandrov di $\N$ e $E$ un banach, allora
\[c(E)\doteqdot\cpa{x:\N\to E,\text{ convergente}}\cong \Bs C(S,E)\]
Questo mostra che $c(E)$ \`e chiuso (e quindi completo) in $\ell_\infty(E)=\Bs(\N,E)$.
\end{example}


Conseguenze:
\begin{proposition}
Lo spazio $(L(X,Y),\normd)$ \`e completo
\end{proposition}
\begin{proof}
Considerando l'inclusione isometrica 
\[R:\funcDef{L(X,Y)}{\Bs(B_X(0,1),Y)}{T}{T\res{B_X(0,1)}}\]
basta vedere che $R(L(X,Y))$ \`e chiuso.

Se $(T_n)_{n\in\N}\subseteq L(X,Y)$ \`e tale che $R(T_n)\to f$ uniformemente in $\Bs(B_X(0,1),Y)$ allora mostriamo che $f$ \`e la restrizione a $B_X(0,1)$ di una qualche lineare $T$.

Mostriamo che le $T_n$ convergono puntualmente per ogni $x\in X$: se $x=0$ ok, se $x\neq 0$
\[T_n(x)=\norm x T_n(x/\norm x)=\norm x R(T_n)(x/\norm x)\to \norm x f(x/\norm x)\]

Sia $T:X\to Y$ definita da $T(x)=\norm x f(x/\norm x)$

[MOSTRARE CHE LA CONVERGENZA \`E UNIFORME, ME LO SONO PERSO]
\end{proof}

\begin{corollary}[Duale di spazio normato \`e banach]\label{CorDualeNormatoEBanach}
Il duale di uno spazio normato \`e sempre banch.
\end{corollary}

\begin{theorem}[Integrazione per serie]\label{ThIntegrazionePerSerie}
    Sia $(X,\Qs,\mu)$ \`e uno spazio di misura e sia $(f_k)_{k\in\N}\subseteq \Lc^1(X,\Qs,\mu)$ tali che
    \[\sum_{k\in \N}\norm{f_k}_1<\infty\]
Allora $\sum_{k\in\N}f_k$ converge q.o. e in norma 1.
\end{theorem}
\begin{proof}
Per ogni $n\in\N$ sia $g_n:X\to \R$ data da
\[g_n(x)=\sum_{k=0}^n\abs{f_k(x)}.\]
Notiamo che $(g_n)$ \`e una successione di funzioni misurabili non negative crescente. Inoltre $g_n\to \sum_{k\in\N}\abs{f_k(x)}$ per definizione di serie.

Per convergenza monotona
\[\sum_{k\in\N}\norm f_1\leftarrow\sum_{k=0}^n\norm{f_k}_1=\inf_X g_nd\mu\to \int_X gd\mu\]
cio\`e $\inf_X gd\mu=\sum{k\in\N}\norm f_1<\infty$, cio\`e $g\in\Lc^1$.

Inoltre $s_n=\sum_{k=0}^nf_k$ \`e una successione dominata da $g$:
\[\abs{s_n(x)}\leq \sum_{k=0}^n\abs{f_k(x)}\leq g(x).\]
Quindi la serie $\sum f_k(x)$ \`e una serie assolutamente convergente per ogni $x$ dove $g<\infty$. Poich\'e $\int g<\infty$ le eccezioni sono trascurabili, quindi quasi ovunque $\sum f_k(x)$ \`e assolutamente convergente.

Sia $f(x)=\sum f_k(x)$ dove la serie converge. Notiamo che
\[\abs{f(x)}\leq \sum_{k\in\N}\abs{f_k(x)}=g(x),\]
quindi $\norm f_1\leq \int gd\mu=\sum_{k\in\N}\norm{f_k}_1$.

Applicando come prima la stima alle code
\[\norm{f-s_n}_1=\norm{\sum_{k=n+1}^\infty f_k}_1\leq \sum_{k>n}\norm{f_k}_1=o(1)\]
dove l'ultimo termine va a 0 perch\'e $\sum\norm {f_k}_1$ \`e convergente.
\end{proof}

\begin{corollary}[Weil]\label{CorTeoremaWeil}
Siano $f_n\in \Lc^1(X,\Qs,\mu)$ convergenti in $\normd_1$. Allora esiste $n_k$ successione strettamente crescente di indici tali che $f_{n_k}$ converge quasi ovunque ed \`e dominata in $\Lc^1$.
\end{corollary}
\begin{proof}
Sia $f$ il limite in $\normd_1$. Data questa convergenza consideriamo una sottosuccessione $n_k$ tale che $\norm{f-f_{n_k}}_1<2^{-k}$. Scrivendo la successione in termini di una somma telescopica
\[f_{n_k}=f_{n_0}+\sum_{j=1}^k(f_{n_j}-f_{n_{j-1}})\]
si ha per il teorema di integrazione per serie\footnote{$\norm{f_{n_0}}_1+\sum_{j=1}^\infty \norm{f_{n_j}-f_{n_{j-1}}}_1\leq \norm{f_{n_0}}_1+\sum_{j=1}^\infty \norm{f_{n_j}-f}_1+ \sum_{j=1}^\infty \norm{f_{n_{j-1}}-f}_1<\infty$} (\ref{ThIntegrazionePerSerie}) $f_{n_k}$ converge quasi ovunque e in $Lc^1$, inoltre \`e dominata da
\[g(x)=\abs{f_{n_0}(x)}+\sum_{j=0}^\infty\abs{f_{n_j}-f_{n_{j-1}}}\geq \abs{f_{n_k}(x)}\]
con $g(x)\in \Lc^1$.
\end{proof}

\begin{proposition}[$L^1$ \`e completo]\label{PrL1Completo}
Se $(X,\Qs,\mu)$ \`e uno spazio di misura, $L^1(X,\Qs,\mu)$ \`e completo.
\end{proposition}
\begin{proof}
Segue immediatamente dal teorema di integrazione per serie (\ref{ThIntegrazionePerSerie}).
\end{proof}

\begin{remark}
La convergenza quasi ovunque di funzioni $\Lc^1(\R,dx)$ \`e \textbf{NON} \`e la convergenza rispetto a una topologia opportuna su $\Lc^1(\R,dx)$.

\begin{proposition}[Propriet\`a di Uhrison]\label{PrProprietaUhrisohn}
Ogni convergenza topologica in $X$ insieme ha la seguente propriet\`a \textbf{di Uhrisohn}: $x_n\to x$ rispetto alla topologia se e solo se per ogni sottosuccessione $x_{n_k}$ esiste una sotto-sottosuccessione $x_{n_{k_j}}\to x$.
\end{proposition}
\begin{proof}
Se $x_n\to x$ converge ok. Se non converge allora esiste un intorno $U$ di $x$ tale che $x_n\notin U$ frequentemente, quindi troviamo una sottosuccessione $x_{n_k}$ che sta sempre fuori da $U$, quindi nessuna sua sotto-sottosuccessione pu\`o convergere a $x$.
\end{proof} 

La convergenza q.o. per successioni in $\Lc^1(\R)$ non ha la propriet\`a di Uhrisohn.
\end{remark}

\begin{definition}[Operatore di composizione]
Se $E$ \`e uno spazio di funzioni con codominio $\R$ e $f:\R\to\R$, definiamo l'operatore di composizione per $f$ come $E\ni u\mapsto f\circ u$.
\end{definition}

\begin{lemma}
Sia $u_k$ una successione che converge a $u$ in $\normd_p$. A meno di sottosuccessione $u_k\to u$ quasi ovunque e dominata in $\Lc^p$.
\end{lemma}
\begin{proof}
Teorema di Weil (\ref{CorTeoremaWeil}) in $\Lc^p$.
\end{proof}

\begin{proposition}
Lo spazio $L^p(X,\Qs,\mu)$ per $0\leq p<\infty$ \`e completo.
\end{proposition}
\begin{proof}
$L^p$ ed $L^1$ NON sono isomorfi come spazi di Banach in generale\footnote{cursiosit\`a non banale da vedere}, ma esiste un omeomorfismo localmente Lipschitz e questo basta a mostrare la completezza: se $u_k$ \`e una successione di Cauchy in $L^p$, se $\Phi$ \`e Lipschitz allora $\Phi(u_k)$ \`e ancora di Cauchy in $L^1$ e quindi converge, poi torno indietro con $\Phi\ii$, che mantiene il limite per continuit\`a.


Consideriamo
\[\Phi:\funcDef{\Lc^p}{\Lc^1}{u}{\abs{u}^p\sgn(u)}\]
Chiaramente \`e invertibile mandando $v\in L^1$ in $\abs{v}^{1/p}\sgn v$. La mappa $\Phi$ \`e l'operatore di composizione con la funzione $f(t)=\abs{t}^p\sgn t$. La continuit\`a degli operatori di composizione \`e un fatto generale. Se $u_k\to u$ converge in $\normd_p$ allora per il lemma a meno di sottosuccessione converge q.o. e dominata, quindi componendo con $f$ abbiamo ancora convergenza quasi ovunque per continuit\`a ($f(u_k)\to f(u)$ q.o.). Se $\abs{u_k}\leq g$ in $\Lc^p$ allora $\abs{u_k}^p\leq g^p$ in $\Lc^1$, similmente per $\Phi\ii$, quindi effettivamente $\Phi$ \`e un omeomorfismo.


Mostriamo ora che $\Phi$ \`e localmente lipschitz: siano $u,v\in \Lc^p(X)$
\[\abs{\Phi(u)-\Phi(v)}_1=\int_X\abs{f(u(x))-f(v(x))}d\mu(x)\]
ma se $t<s$ allora $\abs{f(t)-f(s)}\leq \sup_{t\leq \xi\leq s}\abs{f'(\xi)}\abs{t-s}$ e $\abs{f'(xi)}=p\abs{xi}^{p-1}\leq p(\max\cpa{\abs t,\abs s})^p$, quindi
\begin{align*}
    \abs{\Phi(u)-\Phi(v)}_1\leq& p\int_X\max\cpa{\abs{u(x)}^{p-1},\abs{v(x)}^{p-1}}\abs{u(x)-v(x)}d\mu\leq\\
    \leq &p\int_X\pa{\abs{u(x)}^{p-1}+\abs{v(x)}^{p-1}}\abs{u(x)-v(x)}d\mu\pasgnl\leq{H\"older}\\
    \leq&p\pa{\pa{\int_X\abs{u}^{(p-1)q}}^{1/q}+\pa{\int_X\abs{v}^{(p-1)q}}^{1/q}}\pa{\int_X\abs{u-v}^p}^{1/p}=\\
    \pasgnlmath={p-1=p/q}&p(\norm u_p^{p-1}+\norm v_p^{p-1})\norm{u-v}_p
\end{align*}
quindi $\Phi$ \`e Lipschitz di costante $2pR^{p-1}$ sulla palla $B_{L^p}(0,R)\subseteq L^p$
\end{proof}



\begin{proposition}
Lo spazio $L^\infty(X,\Qs,\mu)$ \`e completo
\end{proposition}
\begin{proof}
~[NON HO VISTO, RIGUARDA I PDF]
\end{proof}

$\norm f_{C^1}=\norm f_{\infty,\Omega}+\sum_{i=1}^n\norm{\del_i f}_{\infty,\Omega}$. Questa norma rende continua l'immaersione $C^1_b\to (C_b^0)^{n+1}$ data da $f\mapsto(f,\del_1f,\cdots,\del_n f)$

\begin{proposition}
Sia $\Omega\subseteq \R^n$ aperto. Lo spazio 
\[C^k_b(\Omega)=\cpa{f:\Omega\to\R\mid \text{di classe $C^k$ con derivate limitate su $\Omega$ fino all'ordine $k$}}\]
\`e completo.
\end{proposition}
\begin{proof}
Il caso $k=1$ \`e una conseguenza del teorema di limite sotto il segno di derivata, infatti se $f_k:\Omega\to \R$, $\del_i f_k:\Omega\to\R$ \`e tale che $\del_i f_k\to g_i$ uniformemente in $\Omega$ e $f_k\to f$ puntualmente in $\Omega$ allora esiste $\del_i f$ e vale $g_i$. Se poi $f_k\in C^1(\Omega)$ allora la $g_i$ \`e continua perch\'e limite uniforme di $\del_i f_k$ continue, quindi per il teorema del differenziale totale la $f$ \`e anche $C^1$.

Per il teorema di limite sotto il segno di derivata, l'immersione $C^1_b\to (C_b^0)^{n+1}$ ha immagine chiusa, infatti una successione $(f_k,\del_1f_k,\cdots,\del_nf_k)$ nell'immagine convergente a $(f,g_1,\cdots, g_n)$ \`e proprio una delle ipotesi del teorema di convergenza sotto segno di derivata, quindi $f_k\to f$ in $C^1$
\end{proof}

\section{Duali di spazi concreti}
\subsection{Spazi \texorpdfstring{$\ell_p$}{lp}}
\begin{definition}[Norma estesa]
Sia $\sigma:\K^\N\to [0,\infty]$ una \textbf{norma estesa}, cio\`e
\begin{enumerate}
    \item $\sigma(x+y)\leq \sigma(x)+\sigma(y)$
    \item $\sigma(\la x)=\abs\la\sigma(x)$
    \item $\sigma(x)=0\coimplies x=0$
\end{enumerate}
\end{definition}
\noindent
Inoltre supponiamo che 
\begin{enumerate}
    \item[4.] per ogni $n\in\N$ esista $C_n$ tale che per ogni $x\in\K^\N$ si abbia $\abs{x(n)}\leq C_n\sigma(x)$
    \item[5.] $\sigma$ \`e LSC\footnote{semicontinua inferiormente} rispetto alla convergenza puntale, cio\`e
    \[x^\nu\in \K^\N,\ \forall i\ x^\nu(i)\to x(i)\implies \sigma(x)\leq \liminf_{\nu\to+\infty}\sigma(x^\nu)\]
\end{enumerate}
\begin{example}
La funzione $\sigma(x)=(\sum\abs{x_i}^p)^{1/p}$ \`e una norma estesa su $\K^\N$ che ha propriet\`a indicate. Anche $\sigma(x)=\norm x_\infty$ ha queste propriet\`a.
\end{example}

\begin{remark}
Le propriet\`a 4. e 5. sono equivalenti a dire che $\cpa{\sigma\leq 1}$ \`e compatto in $\K^\N$, infatti
\[4.\coimplies \cpa{\sigma\leq 1}\subseteq \prod_n\ol{B(0,C_n)}\subseteq \K^\N\]
e $5.$ equivale a chiedere $\cpa{\sigma\leq 1}$ chiuso, quindi insieme dicono che $\cpa{\sigma\leq 1}$ \`e un chiuso in un compatto ($\prod_n\ol{B(0,C_n)}$ \`e prodotto di compatti).
\end{remark}

\begin{definition}[Dominio di finitezza]
Definiamo il \textbf{dominio di finitezza} della norma estesa $\sigma$ come
\[\ell_\sigma=\cpa{x\in\K^\N\mid \sigma<+\infty}\]
\end{definition}

\begin{exercise}
Il dominio di finitezza $\ell_\sigma$ \`e uno spazio di Banach e $\sigma$ induce la norma.
\end{exercise}
\begin{proof}
Traccia:
\begin{itemize}
    \item Verificare che $\ell_\sigma$ \`e uno spazio vettoriale
    \item $\sigma$ \`e una norma
    \item Verificare la completezza: 
    \begin{itemize}
        \item Sia $(x^\nu)_\nu\subseteq \ell_\sigma$ di Cauchy per $\sigma$. Allora per ogni $n\in\N$
        \[\pa{x^\nu(n)}_\nu\subseteq \K\]
        \`e una successione di Cauchy in $\K$, quindi esiste $x\in\K^\N$ tale che $x^\nu\to x$ puntualmente.
        \item Verificare che $x\in\ell_\sigma$: essendo di Cauchy, $x^\nu$ \`e limitata, cio\`e $\sigma(x^\nu)\leq R$ per qualche $R\in \R$, dunque $\sigma(x)\leq R$ perch\'e $\cpa{\sigma\leq R}$ \`e chiuso.
        \item Verficare che $\sigma(x^\nu-x)\to 0$: Per ogni $\e>0$ esiste $n\in\N$ tale che per ogni $p,q\geq n$ vale $\sigma(x^p-x^q)\leq \e$. Notiamo che $x^p-x^n\to x-x^n$ puntualmente, quindi per la semicontinuit\`a si ha che per ogni $\e>0$ esiste $n\in\N$ tale che per ogni $q\geq n$ vale
        \[\sigma(x-x^q)\leq\liminf_{p\to +\infty}\sigma(x^p-x^q)\leq  \e\]
        cio\`e $\sigma(x-x^q)\to 0$ in norma $\sigma$.
    \end{itemize}
\end{itemize}
\end{proof}

\begin{remark}
Questa \`e una seconda dimostrazione della completezza di $\ell_p$ per $1\leq p\leq \infty$.
\end{remark}

\begin{remark}
Funziona anche l'analogo per paranorme, quindi in realt\`a segue anche la completezza di $\ell_p$ per $0<p\leq 1$.
\end{remark}

\begin{proposition}[Duali di $\ell_p$]\label{PrDualilp}
Se $p$ e $q$ sono esponenti coniugati ($\frac1p+\frac1q=1$, $\frac1\infty\doteqdot 0$) allora vale l'isometria $(\ell_p)^\ast\cong \ell_q$.
\end{proposition}
\begin{proof}
Esiste una inclusione lineare isometrica data da
\[\Phi:\funcDef{\ell_q}{(\ell_p)^\ast}{y}{\Phi_y: x\mapsto \sum_{i=0}^\infty y_ix_i},\]
dove la serie in esame converge assolutamente per la disuguaglianza di H\"older:
\[\sum\abs{x_iy_i}\leq \norm x_p\norm y_q.\]
Effettivamente $\Phi_y:\ell_p\to \K$ \`e lineare e continua per $\normd_{(\ell_p)^\ast}$, infatti
\[\norm{\Phi_y}_{(\ell_p^\ast)}=\sup_{\norm x_p\leq 1}\abs{\sum_{i=0}^\infty x_iy_i}\leq \sup_{\norm x_p\leq 1}\norm x_p\norm y_q=\norm y_q.\]
La stessa disuguaglianza mostra che $\Phi$ stessa \`e un elemento di $L(\ell_q,(\ell_p)^\ast)$ di norma minore o uguale a $1$.

Resta da mostrare che $\Phi$ \`e isometrica e surgettiva.

\setlength{\leftmargini}{0cm}
\begin{itemize}
\item[$\boxed{1< p<\infty}$] Per mostrare che $\norm{\Phi_y}_{\ell_p^\ast}=\norm y_q$ per ogni $y\in \ell_q$ consideriamo $x\in\K^\N$ dato da $x_i=\ol{\sgn y_i}\abs{y_i}^{q-1}$. Con questa scelta si ha che
\[x_iy_i=\ol{\sgn y_i}{\sgn y_i}\abs{y_i}^q=\abs{y_i}^q,\]
inoltre
\[\norm x_p^p=\sum_{i=0}^\infty \abs{x(i)}^p=\sum_{i=0}^\infty \abs{y_i}^{(q-1)p}=\sum_{i=0}^\infty \abs{y_i}^{q}=\norm y_q^q,\]
cio\`e $x\in \ell_p$ e 
\[\norm{\Phi_y}_{\ell_p^\ast}\geq \frac{\Phi_y(x)}{\norm x_p}=\frac{\sum_{i=0}^\infty \abs{y_i}^q}{(\norm y_q)^{q/p}}=\norm{y}_q^{q-q/p}=\norm{y}_q,\]
d'altronde sappiamo che vale anche l'altra disuguaglianza in generale, quindi abbiamo $\norm{\Phi_y}_{\ell_p^\ast}=\norm y_q$ come voluto.
\item[$\boxed{p=\infty,\ q=1}$] Sia $x_i=\ol{\sgn y_i}$. Segue che $\norm x_\infty\leq 1$ quindi \`e un elemento valido e
\[\Phi_y(x)=\norm y_1,\]
da cui segue $\norm{\Phi_y}_{\ell_\infty^\ast}\geq \norm y_1$ come voluto.
\item[$\boxed{p=1,\ q=\infty}$] In generale $\norm{\Phi_y}_{\ell_1^\ast}$ non \`e raggiunto come $\Phi_y(x)$ per qualche $x$. La conclusione per\`o vale comunque.
\end{itemize}
Mostriamo ora che l'inclusione \`e surgettiva per $1\leq p<\infty$. Per ogni $\vp\in \ell_p^\ast$ cerchiamo $y\in\ell_q$ tale che $\vp=\Phi_y$. C'\`e un solo $y$ possibile, basta valutare $\vp$ negli $e_i=(\delta_{ij})_j$. Per ogni $m\in\N$ sia $P_m:\K^\N\to \K^m$ il proiettore sulle prime $m$-entrate. Considerando $P_m$ come operatore $P_m:\ell_p\to\K^m\subseteq \ell_p$ restringendo il dominio, definiamo $\vp_m=\vp\circ P_m=P_m^\ast\vp$. Infine, sia 
\[y_m=P_m y=(y(0),y(1),\cdots, y(m-1), 0,0,\cdots)=\sum_{i=0}^{m-1}y_ie_i,\] 
e notiamo che
\[\vp_m=\Phi_{y_m}\]
infatti sono entrambi elementi di $\ell_p^\ast$ e 
\begin{align*}
    \vp_m(e_k)=\vp(P_m(e_k))=&\begin{cases}
        \vp(e_k) &\text{se }k<m\\
        0 &\text{se }k\geq m
        \end{cases}\\
    \Phi_{y_m}(e_k)=\sum_{i=0}^\infty y_m(i)e_k(i)=y_m(i)=&\begin{cases}
        \vp(e_k) &\text{se }k<m\\
        0 &\text{se }k\geq m
        \end{cases}
\end{align*}
quindi $\vp_m$ e $\Phi_{y_m}$ coincidono su $(e_k)$, quindi sullo span di questi e quindi sulla chiusura di questo span, che \`e $\ell_p$ se $p<\infty$.

Essendo $\Phi$ isometrica
\[\norm{y_m}_q=\norm{\Phi_{y_m}}_{\ell_p^\ast}=\norm{\vp_m}_{\ell_p^\ast}\leq \norm\vp_{\ell_p^\ast}\]
quindi $\sum_{i=0}^{m-1}\abs{y(i)}^q\leq \norm\vp_{\ell_p^\ast}^q$ per ogni $m$, dunque passando al $\sup$ in $m$ 
\[\norm y_q\leq \norm\vp_{\ell_p^\ast}\]
e quindi $y$ era un elemento valido di $\ell_q$.
\setlength{\leftmargini}{0.5cm}
\end{proof}

\begin{proposition}[Convergenza forte e debole coincidono su $\ell_1$]\label{ConvergenzaForteEDeboleCoincidonoSul1}
La convergenza debole e la convergenza in norma per $\ell_1$ sono la stessa cosa.
\end{proposition}
\begin{proof}
Poich\'e la topologia debole \`e meno fine della topologia forte basta mostrare che convergenza debole implica convergenza in $\normd_1$.

Sia $f_n\to f$ in $w$-$\ell_1$, cio\`e per ogni funzionale $\phi$ lineare continuo su $\ell_1$ si ha che $\ps{\phi,f_n}\to\ps{\phi,f}$. Dunque, ricordando (\ref{PrDualilp}) che $\ell_\infty=\ell_1^\ast$, per ogni $\vp\in\ell_\infty$ si ha
\[\sum_i\vp(i)f_n(i)\to \sum_i\vp(i)f(i).\]
Notiamo che, portando $f$ al primo membro possiamo supporre senza perdita di generalit\`a $f=0$. Per la propriet\`a di Uhrison (\ref{PrProprietaUhrisohn}) basta provare che esiste una sottosuccessione di $f_n$ che converge a $0$ infatti $f_n\to 0$ se e solo se per ogni sottosuccessione $f_{n_k}$ esiste una sotto-sottosuccessione $f_{h_{k_j}}\to 0$.

Nel caso particolare di successioni $f_n$ a supporto disgiunto la tesi \`e facile: Se siamo in questo caso scegliamo $\vp\in\ell_\infty$ data da 
\[\vp=\sum \ol{\sgn f_i}\text{ dove }(\sgn f_i)(x)=\sgn(f_i(x))=\begin{cases}
    f_i(x)/\abs{f_i(x)} & f_i(x)\neq 0\\
    0 & f_i(x)=0
\end{cases}\] 
in modo tale che $\ps{\vp,f_n}=\ps{\ol{\sgn(f_n)},f_n}=\norm{f_n}_1$ e stesso per $f$, quindi in questo caso \`e chiaro che convergenza debole implica convergenza in $\normd_1$.

Assumiamo dunque che $f_n\to 0$ debolmente (e quindi puntualmente guardando i funzionali che estraggono la $n$-esima entrata). Basta provare che esiste una successione $(g_j)_{j\geq 0}\subseteq \ell_1$ e una sottosuccessione $f_{n_j}$ tale che $\norm{f_{n_j}-g_j}_1\to 0$ e $g_j$ hanno supporto disgiunto. 

Costruiamo le $g_j$ per induzione. Notiamo che 
\begin{itemize}
    \item per ogni $f_n$ si ha che $\norm{f_n\chi_{\N\bs[0,T]}}_1\to 0$ per $T\to\infty$
    \item per ogni $T$ si ha $\norm{f_n\chi_{[0,T]}}_1\to 0$ per $n\to \infty$
\end{itemize}
quindi per costruire le $g_j$ basta alternare questi fatti prendendo opportuni limiti (credo?).
\end{proof}

\begin{remark}
Questo \`e un esempio dove due topologie diverse hanno ``le stesse successioni convergenti".
\end{remark}