\chapter{Distribuzioni}
\begin{definition}[Distribuzione]
Fissato $\Omega\subseteq \R^n$ aperto, una \textbf{distribuzione} \`e una forma lineare e continua su $\Dc(\Omega)$. Lo spazio delle distribuzioni \`e dunque il duale topologico di $\Dc(\Omega)$, cio\`e $\Dc(\Omega)^\ast$, che per\`o in questo contesto viene spesso indicato $\Dc'(\Omega)$ per ragioni storiche.
\end{definition}

\begin{remark}
$u:\Dc(\Omega)\to \R$ lineare \`e una distribuzione SE \`e continua, cio\`e se per ogni $K\in k(\Omega)$ si ha che
\[u\res{C^\infty_K}:C_K^\infty\to \R\]
\`e continua, o equivalentemente se per ogni $K\in k(\Omega)$ esistono $m\in\N$ e $C>0$ tali che
\[\abs{\ps{u,f}}\leq C p_m(f)\quad \forall f\in C_K^\infty.\]
\end{remark}

\begin{definition}[Ordine di una distribuzione]
Se $m\in\N$ \`e tale che per ogni $K\in k(\Omega)$ esiste $C>0$ tale che
\[\abs{\ps{u,f}}\leq C p_m(f)\quad \forall f\in C_K^\infty.\]
si dice che $u$ ha \textbf{ordine minore o uguale a $m$}. Se non esiste un tale $m$ diciamo che $u$ ha ordine $\infty$, mentre se esite allora l'\textbf{ordine di $u$} \`e il minimo tale $m$. Indichiamo l'ordine di $u$ con $\ord(u)$.
\end{definition}

\begin{remark}
Intuitivamente l'ordine \`e ``il massimo ordine di derivate" che pu\`o apparire scrivendo $u$ esplicitamente.
\end{remark}

\begin{example}
La valutazione in un punto \`e una distribuzione di ordine $0$. La valutazione della prima derivata in un punto \`e una distribuzione di ordine $1$.
\end{example}


\begin{remark}
Valgono:
\begin{itemize}
\item Per ogni $K\in k(\Omega)$ e per ogni $f_j\to 0$ in $C^\infty_K$ vale $\ps{u,f_j}\to0$.
\item Per ogni $f_j\to 0$ in $\Dc(\Omega)$, $\ps{u,f_j}\to 0$.
\item Per (\ref{PrTopologizzazioneInNormeDiLimiteInduttivoDOmega}) esistono $\sigma,\mu$ tali che
\[\abs{\ps{u,f}}\leq p_{\sigma,\mu}(f)\qquad \forall f\in \Dc(\Omega)\]
\end{itemize}
\end{remark}

\begin{definition}[Integrabili su compatti]
Definiamo le funzioni \textbf{integrabili su compatti} $L^1_{loc}(\Omega)$ come le funzioni $u$ su $\Omega$ tali che per ogni $K\in k(\Omega)$ si ha $u\res K\in L^1(K)$.
\end{definition}


\begin{definition}[Inclusione delle localmente integrabili nelle distribuzioni]
Definiamo
\[T:\funcDef{L^1_{loc}(\Omega)}{\Dc'(\Omega)}{u}{T_u:\funcDef{\Dc(\Omega)}{\R}{f}{\int_\Omega fu\,dx}}\]
\end{definition}

\begin{proposition}\label{PrInclusioneLocalmenteIntegrabiliInDistribuzioni}
La mappa $T$ \`e ben definita e iniettiva.
\end{proposition}
\begin{proof}
L'integrale $\int_\Omega fu\,dx$ \`e ben definito perch\'e $f$ ha supporto compatto ed \`e continua (quindi $uf=u\res K f$ \`e integrabile).
$T_u$ \`e continua su ogni $C^\infty_K$ perch\'e
\[\abs{\ps{T_u,f}}\leq\norm{\mu_K}_1\norm{f}_{\infty}=\norm{\mu\res K}p_0(f).\]
L'iniettivit\`a \`e evidente perch\'e se $u\neq v$ in $L^1_{loc}$ allora esiste un insieme di misura non negativa dove non coincidono, opportune mollificazioni della caratteristica di questo insieme mostrano che $T_u\neq T_v$.
\end{proof}

\begin{example}
Le seguenti sono distribuzioni:
\begin{enumerate}
\item \textbf{Valutazioni di derivate}: Sia $x_0\in\Omega$ allora le seguenti mappe sono distribuzioni per ogni $\al$
\[\funcDef{\Dc(\Omega)}{\R}{f}{\del^\al f(x_0)}\]
in quanto $\abs{\del^\al f(x_0)}\leq p_m(f)$ per $m=\abs{\al}$.
\item \textbf{Integrale contro $u\in L_{loc}^1$ fissata}: l'immagine di $T$, cio\`e le mappe della forma
\[T_u:\funcDef{\Dc(\Omega)}{\R}{f}{\int_\Omega fu\,dx}\]
sono distribuzioni.
\item Se $(x_j)\subseteq\Omega$ con $x_j$ che esce da ogni compatto definitivamente (va verso il bordo) allora
\[\funcDef{\Dc(\Omega)}{\R}{f}{\sum_{i=0}^\infty\del^{\al_i} f(x_i)}\]
\`e una distribuzione (qualunque sia la successione degli $\al_i\in\N^n$ che consideriamo, tanto su ogni compatto la somma \`e finita).
\end{enumerate}
\end{example}

\begin{definition}[Bracket di Iverson]
Il \textbf{Bracket di Iverson} per una condizione booleana $\vp$ su un insieme $A$ \`e la funzione caratteristica di quella condizione, cio\`e
\[[x]=\chi_{\cpa{x\in A\mid \vp(x)}}.\]
\end{definition}

\begin{definition}[]
Dato $\Omega\subseteq\R^n$ aperto definiamo
\[\Theta=\cpa{\theta:\Omega\times \N^n\to[0,\infty), \text{``localmente finita"}}\]
cio\`e per ogni $x\in\Omega$ esiste $U$ intorno di $x$ tale che per ogni $y\in U$
\[\cpa{\al\in\N^n\mid \theta(y,\al)\neq 0}\quad\text{\`e finito}.\]
\end{definition}

\begin{remark}
Se $\theta\in\Theta$ allora per ogni $K\in k(\Omega)$ si ha che esiste $N\in\N$ tale che per ogni $x\in K$ e per ogni $\abs{\al}\geq N$ vale $\theta(x,\al)=0$.
\end{remark}
\begin{remark}
Ogni $\theta\in\Theta$ \`e maggiorato da una $\wt \theta$ della forma
\[\wt \theta(x,\al)=\sigma(x)[\abs{\al}\leq \mu(x)]\]
dove $\sigma,\mu\in C^0(\Omega)_+$ e $[\cdot]$ \`e il Bracket di Iverson,
\[\mu(x)=\max\cpa{\abs{\al}\mid \theta(x,\al)\neq 0}\]
eccetera (vedi capitolo precedente).
\end{remark}

\begin{notation}
Per ogni $\theta\in \Theta$ e $u\in \Dc(\Omega)$ poniamo
\[\norm u_\theta=\norm{\theta\cdot \del^\bullet u(\bullet)}_{\infty,\Omega\times \N^n}\]
\end{notation}

\section{Estensioni e operazioni sulle distribuzioni}
\subsection{Estensioni}
Possiamo considerare estensioni di operatori su $\Dc(\Omega)$ a operatori su $\Dc'(\Omega)$ tramite le inclusioni
\[\Dc(\Omega)\subseteq C^1(\Omega)\subseteq L^1_{loc}\overset T\subseteq \Dc'(\Omega)\]
dove come prima
\[T:\funcDef{L^1_{loc}(\Omega)}{\Dc'(\Omega)}{u}{T_u:\funcDef{\Dc(\Omega)}{\R}{f}{\int_\Omega fu\,dx}}\]
\begin{remark}
Ponendo $\ps{T_u,\vp}=\int_\Omega u\vp dx$, per ogni $K\in k(\Omega)$ e $\vp\in C^\infty_K$ si ha
\[\abs{\ps{T_u,\vp}}\leq C_K\norm\vp_\infty\quad\text{dove }C_K=\int_K\abs udx=\norm u_{1,K}\text{ e }\norm\vp_\infty=p_0(\vp).\]
\end{remark}

\begin{exercise}
La mappa lineare $T:L^1_{loc}(\Omega)\to \Dc'(\Omega)$ \`e continua rispetto alle topologie
\begin{itemize}
    \item Su $L^1_{loc}(\Omega)$ consideriamo la topologia di spazio di Fr\'echet indotta dalle norme $\cpa{\normd_{1,K}}_{K\in k(\Omega)}$.
    \item Su $\Dc'(\Omega)$ consideriamo la topologia debole $\sigma(\Dc'(\Omega),\Dc(\Omega))$.
\end{itemize}
\end{exercise}
\begin{proof}
$T$ \`e continua per queste topologie se e solo se per ogni $\vp\in \Dc(\Omega)$ si ha che
\[\funcDef{L^1_{loc}(\Omega)}{\R}{u}{\ps{T_u,\vp}}\]
\`e continua e questo \`e vero se e solo se per ogni $K\in k(\Omega)$ esiste $C$ tale che per $u\in C^\infty_K$
\[\abs{\ps{T_u,\vp}}\leq C \norm{u}_{1,K}=C\int_K \abs{u}dx\]
ma questo \`e vero perch\'e
\[\abs{\ps{T_u,\vp}}=\abs{\int_\Omega u\vp dx}\leq \int_K\abs udx \norm\vp_{\infty}=\norm{u}_{1,K}\norm{\vp}_\infty.\]
\end{proof}

\begin{notation}
Se non c'\`e pericolo di confusione consideriamo $T$ come una inclusione e scriviamo $L^1_{loc}\subseteq \Dc'(\Omega)$ e $T_u=u$. Spesso si usa anche $u(\vp)$ al posto di $\ps{T_u,\vp}=\ps{u,\vp}$.
\end{notation}


\begin{definition}[Funzioni nulle al bordo]
Definiamo $C_0(X)$ come
\[C_0(X)=\cpa{f\in C(X)\mid \lim_{x\to\infty}f(x)= 0\text{ in }\wt X}\]
dove $\wt X=X\cup \cpa\infty$ \`e la compattificazione di Alexandroff.
\end{definition}

\begin{proposition}[Distribuzioni di ordine limitato si estendono a $(C_0^m)^\ast$]\label{PrDistribuzioniDiOrdineFinitoSiEstendonoAFunzionaliContinuiSuRegolariNulleAlBordo}
Le distribuzioni di ordine minore o uguale a $m$ si estendono a funzionali lineari continui su tutto 
\[C^m_0(\Omega)=\ol{C^\infty_C(\Omega)}^{C^m(\Omega)}=\cpa{f\in C^m(\Omega)\mid \forall \abs\al\leq m,\ \del^\al f\in C^0_0(\Omega)}.\]
\end{proposition}
\begin{proof}
Se $u\in \Dc'(\Omega)$ ha ordine $\leq m$ allora \`e continua per la topologia indotta da $C^m(\Omega)$ (infatti per ogni $K\in k(\Omega)$ esiste $C_K$ tale che $\abs{u(\vp)}\leq C_Kp_m(\vp)$). Quindi si estende per continuit\`a in modo unico a una forma lineare continua sulla chiusura (per la topologia di $C^m$), cio\`e $C^m_0(\Omega)\subseteq C^m(\R^n)$:

Fissiamo $K\in k(\Omega)$ e sia $dist(K,\Omega^c)=\e$. Sia $\vp$ funzione $C^\infty$ non negativa con supporto contenuto in $B(0,\e/4)$ e tale che $\int \vp=1$. Sia 
\[\eta=\vp\ast\chi_{K+\frac\e4B}.\]
Per costruzione $\eta=1$ su $K$ e $\eta=0$ in $\Omega\bs K+\frac\e2 B$.
Quindi dato $K\in k(\Omega)$ esiste $\eta\in C^\infty_C(\Omega)$ con $0\leq \eta\leq 1$ e $\eta\res K=1$.


Concludere mostrando che la chiusura in $C^m(\Omega)$ di $C_C^\infty(\Omega)$ \`e $C^m_0(\Omega)$ usando l'approssimazione via convoluzioni e la moltiplicazione per $\eta$.
\end{proof}

\subsection{Derivazione}
\begin{proposition}
L'operatore $\del_i$ di derivazione su $\Dc(\Omega)$ si estende ad un operatore su $\Dc'(\Omega)$ nel senso di sopra ponendo
\[\del_i=-\del_i^\ast:\funcDef{\Dc'(\Omega)}{\Dc'(\Omega)}{u}{-u\circ \del_i}\]
\end{proposition}
\begin{proof}
Per $u\in C^1(\Omega)$ e $\del_i u\in C^0(\Omega)$ si ha che $u$ e $\del_i u$ appartengono a $L^1_{loc(\Omega)}$. Le distribuzioni $T_u$ e $T_{\del_i u}$ sono legate dalla relazione data dall'integrazione per parti\footnote{i termini al bordo spariscono perch\'e tutto ha supporto compatto.}:
\[\ps{T_u,\del_i\vp}=\int_\Omega u\del_i\vp dx=-\int_\Omega\del_i u\vp dx=-T_{\del_i u} \vp\]
cio\`e $T_{\del_iu}=-T_u\circ \del_i$ (qu\`i $\del_i$ \`e inteso in senso classico).
\smallskip

\noindent
Definendo quindi
\[\del_i:\funcDef{\Dc'(\Omega)}{\Dc'(\Omega)}{u}{-u\circ \del_i}\]
abbiamo una estensione di $\del_i$ a $\Dc'(\Omega)$. Il codominio \`e effettivamente $\Dc'(\Omega)$ perch\'e $\del_i:\Dc(\Omega)\to\Dc(\Omega)$ \`e lineare e continua, quindi $-u\circ \del_i$ \`e composizione di due mappe lineari e continue.
\end{proof}



\begin{remark}
Pi\`u in generale \`e definito $\del^\al$ su $\Dc'(\Omega)$ e vale $\ps{\del^\al u,\vp}=(-1)^{\abs\al}\ps{u,\del^\al\vp}$.
\end{remark}



\begin{remark}
$\ord(\del_i u)\leq\ord(u)+1$.
\end{remark}



\subsection{Moltiplicazione per funzione liscia}
\begin{notation}
Definiamo $\Ec(\Omega)=C^\infty(\Omega)$.
\end{notation}

\begin{remark}
Se $f\in \Ec(\Omega)$ \`e definito un operatore lineare
\[M_f:\funcDef{\Dc(\Omega)}{\Dc(\Omega)}{\vp}{f\vp}\]
\end{remark}

\begin{remark}
$M_f$ \`e continuo.
\end{remark}
\begin{proof}
A livello dei $C^\infty_K$ il supporto resta contenuto in $K$ dopo la moltiplicazione e
\begin{align*}
    p_{m,K}(M_f(\vp))=&p_{m,K}(f\vp)=\max_{\abs\al\leq m}\norm{\del^\al (f\vp)}_\infty=\\
    =&\max_{\abs\al\leq m}\norm{\sum_{\beta\leq \al}\binom\al\beta\del^\beta f\del^{\al-\beta}\vp}_\infty\leq\\
    \leq& (2^m p_{m,K}(f))p_m(\vp).
\end{align*}
\end{proof}

\begin{remark}
Applicando $T$ \`e definito un operatore di moltiplicazione sulle distribuzioni
\[M_f:\funcDef{\Dc'(\Omega)}{\Dc'(\Omega)}{u}{fu}\]
dove
\[(fu)(\vp)=u(\vp f).\]
\end{remark}




\section{Distribuzioni di ordine limitato come misure}
\begin{remark}
C'\`e una immersione isometrica
\[\funcDef{C^m_0(\Omega)}{C^0_0(\wt \Omega)^N}{\vp}{(\del^\al\vp)_{\abs\al\leq m}}\]
dove $N=\#\cpa{\al\in\N^n\mid \abs\al\leq m}$ e $\wt \Omega$ \`e la compattificazione di $\Omega$ a un punto.
\smallskip

\noindent
Segue che $C^m_0(\Omega)^\ast\inj (C^0_0(\Omega)^\ast)^N$ per Hahn-Banach (\ref{ThHahnBanach}).
\end{remark}

\begin{fact}
Se $u\in \Dc'(\Omega)$ con $\ord(u)\leq m$, quindi tale che si estende a $u\in C^m_0(\Omega)^\ast$ (\ref{PrDistribuzioniDiOrdineFinitoSiEstendonoAFunzionaliContinuiSuRegolariNulleAlBordo}), allora grazie a $C^m_0(\Omega)^\ast\inj (C^0_0(\Omega)^\ast)^N$ otteniamo che esistono $N$ misure di Radon\footnote{boreliane, finite sui compatti, regolari da fuori sui Boreliani e regolari da dentro per gli aperti.} $\cpa{\mu_\al}_{\abs\al\leq m}$ tali che
\[\ps{u,\vp}=\sum_{\abs\al\leq m}\int_\Omega \vp d\mu_\al=\sum_{\abs\al\leq m}\vp\rho_\al d\nu_\al\]
con $\nu_\al\geq 0$ e $\abs\rho\leq 1$.
\end{fact}

\begin{remark}
Queste misure non sono uniche perch\'e abbiamo usato Hahn-Banach (\ref{ThHahnBanach}) per estendere $u$ da $C^m_0(\Omega)$ a $C^0_0(\wt \Omega)^N$.
\end{remark}


\begin{remark}
Le distribuzioni di ordine $0$ sono misure di Radon, cio\`e
\[\ord(u)=0\implies u(\vp)=\int_\Omega\vp d\mu\quad \mu\text{ misura relativa finita sui compatti.}\]
\end{remark}

\begin{fact}
    Ogni distribuzione positiva $u\in \Dc'(\Omega)$, cio\`e tale che $u(\vp)\geq 0$ per ogni $\vp\in \Dc(\Omega)$ tale che $\vp\geq 0$ ha ordine 0.
\end{fact}
\begin{proof}
Per ogni $K\in k(\Omega)$ sia $\eta\in C^\infty(\Omega)$ con $0\leq \eta\leq 1$ con $\supp\eta\subseteq \Omega$ e $\eta=1$ su $K$. Allora per ogni $\vp\in C^\infty_K$ vale
\[\norm\vp_\infty\eta\pm \vp\geq 0\]
infatti su un punto di $K$ $\eta=1$ e quindi applico la definizione di $\normd_\infty$, mentre su un punto che non appartiene a $K$ abbiamo $\vp=0$ e quindi la disuguaglianza continua a valere.
Dunque 
\[u(\norm\vp_\infty\eta\pm\vp)=\norm\vp_\infty u(\eta)\pm u(\vp)\geq 0\]
e quindi $\abs{u(\vp)}\leq u(\eta)\norm\vp_\infty=C_Kp_0(\vp)$, cio\`e $u$ ha ordine $0$.
\end{proof}


\section{Successioni di distribuzioni}
\begin{proposition}[]\label{PrSuccessioniDiDistribuzioni}
Sia $(u_j)\subseteq \Dc'(\Omega)$ una successione convergente puntualmente, cio\`e $(u_j(\vp))$ converge in $\R$ per ogni $\vp\in \Dc(\Omega)$. Allora il limite
\[u(\vp)=\lim_{j}u_j(\vp)\]
definisce una distribuzione $u$. Inoltre per ogni $K\in k(\Omega)$ esiste $C_K\geq 0$ e $m\in\N$ tale che per ogni $j\geq 0$
\[\abs{u_j(\vp)}\leq C_Kp_m(\vp)\quad\forall \vp\in C^\infty_K.\]
\end{proposition}
\begin{proof}
Sia $K\in k(\Omega)$. Notiamo che $u_j\res{C^\infty_K}$ \`e una successione puntualmente limitata su $C^\infty_K$ (per ogni $\vp\in C^\infty_K$ si ha $\abs{u_j(\vp)}\leq C_\vp$). Siccome $C^\infty_K$ \`e uno spazio di Fr\'echet (e quindi di Baire) per Banach-Steinhaus (\ref{ThBanachSteinhausUniformeLimitatezza}) la successione $(u_j\res{C^\infty_K})$ \`e limitata in $C^\infty_K$, cio\`e vale la disuguaglianza affermata.

Allora questa stima vale anche per $u$ limite puntuale, il quale \`e anche ovviamente lineare, quindi $u\in \Dc'(\Omega)$.
\end{proof}

\begin{corollary}
    Se $(\vp_j)\subseteq\Dc(\Omega)$ \`e una successione convergente in $\Dc(\Omega)$ a $\vp$ allora
    \[u_j(\vp_j)\to u(\vp)\]
\end{corollary}
\begin{proof}
$\vp_j\to \vp$ implica che esiste $K\in k(\Omega)$ tale che $\vp_j\in C^\infty_K$ e $\vp_j\to \vp$ in $C^\infty_K$ e ora che sappiamo che tutte le funzioni hanno supporto nello stesso $K$ possiamo usare la disguguaglianza
\[\abs{u_j(\vp_j)}\leq C_Kp_m(\vp_j)\to 0\]
dove per l'ultimo limite ho supposto senza perdita di generalit\`a $\vp=0$ (altrimenti sostituisco $\vp_j$ con $\vp_j-\vp$).
\end{proof}


\section{Distribuzioni sono un fascio}
\begin{proposition}\label{PrDistribuzioniSonoUnFascio}
Il funtore $\Dc':(\text{Aperti di }\R^n)^{op}\to (SVTLC)$ definisce un fascio, cio\`e:
\begin{enumerate}
    \item Per ogni aperto $\Omega$ di $\R^n$ \`e ben definito $\Dc'(\Omega)$.
    \item Per ogni contenimento $U\subseteq V\subseteq \R^n$ di aperti abbiamo una funzione di restrizione
    \[\rho^V_U:\Dc'(V)\to\Dc'(U)\]
    tale che $\rho^U_U=id_{\Dc'(U)}$ e se $U\subseteq V\subseteq W$ allora 
    \[\rho^V_U\circ\rho^W_V=\rho^W_U.\]
    \item Se $\Omega$ aperto ammette un ricoprimento aperto $\cpa{\Omega_i}$ e $u\in \Dc'(\Omega)$ allora $\rho^\Omega_{\Omega_i}(u)\doteqdot u_i=0$ per ogni $i$ implica che $u=0$.
    \item Se $\Omega$ aperto ammette un ricoprimento aperto $\cpa{\Omega_i}$ e per ogni $i$ abbiamo $u_i\in \Dc'(\Omega_i)$ tali che 
    \[\rho^{\Omega_i}_{\Omega_{i}\cap \Omega_j}(u_i)=\rho^{\Omega_j}_{\Omega_{i}\cap \Omega_j}(u_j)\]
    per ogni coppia $i,j$ allora esiste $u\in \Dc'(\Omega)$ tale che $\rho^\Omega_{\Omega_i}(u)= u_i$.
\end{enumerate}
\end{proposition}
\begin{proof}
Mostriamo le varie propriet\`a:
\setlength{\leftmargini}{0cm}
\begin{enumerate}
    \item Ovvio.
    \item Dati due aperti $U\subseteq V\subseteq \R^n$ esiste una inclusione
    \[\Dc(U)=\bigcup_{K\in k(U)}C^\infty_K\inj \Dc(V)=\bigcup_{K\in k(V)}C^\infty_K\]
    e quindi un operatore di restrizione
    \[\rho^V_U:\Dc'(V)\to \Dc'(U).\]
    Questo operatore chiaramente rispetta le due propriet\`a.
    \item Sia $u\in \Dc(\Omega)$ tale che
\[u\res{\Omega_j}=\rho^\Omega_{\Omega_j}(u)=0.\]
Per ogni $\vp$ in $\Dc(\Omega)$ esiste $F\subseteq I$ finito tale che $K=\supp \vp\subseteq \bigcup_{j\in F}\Omega_j$. Esiste inoltre una partizione di unit\`a $\cpa{\eta_j}_{j\in F}\subseteq \Dc(\Omega)$ tale che
\[\eta_j\in \Dc(\Omega_j)\quad\text e\quad \sum_{j\in F}\eta_j=1\text{ su }K.\]
Allora $f=\sum_{j\in F} f\eta_j$ e $u(f)=\sum_{j\in F}u(f\eta_j)=0$ in quanto $u\res{\Omega_j}=0$.
\item Sia $\cpa{\Omega_i}$ un ricoprimento aperto di $\Omega$ e per ogni $i$ abbiamo $u_i\in \Dc'(\Omega_i)$ tali che 
\[\rho^{\Omega_i}_{\Omega_{i}\cap \Omega_j}(u_i)=\rho^{\Omega_j}_{\Omega_{i}\cap \Omega_j}(u_j)\]
per ogni coppia $i,j$. Definiamo $u(\vp)$ per $\vp\in\Dc(\Omega)$ come segue:

Sia $K\in k(\Omega)$ tale che $\vp\in C^\infty_K$ e sia $F\subseteq I$ finito tale che $K\subseteq \bigcup_{i\in F}\Omega_i$. Siano $\cpa{\eta_j}_{j\in F}\subseteq C^\infty(\Omega)$ tali che $\supp\eta_j\subseteq \Omega_j$, $0\leq \eta_j\leq 1$ e $\sum_{j\in F}\eta_j=1$ su $K$.
Poniamo
\[u(\vp)=\sum_{j\in F}u_j(\vp\eta_j)\]
Notiamo che $\vp\eta_j\in\Dc(\Omega_j)$ quindi ha senso valutare $u_j$ nel prodotto.
La definizione non dipende dalla famiglia $\cpa{\eta_j}$ in quanto se $\eta_j'$ ha le stesse propriet\`a allora
\begin{align*}
    \sum_{j\in F}u_j(\vp\eta_j)=&\sum_{j\in F}u_j\pa{\sum_{i\in F}\vp\eta_j\eta_i'}=\sum_{i,j\in F}u_j(\vp\eta_j\eta_i')\pasgnl={ipotesi}\\
    =&\sum_{i,j\in F}u_i(\vp\eta_j\eta_i')=\sum_{i\in F}u_i(\vp\eta_i').
\end{align*}
Per costruzione $u$ eredita la linearit\`a e la continuit\`a delle $u_i$, quindi \`e un elemento di $\Dc'(\Omega)$.
\end{enumerate}
\setlength{\leftmargini}{0.5cm}
\end{proof}

\begin{remark}
Si pu\`o considerare pi\`u in generale il fascio delle distribuzioni su una variet\`a $C^\infty$ di dimensione $n$, basta incollare i fasci di distribuzioni su un ricoprimento di aperti omeomorfi a $\R^n$.
\end{remark}

\subsection{Distribuzioni a supporto compatto}
\begin{definition}[Supporto di una distribuzione]
Fissiamo una distribuzione $u\in \Dc'(\Omega)$. Sia $\Omega_0$ il pi\`u grande aperto tale che $u\res{\Omega_0}=0$. Il chiuso $\Omega\bs \Omega_0$ si dice \textbf{supporto} di $u$ e si indica $\supp(u)$.
\end{definition}
\begin{remark}
$\Omega_0$ \`e ben definito in quanto \`e l'unione di tutti gli aperti dove $u$ si restringe alla mappa nulla: Poich\'e $\Dc'$ \`e un fascio (\ref{PrDistribuzioniSonoUnFascio}) e per definizione $u\res{\Omega_0}$ ha tutte le restizioni a $U\subseteq \Omega_0$ aperto banali, $u\res{\Omega_0}=0$.
\end{remark}

\begin{definition}[Distribuzione a supporto compatto]
Se $\supp(u)$ \`e compatto, $u$ si dice \textbf{a supporto compatto}. Scriviamo l'insieme delle distribuzioni a supporto compatto con $\Dc'_C(\Omega)$.
\end{definition}

\begin{proposition}\label{PrSupportoCompattoEContinuita}
Se $u\in \Dc_C'(\Omega)$ e $K\in k(\Omega)$ allora valgono le implicazioni dall'alto verso il basso
\begin{enumerate}
    \item $\supp(u)\subseteq int(K)$.
    \item Esistono $C\geq 0$ e $m\in\N$ tali che per ogni $\vp\in\Dc(\Omega)$ si ha 
    \[\abs{u(\vp)}\leq Cp_{m,K}(\vp),\]
    cio\`e $u$ \`e continua.
    \item $\supp(u)\subseteq K$.
\end{enumerate}
\end{proposition}
\begin{proof}
    Mostriamo le due implicazioni
\setlength{\leftmargini}{0cm}
\begin{itemize}
\item[$\boxed{1.\implies2.}$] Siano $\supp(u)\subseteq int(K)$ e $\psi\in C^\infty(\Omega)$ con $\supp(\psi)\subseteq K$ e $\psi=1$ su un intorno $U$ di $\supp(u)$.

Allora per ogni $\vp\in \Dc(\Omega)$ si ha che $(1-\psi)\vp$ \` e nulla su $U$, quindi
\[\cpa{(1-\psi)\vp\neq 0}\subseteq \Omega\bs U\implies \supp((1-\psi)\vp)=\ol{\cpa{(1-\psi)\vp\neq 0}}\subseteq \Omega\bs U\]
Segue che $(1-\psi)\vp$ e $u$ hanno supporto disgiunto, dunque
\[0=u((1-\psi)\vp)=u(\vp)-u(\psi\vp),\]
cio\`e per ogni $\vp\in \Dc(\Omega)$ si ha
\[u(\vp)=u(\psi\vp).\]
Per continuit\`a di $u$ come elemento di $\Dc'(\Omega)$, poich\'e $\psi\vp\in C^\infty_K$, esistono $m\in \N$ e $C\geq 0$ tali che
\[\abs{u(\vp)}=\abs{u(\psi\vp)}\leq Cp_{m,K}(\psi\vp)\leq C'p_{m,K}(\vp)\]
dove l'ultima stima \`e un conto gi\`a visto che usa la formula di Leibnitz\footnote{\[p_{m,K}(\psi\vp)=\max_{\abs\al\leq m}\norm{\del^\al(\psi\vp)}_{\infty,K}=\max_{\abs\al\leq m}\norm{\sum_{\beta\leq \al}\binom \al\beta\del^{\al-\beta}\psi\del^{\beta} vp}_{\infty,K}\leq 2^mp_m(\psi)p_{m,K}(\vp).\]}, quindi $u$ \`e continua per la topologia indotta\footnote{e quindi potremmo estendere $u$ con Hahn-Banach (\ref{ThHahnBanach}).} da $\Ec(\Omega)$ su $\Dc(\Omega)$.
\item[$\boxed{2.\implies3.}$] Se per ogni $\vp\in \Dc(\Omega)$ vale $\abs{u(\vp)}\leq Cp_{m,K}(\vp)$ allora in particolare vale se $\vp$ ha supporto in $\Omega\bs K$, ma in tal caso $p_{m,K}(\vp)=0$, cio\`e $u$ \`e nulla su $\Dc(\Omega\bs K)$ e quindi il supporto \`e contenuto in $K$.
\end{itemize}
\setlength{\leftmargini}{0.5cm}
\end{proof}
\begin{corollary}
Le distribuzioni a supporto compatto in $\Omega$ si possono identificare con gli elementi di\footnote{la topologia su $\Ec(\Omega)$ \`e quella indotta dalle seminorme $\cpa{p_{m,K}}_{m\in\N,\ K\in k(\Omega)}$.} $\Ec'(\Omega)=(C^\infty(\Omega))^\ast$.
\end{corollary}
\begin{proof}
Se $u$ ha supporto compatto \`e continua per la topologia indotta da $\Ec(\Omega)$ su $\Dc(\Omega)$ e quindi per Hahn-Banach (\ref{ThHahnBanach}) si estende ad una forma lineare continua su tutto $\Ec(\Omega)$. Questa estensione \`e in realt\`a unica perch\'e $\Dc(\Omega)$ \`e denso in $\Ec(\Omega)$. In questo senso possiamo identificare $\Ec'(\Omega)$ con $\Dc_C'(\Omega)$ come spazi vettoriali.
\end{proof}

\begin{remark}
Se $u\in \Dc_C'(\Omega)$ allora ha anche ordine finito per il punto 1. della proposizione sopra (\ref{PrSupportoCompattoEContinuita}).
\end{remark}

\begin{example}[Non vale 3.$\implies$2. di (\ref{PrSupportoCompattoEContinuita})]
Sia $n=1$, $\Omega=\R$ e consideriamo $u\in\Dc'(\R)$ tale che
\[u(\vp)=\sum_{k\geq 1}\frac1k\pa{\vp\pa{\frac1k}-\vp(0)},\quad \forall \vp\in \Dc(\R).\]
La serie \`e assolutamente convergente perch\'e $\abs{\vp(\frac1k)-\vp(0)}\leq\norm{\dot\vp}_\infty \frac1k$ per Lagrange e 
\[\sum_{k\geq 1}\frac1k\abs{\vp\pa{\frac1k}-\vp(0)}\leq \pa{\sum_{k\geq 1}\frac1{k^2}}\norm{\dot\vp}_\infty.\]
Da questa scrittura si vede anche che $u$ dipende da $\vp$ con continuit\`a rispetto alla norma $\norm{\del\bullet}_{\infty}$.

Se $\vp$ ha supporto disgiunto da $K=\cpa0\cup\cpa{\frac1k}_{k\geq0}$ allora $\supp(u)\subseteq K$ (cio\`e vale la condizione 3.).

Eppure non vale la condizione 2. per $K$ infatti per $m\in \N$ sia $\vp_m$ una funzione $\Dc(\Omega)$ tale che $\vp_m=0$ su un intorno di $[0,\frac1{m+1}]$ e $\vp_m=1$ su un intorno di $[\frac1m,1]$, allora
\[\vp_m\pa{\frac1k}=\chi_{\cpa{k\leq m}},\quad \vp_m^{(j)}(x)=0\ \forall x\in K,\ \forall j\geq 1\]
perci\`o $p_{m,K}(\vp_m)=\norm{\vp_m}_{\infty}=1$ quindi
\[u(\vp_m)=\sum_{k=1}^m\frac1k\]
cio\`e $u$ non \`e limitata e quindi non esistono $m,C$ tali che 
\[\abs{u(\vp)}\leq C p_{m,K}(\vp)\quad \forall \vp\in \Dc(\R)\]
\end{example}


\begin{example}
Se $K$ \`e un singoletto $\cpa{x_0}$ per $x_0\in \Omega$ allora le implicazioni di (\ref{PrSupportoCompattoEContinuita}) si possono invertire: Se $\supp(u)=\cpa{x_0}$ allora esistono $m\in \N$ e costanti $\cpa{c_\al}_{\abs\al\leq m}$ tali che per ogni $\vp\in \Dc(\Omega)$
\[u(\vp)=\sum_{\abs{\al}\leq m}c_\al\del^\al \vp(x_0).\]
Risulta $c_\al=u\pa{\frac{(x-x_0)^\al}{\al!}}$:
\[u\pa{\frac{(x-x_0)^\al}{\al!}}=\sum_{\abs\beta\leq m}\frac{c_\beta}{\al!}\del^\beta((x-x_0)^\al)=\frac{c_\al}{\al!} \del^\al(x-x_0)^\al=c_\al\cdot 1.\]
\end{example}
\begin{proof}
Senza perdita di generalit\`a sia $x_0=0$ e $u\in\Dc'(\Omega)$ con $\supp(u)=\cpa{0}$. Scegliamo $\eta\in C^\infty(\R^n)$ con $\supp(\eta)\subseteq B(0,2)$ e $\eta\res{B(0,1)}=1$.

Definiamo
\[\eta_\e(x)=\eta\pa{\frac x\e}\quad\leadsto\quad \supp(\eta_\e)\subseteq B(0.2\e),\ \eta_\e\res{B(0,\e)}=1\]
Quindi 
\[\del^\al\eta_\e(x)=\e^{-\abs\al}\del^\al\eta\pa{\frac x\e}.\]
Per ogni $\vp\in \Dc(\Omega)$, $\eta_\e\vp\in \Dc(\Omega)$ e $(1-\eta_\e)\vp$ ha supporto su $\Omega\bs B(0,\e)$, quindi la $u$ su annulla su questa funzione. Dunque per ogni $\e>0$
\[u(\vp)=u(\eta_\e\vp)\]
Mostriamo il seguente caso particolare: se $\vp\in\Dc(\Omega)$ \`e tale che $\del^\al\vp(0)=0$ per ogni $\abs\al\leq m$ per un qualche $m$ allora $u(\vp)=0$. In queste ipotesi si ha grazie alla continuit\`a di $u$ su $C^\infty_{\ol{B(0,\e_0)}}$ e al fatto che $\eta_\e\vp\in \Dc(B(0,\e_0))$ per ogni $\e<\e_0$ che
\[\abs{u(\vp)}=\abs{u(\eta_\e\vp)}\leq C_0 p_{m,B(0,\e_0)}(\eta_\e\vp)\]
Si conclude che $u(\vp)=0$ osservando che 
\[p_{m,B(0,\e_0)}(\eta_\e\vp)\overset{\supp(\eta_\e)\subseteq B(0,\e)}=p_{m}(\eta_\e\vp)=o(1)\] 
per $\e\to0$, cio\`e $\eta_\e\vp\to 0$ in $C^m(\R^n)$: poich\'e $\del^\al\eta_\e(x)=\e^{-\abs\al}\del^\al \eta\pa{\frac x\e}$ si ha $p_m(\eta_\e)=\e^{-m}p_m(\eta)$. D'altra parte dalla formula di Taylor, poich\'e $\del^\al\vp(0)=0$ per ogni $\abs\al\leq m$, si ha
\[\abs{\vp(x)}=O(\abs{x}^{m+1}),\qquad \abs{\del^\al\vp(x)}=O(\abs{x}^{m-\abs\al+1})\]
per $x\to 0$. Allora
\[p_m(\eta_\e\vp)\leq 2^m \max_{\smat{\abs\la\leq m\\ \beta\leq\al\\x\in B(0,2\e)}}\abs{\del^{\al-\beta }\eta_\e\del^\beta\vp}=O(\e^{-m+\abs\beta}\e^{m-\abs\beta+1})=O(\e)\]
Quindi $u(\vp)=0$ per ogni $\vp\in \Dc(\Omega)$ con $\del^\al\vp(0)=0$ per $\abs\al\leq m$.

\bigskip

Ora consideriamo $\vp$ qualunque. Dalla formula di Taylor
\[\vp(x)=\sum_{\abs\al\leq m}\frac1{\al!}\del^\al\vp(0)x^\al + \rho(x)\]
con $\rho\in \Dc(\Omega)$ tale che $\del^\al\rho(0)=0$ per ogni $\abs\al\leq m$. Mettendo tutto insieme abbiamo finito perch\'e $u(\rho)=0$ e
\[u(\vp)=\sum_{\abs\al\leq m}\frac1{\al!}c_\al\del^\al\vp(0)\]
con $c_\al=u(x^\al)$.
\end{proof}


















\begin{exercise}
[Convoluzione di una funzione $C^\infty$ a supp.cpt. e una distribuzione.]
Per ogni $f\in \Dc(\R^n)$ \`e definita la mappa
\[\funcDef{\Ec(\R^n)}{\Ec(\R^n)}{\vp}{f\ast\vp}\]
Per ogni $f\in \Ec(\R^n)$, la convoluzione induce una mappa.
\[\Dc(\R^n)\to \Ec(\R^n)\]
Queste mappe sono continue e lineari. Restano definite le trasposte
\[\Ec'(\R^n)\to\Ec'(\R^n),\qquad \Ec'(\R^n)\to \Dc'(\R^n)\]
tali che $u\mapsto \wt u=u\circ (f\ast\bullet)$, cio\`e\footnote{dove $g$ appartiene a $\Ec(\R^n)$ nel primo caso e a $\Dc(\R^n)$ nel secondo.} $\wt u(g)=u(f\ast g)$.

Tenendo presente le propriet\`a della convoluzione questo fornisce una estensione dell'operazione di convoluzione alle distribuzioni.

Cosa si pu\`o dire sulla continuit\`a dell'operazione (per esempio con $(\Ec',\sigma(\Ec',\Ec))$ e $(\Dc',\sigma(\Dc',\Dc))$)?
\end{exercise}













