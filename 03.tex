\chapter{Limitatezza e Banach-Steinhaus}

\section{Limitatezza}
\begin{definition}[Insieme limitato]
Un sottoinsieme $S$ di uno SVT $X$ con $\Uc$ intorni di $0$ \`e \textbf{limitato} se \`e assorbito da ogni elemento di $\Uc$, cio\`e\footnote{questa condizione \`e equivalente a chiedere $tU\supseteq S$ per ogni $t$ con $\abs t\geq n$ o a chiedere che l'assorbimento valga per elementi di una pre-base di intorni di $0$ al posto di tutti gli elementi di $\Uc$.} per ogni $U\in\Uc$ esiste $n\in\N$ tale che $nU\supseteq S$.
\end{definition}
\begin{remark}
Valgono le seguenti propriet\`a
\begin{enumerate}
    \item Se $S$ \`e limitato allora anche $\ol S$ lo \`e, basta considerare intorni chiusi.
    \item Se $S$ e $S'$ sono limitati, $S\cup S'$ lo \`e.
    \item Ogni compatto \`e limitato, basta scegliere un intorno limitato di $x$ per ogni $x\in K$ e poi estrarre un sottoricoprimento finito. Un tale intorno esiste scalando intorni di $0$ bilanciati.
    \item Ogni $T:X\to Y$ lineare e continua tra SVT \`e limitata, cio\`e per ogni $S\subseteq X$ limitato, $T(S)$ \`e limitato. In generale non vale il viceversa ma vale se $X$ e $Y$ sono normati.
\end{enumerate}
\end{remark}

\begin{proposition}[Limitatezza in SVTLC]\label{PrLimitatezzaInSVTLC}
Se $(X,\Pc)$ \`e SVTLC allora $S\subseteq X$ \`e limitato se e solo se per ogni seminorma $p\in \Pc$, $p$ \`e limitata su $S$.
\end{proposition}
\begin{proof}
$p$ limitata su $S$ significa che 
\[S\subseteq B_p(0,R_p)=\frac{R_p}\e B_p(0,\e)\]
e le palle $\cpa{B_p(0,\e)}_{p\in\Pc, \e>0}$ sono una prebase di intorni di $0\in X$.
\end{proof}
\begin{corollary}
Se $(X,\normd)$ \`e normato allora $S$ \`e limitato se e solo se $\exists R>0$ tale che $S\subseteq B(0,R)$.
\end{corollary}

\begin{exercise}
Se $X$ \`e I-numerabile e $T:X\to Y$ lineare tale che per ogni $x_k\to 0$ in $X$ esiste $x_{k_j}$ tale che $T(x_{k_j})$ limitata allora $T$ \`e continua.
\end{exercise}

\begin{proposition}[Caratterizzazione sequenziale della limitatezza]\label{PrCaratterizzazioneSequenzialeLimitatezza}
Se $X$ SVT e $S\subseteq X$, $S$ \`e limitato se e solo se per ogni $(s_k)$ successione in $S$ e per ogni $(\al_k)$ successione in $\K$ infinitesima, si ha $\al_k s_k\to 0$.
\end{proposition}
\begin{proof}
Sia $S$ limitato, $(s_k)$ successione in $S$ e $(\al_k)$ successione infinitesima in $\K$. Sia $U$ intorno bilanciato di $0$ e sia $n$ tale che $S\subseteq nU$. Notiamo che definitivamente $\abs{\al_k}<\frac1n$, quindi
\[\al_k s_k\in \al_k S\subseteq \al_k nU\pasgnl\subseteq{k grande} U.\]

Supponiamo ora $S$ non limitato, allora esiste $U\in\Uc_X$ che non assorbe $S$, cio\`e per ogni $n\in\N$ esiste $s_n\in S\bs nU$. Dunque $(s_n)$ \`e una successione in $S$ tale che $\frac1n s_n\notin U$ per costruzione, dunque $\frac1n s_n$ non tende a $0\in X$ nonostante $\frac1n$ sia infinitesima.
\end{proof}

\begin{proposition}\label{PrSuccessioniCauchyLimitate}
Le successioni di Cauchy sono limitate.
\end{proposition}
\begin{proof}
Sia $(x_k)$ una successione di Cauchy in $X$, cio\`e per ogni $U\in \Uc_X$ esiste $n\in\N$ tale che per ogni $p,q\geq n$ si ha $x_p-x_q\in U$.

Fissiamo $U\in \Uc_X$ e sia $V$ bilanciato tale che $V+V\subseteq U$. Per la definizione di successione di Cauchy esiste $n_0$ tale che $x_k-x_{n_0}\in V$ per ogni $k\geq n_0$, cio\`e $x_k\in x_{n_0}+V$.

Inoltre, esiste $m$ tale che $x_k\in mV$ per ogni $k\leq n_0$ dato che un insieme finito \`e limitato. Allora per ogni $k\in\N$ si ha $x_k\in mV+V$, infatti se $k\leq n_0$ allora abbiamo $mV$, se $k>n_0$ allora $x_{n_0}\in mV$ e $x_k\in x_{n_0}+V\subseteq mV+V$.

Poich\'e $V$ \`e bilanciato, $mV+V\subseteq mV+mV=m(V+V)\subseteq mU$.
\end{proof}

\section{Spazi di Baire e II-categoria}
\begin{theorem}[Baire]\label{ThBaire}
Se $\cpa{A_k}_{k\geq 0}$ \`e una famiglia numerabile di aperti densi di uno spazio metrico completo allora $\bigcap A_k$ \`e denso.
\end{theorem}
\begin{proof}
Per induzione si definisce una successione di palle chiuse di $X$ dove $B_0$ \`e arbitraria e
\[B_k=\ol{B(x_k,r_k)}\text{ tali che }B_{k+1}\subseteq B_k\cap A_k\text{ e }r_k=o(1)\]
che possiamo fare perch\'e $A_k$ \`e un aperto denso.

Allora la successione dei centri \`e una successione di Cauchy, infatti se $p,q\geq n$ si ha $x_p,x_q\in B_n$ e quindi $d(x_p,x_q)\leq 2r_n$. Dunque $x_n\to x^\ast$ in $X$ per completezza. Inoltre, poich\'e $x_k\in B_n$ definitivamente, $x^\ast=\lim x_k\in B_n$ per ogni $n$ (dato che $B_n$ \`e chiuso). In particolare $x^\ast\in B_{n+1}\subseteq A_n$ per ogni $n$ e quindi $x^\ast\in \bigcap A_n$. Per costruzione $x^\ast\in B_0$, quindi per ogni palla $B_0$ abbiamo mostrato che $B_0\cap \bigcap A_n\neq \emptyset$, cio\`e $\bigcap A_n$ \`e denso.
\end{proof}

\begin{exercise}
La stessa conclusione vale se $X$ \`e localmente compatto al posto di metrico completo.
\end{exercise}

\begin{definition}[Spazio di Baire]
Uno spazio topologico \`e \textbf{di Baire} se ogni intersezione numerabile di aperti densi \`e densa.
\end{definition}

\begin{remark}
Ogni aperto non vuoto di $X$ di Baire \`e ancora di Baire. Basta verificare che ogni aperto denso di $A$ \`e della forma $A\cap U$ con $U$ aperto denso di $X$.
\end{remark}


\begin{definition}[Sottoinsieme di I- e II-categoria]
Un sottoinsieme $S$ di $X$ \`e di \textbf{I-categoria (di Baire) in $X$} se \`e unione numerabile di insiemi $(E_i)_{i\in\N}$ con $int(\ol{E_i})=\emptyset$.

Inoltre $S$ \`e di \textbf{II-categoria (di Baire) in $X$} se non \`e di I-categoria.
\end{definition}

\begin{remark}
Se $X$ \`e di Baire e $S\subseteq X$ \`e di I-categoria allora $X\bs S$ \`e di II-categoria in quanto $X$ stesso \`e di II-categoria (se $X=\bigcup E_i$ con $E_i$ chiusi a parte interna vuota allora $\emptyset=\bigcap E_i^c$ con $E_i^c$ aperti densi, ma questo \`e assurdo perch\'e $X$ di Baire).
\end{remark}

\section{Teorema di Banach-Steinhaus}

\begin{definition}[Famiglia equicontinua]
Una famiglia $\Gamma$ di operatori lineari continui fra SVT $X$ e $Y$ \`e \textbf{equicontinua} se per ogni $U\in\Uc_Y$ esiste $V\in \Uc_X$ tale che per ogni $T\in\Gamma$, $T(V)\subseteq U$.
\end{definition}

\begin{remark}
Possiamo riformulare la condizione nei seguenti modi: per ogni $U\in\Uc_Y$ esiste $V\in\Uc_X$ tale che
\[\forall T\in\Gamma,\ V\subseteq T\ii(U)\coimplies V\subseteq \bigcap_{T\in \Gamma}T\ii(U)\doteqdot \Gamma\ii(U).\]
Equivalentemente la condizione predica che per ogni $V\in\Uc_Y$ si abbia $\Gamma\ii(V)\in\Uc_X$.
\end{remark}

\begin{remark}
Se $T:X\to Y$ fra spazi normati, la norma degli operatori
\[\norm T=\norm T_{\infty,B(0,1)}=\text{migliore costante di Lipschitz per }T.\]
\end{remark}

\begin{example}
Se $X$ e $Y$ sono normati, $\Gamma$ \`e equicontinua se e solo se $\Gamma$ \`e limitato in $L(X,Y)$ rispetto alla norma degli operatori.
\end{example}

\begin{theorem}[Banach-Steinhaus / Uniforme limitatezza]\label{ThBanachSteinhausUniformeLimitatezza}
Siano $X,Y$ SVT, $S\subseteq X$ di seconda categoria e $\Gamma\subseteq L(X,Y)$ con $\Gamma$ puntualmente limitata su $S\subseteq X$, cio\`e per ogni $s\in S$, $\Gamma(s)=\bigcup_{T\in \Gamma} T(s)$ \`e limitato in $Y$.

Allora $\Gamma$ \`e equicontinua.
\end{theorem}
\begin{proof}
Sia $U\in\Uc_Y$ e consideriamo $V\in\Uc_Y$ chiuso tale che $V-V\subseteq U$. Per ipotesi, per ogni $x\in S$ si ha che $\Gamma(x)$ \`e limitato in $Y$, quindi viene assorbito da $V$, cio\`e esiste $n_x\in\N$ tale che per ogni $T\in\Gamma$ si ha $T(x)\in n_x V$, cio\`e tale che
\[x\in\bigcap_{T\in \Gamma}n_xT\ii(V)=n_x\bigcap_{T\in \Gamma}T\ii(V)=n_x\Gamma\ii(V).\]
Dunque $S\subseteq \bigcup_{n\in\N}n\Gamma\ii(V)$. Notiamo che poich\'e $V$ \`e chiuso, $T\ii(V)$ \`e chiuso e quindi anche $\Gamma\ii(V)$ lo \`e perch\'e intersezione di chiusi. Poich\'e $S$ \`e di seconda categoria anche l'unione delle versioni riscalate di $\Gamma\ii(V)$ lo \`e, dunque questo insieme non \`e unione numerabile di chiusi con parte interna vuota, quindi almeno uno tra gli $n\Gamma\ii(V)$ ha parte interna non vuota, quindi anche $\Gamma\ii(V)$ ha parte interna non vuota scalando per $\frac1n$.

Quindi $\Gamma\ii(V)$ \`e intorno di qualche suo punto, dunque\footnote{se $a_0\in int(A)$ allora $A-a_0\subseteq A-A$ \`e un intorno di $0$.} $\Gamma\ii(V)-\Gamma\ii(V)$ \`e un intorno di $0$. 

Ricordando che $V-V\subseteq U$ si ha
\[T\ii(U\supseteq T\ii(V-V)=T\ii(V)-T\ii(V))\supseteq \Gamma\ii(V)-\Gamma\ii(V)\]
quindi passando all'intersezione su $T\in\Gamma$ si ha
\[\Gamma\ii(U)\supseteq \Gamma\ii(V)-\Gamma\ii(V)\in\Uc_X,\]
cio\`e abbiamo mostrato che per ogni $U\in\Uc_Y$ si ha $\Gamma\ii(U)\in\Uc_X$, che \`e equivalente all'equicontinuit\`a di $\Gamma$.
\end{proof}

\begin{corollary}[Sottoinsiemi limitati di operatori]\label{CorPuntualmenteLimitatoImplicaLimitatoPerNormaOperatore}
    Se $X$ e $Y$ sono Banach e $\Gamma\subseteq L(X,Y)$ \`e puntualmente limitata in $X$ (o volendo anche un sottoinseme di $X$ di II-categoria) allora $\Gamma$ \`e un insieme limitato in $L(X,Y)$.
\end{corollary}
\begin{proof}
Diretta applicazione di Banach-Stenhaus (\ref{ThBanachSteinhausUniformeLimitatezza}) notando che spazi di Banach sono in particolare SVT e che equicontinuit\`a per la norma su $L(X,Y)$ significa limitatezza. **************
\end{proof}

\begin{exercise}
Siano $X,Y$ SVT. Trovare la topologia meno fine $\tau$ di SVT su $L(X,Y)$ per la quale
\[\Gamma\text{ puntualmente limitato in $L(X,Y)$}\coimplies \Gamma\text{ limitato nella topologia $\tau$}.\]
\end{exercise}


\begin{corollary}\label{CorConvergenzaLineariSePuntualmenteConvergente}
Siano $X$ e $Y$ Banach e sia $(T_n)\subseteq L(X,Y)$ puntualmente convergente. Allora il limite $T$ \`e ancora lineare, continuo e con norma
\[\norm T\leq \liminf_{n\to\infty}\norm{T_n}.\]
\end{corollary}
\begin{proof}
Per il corollario precedente (\ref{CorPuntualmenteLimitatoImplicaLimitatoPerNormaOperatore}) si ha che $(T_n)$ sono limitati in $\normd$ e il limite puntuale \`e lineare in quanto
\[T_n(\al x+\beta y)=\al T_n(x)+\beta T_n(y)\to \al T(x)+\beta T(y).\]
Questo mostra che $T$ \`e limitato e lineare, quindi $T\in L(X,Y)$.

Inoltre per ogni $x\in X$ si ha
\[\norm{T(x)}=\lim_{n}\norm{T_n(x)}\leq \pa{\sup_n\norm{T_n}}\norm x,\]
quindi $\norm T\leq \sup_n\norm{T_n}$. Ragionando analogamente per una sottosuccessione di $(T_n)$ che in norma converge a $\liminf_n\norm{T_n}$ ricaviamo
\[\norm T\leq \liminf_n\norm{T_n}.\]
\end{proof}

\begin{remark}
In generale NON vale $T_n\to T$ in $\normd$.
\end{remark}

\begin{proposition}[Bilineare separatamente continua \`e continua]\label{PrBilineareSeparatamenteContinuaEContinua}
Sia $b:X\times Y\to Z$ bilineare e separatamente continua, cio\`e per ogni $x\in X, y\in Y$ si ha che $b(x,\cdot):Y\to Z$ e $b(\cdot, y):X\to Z$ sono lineari e continue. Allora $b$ \`e continua, cio\`e
\[\sup_{\norm x\leq 1, \norm y\leq 1}\norm{b(x,y)}<\infty.\]
\end{proposition}
\begin{proof}
Consideriamo la famiglia
\[\Gamma=\cpa{b(x,\cdot):Y\to Z}_{x\in X,\ \norm x\leq 1}\subseteq L(Y,Z).\]
Per ipotesi $\Gamma$ \`e puntualmente limitata in $Y$, infatti per ogni $y\in Y$
\[\sup_{b(x,\cdot)\in\Gamma} \norm{b(x,\cdot)}_{L(Y,Z)}=\sup_{\norm x\leq 1}\norm{b(x,y)}_Z= \norm{b(\cdot,y)}_{L(X,Z)}\norm y<\infty\]
Allora $\Gamma$ \`e limitata in $\normd_{L(Y,Z)}$, cio\`e per ogni $x\in X$ tale che $\norm x\leq 1$ si ha
\[\norm{b(x,y)}_Z\leq M\norm y\]
e quindi al variare di $y$ con $\norm y\leq 1$ troviamo $\norm b_{L(X\times Y,Z)}\leq M$.
\end{proof}

\begin{exercise}
Esiste una isometria lineare
\[\funcDef{L(X,L(Y,Z))}{L^2(X\times Y,Z)}{T}{(x,y)\mapsto T(x)(y)}\]
dove $L^2(X\times Y,Z)$ sono le bilineari.
\end{exercise}

\begin{proposition}[w$^\ast$-limitato vs limitato in $\normd_{X^\ast}$]\label{PrLimitatoInDeboleStarEquivaleLimitatoInNormaDuale}
Sia $Y=\K$ e $X$ Banach. Sia $\Gamma\subseteq X^\ast$, allora $\Gamma$ \`e w$^\ast$-limitato se e solo se \`e limitato in $\normd_{X^\ast}$.
\end{proposition}
\begin{proof}
Essere limitato nella topologia debole$^\ast$ significa ``essere assorbito da ogni intorno w$^\ast$ di $X^\ast$" cio\`e, usando intorni di prebase, essere assorbiti da insiemi della forma
\[\cpa{f\in X^\ast\mid \abs{f(x)}<1}\]
per $x\in X$. Notiamo che $\Gamma$ viene assorbito da $\cpa{f\in X^\ast\mid \abs{f(x)}<1}$ significa $\Gamma(x)$ limitato in $\K$. Per il corollario (\ref{CorPuntualmenteLimitatoImplicaLimitatoPerNormaOperatore}) si ha che $\Gamma$ \`e limitato in $L(X,\K)=X^\ast$.

L'altra implicazione \`e ovvia perch\'e la norma operatore gi\`a rende continui gli operatori e indebolire la topologia non pu\`o trasformare un insieme limitato in uno non limitato.
\end{proof}

\begin{remark}
Se $E\subseteq F$ \`e un sottospazio allora $\Gamma\subseteq E$ \`e limitato in $F$ se e solo se \`e limitato in $E$ per la topologia indotta.
\end{remark}

\begin{proposition}
Sia $\Gamma\subseteq X$, allora $\Gamma$ \`e w-limitato se e solo se \`e $\normd$-limitato.
\end{proposition}
\begin{proof}
Se $\Gamma$ \`e $\sigma(X,X^\ast)$-limitato allora tramite l'immersione isometrica $X\to X^{\ast\ast}$ troviamo un insieme $\sigma(X^{\ast\ast},X^\ast)$-limitato. A questo punto basta applicare la proposizione precedente (\ref{PrLimitatoInDeboleStarEquivaleLimitatoInNormaDuale}).
\end{proof}

\subsection{Teorema della mappa aperta}

\begin{lemma}\label{LmPerMappaAperta}
Siano $X$ e $Y$ spazi di Banach, $B$ palla unitaria chiusa di $X$, $T\in L(X,Y)$, $U$ limitato, $U\subseteq Y$ tali che se $0<t<1$ allora
\[U\subseteq TB+tU.\]
Allora si ha $(1-t)U\subseteq TB$.
\end{lemma}
\begin{proof}
RIPESCALA DA DENTRO LA DIMOSTRAZIONE MAPPA APERTA

per ogni $u_0\in U$ costruisco $(x_k)\subseteq B$ e $(u_k)\subseteq U$ tali che $u_k=Tx_k+t u_{k+1}$ per ogni $k\geq 0$.
\end{proof}

\begin{remark}
Se $U$ \`e un intorno di $0$ limitato in $Y$, o anche $U$ assorbente, allora $T$ \`e surgettivo.
\end{remark}


\begin{exercise}[Teorema di estensione di Tietze]\label{ExThEstensioneTietze}
Se $X$ \`e T4, $Y\subseteq X$ chiuso, $f\in C^0(Y,\R)$, allora $f$ si estende ad una continua su $X$.
\end{exercise}
\begin{proof}
Basta il caso di $f$ limitata:

la tesi \`e che l'operatore di restrizione (il quale \`e lineare e continuo)
\[R:C_b^0(X)\to C_b^0(Y)\]
\`e surgettivo. Basta applicare il lemma (\ref{LmPerMappaAperta}) come segue:
\[3B_{C_b(Y)}\subseteq R\pa{B_{C_b(X)}}+2B_{C_b(Y)}\]
e chiamiamo $U=3B_{C_b(Y)}$, $T=(2/3)\cdot$. Per il lemma di Urysohn, se $F_0$ e $F_1$ sono chiusi disgiunti di $X$ esiste $f$ tale che $F_0=\cpa{f=0}$ e $F_1=\cpa{f=1}$.
\end{proof}

\begin{theorem}[Mappa aperta]\label{ThMappaAperta}
Siano $X,Y$ Banach e $T:X\to Y$ lineare continuo e tale che $T(X)$ \`e di II-categoria in $Y$ (per esempio $T$ surgettivo). Allora $T$ \`e una mappa aperta.
\end{theorem}
\begin{proof}
Sia $B$ la palla unitaria chiusa di $X$. Basta mostrare che $T(B)$ \`e un intorno di $0$ in $Y$ (per omotetia e traslazione seguir\`a che $T$ manda intorni di $x$ in intorni di $T(x)$, cio\`e \`e aperta). Notiamo che 
\[X=\bigcup_n nB\implies T(X)=\bigcup_n nT(B)\]
Per ipotesi $T(X)$ \`e di II-categoria in $Y$, quindi per qualche $n$ si ha che $\ol{n T(B)}$ ha parte interna non vuota e quindi $\ol{T(B)}$ stesso ha parte interna non vuota. Poich\'e\footnote{ricorda che in generale se $f$ \`e continua allora $f(\ol A)\subseteq \ol{f(A)}$. 

In questo caso la mappa \`e $(x,y)\mapsto x-y$ e usiamo il fatto che $T$ \`e lineare e \[\ol{T(B)\times T(B)}=\ol{T(B)}\times \ol{T(B)}.\]} 
\[\ol{T(B)}-\ol{T(B)}\subseteq \ol{T(B-B)}=\ol{T(2B)}=2\ol{T(B)}\]
si ha che $\ol{T(B)}$ \`e un intorno di $0\in Y$.


Mostriamo ora che $T(B)$ stesso \`e un intorno di $0$. Poich\'e la chiusura \`e l'intersezione degli aperti che contengono $T(B)$ si ha in particolare che
\[\ol{T(B)}=T(B)+\frac12\ol{T(B)}.\]
Se $y_0\in\ol{T(B)}$ allora esistono $x_0\in B$ e $y_1\in \frac12\ol{T(B)}$ tali che $y_0=T(x_0)+\frac12 y_1$.

Iterando troviamo $y_2\in \ol{T(B)}$ tale che $y_1=T(x_1)+\frac12 y_1$ per $x_1\in B$. Questo definisce due successioni $(x_n)\subseteq B$ e $(y_n)\subseteq \ol{T(B)}$ tali che $y_n=T(x_n)+\frac12 y_{n+1}$, quindi
\[y_0=T(x_0)+\frac12 T(x_1)+\cdots+2^{-n}T(x_n)+2^{-n-1}y_{n+1}=T\pa{\sum_{i=0}^n2^{-i}x_i}+2^{-n-1}y_{n+1}.\]
Poich\'e $X$ \`e completo la serie $\sum_{i=0}^n2^{-i}x_i$ converge as un punto $x^\ast\in 2B$ (perch\'e assolutamente convergente e $\sum_{n\geq 0} 2^{-n}=2$). 

Siccome $T$ \`e continua, $T(B)$ \`e limitato e quindi $\ol{T(B)}$ \`e limitato, quindi 
\[\norm{2^{-n-1}y_{n+1}}\leq 2^{-n-1}\norm T\to 0,\]
quindi (per continuit\`a di $T$) si ha $y_0=T(x^\ast)$, cio\`e
\[\ol{T(B)}\subseteq 2T(B),\]
in particolare $T(B)$ \`e un intorno di $0$ per omotetia.
\end{proof}

\begin{remark}[Lineare continuo allora omeo se e solo se bigettivo]
Un operatore lineare continuo \`e un omeomorfismo se e solo se \`e bigettivo. Questo \`e immediato da mappa aperta (\ref{ThMappaAperta}).
\end{remark}

\begin{remark}
Se $T:X\to Y$ lineare continuo allora induce
% https://q.uiver.app/#q=WzAsMyxbMCwwLCJYIl0sWzEsMCwiWSJdLFswLDEsIlgvXFxrZXIgVCJdLFswLDIsIlxccGkiLDJdLFswLDEsIlQiXSxbMiwxLCJcXHd0IFQiLDJdXQ==
\[\begin{tikzcd}
	X & Y \\
	{X/\ker T}
	\arrow["T", from=1-1, to=1-2]
	\arrow["\pi"', from=1-1, to=2-1]
	\arrow["{\wt T}"', from=2-1, to=1-2]
\end{tikzcd}\]
con $\wt T$ lineare continua iniettiva. Se $T$ \`e surgettiva allora per il teorema della mappa aperta (\ref{ThMappaAperta}) $\wt T$ \`e un omeomorfismo lineare.
\end{remark}

\begin{remark}
Se $T:X\to Y$ \`e lineare e continua allora
\[\text{aperta $\coimplies$ surgettiva $\coimplies$ identificazione.}\]
\end{remark}

\begin{theorem}[Grafico chiuso]\label{ThGraficoChiuso}
Siano $X,Y$ Banach, $T:X\to Y$ lineare. Allora $T$ \`e continua se e solo se
\[\Gamma=\cpa{(x,T(x))\in X\times Y\mid x\in X}\]
\`e chiuso.
\end{theorem}
\begin{proof}
Data una qualsiasi mappa continua $f:X\to Y$ con $Y$ Hausdorff si ha che $\Gamma$ \`e la preimmagine della diagonale di $Y\times Y$ rispetto alla mappa $id_Y\times f$. Poich\'e $Y$ \`e un Banach (e quindi metrico e quindi $T_2$) effettivamente abbiamo la prima implicazione.
\smallskip

Supponiamo ora che $\Gamma$ sia chiuso. Poich\'e $X\times Y$ \`e prodotto di Banach esso stesso \`e banach e quindi $\Gamma$ \`e banach perch\'e chiuso di un Banach. Osserviamo ora che
\[T(x)=P_Y((x,T(x)))=P_Y((P_X\res{\Gamma})\ii(x))\implies T=P_Y\circ(P_X\res{\Gamma})\ii.\]
Poich\'e $P_X\res{\Gamma}$ \`e bigettiva, continua e lineare, per il teorema della mappa aperta (\ref{ThMappaAperta}) essa \`e un omeomorfismo, quindi $T$ \`e continuo in quanto composizione di $P_Y$ e $P_X\res{\Gamma}\ii$ continue.
\end{proof}

\begin{exercise}
Sia $T:X\to Y$ lineare fra Banach. Controntare la continuit\`a di $T$ con le topologie forti e deboli di $X$ e $Y$
\begin{align*}
(X,w)&\to (Y,w)\\
(X,w)&\to (Y,s)\\
(X,s)&\to (Y,w)\\
(X,s)&\to (Y,s)
\end{align*}
\end{exercise}
\begin{proof}
Hint: usare grafico chiuso (\ref{ThGraficoChiuso}) ricordando che sottospazi vettoriali di Banach sono chiusi forti se e solo se sono chiusi deboli e osservando chi \`e la topologia debole di $X\times Y$ (topologia prodotto)

Tre di queste nozioni sono equivalenti e una no. Quella diversa \`e pi\`u forte? Pi\`u debole?
\end{proof}


\begin{definition}[Forte iniettivit\`a]
Una mappa $T:X\to Y$ lineare continua \`e \textbf{fortemente iniettiva} se esiste $c>0$ tale che 
\[\forall x\in X\qquad \norm{T(x)}\geq c\norm x.\]
\end{definition}

\begin{proposition}
Se $X$ e $Y$ sono banach e $T:X\to Y$ lineare continua, $T$ \`e fortemente iniettiva se e solo se $T$ \`e iniettiva e $\imm T$ \`e chiuso.
\end{proposition}
\begin{proof}
Diamo le implicazioni
\setlength{\leftmargini}{0cm}
\begin{itemize}
\item[$\boxed{\implies}$] Iniettiva ok. Sia $T':X\to \imm T\subseteq Y$ la stessa mappa di $T$ ma con codominio ristretto. Notiamo che $T'$ \`e invertibile perch\'e iniettiva e surgettiva per costruzione e che ha inversa continua per la disuguaglianza in ipotesi, quindi $\imm(T)$ \`e Banach perch\'e $X$ \`e Banach e quindi $\imm T$ \`e chiuso in $Y$.
\item[$\boxed{\impliedby}$] Se $T$ \`e iniettiva con immagine chiusa allora $T':X\to \imm T$ \`e invertibile. Inoltre, poich\'e $\imm T$ \`e Banach perch\'e chiuso di $Y$, si ha che per mappa aperta (\ref{ThMappaAperta}) vale $(T')\ii$ continua, cio\`e $T$ fortemente iniettiva.
\end{itemize}
\setlength{\leftmargini}{0.5cm}
\end{proof}

\begin{proposition}[Retrazioni e sezioni per lineari continue]\label{PrRetrazioniSezioniPerOperatoriLineariContinui}
Sia $T\in L(X,Y)$ con $X,Y$ Banach. Allora $T$ \`e una\footnote{cio\`e esiste $S:Y\to X$ tale che $T$ \`e l'inversa destra / sinistra di $S$.}
\begin{itemize}
    \item inversa destra $\coimplies$ iniettiva e $\imm T$ \`e complementato\footnote{cio\`e esiste $V\subseteq Y$ tale che $Y=\imm T\oplus V$.}
    \item inversa sinistra $\coimplies$ surgettivo e $\ker T$ \`e complementato.
\end{itemize}
\end{proposition}
\begin{proof}
Se $T:X\to Y$ e $S:Y\to X$ sono una coppia tale che $S\circ T=id_X$ allora $T\circ S=P$ \`e un proiettore lineare continuo, infatti
\[P^2=(T\circ S)\circ (T\circ S)=T\circ id_X \circ S=T\circ S.\]
Quindi $Y=\ker P\oplus \imm P$ e $\ker P=\ker S$, $\imm P=\imm T$, ovvero
\[Y=\ker S\oplus \imm T\]
come volevamo.
\medskip

Viceversa, se $T$ \`e iniettivo e $\imm T$ \`e complementata (rispettivamente $S$ \`e surgettivo e $\ker S$ complementato) allora considero un proiettore $P_{\imm T}$ (ok per la decomposizione in somma diretta) e definisco $S=(T')\ii\circ P_{\imm T}$ che \`e inversa sinistra di $T$ (rispettivamente definisco un proiettore $Q$ su $\ker S$ con $id_Y-Q$ proiettore sul supplementare $V$ fissato di $\ker S$, a questo punto considero $S\res V\ii$, che diventa inversa destra).
\end{proof}

\begin{proposition}[Norme confrontabili su Banach sono equivalenti]\label{PrNormeConfrontabiliSuBanachSonoEquivalenti}
Due norme si Banach confrontabili sullo stesso $\K$-spazio vettoriale sono equivalenti.
\end{proposition}
\begin{proof}
Caso particolare di ``$T$ bigettiva continua" scegliendo $T=id_X$.
\end{proof}

\begin{exercise}
Su uno spazio normato $X$ di dimensione infinita esistono sempre forme lineari non continue. 
\end{exercise}

\begin{remark}
Esistono $L:X\to X$ lineari bigettive non continue
\end{remark}
\begin{proof}
Fisso $f$ forma discontinua e fisso $u\in X$, definiamo
\[L(x)=x+f(x)u\]
e notiamo che
\begin{align*}
L^2(x)=&L(x+f(x)u)=L(x)+f(x)L(u)=\\
=&x+f(x)u +f(x)(u+f(u)u)=\\
=&x+(2f(x)+f(x)f(u))u.
\end{align*}
Se $u$ \`e tale che $f(u)=-2$ allora $L^2=id_X$, cio\`e $L$ involuzione. In particolare $L$ \`e bigettiva ma continua se e solo se $f$ lo \`e, e non lo \`e quindi $L$ non continua su $(X,\normd_1)$ Banach.

Poniamo $\norm x_2=\norm{L(x)}_2$. Notiamo che $\normd_2$ rende $X$ Banach in quanto $L:(X,\normd_1)\to (X,\normd_2)$ \`e una isometria. Notiamo dunque che $\normd_1$ e $\normd_2$ sono norme che rendono $X$ banach e che non sono equivalenti ($L$ \`e discontinua per $\normd_1$ ma continua per $\normd_2$).
\end{proof}

\begin{exercise}
Siano $\normd_1,\ \normd_2$ norme sullo stesso $X$ e sia $\normd_3=\normd_1+\normd_2$. Allora
\begin{enumerate}
    \item Una successione $(x_n)$ converge a $x\in X$ in $\normd_3$ se e solo se converge a $x$ in $\normd_1$ e $|normd_2$.
    \item $(x_n)$ \`e di Cauchy in $X$ se e solo se \`e di Cauchy sia per $\normd_1$ che per $\normd_2$.
\end{enumerate}
\end{exercise}

\begin{exercise}
TROVA L'IMBROGLIO:

\noindent
``\textbf{Proposizione.}" Tutte le norme di Banach sullo stesso $X$ sono equivalenti.
\begin{proof}[``Dimostrazione"]
    Siano $\normd_1$ e $\normd_2$ di Banach. Notiamo che $\normd_3$ \`e pi\`u fine della altre due e che $(x_n)$ \`e di Cauchy per $\normd_3$ se e solo se lo \`e per le altre due, quindi per il punto 1. della proposizione precedente la successione converge in $\normd_3$. Segue dunque che, poich\'e $\normd_3$ \`e pi\`u fine allora \`e confrontabile con le altre due, quindi le tre norme sono equivalenti.
\end{proof}
\end{exercise}


\section{SVT I-numerabili e paranorme}
\begin{definition}[Paranorma]
Una \textbf{paranorma} sull $\K$-spazio vettoriale $X$ \`e una funzione $q:X\to[0,\infty)$ tale che
\begin{enumerate}
    \item $q(x+y)\leq q(x)+q(y)$
    \item $q(\la x)\leq q(x)$ per ogni $x\in X$ e $\la\in\K$ tale che $\abs{\la}\leq 1$
    \item Se $\la_k\to 0$ in $\K$ allora $q(\la_k x)\to 0$
\end{enumerate}
Inoltre $q$ \`e \textbf{definita} se vale
\[q(x)=0\coimplies x=0.\]
\end{definition}

\begin{remark}
Dalla propriet\`a $2.$ segue che $q(\la x)=q(x)$ se $\abs\la=1$ e che $q(\la x)\leq q(\mu x)$ se $\abs\la\leq\abs\mu$. In particolare $q(x)=q(-x)$.

Quindi $d(x,y)=q(x-y)$ \`e una (semi)distanza su $X$ (distanza se $q$ definita).
\end{remark}

\begin{exercise}
Dimostrare che $(X,d)$ \`e uno SVT per $d$ indotta da paranorma $q$.
\end{exercise}

\begin{exercise}
Sia $X$ un $K$-SVT I-numerabile. Allora la sua topologia proviene da una paranorma (la quale \`e definita sse $X$ \`e $T_0$).
\end{exercise}
\begin{proof}
TRACCIA
\begin{itemize}
    \item Sia $\cpa{U_n}_{n\geq 0}$ base numerabile di intorni bilanciati di $0$ tali che $U_{n+1}+U_{n+1}\subseteq U_n$.
    \item Estendiamo la successione per $n<0$ ponendo $U_k=U_{k+1}+U_{k+1}$ per ogni $k<0$.
    
    Nota che $U_{k+1}+U_{k+1}\subseteq U_k$ per ogni $k\in \Z$ e gli $\cpa{U_k}_{k\in\Z}$ sono intorni bilanciati. 
    \item Poniamo
    \[q(x)=\inf\cpa{\sum_{i=1}^r2^{-ki}\mid r\in\N,(k_1,\cdots, k_r)\in\Z^r\ t.c.\ x\in U_{k_1}+U_{k_2}+\cdots+U_{k_r}}\]
    Mostra che $q$ \`e una paranorma su $X$.
    \item Nota che $\cpa{q<2^{-n-1}}\subseteq U_n\subseteq \cpa{q\leq 2^{-n}}$ e quindi $q$ induce la topologia di $X$.
\end{itemize}
\end{proof}