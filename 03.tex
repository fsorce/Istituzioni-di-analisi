\chapter{Teorema di Hahn-Banach}

Il teorema di Hanh-Banch ci permetter\`a di costruire funzionali lineari continui.

\filosofia{Funzionali sono i surrogati delle coordinate, che non ci sono in generale, e anche quando ci sono possono essere pi\`u complicate di quanto non valga la pena.}

\section{Teorema di Hahn-Banach reale}
\begin{definition}[Funzione sublineare]
Una funzione $p:X\to \R$ \`e 
\begin{itemize}
    \item \textbf{positivamente omogena} se per $t\in\R$, $t\geq 0$ abbiamo $p(tu)=tp(u)$,
    \item \textbf{subadditiva} se per ogni $u,v\in X$ vale $p(u+v)\leq p(u)+p(v)$,
    \item \textbf{sublineare} se \`e subadditiva e positivamente omogenea.
\end{itemize}
\end{definition}


\filosofia{Pillola filosofica: Teorema di esistenza senza buon criterio per scegliere un candidato spesso chiama l'uso di scelta.}

\begin{theorem}[Hahn-Banach]\label{ThHahnBanach}
Siano $X$ uno spazio vettoriale reale, $M\subseteq X$ sottospazio, $p:X\to \R$ sublineare, $f:M\to \R$ lineare tale che $f\leq p$ su $M$.

Allora $f$ si estende a $F:X\to \R$ lineare tale che $F\leq p$.
\end{theorem}
\begin{proof}
Vogliamo applicare il lemma di Zorn. Sia
\[\Ms=\cpa{g\in N'\mid g\leq p,\ M\subseteq N\subseteq X}\]
Notiamo che $\Ms$ \`e ordinato secondo l'inclusione dei sottografici, cio\`e
\[g\preceq h\coimplies \Gamma g\subseteq \Gamma h\coimplies \begin{cases}
\dom g\subseteq \dom h\\
g(x)\leq h(x)\quad\forall x\in \dom g
\end{cases}\]
Condizione delle catene vale:
\[\text{se }\cpa{g_\al}_{\al\in \Lambda}\text{ catena in }\Ms\]
allora $\bigcup_\al\Gamma g_\al$ \`e ancora il grafico di una funzione lineare minore di $p$.

Dunque per il lemma di Zorn esiste un elemento massimale in $\Ms$. Per concludere basta mostrare che un massimale di $\Ms$ \`e definito su tutto $X$, cio\`e vogliamo mostrare che se $g\in \Ms$ \`e tale che $\dom g\neq X$ allora esiste $g'\in\Ms$ che estende $g$.

Sia dunque per assurdo $x\in X\bs N$ dove $N=\dom g$. Vogliamo estendere $g$ a $h:N\oplus \ps x\to \R$ con $h\leq p$. In quanto estensione \[h(u+tx)=h(u)+th(x)=g(u)+th(x),\] dove $u$ generico elemento di $N$. Sia $\al=h(x)$ e cerchiamo un opportuno $\al$ in modo tale che $h\leq p$.

Chiediamo che $\forall u\in N,\ \forall t\in \R$\[g(u+tx)\leq p(u+tx),\]
o equivalentemente per ogni $t>0$ chiediamo
\[\begin{cases}
    h(u+tx)\leq p(u+tx)\\
    h(v-tx)\leq p(v-tx)
\end{cases}\]
equivalentemente
\[\begin{cases}
    g(u/t)+\al\leq p(u/t+x)\\
    g(v/t)-\al\leq p(v/t-x)
\end{cases}\]
dunque vogliamo
\[-p(v/t-x)+g(v/t)\leq \al\leq p(u/t+x)-g(u/t)\]
cio\`e \[\sup_{v\in N}-p(v-x)+g(v)=m_\ast\leq \al\leq m^\ast=\inf_{u\in N}p(v+x)-g(u),\]
dunque un tale $\al$ esiste solo se $m_\ast\leq m^\ast$. Questo \`e vero perch\'e
\[g(u)+g(v)=g(u+v)\leq p(u+v)=p(u+x+v-x)\leq p(u+x)+p(v-x).\]
\end{proof}

\begin{remark}
Non serve questo teorema per spazi di dimensione finita o spazi di Hilbert, in quanto in quei casi abbiamo estensioni canoniche (se $dom f=N$, considero la proiezione orgonale su $N$ e poi applico $f$).
\end{remark}

\begin{corollary}[Hahn-Banach per spazi normati]\label{CorHahnBanachPerSpaziNormati}
Se $(X,\normd)$ \`e spazio normato reale e $Y$ \`e sottospazio lineare allora ogni funzione continua su $Y$ si estende ad una su $X$ con la stessa norma.
\end{corollary}
\begin{proof}
Se $f\in Y^\ast$, per la definizione di norma duale si ha 
\[f(x)\leq \norm f_{Y^\ast}\norm x\doteqdot p(x),\]
quindi $f$ si estende a $F:X\to \R$ lineare con $F(x)\leq \norm f_{Y^\ast}\norm x$, cio\`e $\norm F_{X^\ast}\leq \norm f_{Y^\ast}$. Poich\'e $F$ estende $f$ in realt\`a abbiamo uguaglianza tra le norme\footnote{consideriamo la stessa successione in $Y$ che realizza la definizione di $\norm f_{Y^\ast}$}.
\end{proof}

\begin{remark}
Se $X$ \`e di Hilbert, una estensione di $f\in Y^\ast$ \`e data dal proiettore ortogonale su\footnote{stiamo supponendo $Y$ chiuso a meno di passare alla chiusura} $Y$ $P:X\to \ol Y$. A questo punto definendo $F=f\circ P$.
\end{remark}

\begin{corollary}[ricostruire norma tramite funzionali]\label{CorRicostruireNormaTramiteFunzionali}
Se $(X,\normd)$ \`e spazio normato reale e $Y$ \`e sottospazio lineare e $x\in X$, allora la norma di $x$ si pu\`o ricostruire dalla norma duale di $X^\ast$, in particolare\footnote{dove $\ps{f,x}=f(x)$ quando $f$ \`e forma lineare, come in questo caso.}
\[\norm x=\max_{\norm f_{X^\ast}\leq1}\ps{f,x}\]
\end{corollary}
\begin{proof}
Se $f\in X^\ast$ e $\norm f\leq 1$ allora
\[\ps{f,x}\leq \norm f\norm x\leq \norm x\implies \norm x\leq \max_{\norm f\leq 1}\ps{f,x}.\]
D'altra parte, per il corollario precedente (\ref{CorHahnBanachPerSpaziNormati}) nel caso particolare di $Y=x\R$, il funzionale lineare continuo \[\phi:\funcDef{x\R}{\R}{\la x}{\la\norm x}\]
si estende a tutto $X$ con la stessa norma. Se $x=0$ allora $\norm \phi=0$ per linearit\`a, altrimenti $\norm \phi=1$ su $x\R$. In ogni caso $\norm \phi\leq 1$, quindi per ogni $x\in X$ esiste $f\in X^\ast$ tale che $\norm f\leq 1$ e $\ps{f,x}=\norm x$.
\end{proof}

\begin{definition}[Operatore aggiunto]
Per $T:X\to Y$ lineare continua tra spazi normati, si definisce l'\textbf{operatore aggiunto o trasposto} di $T$ come
\[T^\ast:\funcDef{Y^\ast}{X^\ast}{f}{f\circ T}\]
\end{definition}

\begin{proposition}[Norma dell'aggiunto]
La norma di $T^\ast$ coincide con la norma di $T$, in particolare $T^\ast$ \`e continuo.
\end{proposition}
\begin{proof}
Segue dai corollari di Hahn-Banach sopra, infatti
\begin{align*}
    \norm{T^\ast}_{L(Y^\ast,X^\ast)}=&\sup_{f\in Y^\ast,\norm f\leq 1}\norm{T^\ast f}_{X^\ast}=\sup_{f\in Y^\ast,\norm f\leq 1}\sup_{\norm x\leq 1,x\in X}\ps{T^\ast f,x}=\\
    =&\sup_{\norm f\leq 1,\norm x\leq 1}\abs{f,Tx}=\sup_{\norm x\leq 1}\sup_{\norm f\leq 1}\abs{\ps{f,Tx}}\pasgnl={(\ref{CorRicostruireNormaTramiteFunzionali})}\\
    =&\sup_{\norm x\leq 1}\norm{Tx}=\norm T_{L(X,Y)}.
\end{align*}
\end{proof}


\subsection{Inclusione isometrica nel biduale}
\begin{proposition}[Inclusione isometrica nel biduale]\label{PrInclusioneIsometricaNelBiduale}
Sia $(X,\normd)$ uno spazio normato reale e consideriamo la mappa
\[i_X:\funcDef{X}{X^{\ast\ast}}{x}{val_x}\]
Essa \`e una inclusione isometrica.
\end{proposition}
\begin{proof}
\`E immediato vedere che $i_X$ \`e lineare e continua\footnote{$\ps{val_x,f+\la g}=f(x)+\al g(x)=\ps{val_x,f}+\la\ps{val_x,g}$ e $\norm{val_x}\leq \norm x$ in quanto $\abs{\ps{val_x,f}}=\abs{\ps{f,x}}\leq \norm f\norm x$.} Per\`o sappiamo che per ogni $x\in X$ esiste $f\in X$ tale che $\norm f\leq 1$ e $\norm x=\ps{f,x}$, cio\`e $\norm{val_x}=\norm x$, ovvero $i_X:X\to X^{\ast\ast}$ \`e una inclusione isometrica.
\end{proof}

\begin{definition}[Spazio riflessivo]
Uno spazio normato $(X,\normd)$ \`e \textbf{riflessivo} se $i_X:X\to X^{\ast\ast}$ \`e surgettiva, ovvero se $i_X$ \`e una isometria.
\end{definition}

\begin{remark}
Esistono spazi di Banach non riflessivi ma isometrici al loro biduale. Nella definizione chiediamo che la mappa canonica $i_X$ sia una isometria.
\end{remark}

\begin{example}
Sia $c_0=\cpa{x\in \R^\N\mid x(n)=o_n(1)}$. Questo \`e un sottospazio chiuso di \[\ell_\infty=\cpa{x\in \R^\N\mid \norm x_\infty< \infty}\]
Se $\wh\N$ \`e la compattificazione di $\N$ ad un punto ($\wh \N=\N\cup\cpa{\infty}$) allora $c_0$ sono le funzioni $\wh \N\to \R$ continue che valgono $0$ in $\infty$ ristrette a $\N$.


Risulta che l'inclusione $c_0\inj \ell_\infty$ \`e l'inclusione nel biduale, infatti $c_0^\ast$ si pu\`o identificare con
\[\ell_1=\cpa{f\in\R^\N\mid \norm f_1=\sum\abs{f_n}<\infty}\]
identificando $f\in \ell_1$ con $\wt f(x)=\sum f_n x_n$ (che converge perch\'e assolutamente convergente). Risulta che questa identificazione \`e una isometria. 

Con un processo analogo identifichiamo $\ell_1^\ast$ con $\ell_\infty$.

% https://q.uiver.app/#q=WzAsMyxbMCwwLCJjXzAiXSxbMiwwLCJcXGVsbF9cXGluZnR5Il0sWzEsMSwiY18wXntcXGFzdFxcYXN0fSJdLFswLDIsImlfe2NfMH0iLDJdLFswLDEsIlxcc3Vic2V0ZXEiLDFdLFsxLDIsIlxcY29uZyIsMV1d
\[\begin{tikzcd}
	{c_0} && {\ell_\infty} \\
	& {c_0^{\ast\ast}}
	\arrow["\subseteq"{description}, from=1-1, to=1-3]
	\arrow["{i_{c_0}}"', from=1-1, to=2-2]
	\arrow["\cong"{description}, from=1-3, to=2-2]
\end{tikzcd}\]
\end{example}

\begin{remark}
Se $X$ \`e Hilbert allora $X\inj X^{\ast\ast}$ \`e surgettiva tramite l'isomorfismo di Riesz
\[x\mapsto \ps{\cdot,x}\mapsto \ps{\cdot,\ps{\cdot,x}}=val_x\]
\end{remark}

\begin{remark}
Se $X$ normato, $i_X:X\to X^{\ast\ast}$ ci permette di costruire un completamento considerando $\ol{i_X(X)}$ in $X^{\ast\ast}$ in quanto il biduale \`e completo.
\end{remark}


\subsection{Sulle ipotesi del teorema di Hahn-Banach}

Il funzionale $p$ nelle ipotesi \`e positivamente omogeneo e subadditivo (cio\`e sublineare).

\begin{remark}
Una funzione $f$ \`e subadditiva se, detto $\Gamma$ il grafico di $f$, $\Gamma+(x,f(x))$ sta sempre sopra $\Gamma$.
\end{remark}

\begin{exercise}
Mostra le seguenti implicazioni
\begin{itemize}
    \item Positivamente omogeneo e subadditivo implica convesso
    \item Positivamente omogeneo e convesso implica subadditivo (e quindi sublineare)
    \item Subadditivo, convesso e $p(0)\leq 0$ implica positivamente omogeneo
\end{itemize}
\end{exercise}

\begin{exercise}
Trovare $f:\R\to \R$ che sia subadditiva, convessa ma non positivamente omogenea.
\end{exercise}

\begin{exercise}
Nel teorema di Hahn-Banach si pu\`o prendere pi\`u in generale $p$ convesso?

Si, ma si riconduce al caso standard trovando un nuovo funzionale $p_0$ che sia sublineare e tale che $f\leq p_0\leq p$.
\end{exercise}

\section{Estensioni e altre versioni di Hahn-Banach}

\subsection{Teorema di Hahn-Banch complesso}
\begin{theorem}[Hahn-Banach complesso]\label{ThHahnBanachComplesso}
    Sia $X$ un $\C$-spazio vettoriale normato, $Y\subseteq X$ un suo sottospazio vettoriale e $f\in Y^\ast$, allora $f$ si estende ad un funzionale lineare su $X$ con uguale norma.
\end{theorem}
\begin{proof}
Sia $(X_0,\normd)$ lo spazio normato reale ottenuto da $X$ per restrizione degli scalari e sia $f_0=\Real f$. Notiamo che $f_0$ \`e un funzionale lineare continuo reale su $Y$, che quindi possiamo estendere a $\wt f_0\in X^\ast$ mantenendo la norma. Definiamo \[\wt f(x)=\wt f_0(x)-i\wt f_0(ix).\]
Notiamo che $\wt f\res Y=f$, infatti 
\[f(y)=\Real(f(y))+i\Imag(f(y))=\Real(f(y))-i\Imag(if(iy))=\Real(f(y))-i\Real(f(iy)).\]
Si ha anche che $\wt f$ \`e $\C$-lineare e che $\norm{\wt f}_{X^\ast}=\norm f_{Y^\ast}$

(COMPLETA PER ESERCIZIO)
\end{proof}





\subsection{Teoremi di separazione dei convessi}

\begin{proposition}[Funzionali di Minkowski sono sublineari]\label{PrFunzionaliMinkowskiSonoSublineari}
Se $C$ \`e convesso e $0\in C$ allora $p_C$ \`e sublineare.
\end{proposition}
\begin{proof}
Dimostriamo le due propriet\`a:
\setlength{\leftmargini}{0cm}
\begin{itemize}
    \item[$\boxed{\text{pos.omo.}}$] Per ogni $\la>0$, $x\in X$ si ha che
    \[p_C(\la x)=\inf\cpa{t>0\mid \la x\in tC}=\inf\cpa{\la s>0\mid \la x\in\la sC}=\la p_C(x)\]
    \item[$\boxed{\text{subadd.}}$] Per ogni $x,y\in X$ siano $a$ e $b$ tali che
    \[a>p_C(x),\quad b>p_C(y).\]
    Se uno tra $p_C(x)$ e $p_C(y)$ \`e infinito allora la tesi vale trivialmente. Supponiamo dunque che questo non sia il caso.
    Allora $x\in aC$ e $y\in bC$, cio\`e $x/a,y/b\in C$. Notiamo che
    \[\frac{x+y}{a+b}=\frac a{a+b}\frac xa+\frac b{a+b}\frac yb\]
    dunque $\frac{x+y}{a+b}\in C$ per convessit\`a, cio\`e $x+y\in (a+b)C$ e quindi $p_C(x+y)\leq a+b$. Passando all'estremo inferiore per $a>p_C(x)$ e $b>p_C(y)$ troviamo
    \[p_C(x+y)\leq p_C(x)+p_C(y)\]
\end{itemize}
\setlength{\leftmargini}{0.5cm}
\end{proof}


\begin{remark}
    Se $C$ \`e un disco, cio\`e \`e assorbente, bilanciato e  convesso allora $p_C$ \`e una seminorma.
\end{remark}

\begin{exercise}
Se $X$ SVT, $F:X\to \K$ lineare non continua allora per ogni aperto $A$ non vuoti si deve avere $F(A)=\K$.
\end{exercise}

\begin{lemma}\label{LmFunzionaleLineareNonNulloSuSVTMappaAperta}
    Ogni funzionale lineare non nullo su uno SVT \`e una mappa aperta
\end{lemma}
\begin{proof}
Sia $F\neq0$ lineare con $F:X\to \K$. Vogliamo mostrare che $F$ manda intorni di $x\in X$ in intorni di $F(x)\in\K$. Poich\'e $X$ \`e SVT, basta mostrare che $F(U)$ \`e intorno di $0\in\K$ per ogni $U$ intorno di $0\in \K$. In realt\`a basta prendere una base di intorni di $0$, quindi consideriamo gli $U$ bilanciati. Notiamo che $F(U)$ \`e un insieme bilanciato di $\K$, infatti se $\la\in\K$ e $\abs\la\leq 1$ allora $\la F(U)=F(\la U)\subseteq F(U)$, quindi abbiamo le seguenti possibilit\`a:
\begin{itemize}
    \item $F(U)=\cpa0$, ma allora $F=0$ assurdo
    \item $F(U)$ \`e un disco, dunque \`e intorno di $0$ ok.
    \item $F(U)=\K$ ok. 
\end{itemize}
\end{proof}
\begin{corollary}[Discontinuit\`a per funzionali lineari]\label{CorDiscontinuitaFunzionaliLineari}
$F:X\to \K$ lineare \`e discontinua se e solo se \`e surgettiva su ogni aperto non vuoto.
\end{corollary}
\begin{proof}
Se $F$ non \`e surgettiva su un aperto non vuoto, a meno di traslazione $F$ non \`e surgettiva su un intorno di $0$, quindi non \`e surgettiva su un qualche aperto bilanciato. Quindi esiste un elemento che non \`e nella immagine, ma allora $F$ non assume valori di modulo superiore a questo valore non raggiunto.
\end{proof}


\begin{theorem}[Separazione di convessi]\label{ThSeparazioneDiConvessi}
Valgono i seguenti teoremi:
\setlength{\leftmargini}{0cm}
\begin{itemize}
    \item Siano $X$ un $\R$-SVT, $A$ un suo aperto convesso non vuoto e $B$ un convesso non vuoto disgiunto da $A$. Allora esistono $F\in X^\ast$ e $\gamma\in \R$ tali che per ogni $a\in A,\ b\in B$ si ha
    \[\ps{F,a}<\gamma\leq\ps{F,b},\]
    cio\`e $A\subseteq \cpa{F<\gamma}$ e $B\subseteq \cpa{F\geq \gamma}$.
    \item Sia $X$ un $\R$-SVTLC\footnote{La locale convessit\`a serve, infatti esistono SVT metrizzabili che non hanno funzionali lineari continui e in tal caso la tesi non vale neanche per $K=\cpa x$ e $C=\cpa y$.}, $K$ convesso compatto e $C$ convesso chiuso disgiunti. Allora esistono $F\in X^\ast$, $\gamma_1,\gamma_2\in\R$, $\gamma_1<\gamma_2$ tali che per ogni $x\in K$ e per ogni $y\in C$ vale
    \[\ps{F,x}\leq \gamma_1<\gamma_2\leq \ps{F,y}\]
    ovvero $K\subseteq \cpa{F\leq \gamma_1}$ e $C\subseteq \cpa{F\geq \gamma_2}$.
\end{itemize}
\setlength{\leftmargini}{0.5cm}
\end{theorem}
\begin{proof}
Diamo le due dimostrazioni
\setlength{\leftmargini}{0cm}
\begin{itemize}
    \item Sia $x_0\in B-A=\cpa{b-a\mid a\in A,\ b\in B}$. Poich\'e $A\cap B=\emptyset$, $x_0\neq 0$. Sia 
    \[C=A-B+x_0=\bigcup_{b\in B}(A-b+x_0).\]
    Dalla definizione \`e evidente che $C$ \`e un aperto (unione di traslati di $A$ che \`e aperto) e contiene $0$. $C$ \`e convesso perch\'e la somma algebrica di due convessi \`e un convesso (quindi $A-B$ convesso e traslare un convesso lo lascia convesso). Essendo aperto in particolare \`e assorbente per (\ref{PrProprietaIntorni0}).

    Quindi il funzionale di Minkowski associato $p_C$ \`e un funzionale sublineare $X\to \R$ (non raggiunge $+\infty$ perch\'e assorbente). Sia $f_0:\R x_0\to \R$ il funzionale lineare definito da $\ps{f_0,x_0}=1$. Poich\'e $0\notin A-B$, $x_0\notin C$ e quindi\footnote{ricorda che $\cpa{p_C<1}\subseteq C$} $p_C(x_0)\geq 1$. Applicando il teorema di Hahn-Banach (\ref{ThHahnBanach}) $f_0$ si estende a $F:X\to \R$ con $F\leq p_C$ in $X$. Per ogni $a\in A,\ b\in B$, poich\'e $a-b+x_0\in C$, si ha
    \[F(a)-F(b)+1=F(a-b+x_0)\leq p_C(a-b+x_0)\leq 1\]
    cio\`e $F(a)\leq F(b)$. Ponendo $\gamma=\sup_A F$ abbiamo le disuguaglianze volute se mostriamo che $F(a)<\gamma$ per ogni $a\in A$. Per il lemma (\ref{LmFunzionaleLineareNonNulloSuSVTMappaAperta}) si ha che $F$ \`e una mappa aperta, quindi $F(A)$ \`e un aperto di $\R$ tale che $\sup F(A)\leq \gamma$, ma allora il valore $\gamma$ non \`e raggiunto.


    Concludiamo notando che $F$ \`e continuo\footnote{volendo anche perch\'e limitato su intorno di $0$ o anche perch\'e non \`e surgettiva sull'aperto $A$. Vedi esercizio sopra per l'ultima.} perch\'e \`e limitato superiormente sull'aperto $A$.
    \item Sia $V$ intorno convesso di $0$ tale che $(K+V)\cap C=\emptyset$, basta usare (\ref{PrSeparoCompattoEChiusoDisgiunti}) e poi notare che in questo caso abbiamo una base di intorni convessi. Evidentemente $K+V$ \`e aperto e convesso\footnote{somma di convessi \`e convessa}. Per il primo punto esiste $F\in X^\ast$ e $\gamma\in\R$ tale che per ogni $x\in K+V$ e $y\in C$
    \[\ps{F,x}<\gamma\leq \ps{F,y}.\]
    Sia $\gamma_1=\max_{x\in K}\ps{F,x}$, allora $\gamma_1<\gamma$ e quindi se $x\in K$
    \[\ps{F,x}\leq \gamma_1<\gamma\leq \ps{F,y}\]
    che \`e la tesi a meno di definire $\gamma_2=\gamma$.
\end{itemize}
\setlength{\leftmargini}{0.5cm}
\end{proof}




\section{Parentesi esercizi}


\begin{definition}[Misura non atomica]
    Uno spazio di misura $(X,\Qs,\mu)$ \`e \textbf{non-atomico} se per ogni $A\in\Qs$ di misura positiva contiene $B\in\Qs$ di misura positiva strettamente minore.
\end{definition}

\begin{exercise}[Sierpinski]
Se $(X,\Qs,\mu)$ \`e non-atomico allora \`e divisibile, cio\`e per ogni $A\in \Qs$ e per ogni $\la\in [0,\mu(A)]$ esiste $B\subseteq A$, $B\in\Qs$, tale che $\mu(B)=\la$.

Inoltre, vedendo la misura come funzione $\mu:\Qs\to [0,\mu(X)]$, esiste una inversa destra monotona crescente per inclusione $E:[0,\mu(X)]\to \Qs$, cio\`e si ha $\mu\circ E=id$ e per ogni $t\in [0,\mu(X)]$ abbiamo $\mu(E_t)=t$ e $E_t\subseteq E_{t'}$ per ogni $t\leq t'$.
\end{exercise}
\begin{proof}
Vogliamo applicare Zorn all'insieme delle inverse destre monotone parziali, cio\`e
\[\Gamma=\cpa{E:S\to \Qs\mid S\subseteq [0,\mu(X)], E\text{ monot. cresc. per $\subseteq$, }\mu(E(t)=t\ \forall t\in S)}\]
Chiaramente la condizione sulle catene funziona quindi $\Gamma$ ha un elemento massimale. Mostriamo poi che il dominio del massimale \`e chiuso e che \`e denso, e quindi deve essere tutto. (CONCLUDERE PER ESERCIZIO)
\end{proof}

\begin{exercise}
Sia $(X,\Qs,\mu)$ uno spazio di misura e sia $0<p\leq1$. Definiamo
\[\Lc^p(X)=\cpa{f:X\to\R\mid f\text{ misurabile, }\int_X\abs{f}^pd\mu<\infty}\]
e sia $q:\Lc^p\to [0,\infty)$ con $q(f)=\int_X\abs{f}^pd\mu=\norm f_p^p$.

Notiamo che $q(f+g)\leq q(f)+q(g)$, che $q(\la f)=\abs\la q(f)$ e che $q(f)=0$ se e solo se $f=0$ q.o.. Dunque $q$ definisce una semidistanza $d_q(f,g)=q(f-g)$, che induce una distanza sul quoziente
\[L^p(X)=\quot{\Lc^p(X)}{\ol{\cpa0}}\]
Questa distanza rende $L^p(X)$ uno SVT metrico completo omeomorfo a $L^1(X)$.


Mostrare che se $(X,\Qs,\mu)$ \`e non-atomico e $p<1$ allora $L^p(X)$ non ha funzionali lineari continui diversi da $0$ e non ha aperti convessi diversi da $L^p(X)$.
\end{exercise}


\begin{exercise}
    Sia $X=\N$ con la misura di cardinalit\`a. In questo caso $L^p(\N)=\ell_p$ con la definizione di prima. Questo \`e uno SVT metrico completo ma la misura \`e puramente atomica (misura ricostruibile dai singoletti). Mostra che $(\ell_p)^\ast=(\ell_1)^\ast$.
\end{exercise}
\begin{proof}
Nota che se $0<p\leq q\leq \infty$ allora $\ell_p\subseteq \ell_q$ e l'inclusione \`e una mappa continua, quindi una mappa lineare su $\ell_q$ restituisce una mappa lineare su $\ell_p$, quindi abbiamo $(\ell_p)^\ast\supseteq(\ell_1)^\ast$, va mostrato che non ce ne sono altri.(CONCLUDERE PER ESERCIZIO)
\end{proof}