\chapter{Topologia}


\begin{proposition}[Topologia iniziale]\label{PrTopologiaInizialeEsiste}
    Sia $X$ un insieme e $\Fc$ una famiglia di mappe a valori in uno spazio topologici. Notazione:
    \[\Fc=\cpa{f_j:X\to (Y_j,\tau_j)}_{j\in I}.\]
    Allora esiste la topologia meno fine su $X$ che rende continue le mappe $f_j$. Una prebase di questa topologia \`e data da
    \[\cpa{f_j\ii(A)\mid j\in I, A\in \tau_j}.\]
    In realt\`a basterebbe prendere una prebase per $\tau_j$ al posto di tutta la topologia.

    Questa topologia \`e detta \textbf{topologia iniziale della famiglia $\Fc$} e si denota $\tau_\Fc$.
\end{proposition}

\begin{remark}[Propriet\`a universale della topologia iniziale]\label{PrProprietaUniversaleTopologiaIniziale}
Data una mappa $\vp:(Z,\tau_Z)\to (X,\tau_\Fc)$ essa \`e continua se e solo se $f\circ \vp$ \`e continua per ogni $f\in \Fc$.
\end{remark}
\begin{proof}
Se $\vp$ \`e continua allora $f\circ \vp$ \`e composizione di continue. Se sappiamo che $f\circ \vp$ \`e continua per ogni $f\in\Fc$ allora, se $A$ \`e un aperto di $X$ per continuit\`a di $f\circ \vp$ abbiamo
\[\tau_Z\ni(f\circ \vp)\ii(A)=\vp\ii(f\ii(A))\]
cio\`e le preimmagini tramite $\vp$ di aperti di prebase sono aperti di $Z$, quindi $\vp$ \`e continua.
\end{proof}


\begin{proposition}[Transitivit\`a della topologia iniziale]\label{PrTransitivitaTopologiaIniziale}
Supponiamo di avere una famiglia di mappe $\Fc'=\cpa{f_i:X\to Y_i}_{i\in I}$ e per ogni $i\in I$ sia $\Gc_i=\cpa{g_{ij}:Y_i\to Z_{ij}}_{j\in J_i}$ una famiglia di mappe. Su ogni $Y_i$ consideriamo la topologia iniziale determinata da $\Gc_i$. Allora la topologia iniziale data da $\Fc'$ su $X$ coincide con la topologia iniziale su $X$ definita da $\Fc=\cpa{g_{ij}\circ f_i\mid i\in I,\ j\in J_i}$.
\end{proposition}
\begin{proof}
Entrambe le topologie in esame sono generate dagli insiemi $(g_{ij}\circ f_i)\ii(A)$ al variare di $i\in I$, $j\in J_i$ e $A\in \tau_{Z_{ij}}$, infatti
\[\text{prebase per $\Fc\to$ }(g_{ij}\circ f_i)\ii(A)=f_j\ii(g_{ij}\ii(A))\text{ $\leftarrow$ prebase per $\Fc'$}.\]
\end{proof}


\chapter{Duali di \texorpdfstring{$\ell_p$}{lp}}
\section{Norme estese}
\begin{definition}[Norma estesa]
Sia $\sigma:\K^\N\to [0,\infty]$ una \textbf{norma estesa}, cio\`e
\begin{enumerate}
    \item $\sigma(x+y)\leq \sigma(x)+\sigma(y)$
    \item $\sigma(\la x)=\abs\la\sigma(x)$
    \item $\sigma(x)=0\coimplies x=0$
\end{enumerate}
\end{definition}
\noindent
Inoltre supponiamo che 
\begin{enumerate}
    \item[4.] per ogni $n\in\N$ esista $C_n$ tale che per ogni $x\in\K^\N$ si abbia $\abs{x(n)}\leq C_n\sigma(x)$
    \item[5.] $\sigma$ \`e LSC\footnote{semicontinua inferiormente} rispetto alla convergenza puntale, cio\`e
    \[x^\nu\in \K^\N,\ \forall i\ x^\nu(i)\to x(i)\implies \sigma(x)\leq \liminf_{\nu\to+\infty}\sigma(x^\nu)\]
\end{enumerate}
\begin{example}
La funzione $\sigma(x)=(\sum\abs{x_i}^p)^{1/p}$ \`e una norma estesa su $\K^\N$ che ha propriet\`a indicate. Anche $\sigma(x)=\norm x_\infty$ ha queste propriet\`a.
\end{example}

\begin{remark}
Le propriet\`a 4. e 5. sono equivalenti a dire che $\cpa{\sigma\leq 1}$ \`e compatto in $\K^\N$, infatti
\[4.\coimplies \cpa{\sigma\leq 1}\subseteq \prod_n\ol{B(0,C_n)}\subseteq \K^\N\]
e $5.$ equivale a chiedere $\cpa{\sigma\leq 1}$ chiuso, quindi insieme dicono che $\cpa{\sigma\leq 1}$ \`e un chiuso in un compatto ($\prod_n\ol{B(0,C_n)}$ \`e prodotto di compatti).
\end{remark}

\begin{definition}[Dominio di finitezza]
Definiamo il \textbf{dominio di finitezza} della norma estesa $\sigma$ come
\[\ell_\sigma=\cpa{x\in\K^\N\mid \sigma<+\infty}\]
\end{definition}

\begin{exercise}
Il dominio di finitezza $\ell_\sigma$ \`e uno spazio di Banach e $\sigma$ induce la norma.
\end{exercise}
\begin{proof}
Traccia:
\begin{itemize}
    \item Verificare che $\ell_\sigma$ \`e uno spazio vettoriale
    \item $\sigma$ \`e una norma
    \item Verificare la completezza: 
    \begin{itemize}
        \item Sia $(x^\nu)_\nu\subseteq \ell_\sigma$ di Cauchy per $\sigma$. Allora per ogni $n\in\N$
        \[\pa{x^\nu(n)}_\nu\subseteq \K\]
        \`e una successione di Cauchy in $\K$, quindi esiste $x\in\K^\N$ tale che $x^\nu\to x$ puntualmente.
        \item Verificare che $x\in\ell_\sigma$: essendo di Cauchy, $x^\nu$ \`e limitata, cio\`e $\sigma(x^\nu)\leq R$ per qualche $R\in \R$, dunque $\sigma(x)\leq R$ perch\'e $\cpa{\sigma\leq R}$ \`e chiuso.
        \item Verficare che $\sigma(x^\nu-x)\to 0$: Per ogni $\e>0$ esiste $n\in\N$ tale che per ogni $p,q\geq n$ vale $\sigma(x^p-x^q)\leq \e$. Notiamo che $x^p-x^n\to x-x^n$ puntualmente, quindi per la semicontinuit\`a si ha che per ogni $\e>0$ esiste $n\in\N$ tale che per ogni $q\geq n$ vale
        \[\sigma(x-x^q)\leq\liminf_{p\to +\infty}\sigma(x^p-x^q)\leq  \e\]
        cio\`e $\sigma(x-x^q)\to 0$ in norma $\sigma$.
    \end{itemize}
\end{itemize}
\end{proof}

\begin{remark}
Questa \`e una seconda dimostrazione della completezza di $\ell_p$ per $1\leq p\leq \infty$.
\end{remark}

\begin{remark}
Funziona anche l'analogo per paranorme, quindi in realt\`a segue anche la completezza di $\ell_p$ per $0<p\leq 1$.
\end{remark}




\section{Duali di \texorpdfstring{$\ell_p$}{lp}}

\begin{proposition}[Duali di $\ell_p$]\label{PrDualilp}
Se $p$ e $q$ sono esponenti coniugati ($\frac1p+\frac1q=1$, $\frac1\infty\doteqdot 0$) allora vale l'isometria $(\ell_p)^\ast\cong \ell_q$.
\end{proposition}
\begin{proof}
Esiste una inclusione lineare isometrica data da
\[\Phi:\funcDef{\ell_q}{(\ell_p)^\ast}{y}{\Phi_y: x\mapsto \sum_{i=0}^\infty y_ix_i},\]
dove la serie in esame converge assolutamente per la disuguaglianza di H\"older:
\[\sum\abs{x_iy_i}\leq \norm x_p\norm y_q.\]
Effettivamente $\Phi_y:\ell_p\to \K$ \`e lineare e continua per $\normd_{(\ell_p)^\ast}$, infatti
\[\norm{\Phi_y}_{(\ell_p^\ast)}=\sup_{\norm x_p\leq 1}\abs{\sum_{i=0}^\infty x_iy_i}\leq \sup_{\norm x_p\leq 1}\norm x_p\norm y_q=\norm y_q.\]
La stessa disuguaglianza mostra che $\Phi$ stessa \`e un elemento di $L(\ell_q,(\ell_p)^\ast)$ di norma minore o uguale a $1$.

Resta da mostrare che $\Phi$ \`e isometrica e surgettiva.

\setlength{\leftmargini}{0cm}
\begin{itemize}
\item[$\boxed{1< p<\infty}$] Per mostrare che $\norm{\Phi_y}_{\ell_p^\ast}=\norm y_q$ per ogni $y\in \ell_q$ consideriamo $x\in\K^\N$ dato da $x_i=\ol{\sgn y_i}\abs{y_i}^{q-1}$. Con questa scelta si ha che
\[x_iy_i=\ol{\sgn y_i}{\sgn y_i}\abs{y_i}^q=\abs{y_i}^q,\]
inoltre
\[\norm x_p^p=\sum_{i=0}^\infty \abs{x(i)}^p=\sum_{i=0}^\infty \abs{y_i}^{(q-1)p}=\sum_{i=0}^\infty \abs{y_i}^{q}=\norm y_q^q,\]
cio\`e $x\in \ell_p$ e 
\[\norm{\Phi_y}_{\ell_p^\ast}\geq \frac{\Phi_y(x)}{\norm x_p}=\frac{\sum_{i=0}^\infty \abs{y_i}^q}{(\norm y_q)^{q/p}}=\norm{y}_q^{q-q/p}=\norm{y}_q,\]
d'altronde sappiamo che vale anche l'altra disuguaglianza in generale, quindi abbiamo $\norm{\Phi_y}_{\ell_p^\ast}=\norm y_q$ come voluto.
\item[$\boxed{p=\infty,\ q=1}$] Sia $x_i=\ol{\sgn y_i}$. Segue che $\norm x_\infty\leq 1$ quindi \`e un elemento valido e
\[\Phi_y(x)=\norm y_1,\]
da cui segue $\norm{\Phi_y}_{\ell_\infty^\ast}\geq \norm y_1$ come voluto.
\item[$\boxed{p=1,\ q=\infty}$] In generale $\norm{\Phi_y}_{\ell_1^\ast}$ non \`e raggiunto come $\Phi_y(x)$ per qualche $x$ \footnote{per esempio $y_i=1-2^{-i}$ perch\'e in tal caso $\Phi_y(x)=\sum (1-2^{-i})x_i<\sum \abs{x_i}=\norm x_1$}. La conclusione per\`o vale comunque.
\end{itemize}
Mostriamo ora che l'inclusione \`e surgettiva per $1\leq p<\infty$. Per ogni $\vp\in \ell_p^\ast$ cerchiamo $y\in\ell_q$ tale che $\vp=\Phi_y$. C'\`e un solo $y$ possibile, basta valutare $\vp$ negli $e_i=(\delta_{ij})_j$. Per ogni $m\in\N$ sia $P_m:\K^\N\to \K^m$ il proiettore sulle prime $m$-entrate. Considerando $P_m$ come operatore $P_m:\ell_p\to\K^m\subseteq \ell_p$ restringendo il dominio, definiamo $\vp_m=\vp\circ P_m=P_m^\ast\vp$. Infine, sia 
\[y_m=P_m y=(y(0),y(1),\cdots, y(m-1), 0,0,\cdots)=\sum_{i=0}^{m-1}y_ie_i,\] 
e notiamo che
\[\vp_m=\Phi_{y_m}\]
infatti sono entrambi elementi di $\ell_p^\ast$ e 
\begin{align*}
    \vp_m(e_k)=\vp(P_m(e_k))=&\begin{cases}
        \vp(e_k) &\text{se }k<m\\
        0 &\text{se }k\geq m
        \end{cases}\\
    \Phi_{y_m}(e_k)=\sum_{i=0}^\infty y_m(i)e_k(i)=y_m(i)=&\begin{cases}
        \vp(e_k) &\text{se }k<m\\
        0 &\text{se }k\geq m
        \end{cases}
\end{align*}
quindi $\vp_m$ e $\Phi_{y_m}$ coincidono su $(e_k)$, quindi sullo span di questi e quindi sulla chiusura di questo span, che \`e $\ell_p$ se $p<\infty$.

Essendo $\Phi$ isometrica
\[\norm{y_m}_q=\norm{\Phi_{y_m}}_{\ell_p^\ast}=\norm{\vp_m}_{\ell_p^\ast}\leq \norm\vp_{\ell_p^\ast}\]
quindi $\sum_{i=0}^{m-1}\abs{y(i)}^q\leq \norm\vp_{\ell_p^\ast}^q$ per ogni $m$, dunque passando al $\sup$ in $m$ 
\[\norm y_q\leq \norm\vp_{\ell_p^\ast}\]
e quindi $y$ era un elemento valido di $\ell_q$.
\setlength{\leftmargini}{0.5cm}
\end{proof}


\begin{proposition}\label{PrDualec0El1}
Si ha che $\ell_1\cong c_0^\ast$
\end{proposition}
\begin{proof}
Consideriamo
\[\Phi:\funcDef{\ell_1}{c_0^\ast}{y}{x\mapsto \sum_{i=0}^\infty x_i y_i}\]
Allora $\Phi$ \`e lineare e $\abs{\Phi_y(x)}\leq \norm x_\infty\norm y_1$. $\Phi$ \`e isometrica
\[\norm{\Phi_y}_{c_0^\ast}=\sup_{\norm x_\infty\leq 1, x\in c_0}\sum x_i y_i=\norm y_1,\]
infatti l'estremo superiore si realizza con la successione $x^n=\ol{\sgn y}\chi_{[0,n]}$ ($\Phi_y(x^n)=\sum_{i=0}^n\abs{y_i}$ e passando al limite in $n$ troviamo proprio $\norm y_1$). Inoltre $\Phi$ \`e surgettiva infatti $(e_k)_k\in\N\subseteq c_0$ genera un sottospazio denso.
\end{proof}

\subsection{\texorpdfstring{$\ell_1,\ c_0$}{l1, c0} e \texorpdfstring{$\ell_\infty$}{linf}}
\begin{definition}[Finita additivit\`a]
Una funzione $\mu:\Ps(S)\to\K$ \`e \textbf{finitamente additiva} se per ogni $A,B\subseteq S$ disgiunti, $\mu(A\cup B)=\mu(A)+\mu(B)$.
\end{definition}

\begin{remark}
Domanda: $c_0$ \`e un duale? Cio\`e, esiste $X$ Banach tale che $X^\ast$ \`e linearmente omeomorfo a $c_0$?

NO! Perch\'e $c_0$ non \`e complementato in $\ell_\infty$ (difficile da mostrare). Questo basta per (\ref{LmDualeComplementatoNelTriduale}).
\end{remark}

\begin{lemma}\label{LmSuccessioneInl1}
Se $X$ \`e un sottospazio $\infty$-dimensionale di $\ell_1$ allora esiste una successione $(x_k)\subseteq X$ e una successione di naturali $(T_k)\subseteq \N$ strettamente crescente tali che 
\[\begin{cases}
\norm {x_k}_1=1\\
\norm{x_k}_{1,[0,T_k]}=\sum_{i=0}^{T_k}\abs{x_k(i)}\geq 3/4\\
x_{k+1}\res{[0,T_k]}=0
\end{cases}\]
\end{lemma}
\begin{proof}
Scegliamo $x_0\in X$ di norma 1 e $T_0\in \N$ che abbia la seconda propriet\`a. Supponiamo ora di aver definito $x_0,\cdots, x_k$ e di avere $T_0<\cdots, T_k$, allora
\[X\cap \cpa{x\in\ell_1\mid x\res{[0,T_k]}=0}\neq \emptyset\]
in qunato intersezione fra un sottospazo di dimensione infinita e dei sottospazi di codimensione finita, infatti quell'intersezione si pu\`o scrivere come
\[\bigcap_{0\leq i\leq T_k}\cpa{x\in X\mid x(i)=0}.\]
Prendendo un elemento $x_{k+1}$ normalizzato in questa intersezione abbiamo esteso la successione. Per scegliere $T_{k+1}$ basta prenderlo maggiore di $T_k$ e tale che 
\[\norm{x_{k+1}}_{1,[0,T_{k+1}]}\geq 3/4.\]
\end{proof}

\begin{proposition}[]\label{PrSottospazioChiusoDil1Contienel1}
Se $Y\subseteq \ell_1$ \`e un sottospazio chiuso di dimensione infinita allora $Y$ contiene una copia di $\ell_1$.

\filosofia{\textsl{Se guardi a lungo dentro $\ell_1$, $\ell_1$ guarda dentro di te.}}
\end{proposition}
\begin{proof}
Sia $X$ sottospazio chiuso di dimensione infinita di $\ell_1$ e siano $(x_k)\subseteq \ell_1$ e $(T_k)\subseteq \N$ come nel lemma (\ref{LmSuccessioneInl1}) Definiamo l'operatore lineare
\[L:\funcDef{\ell_1}{X}{\la}{\sum_{k=0}^\infty\la_k x_k}\]
$L$ \`e ben definita perch\'e la serie \`e assolutamente convergente rispetto a $\normd_1$
\[\norm{\sum_{k=0}^\infty\la_k x_k}_1\leq \sum_{k=0}^\infty\abs{\la_k} \norm{x_k}_1\leq \norm \la_1.\]
Notiamo anche che $X$ chiuso e quindi $L$ continuo di norma $\norm L\leq 1$.

Sia $I_k=[0,T_k]$ e notiamo che
\begin{align*}
\norm{L\la}=&\norm{\sum_{k=0}^\infty\la_k x_k}\geq\norm{\sum_{k=0}^\infty\la_k x_k\res{I_k}}-\sum_{k=0}^\infty\norm{\la_k x_k\res{I_k^c}}=\\
=&\sum_{k=0}^\infty\abs{\la_k} \norm{x_k\res{I_k}}_1-\sum_{k=0}^\infty\abs{\la_k} \norm{x_k\res{I_k^c}}\geq\\
\geq&\frac34\norm\la_1-\frac14\norm{\la}_1=\frac12\norm\la_1
\end{align*}
dunque $L:\ell_1\to X$ \`e fortemente iniettivo e quindi \`e un isomorfismo con l'immagine in quanto questa \`e chiusa.
\end{proof}


\begin{exercise}
$c_0$ non \`e un duale.
\end{exercise}
\begin{proof}
Segue dalla proposizione (\ref{PrSottospazioChiusoDil1Contienel1}): se esistesse $X$ tale che $X^\ast\cong c_0$ allora $\iota_X:X\inj X^{\ast\ast}\cong \ell_1$ e quindi per la proposizione $X$ contiene un sottospazio $Y$ isomorfo a $\ell_1$, ma allora da $Y\subseteq X$ segue 
\[\ell_\infty\cong \ell_1^\ast\cong Y^\ast\pasgnl\cong{(\ref{PrDualeDiSottospaziEDualeQuoziente})} X^\ast/\Ann(Y)\cong c_0/\Ann Y\]
ma $\ell_\infty$ non \`e separabile mentre $c_0$ \`e separabile e ogni quoziente di un separabile deve essere separabile.
\end{proof}


\begin{proposition}\label{PrDualelinftyContienel1}
Si ha che $\ell_1\inj \ell_\infty^\ast$ \`e una immersione isometrica NON surgettiva.
\end{proposition}
\begin{proof}
L'iniezione \`e chiara. Consideriamo le funzioni che hanno limite (le costanti a meno di una infinitesima)
\[c=\cpa{x\in\ell_\infty\mid \exists \lim_{i\to\infty}x_i}\cong c_0\oplus \R\]
Esiste un funzionale su $c$ dato da
\[\la:\funcDef{c}{\K}{y}{\lim y_i}.\]
Questo \`e continuo perch\'e $\norm \la\leq 1$ (perch\'e $\abs{\lim y_i}\leq \norm y_\infty$). Per il teorema di Hahn-Banach (\ref{CorHahnBanachPerSpaziNormati}) si estende ad un funzionale continuo in $\ell_\infty$.

Consideriamo
\[ba=\cpa{\mu:\Ps(\N)\to\K\mid \text{limitate e finitamente additive.}}\subseteq (\Bs(\Ps(\N),\K),\normd_\infty)\]
e la mappa
\[\Psi:\funcDef{\ell_\infty^\ast}{ba}{y}{A\mapsto y(\chi_A)}\]
Notiamo che $\Psi$ \`e surgettiva: se $\mu\in ba$ e definiamo una funzione lineare su $\ell_\infty$ come segue
\begin{itemize}
    \item Se $x\in\ell_\infty$ \`e della forma $x=\sum c_i\chi_{A_i}$, cio\`e $x(\N)\subseteq \cpa{\sum_{i\in J}c_i\mid J\subseteq \cpa{0,\cdots, n}}$ \`e finito, quindi
    \[x=\sum_{c\in\K}c\chi_{\cpa{x=0}}\text{ \`e una somma finita}\]
    Sia $S$ il sottoinsieme di $\ell_\infty$ delle successioni di questa forma.
    Mostriamo che $\ol{S}=\ell_\infty$ dove la chiusura \`e presa rispetto a $\normd_\infty$. Per ogni $x\in \ell_\infty$ con $x:\N\to\R$ si ha che
    \[x-2^{-n}\leq x^n=\frac{\floor{2^n x}}{2^n}\leq x\]
    \item Per $x\in S$ dato da $x=\sum_{i=1}^n c_i\chi_{A_i}$ poniamo $\ps{\mu,x}=\sum_{i=1}^n c_i\mu(A_i)$. Chiaramente abbiamo linearit\`a e la buona definizione segue dal fatto che questa espressione coincide con $\sum_{c\in\K}c\mu\pa{\cpa{x=0}}$ per finita additivit\`a, ma questa forma \`e univocamente determinata da $x$.
    \item Vale che $\abs{\ps{\mu,x}}\leq \norm x_\infty\pa{\sum_{c\in \K} \abs{\mu(\cpa{x=c})}}$ infatti per ogni $c$, se $\mu(\cpa{x=c})\neq 0$ allora $\abs c\leq \norm x_\infty$ per definizione.
    \item Per ogni $\mu$ esiste $C$ tale che per ogni $x\in S$ si ha 
    \[\abs{\ps{\mu,x}}\leq C\norm x_\infty\]
    (VERIFICARE!)
    \item Quindi $\mu$ si estende alla chiusura di $S$, che \`e tutto $\ell_\infty$. 
\end{itemize}
Notiamo che $ba=\ell_\infty^\ast=\ell_1^{\ast\ast}=c_0^{\ast\ast\ast}$, quindi $\ell_1\inj \ell_1^{\ast\ast}=ba$ \`e complementato per il lemma (\ref{LmDualeComplementatoNelTriduale}).
\end{proof}



\begin{proposition}[Convergenza forte e debole coincidono su $\ell_1$]\label{PrConvergenzaForteEDeboleCoincidonoSul1}
La convergenza debole e la convergenza in norma per $\ell_1$ sono la stessa cosa.
\end{proposition}
\begin{proof}
Poich\'e la topologia debole \`e meno fine della topologia forte basta mostrare che convergenza debole implica convergenza in $\normd_1$.

Sia $f_n\to f$ in $w$-$\ell_1$, cio\`e per ogni funzionale $\phi$ lineare continuo su $\ell_1$ si ha che $\ps{\phi,f_n}\to\ps{\phi,f}$. Dunque, ricordando (\ref{PrDualilp}) che $\ell_\infty=\ell_1^\ast$, per ogni $\vp\in\ell_\infty$ si ha
\[\sum_i\vp(i)f_n(i)\to \sum_i\vp(i)f(i).\]
Notiamo che, portando $f$ al primo membro possiamo supporre senza perdita di generalit\`a $f=0$. Per la propriet\`a di Uhrison (\ref{PrProprietaUhrisohn}) basta provare che esiste una sottosuccessione di $f_n$ che converge a $0$ infatti $f_n\to 0$ se e solo se per ogni sottosuccessione $f_{n_k}$ esiste una sotto-sottosuccessione $f_{h_{k_j}}\to 0$.

Nel caso particolare di successioni $f_n$ a supporto disgiunto la tesi \`e facile: Se siamo in questo caso scegliamo $\vp\in\ell_\infty$ data da 
\[\vp=\sum \ol{\sgn f_i}\text{ dove }(\sgn f_i)(x)=\sgn(f_i(x))=\begin{cases}
    f_i(x)/\abs{f_i(x)} & f_i(x)\neq 0\\
    0 & f_i(x)=0
\end{cases}\] 
in modo tale che $\ps{\vp,f_n}=\ps{\ol{\sgn(f_n)},f_n}=\norm{f_n}_1$ e stesso per $f$, quindi in questo caso \`e chiaro che convergenza debole implica convergenza in $\normd_1$.

Assumiamo dunque che $f_n\to 0$ debolmente (e quindi puntualmente guardando i funzionali che estraggono la $n$-esima entrata). Basta provare che esiste una successione $(g_j)_{j\geq 0}\subseteq \ell_1$ e una sottosuccessione $f_{n_j}$ tale che $\norm{f_{n_j}-g_j}_1\to 0$ e $g_j$ hanno supporto disgiunto. 

Costruiamo le $g_j$ per induzione. Notiamo che 
\begin{itemize}
    \item per ogni $f_n$ si ha che $\norm{f_n\chi_{\N\bs[0,T]}}_1\to 0$ per $T\to\infty$
    \item per ogni $T$ si ha $\norm{f_n\chi_{[0,T]}}_1\to 0$ per $n\to \infty$
\end{itemize}
quindi per costruire le $g_j$ basta alternare questi fatti prendendo opportuni limiti (credo?).
\end{proof}

\begin{remark}
Questo \`e un esempio dove due topologie diverse hanno ``le stesse successioni convergenti".
\end{remark}