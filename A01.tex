\chapter{Topologia}


\begin{proposition}[Topologia iniziale]\label{PrTopologiaInizialeEsiste}
    Sia $X$ un insieme e $\Fc$ una famiglia di mappe a valori in uno spazio topologici. Notazione:
    \[\Fc=\cpa{f_j:X\to (Y_j,\tau_j)}_{j\in I}.\]
    Allora esiste la topologia meno fine su $X$ che rende continue le mappe $f_j$. Una prebase di questa topologia \`e data da
    \[\cpa{f_j\ii(A)\mid j\in I, A\in \tau_j}.\]
    In realt\`a basterebbe prendere una prebase per $\tau_j$ al posto di tutta la topologia.

    Questa topologia \`e detta \textbf{topologia iniziale della famiglia $\Fc$} e si denota $\tau_\Fc$.
\end{proposition}

\begin{remark}[Propriet\`a universale della topologia iniziale]\label{PrProprietaUniversaleTopologiaIniziale}
Data una mappa $\vp:(Z,\tau_Z)\to (X,\tau_\Fc)$ essa \`e continua se e solo se $f\circ \vp$ \`e continua per ogni $f\in \Fc$.
\end{remark}
\begin{proof}
Se $\vp$ \`e continua allora $f\circ \vp$ \`e composizione di continue. Se sappiamo che $f\circ \vp$ \`e continua per ogni $f\in\Fc$ allora, se $A$ \`e un aperto di $X$ per continuit\`a di $f\circ \vp$ abbiamo
\[\tau_Z\ni(f\circ \vp)\ii(A)=\vp\ii(f\ii(A))\]
cio\`e le preimmagini tramite $\vp$ di aperti di prebase sono aperti di $Z$, quindi $\vp$ \`e continua.
\end{proof}


\begin{proposition}[Transitivit\`a della topologia iniziale]\label{PrTransitivitaTopologiaIniziale}
Supponiamo di avere una famiglia di mappe $\Fc'=\cpa{f_i:X\to Y_i}_{i\in I}$ e per ogni $i\in I$ sia $\Gc_i=\cpa{g_{ij}:Y_i\to Z_{ij}}_{j\in J_i}$ una famiglia di mappe. Su ogni $Y_i$ consideriamo la topologia iniziale determinata da $\Gc_i$. Allora la topologia iniziale data da $\Fc'$ su $X$ coincide con la topologia iniziale su $X$ definita da $\Fc=\cpa{g_{ij}\circ f_i\mid i\in I,\ j\in J_i}$.
\end{proposition}
\begin{proof}
Entrambe le topologie in esame sono generate dagli insiemi $(g_{ij}\circ f_i)\ii(A)$ al variare di $i\in I$, $j\in J_i$ e $A\in \tau_{Z_{ij}}$, infatti
\[\text{prebase per $\Fc\to$ }(g_{ij}\circ f_i)\ii(A)=f_j\ii(g_{ij}\ii(A))\text{ $\leftarrow$ prebase per $\Fc'$}.\]
\end{proof}

\section{Limiti induttivi su spazi topologici}
\begin{proposition}
Sia $\cpa{(X_n,\tau_n)}_{n\in\N}$ una famiglia di spazi topologici con inclusioni continue $X_n\subseteq X_{n+1}$. Allora esiste la pi\`u fine topologia $\tau_\infty$ su $X_\infty=\bigcup_{n\in\N}X_n$ che rende continue le inclusioni $X_n\subseteq X_\infty$.

La topologia in questione \`e
\begin{align*}
    \tau_\infty=&\cpa{A\subseteq X_\infty\mid \forall n\in\N,\ A\cap X_n\in\tau_n}=\\
    =&\cpa{A\subseteq X_\infty\mid A=\bigcup_{n\in\N} A_n,\ A_n\subseteq A_{n+1},\ A_n\in \tau_n}.
\end{align*}
\end{proposition}
\begin{proof}
Poich\'e la continuit\`a delle inclusioni si traduce in ``$\forall n,\ A\cap X_n\in \tau_n$" basta verificare che questa condizione definisce una topologia, ma questo \`e ovvio perch\'e 
\begin{itemize}
    \item $(\bigcup A_i)\cap X_n=\bigcup A_i\cap X_n$, 
    \item $(A\cap B)\cap X_n=(A\cap X_n)\cap (B\cap X_n)$, 
    \item $\emptyset\cap X_n=\emptyset$ e 
    \item $X_\infty\cap X_n=X_n$.
\end{itemize}
\end{proof}
\begin{remark}
Se ogni inclusione $X_n\subseteq X_{n+1}$ \`e inclusione di sottospazio, cio\`e $\tau_n$ \`e la topologia indotta, allora $\tau_\infty$ induce $\tau_n$ come topologia di sottospazio $X_n\subseteq X_\infty$.
\end{remark}
\begin{proof}
Se $A_0\subseteq X_0$ aperto allora esiste $A_1\in \tau_1$ tale che $A_0=A_1\cap X_0$ perch\'e $\tau_0$ \`e la topologia indotta da $\tau_1$. Iterando troviamo $A_n$ aperti inscatolati, quindi $A=\bigcup_n A_n$ e per costruzione $A\cap X_0=A_0$. Per gli indici pi\`u alti \`e uguale.
\end{proof}

\begin{remark}
$f:X_\infty\to Z$ \`e continua se e solo se per ogni $n\in\N$, $f\res{X_n}\to Z$ \`e continua.
\end{remark}

\begin{remark}
Il limite su sottosuccessione $\cpa{X_{n_k}}_{k\geq 0}$ \`e sempre $X_{\infty}$ con la stessa topologia.
\end{remark}

\begin{remark}
In generale un limite induttivo di SVT $X_n\subseteq X_{n+1}$ con inclusioni lineari \`e uno spazio topologico $(X_\infty,\tau_\infty)$ e uno spazio vettoriale, ma NON \`e uno SVT.

Il motivo \`e che la somma su $X_\infty$ non \`e necessariamente continua in quanto in generale $\varinjlim (X_n\times X_n)\neq \varinjlim X_n\times\varinjlim X_n$ anche se vale uguaglianza insiemistica.

$+:\bigcup X_n\times \bigcup X_n\to \bigcup X_n$ \`e tale che la restrizione a $X_n\times X_n$ \`e continua, ma questo non implica la continuit\`a della intera mappa.
\end{remark}



