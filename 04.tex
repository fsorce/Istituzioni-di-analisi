\chapter{Topologie deboli, Limitatezza e Banach-Steinhaus}

\section{Topologie deboli}

\begin{proposition}[Topologia iniziale nel caso SVT]\label{PrTopologiaInizialeCasoSVT}
Sia $X$ uno spazio vettoriale su $\K$ e sia $\Fc:\cpa{T_i:X\to Y_i}$ dove ogni $Y_i$ \`e SVT e $T_i$ \`e lineare, allora la topologia iniziale su $X$ indotta\footnote{vedi (\ref{PrTopologiaInizialeEsiste})} da $\Fc$ rende $X$ uno SVT.
\end{proposition}
\begin{proof}
Voglio verificare che $+$ e $\cdot$ sono mappe continue per la topologia iniziale.
% https://q.uiver.app/#q=WzAsNCxbMCwwLCJYXFx0aW1lcyBYIl0sWzEsMCwiWCJdLFswLDEsIllfaVxcdGltZXMgWV9pIl0sWzEsMSwiWV9pIl0sWzEsMywiVF9pIl0sWzAsMiwiVF9pXFx0aW1lcyBUX2kiLDJdLFswLDEsIisiXSxbMiwzLCIrX2kiLDJdXQ==
\[\begin{tikzcd}
	{X\times X} & X \\
	{Y_i\times Y_i} & {Y_i}
	\arrow["{+}", from=1-1, to=1-2]
	\arrow["{T_i\times T_i}"', from=1-1, to=2-1]
	\arrow["{T_i}", from=1-2, to=2-2]
	\arrow["{+_i}"', from=2-1, to=2-2]
\end{tikzcd}\]
% https://q.uiver.app/#q=WzAsNCxbMCwwLCJcXEtcXHRpbWVzIFgiXSxbMSwwLCJYIl0sWzAsMSwiXFxLXFx0aW1lcyBZX2kiXSxbMSwxLCJZX2kiXSxbMSwzLCJUX2kiXSxbMCwyLCJpZF9cXEtcXHRpbWVzIFRfaSIsMl0sWzAsMSwiXFxjZG90Il0sWzIsMywiXFxjZG90X2kiLDJdXQ==
\[\begin{tikzcd}
	{\K\times X} & X \\
	{\K\times Y_i} & {Y_i}
	\arrow["\cdot", from=1-1, to=1-2]
	\arrow["{id_\K\times T_i}"', from=1-1, to=2-1]
	\arrow["{T_i}", from=1-2, to=2-2]
	\arrow["{\cdot_i}"', from=2-1, to=2-2]
\end{tikzcd}\]
Per la propriet\`a universale della topologia iniziale (\ref{PrProprietaUniversaleTopologiaIniziale}), vogliamo verificare che $T_i\circ +=+_i\circ (T_i\times T_i)$ \`e continua per ogni $i$ e similmente per $T_i\circ \cdot$. Questo \`e vero perch\'e la topologia iniziale \`e rende $T_i$ continua per ogni $i$.
\end{proof}

\begin{remark}
Se ogni $Y_i$ inoltre \`e SVTLC allora anche $X$ lo \`e.
\end{remark}


\begin{definition}[Topologie deboli]
Sia $X$ un $\K$-spazio vettoriale e $\Fc\subseteq X'$ (duale algebrico). La topologia iniziale indotta da $\Fc$ viene detta la \textbf{topologia debole di $\Fc$} e si indica $\sigma(X,\Fc)$.
\end{definition}

\begin{remark}
$\sigma(X,\Fc)=\sigma(X,\Span_\K(\Fc))$ quindi senza perdita di generalit\`a possiamo sempre supporre $\Fc$ sottospazio vettoriale di $X'$.
\end{remark}

\begin{remark}
La famiglia di seminorme associata a $\Fc$ (quella che induce la stessa topologia di $SVTLC$) \`e data da
\[\Pc=\cpa{\abs{f}\mid f\in\Fc}\]
\end{remark}

\begin{remark}
La topologia debole $\sigma(X,\Fc)$ \`e $T_0$ (e quindi Hausdorff perch\'e SVT) se e solo se la famiglia $\Fc$ \`e separante ($\forall x\in X\nz,\ \exists f\in\Fc$ tale che $f(x)\neq0$).
\end{remark}


\begin{lemma}[]\label{LmDualeAlgebricoIndipendenzaEContinuita}
Siano $f_0,\cdots,f_n\in X'_{alg}$ per $X$ un $\K$-spazio vettoriale, allora sono equivalenti
\begin{enumerate}
    \item $f_0=\sum_{i=1}^n\la_if_i$
    \item $\abs{f_0}\leq M\max_{i\in\cpa{1,\cdots, n}}\abs{f_i}$ per qualche $M\geq 0$
    \item $\ker f_0\supseteq \bigcap_{i=1}^n\ker f_i$
\end{enumerate}
\end{lemma}
\begin{proof}
    Diamo le tre implicazioni
    \setlength{\leftmargini}{0cm}
    \begin{itemize}
    \item[$\boxed{1.\implies2.}$] Da $1.$ segue $\abs{f_0}\leq\sum_{i=1}^n\abs{\la_i}\abs{f_i}\leq M\max\abs{f_i}$ per $M=\sum\abs{\la_i}$.
    \item[$\boxed{2.\implies3.}$] Se $x\in \bigcap\ker f_i$, cio\`e $\ps{f_i,x}=0$ per ogni $i$, allora $\ps{f_0,x}\leq M0=0$, cio\`e $f_0(x)=0$ e abbiamo l'inclusione voluta.
    \item[$\boxed{3.\implies1.}$] Sia $F:X\to\K^n$ data da $F=(f_1,\cdots, f_n)$, allora
    \[\ker F=\bigcap\ker f_i\subseteq \ker f_0\]
    quindi abbiamo una fattorizzazione
    % https://q.uiver.app/#q=WzAsMyxbMCwwLCJYIl0sWzAsMSwiXFxLXm4iXSxbMSwwLCJcXEsiXSxbMSwyLCJMIiwyLHsic3R5bGUiOnsiYm9keSI6eyJuYW1lIjoiZGFzaGVkIn19fV0sWzAsMSwiRiIsMl0sWzAsMiwiZl8wIl1d
\[\begin{tikzcd}
	X & \K \\
	{\K^n}
	\arrow["{f_0}", from=1-1, to=1-2]
	\arrow["F"', from=1-1, to=2-1]
	\arrow["L"', dashed, from=2-1, to=1-2]
\end{tikzcd}\]
    dove $L(x_1,\cdots, x_n)=\sum \la_i x_i$ per dei $\la_i$ (in quanto \`e una forma lineare). Ma allora $f_0=L\circ F=\sum \la_i f_i$ come voluto.
    \end{itemize}
    \setlength{\leftmargini}{0.5cm}
\end{proof}

\begin{proposition}[Duale per topologia debole]\label{PrDualePerTopologiaDebole}
Dato $X$ $\K$-spazio vettoriale e $\Fc$ sottospazio di $X'_{alg}$ allora
\[(X,\sigma(X,\Fc))^\ast=\Fc\]
\end{proposition}
\begin{proof}
Sia $f_0\in (X,\sigma(X,\Fc))^\ast$, allora per la proposizione (\ref{PrContinuitaLineariSVT}) esistono $f_1,\cdots, f_n\in\Fc$ e $M\geq0$ tali che per ogni $x\in X$
\[\abs{f_0(x)}\leq M\max_{i}\abs{f_i(x)}.\]
Dunque per il lemma (\ref{LmDualeAlgebricoIndipendenzaEContinuita}) $f_0$ si scrive come combinazione lineare delle $f_i$ e quindi in particolare $f_0\in\Fc$.

L'altra inclusione \`e ovvia per definizione di topologia debole.
\end{proof}

\begin{remark}
Se $X$ ha dimensione infinita, $\sigma(X,\Fc)$ non \`e mai localmente limitata. In particolare ogni intorno di $0$ contiene uno spazio vettoriale di codimensione finita.
\end{remark}
\begin{proof}
Se $U$ intorno di $0$ per $\sigma(X,\Fc)$ allora esistono $f_1,\cdots,f_n\in\Fc$ tali che\footnote{vedi lemma (\ref{LmDualeAlgebricoIndipendenzaEContinuita})}
\[U\supseteq \bigcap_{i=1}^n\cpa{\abs{f_i}<1}\supseteq \bigcap_{i=1}^n \ker f_i\]
e l'intersezione di questi nuclei ha codimensione al massimo $n$.
\end{proof}

\begin{proposition}[Duale di lineare continua \`e debole$^\ast$-continua]\label{PrDualeOperatoreContinuoEDeboleStarContinua}
Se $T:E\to F$ \`e un operatore lineare e continuo allora $T^\ast:F^\ast\to E^\ast$ \`e debole$^\ast$-continua.
\end{proposition}
\begin{proof}
Considera le opportune composizione e la definizione di topologia debole.
\end{proof}

\subsection{Caso degli spazi normati}

\begin{definition}[Topologia debole]
Se $X$ \`e normato, la \textbf{topologia debole} su $X$ \`e la topologia debole associata a $X^\ast$, cio\`e $\sigma(X,X^\ast)$.
\end{definition}

\begin{proposition}
La topologia debole \`e localmente convessa e Hausdorff.
\end{proposition}
\begin{proof}
Per Hahn-Banach (\ref{ThHahnBanach}), il duale $X^\ast$ separa i punti
\end{proof}


\begin{definition}[Topologia debole$^\ast$]
Su $X^\ast$ possiamo considerare la topologia debole associata alle valutazioni $X\subseteq X^{\ast\ast}$, cio\`e scegliendo
\[\Fc=\cpa{val_x\in (X^\ast)'\mid x\in X}.\]
Questa \`e la \textbf{topologia debole$^\ast$} su $X^\ast$ e la indichiamo $\sigma(X^\ast,X)$.
\end{definition}

\begin{remark}
La topologia debole$^\ast$ rende $X^\ast$ uno SVTLC $T_0$ (e quindi Hausdorff), infatti se $f\in X^\ast\nz$ allora esiste $x\in X$ tale che $f(x)\neq 0$.
\end{remark}

\begin{remark}
In generale $\sigma(X^\ast,X)$ \`e meno fine di $\sigma(X^\ast,X^{\ast\ast})$. Abbiamo uguaglianza solo quando $X=X^{\ast\ast}$ in quanto se $X\neq X^{\ast\ast}$ allora dalla proposizione (\ref{PrDualePerTopologiaDebole}) ricaviamo
\[(X^\ast,\sigma(X^\ast,X))^\ast=X\neq X^{\ast\ast}=(X^\ast,\sigma(X^\ast,X^{\ast\ast}))^\ast\]
e quindi in partenza $\sigma(X^\ast,X^{\ast\ast})\neq \sigma(X^\ast,X)$
\end{remark}

\begin{remark}
Poich\'e $(X,\normd)\inj (X^{\ast\ast},\normd)$ isometricamente allora $(X,\sigma(X,X^\ast))$ ha la topologia indotta come sottospazio da\footnote{nota che $X^\ast$ lo si pu\`o pensare come immerso in $X^{\ast\ast\ast}=(X^{\ast\ast})^\ast$, quindi stiamo considerando la topologia debole$^\ast$ su $(X^{\ast})^\ast$} $(X^{\ast\ast},\sigma(X^{\ast\ast},X^\ast))$.
\end{remark}
\begin{proof}
Questo deriva dalla transitivit\`a della topologia iniziale (\ref{PrTransitivitaTopologiaIniziale}) dove la prima famiglia \`e la mappa $X\inj X^{\ast\ast}$ e l'unica altra famiglia sono gli elementi di $X^\ast$ che vanno verso $\K$.
\end{proof}

\section{Spazi di Baire e II-categoria}
\begin{theorem}[Baire]\label{ThBaire}
Se $\cpa{A_k}_{k\in\N}$ \`e una famiglia numerabile di aperti densi di uno spazio metrico completo allora $\bigcap A_k$ \`e denso.
\end{theorem}
\begin{proof}
Per induzione si definisce una successione di palle chiuse di $X$ dove $B_0$ \`e arbitraria e
\[B_k=\ol{B(x_k,r_k)}\text{ tali che }B_{k+1}\subseteq B_k\cap A_k\text{ e }r_k=o(1)\]
che possiamo fare perch\'e $A_k$ \`e un aperto denso.

Allora la successione dei centri \`e una successione di Cauchy, infatti se $p,q\geq n$ si ha $x_p,x_q\in B_n$ e quindi $d(x_p,x_q)\leq 2r_n$. Dunque $x_n\to x^\ast$ in $X$ per completezza. Inoltre, poich\'e $x_k\in B_n$ definitivamente, $x^\ast=\lim x_k\in B_n$ per ogni $n$ (dato che $B_n$ \`e chiuso). In particolare $x^\ast\in B_{n+1}\subseteq A_n$ per ogni $n$ e quindi $x^\ast\in \bigcap A_n$. Per costruzione $x^\ast\in B_0$, quindi per ogni palla $B_0$ abbiamo mostrato che $B_0\cap \bigcap A_n\neq \emptyset$, cio\`e $\bigcap A_n$ \`e denso.
\end{proof}

\begin{exercise}
La stessa conclusione vale se $X$ \`e localmente compatto al posto di metrico completo.
\end{exercise}

\begin{definition}[Spazio di Baire]
Uno spazio topologico \`e \textbf{di Baire} se ogni intersezione numerabile di aperti densi \`e densa.
\end{definition}

\begin{remark}
Ogni aperto non vuoto di $X$ di Baire \`e ancora di Baire. Basta verificare che ogni aperto denso di $A$ \`e della forma $A\cap U$ con $U$ aperto denso di $X$.
\end{remark}


\begin{definition}[Sottoinsieme di I- e II-categoria]
Un sottoinsieme $S$ di $X$ \`e di \textbf{I-categoria (di Baire) in $X$} se \`e unione numerabile di insiemi $(E_i)_{i\in\N}$ con $int(\ol{E_i})=\emptyset$.

Inoltre $S$ \`e di \textbf{II-categoria (di Baire) in $X$} se non \`e di I-categoria.
\end{definition}

\begin{remark}
Se $X$ \`e di Baire e $S\subseteq X$ \`e di I-categoria allora $X\bs S$ \`e di II-categoria in quanto $X$ stesso \`e di II-categoria (se $X=\bigcup E_i$ con $E_i$ chiusi a parte interna vuota allora $\emptyset=\bigcap E_i^c$ con $E_i^c$ aperti densi, ma questo \`e assurdo perch\'e $X$ di Baire).
\end{remark}

\section{Teorema di Banach-Steinhaus}

\begin{definition}[Famiglia equicontinua]
Una famiglia $\Gamma$ di operatori lineari continui fra SVT $X$ e $Y$ \`e \textbf{equicontinua} se per ogni $U\in\Uc_Y$ esiste $V\in \Uc_X$ tale che per ogni $T\in\Gamma$, $T(V)\subseteq U$.
\end{definition}

\begin{remark}
Possiamo riformulare la condizione nei seguenti modi: per ogni $U\in\Uc_Y$ esiste $V\in\Uc_X$ tale che
\[\forall T\in\Gamma,\ V\subseteq T\ii(U)\coimplies V\subseteq \bigcap_{T\in \Gamma}T\ii(U)\doteqdot \Gamma\ii(U).\]
Equivalentemente la condizione predica che per ogni $V\in\Uc_Y$ si abbia $\Gamma\ii(V)\in\Uc_X$.
\end{remark}

\begin{remark}
Se $T:X\to Y$ fra spazi normati, la norma degli operatori
\[\norm T=\norm T_{\infty,B(0,1)}=\text{migliore costante di Lipschitz per }T.\]
\end{remark}

\begin{example}
Se $X$ e $Y$ sono normati, $\Gamma$ \`e equicontinua se e solo se $\Gamma$ \`e limitato in $L(X,Y)$ rispetto alla norma degli operatori.
\end{example}

\begin{theorem}[Banach-Steinhaus / Uniforme limitatezza]\label{ThBanachSteinhausUniformeLimitatezza}
Siano $X,Y$ SVT, $S\subseteq X$ di seconda categoria e $\Gamma\subseteq L(X,Y)$ con $\Gamma$ puntualmente limitata su $S\subseteq X$, cio\`e per ogni $s\in S$, $\Gamma(s)=\bigcup_{T\in \Gamma} T(s)$ \`e limitato in $Y$.

Allora $\Gamma$ \`e equicontinua.
\end{theorem}
\begin{proof}
Sia $U\in\Uc_Y$ e consideriamo $V\in\Uc_Y$ chiuso tale che $V-V\subseteq U$. Per ipotesi, per ogni $x\in S$ si ha che $\Gamma(x)$ \`e limitato in $Y$, quindi viene assorbito da $V$, cio\`e esiste $n_x\in\N$ tale che per ogni $T\in\Gamma$ si ha $T(x)\in n_x V$, cio\`e tale che
\[x\in\bigcap_{T\in \Gamma}n_xT\ii(V)=n_x\bigcap_{T\in \Gamma}T\ii(V)=n_x\Gamma\ii(V).\]
Dunque $S\subseteq \bigcup_{n\in\N}n\Gamma\ii(V)$. Notiamo che poich\'e $V$ \`e chiuso, $T\ii(V)$ \`e chiuso e quindi anche $\Gamma\ii(V)$ lo \`e perch\'e intersezione di chiusi. Poich\'e $S$ \`e di seconda categoria anche l'unione delle versioni riscalate di $\Gamma\ii(V)$ lo \`e, dunque questo insieme non \`e unione numerabile di chiusi con parte interna vuota, quindi almeno uno tra gli $n\Gamma\ii(V)$ ha parte interna non vuota, quindi anche $\Gamma\ii(V)$ ha parte interna non vuota scalando per $\frac1n$.

Quindi $\Gamma\ii(V)$ \`e intorno di qualche suo punto, dunque\footnote{se $a_0\in int(A)$ allora $A-a_0\subseteq A-A$ \`e un intorno di $0$.} $\Gamma\ii(V)-\Gamma\ii(V)$ \`e un intorno di $0$. 

Ricordando che $V-V\subseteq U$ si ha
\[T\ii(U\supseteq T\ii(V-V)=T\ii(V)-T\ii(V))\supseteq \Gamma\ii(V)-\Gamma\ii(V)\]
quindi passando all'intersezione su $T\in\Gamma$ si ha
\[\Gamma\ii(U)\supseteq \Gamma\ii(V)-\Gamma\ii(V)\in\Uc_X,\]
cio\`e abbiamo mostrato che per ogni $U\in\Uc_Y$ si ha $\Gamma\ii(U)\in\Uc_X$, che \`e equivalente all'equicontinuit\`a di $\Gamma$.
\end{proof}

\begin{corollary}[Sottoinsiemi limitati di operatori]\label{CorPuntualmenteLimitatoImplicaLimitatoPerNormaOperatore}
    Se $X$ e $Y$ sono Banach e $\Gamma\subseteq L(X,Y)$ \`e puntualmente limitata in $X$ (o volendo anche un sottoinseme di $X$ di II-categoria) allora $\Gamma$ \`e un insieme limitato in $L(X,Y)$.
\end{corollary}
\begin{proof}
Diretta applicazione di Banach-Stenhaus (\ref{ThBanachSteinhausUniformeLimitatezza}) notando che spazi di Banach sono in particolare SVT e che equicontinuit\`a per la norma su $L(X,Y)$ significa limitatezza. **************
\end{proof}

\begin{exercise}
Siano $X,Y$ SVT. Trovare la topologia meno fine $\tau$ di SVT su $L(X,Y)$ per la quale
\[\Gamma\text{ puntualmente limitato in $L(X,Y)$}\coimplies \Gamma\text{ limitato nella topologia $\tau$}.\]
\end{exercise}


\begin{corollary}\label{CorConvergenzaLineariSePuntualmenteConvergente}
Siano $X$ e $Y$ Banach e sia $(T_n)\subseteq L(X,Y)$ puntualmente convergente. Allora il limite $T$ \`e ancora lineare, continuo e con norma
\[\norm T\leq \liminf_{n\to\infty}\norm{T_n}.\]
\end{corollary}
\begin{proof}
Per il corollario precedente (\ref{CorPuntualmenteLimitatoImplicaLimitatoPerNormaOperatore}) si ha che $(T_n)$ sono limitati in $\normd$ e il limite puntuale \`e lineare in quanto
\[T_n(\al x+\beta y)=\al T_n(x)+\beta T_n(y)\to \al T(x)+\beta T(y).\]
Questo mostra che $T$ \`e limitato e lineare, quindi $T\in L(X,Y)$.

Inoltre per ogni $x\in X$ si ha
\[\norm{T(x)}=\lim_{n}\norm{T_n(x)}\leq \pa{\sup_n\norm{T_n}}\norm x,\]
quindi $\norm T\leq \sup_n\norm{T_n}$. Ragionando analogamente per una sottosuccessione di $(T_n)$ che in norma converge a $\liminf_n\norm{T_n}$ ricaviamo
\[\norm T\leq \liminf_n\norm{T_n}.\]
\end{proof}

\begin{remark}
In generale NON vale $T_n\to T$ in $\normd$.
\end{remark}

\begin{proposition}[Bilineare separatamente continua \`e continua]\label{PrBilineareSeparatamenteContinuaEContinua}
Sia $b:X\times Y\to Z$ bilineare e separatamente continua, cio\`e per ogni $x\in X, y\in Y$ si ha che $b(x,\cdot):Y\to Z$ e $b(\cdot, y):X\to Z$ sono lineari e continue. Allora $b$ \`e continua, cio\`e
\[\sup_{\norm x\leq 1, \norm y\leq 1}\norm{b(x,y)}<\infty.\]
\end{proposition}
\begin{proof}
Consideriamo la famiglia
\[\Gamma=\cpa{b(x,\cdot):Y\to Z}_{x\in X,\ \norm x\leq 1}\subseteq L(Y,Z).\]
Per ipotesi $\Gamma$ \`e puntualmente limitata in $Y$, infatti per ogni $y\in Y$
\[\sup_{b(x,\cdot)\in\Gamma} \norm{b(x,\cdot)}_{L(Y,Z)}=\sup_{\norm x\leq 1}\norm{b(x,y)}_Z= \norm{b(\cdot,y)}_{L(X,Z)}\norm y<\infty\]
Allora $\Gamma$ \`e limitata in $\normd_{L(Y,Z)}$, cio\`e per ogni $x\in X$ tale che $\norm x\leq 1$ si ha
\[\norm{b(x,y)}_Z\leq M\norm y\]
e quindi al variare di $y$ con $\norm y\leq 1$ troviamo $\norm b_{L(X\times Y,Z)}\leq M$.
\end{proof}

\begin{exercise}
Esiste una isometria lineare
\[\funcDef{L(X,L(Y,Z))}{L^2(X\times Y,Z)}{T}{(x,y)\mapsto T(x)(y)}\]
dove $L^2(X\times Y,Z)$ sono le bilineari.
\end{exercise}

\begin{proposition}[w$^\ast$-limitato vs limitato in $\normd_{X^\ast}$]\label{PrLimitatoInDeboleStarEquivaleLimitatoInNormaDuale}
Sia $Y=\K$ e $X$ Banach. Sia $\Gamma\subseteq X^\ast$, allora $\Gamma$ \`e w$^\ast$-limitato se e solo se \`e limitato in $\normd_{X^\ast}$.
\end{proposition}
\begin{proof}
Essere limitato nella topologia debole$^\ast$ significa ``essere assorbito da ogni intorno w$^\ast$ di $X^\ast$" cio\`e, usando intorni di prebase, essere assorbiti da insiemi della forma
\[\cpa{f\in X^\ast\mid \abs{f(x)}<1}\]
per $x\in X$. Notiamo che $\Gamma$ viene assorbito da $\cpa{f\in X^\ast\mid \abs{f(x)}<1}$ significa $\Gamma(x)$ limitato in $\K$. Per il corollario (\ref{CorPuntualmenteLimitatoImplicaLimitatoPerNormaOperatore}) si ha che $\Gamma$ \`e limitato in $L(X,\K)=X^\ast$.

L'altra implicazione \`e ovvia perch\'e la norma operatore gi\`a rende continui gli operatori e indebolire la topologia non pu\`o trasformare un insieme limitato in uno non limitato.
\end{proof}

\begin{remark}
Se $E\subseteq F$ \`e un sottospazio allora $\Gamma\subseteq E$ \`e limitato in $F$ se e solo se \`e limitato in $E$ per la topologia indotta.
\end{remark}

\begin{proposition}\label{PrLimitatoDeboleUgualeLimitatoForte}
Sia $\Gamma\subseteq X$, allora $\Gamma$ \`e w-limitato se e solo se \`e $\normd$-limitato.
\end{proposition}
\begin{proof}
Se $\Gamma$ \`e $\sigma(X,X^\ast)$-limitato allora tramite l'immersione isometrica $X\to X^{\ast\ast}$ troviamo un insieme $\sigma(X^{\ast\ast},X^\ast)$-limitato. A questo punto basta applicare la proposizione precedente (\ref{PrLimitatoInDeboleStarEquivaleLimitatoInNormaDuale}).
\end{proof}
