\chapter{Limitatezza e Banach-Steinhaus}

\section{Limitatezza}
\begin{definition}[Insieme limitato]
Un sottoinsieme $S$ di uno SVT $X$ con $\Uc$ intorni di $0$ \`e \textbf{limitato} se \`e assorbito da ogni elemento di $\Uc$, cio\`e\footnote{questa condizione \`e equivalente a chiedere $tU\supseteq S$ per ogni $t$ con $\abs t\geq n$ o a chiedere che l'assorbimento valga per elementi di una pre-base di intorni di $0$ al posto di tutti gli elementi di $\Uc$.} per ogni $U\in\Uc$ esiste $n\in\N$ tale che $nU\supseteq S$.
\end{definition}
\begin{remark}
Valgono le seguenti propriet\`a
\begin{enumerate}
    \item Se $S$ \`e limitato allora anche $\ol S$ lo \`e, basta considerare intorni chiusi.
    \item Se $S$ e $S'$ sono limitati, $S\cup S'$ lo \`e.
    \item Ogni compatto \`e limitato, basta scegliere un intorno limitato di $x$ per ogni $x\in K$ e poi estrarre un sottoricoprimento finito. Un tale intorno esiste scalando intorni di $0$ bilanciati.
    \item Ogni $T:X\to Y$ lineare e continua tra SVT \`e limitata, cio\`e per ogni $S\subseteq X$ limitato, $T(S)$ \`e limitato. In generale non vale il viceversa ma vale se $X$ e $Y$ sono normati.
\end{enumerate}
\end{remark}

\begin{proposition}[Limitatezza in SVTLC]\label{PrLimitatezzaInSVTLC}
Se $(X,\Pc)$ \`e SVTLC allora $S\subseteq X$ \`e limitato se e solo se per ogni seminorma $p\in \Pc$, $p$ \`e limitata su $S$.
\end{proposition}
\begin{proof}
$p$ limitata su $S$ significa che 
\[S\subseteq B_p(0,R_p)=\frac{R_p}\e B_p(0,\e)\]
e le palle $\cpa{B_p(0,\e)}_{p\in\Pc, \e>0}$ sono una prebase di intorni di $0\in X$.
\end{proof}
\begin{corollary}
Se $(X,\normd)$ \`e normato allora $S$ \`e limitato se e solo se $\exists R>0$ tale che $S\subseteq B(0,R)$.
\end{corollary}

\begin{exercise}
Se $X$ \`e I-numerabile e $T:X\to Y$ lineare tale che per ogni $x_k\to 0$ in $X$ esiste $x_{k_j}$ tale che $T(x_{k_j})$ limitata allora $T$ \`e continua.
\end{exercise}

\begin{proposition}[Caratterizzazione sequenziale della limitatezza]\label{PrCaratterizzazioneSequenzialeLimitatezza}
Se $X$ SVT e $S\subseteq X$, $S$ \`e limitato se e solo se per ogni $(s_k)$ successione in $S$ e per ogni $(\al_k)$ successione in $\K$ infinitesima, si ha $\al_k s_k\to 0$.
\end{proposition}
\begin{proof}
Sia $S$ limitato, $(s_k)$ successione in $S$ e $(\al_k)$ successione infinitesima in $\K$. Sia $U$ intorno bilanciato di $0$ e sia $n$ tale che $S\subseteq nU$. Notiamo che definitivamente $\abs{\al_k}<\frac1n$, quindi
\[\al_k s_k\in \al_k S\subseteq \al_k nU\pasgnl\subseteq{k grande} U.\]

Supponiamo ora $S$ non limitato, allora esiste $U\in\Uc_X$ che non assorbe $S$, cio\`e per ogni $n\in\N$ esiste $s_n\in S\bs nU$. Dunque $(s_n)$ \`e una successione in $S$ tale che $\frac1n s_n\notin U$ per costruzione, dunque $\frac1n s_n$ non tende a $0\in X$ nonostante $\frac1n$ sia infinitesima.
\end{proof}

\begin{proposition}\label{PrSuccessioniCauchyLimitate}
Le successioni di Cauchy sono limitate.
\end{proposition}
\begin{proof}
Sia $(x_k)$ una successione di Cauchy in $X$, cio\`e per ogni $U\in \Uc_X$ esiste $n\in\N$ tale che per ogni $p,q\geq n$ si ha $x_p-x_q\in U$.

Fissiamo $U\in \Uc_X$ e sia $V$ bilanciato tale che $V+V\subseteq U$. Per la definizione di successione di Cauchy esiste $n_0$ tale che $x_k-x_{n_0}\in V$ per ogni $k\geq n_0$, cio\`e $x_k\in x_{n_0}+V$.

Inoltre, esiste $m$ tale che $x_k\in mV$ per ogni $k\leq n_0$ dato che un insieme finito \`e limitato. Allora per ogni $k\in\N$ si ha $x_k\in mV+V$, infatti se $k\leq n_0$ allora abbiamo $mV$, se $k>n_0$ allora $x_{n_0}\in mV$ e $x_k\in x_{n_0}+V\subseteq mV+V$.

Poich\'e $V$ \`e bilanciato, $mV+V\subseteq mV+mV=m(V+V)\subseteq mU$.
\end{proof}

\section{Spazi di Baire e II-categoria}
\begin{theorem}[Baire]\label{ThBaire}
Se $\cpa{A_k}_{k\in\N}$ \`e una famiglia numerabile di aperti densi di uno spazio metrico completo allora $\bigcap A_k$ \`e denso.
\end{theorem}
\begin{proof}
Per induzione si definisce una successione di palle chiuse di $X$ dove $B_0$ \`e arbitraria e
\[B_k=\ol{B(x_k,r_k)}\text{ tali che }B_{k+1}\subseteq B_k\cap A_k\text{ e }r_k=o(1)\]
che possiamo fare perch\'e $A_k$ \`e un aperto denso.

Allora la successione dei centri \`e una successione di Cauchy, infatti se $p,q\geq n$ si ha $x_p,x_q\in B_n$ e quindi $d(x_p,x_q)\leq 2r_n$. Dunque $x_n\to x^\ast$ in $X$ per completezza. Inoltre, poich\'e $x_k\in B_n$ definitivamente, $x^\ast=\lim x_k\in B_n$ per ogni $n$ (dato che $B_n$ \`e chiuso). In particolare $x^\ast\in B_{n+1}\subseteq A_n$ per ogni $n$ e quindi $x^\ast\in \bigcap A_n$. Per costruzione $x^\ast\in B_0$, quindi per ogni palla $B_0$ abbiamo mostrato che $B_0\cap \bigcap A_n\neq \emptyset$, cio\`e $\bigcap A_n$ \`e denso.
\end{proof}

\begin{exercise}
La stessa conclusione vale se $X$ \`e localmente compatto al posto di metrico completo.
\end{exercise}

\begin{definition}[Spazio di Baire]
Uno spazio topologico \`e \textbf{di Baire} se ogni intersezione numerabile di aperti densi \`e densa.
\end{definition}

\begin{remark}
Ogni aperto non vuoto di $X$ di Baire \`e ancora di Baire. Basta verificare che ogni aperto denso di $A$ \`e della forma $A\cap U$ con $U$ aperto denso di $X$.
\end{remark}


\begin{definition}[Sottoinsieme di I- e II-categoria]
Un sottoinsieme $S$ di $X$ \`e di \textbf{I-categoria (di Baire) in $X$} se \`e unione numerabile di insiemi $(E_i)_{i\in\N}$ con $int(\ol{E_i})=\emptyset$.

Inoltre $S$ \`e di \textbf{II-categoria (di Baire) in $X$} se non \`e di I-categoria.
\end{definition}

\begin{remark}
Se $X$ \`e di Baire e $S\subseteq X$ \`e di I-categoria allora $X\bs S$ \`e di II-categoria in quanto $X$ stesso \`e di II-categoria (se $X=\bigcup E_i$ con $E_i$ chiusi a parte interna vuota allora $\emptyset=\bigcap E_i^c$ con $E_i^c$ aperti densi, ma questo \`e assurdo perch\'e $X$ di Baire).
\end{remark}

\section{Teorema di Banach-Steinhaus}

\begin{definition}[Famiglia equicontinua]
Una famiglia $\Gamma$ di operatori lineari continui fra SVT $X$ e $Y$ \`e \textbf{equicontinua} se per ogni $U\in\Uc_Y$ esiste $V\in \Uc_X$ tale che per ogni $T\in\Gamma$, $T(V)\subseteq U$.
\end{definition}

\begin{remark}
Possiamo riformulare la condizione nei seguenti modi: per ogni $U\in\Uc_Y$ esiste $V\in\Uc_X$ tale che
\[\forall T\in\Gamma,\ V\subseteq T\ii(U)\coimplies V\subseteq \bigcap_{T\in \Gamma}T\ii(U)\doteqdot \Gamma\ii(U).\]
Equivalentemente la condizione predica che per ogni $V\in\Uc_Y$ si abbia $\Gamma\ii(V)\in\Uc_X$.
\end{remark}

\begin{remark}
Se $T:X\to Y$ fra spazi normati, la norma degli operatori
\[\norm T=\norm T_{\infty,B(0,1)}=\text{migliore costante di Lipschitz per }T.\]
\end{remark}

\begin{example}
Se $X$ e $Y$ sono normati, $\Gamma$ \`e equicontinua se e solo se $\Gamma$ \`e limitato in $L(X,Y)$ rispetto alla norma degli operatori.
\end{example}

\begin{theorem}[Banach-Steinhaus / Uniforme limitatezza]\label{ThBanachSteinhausUniformeLimitatezza}
Siano $X,Y$ SVT, $S\subseteq X$ di seconda categoria e $\Gamma\subseteq L(X,Y)$ con $\Gamma$ puntualmente limitata su $S\subseteq X$, cio\`e per ogni $s\in S$, $\Gamma(s)=\bigcup_{T\in \Gamma} T(s)$ \`e limitato in $Y$.

Allora $\Gamma$ \`e equicontinua.
\end{theorem}
\begin{proof}
Sia $U\in\Uc_Y$ e consideriamo $V\in\Uc_Y$ chiuso tale che $V-V\subseteq U$. Per ipotesi, per ogni $x\in S$ si ha che $\Gamma(x)$ \`e limitato in $Y$, quindi viene assorbito da $V$, cio\`e esiste $n_x\in\N$ tale che per ogni $T\in\Gamma$ si ha $T(x)\in n_x V$, cio\`e tale che
\[x\in\bigcap_{T\in \Gamma}n_xT\ii(V)=n_x\bigcap_{T\in \Gamma}T\ii(V)=n_x\Gamma\ii(V).\]
Dunque $S\subseteq \bigcup_{n\in\N}n\Gamma\ii(V)$. Notiamo che poich\'e $V$ \`e chiuso, $T\ii(V)$ \`e chiuso e quindi anche $\Gamma\ii(V)$ lo \`e perch\'e intersezione di chiusi. Poich\'e $S$ \`e di seconda categoria anche l'unione delle versioni riscalate di $\Gamma\ii(V)$ lo \`e, dunque questo insieme non \`e unione numerabile di chiusi con parte interna vuota, quindi almeno uno tra gli $n\Gamma\ii(V)$ ha parte interna non vuota, quindi anche $\Gamma\ii(V)$ ha parte interna non vuota scalando per $\frac1n$.

Quindi $\Gamma\ii(V)$ \`e intorno di qualche suo punto, dunque\footnote{se $a_0\in int(A)$ allora $A-a_0\subseteq A-A$ \`e un intorno di $0$.} $\Gamma\ii(V)-\Gamma\ii(V)$ \`e un intorno di $0$. 

Ricordando che $V-V\subseteq U$ si ha
\[T\ii(U\supseteq T\ii(V-V)=T\ii(V)-T\ii(V))\supseteq \Gamma\ii(V)-\Gamma\ii(V)\]
quindi passando all'intersezione su $T\in\Gamma$ si ha
\[\Gamma\ii(U)\supseteq \Gamma\ii(V)-\Gamma\ii(V)\in\Uc_X,\]
cio\`e abbiamo mostrato che per ogni $U\in\Uc_Y$ si ha $\Gamma\ii(U)\in\Uc_X$, che \`e equivalente all'equicontinuit\`a di $\Gamma$.
\end{proof}

\begin{corollary}[Sottoinsiemi limitati di operatori]\label{CorPuntualmenteLimitatoImplicaLimitatoPerNormaOperatore}
    Se $X$ e $Y$ sono Banach e $\Gamma\subseteq L(X,Y)$ \`e puntualmente limitata in $X$ (o volendo anche un sottoinseme di $X$ di II-categoria) allora $\Gamma$ \`e un insieme limitato in $L(X,Y)$.
\end{corollary}
\begin{proof}
Diretta applicazione di Banach-Stenhaus (\ref{ThBanachSteinhausUniformeLimitatezza}) notando che spazi di Banach sono in particolare SVT e che equicontinuit\`a per la norma su $L(X,Y)$ significa limitatezza. **************
\end{proof}

\begin{exercise}
Siano $X,Y$ SVT. Trovare la topologia meno fine $\tau$ di SVT su $L(X,Y)$ per la quale
\[\Gamma\text{ puntualmente limitato in $L(X,Y)$}\coimplies \Gamma\text{ limitato nella topologia $\tau$}.\]
\end{exercise}


\begin{corollary}\label{CorConvergenzaLineariSePuntualmenteConvergente}
Siano $X$ e $Y$ Banach e sia $(T_n)\subseteq L(X,Y)$ puntualmente convergente. Allora il limite $T$ \`e ancora lineare, continuo e con norma
\[\norm T\leq \liminf_{n\to\infty}\norm{T_n}.\]
\end{corollary}
\begin{proof}
Per il corollario precedente (\ref{CorPuntualmenteLimitatoImplicaLimitatoPerNormaOperatore}) si ha che $(T_n)$ sono limitati in $\normd$ e il limite puntuale \`e lineare in quanto
\[T_n(\al x+\beta y)=\al T_n(x)+\beta T_n(y)\to \al T(x)+\beta T(y).\]
Questo mostra che $T$ \`e limitato e lineare, quindi $T\in L(X,Y)$.

Inoltre per ogni $x\in X$ si ha
\[\norm{T(x)}=\lim_{n}\norm{T_n(x)}\leq \pa{\sup_n\norm{T_n}}\norm x,\]
quindi $\norm T\leq \sup_n\norm{T_n}$. Ragionando analogamente per una sottosuccessione di $(T_n)$ che in norma converge a $\liminf_n\norm{T_n}$ ricaviamo
\[\norm T\leq \liminf_n\norm{T_n}.\]
\end{proof}

\begin{remark}
In generale NON vale $T_n\to T$ in $\normd$.
\end{remark}

\begin{proposition}[Bilineare separatamente continua \`e continua]\label{PrBilineareSeparatamenteContinuaEContinua}
Sia $b:X\times Y\to Z$ bilineare e separatamente continua, cio\`e per ogni $x\in X, y\in Y$ si ha che $b(x,\cdot):Y\to Z$ e $b(\cdot, y):X\to Z$ sono lineari e continue. Allora $b$ \`e continua, cio\`e
\[\sup_{\norm x\leq 1, \norm y\leq 1}\norm{b(x,y)}<\infty.\]
\end{proposition}
\begin{proof}
Consideriamo la famiglia
\[\Gamma=\cpa{b(x,\cdot):Y\to Z}_{x\in X,\ \norm x\leq 1}\subseteq L(Y,Z).\]
Per ipotesi $\Gamma$ \`e puntualmente limitata in $Y$, infatti per ogni $y\in Y$
\[\sup_{b(x,\cdot)\in\Gamma} \norm{b(x,\cdot)}_{L(Y,Z)}=\sup_{\norm x\leq 1}\norm{b(x,y)}_Z= \norm{b(\cdot,y)}_{L(X,Z)}\norm y<\infty\]
Allora $\Gamma$ \`e limitata in $\normd_{L(Y,Z)}$, cio\`e per ogni $x\in X$ tale che $\norm x\leq 1$ si ha
\[\norm{b(x,y)}_Z\leq M\norm y\]
e quindi al variare di $y$ con $\norm y\leq 1$ troviamo $\norm b_{L(X\times Y,Z)}\leq M$.
\end{proof}

\begin{exercise}
Esiste una isometria lineare
\[\funcDef{L(X,L(Y,Z))}{L^2(X\times Y,Z)}{T}{(x,y)\mapsto T(x)(y)}\]
dove $L^2(X\times Y,Z)$ sono le bilineari.
\end{exercise}

\begin{proposition}[w$^\ast$-limitato vs limitato in $\normd_{X^\ast}$]\label{PrLimitatoInDeboleStarEquivaleLimitatoInNormaDuale}
Sia $Y=\K$ e $X$ Banach. Sia $\Gamma\subseteq X^\ast$, allora $\Gamma$ \`e w$^\ast$-limitato se e solo se \`e limitato in $\normd_{X^\ast}$.
\end{proposition}
\begin{proof}
Essere limitato nella topologia debole$^\ast$ significa ``essere assorbito da ogni intorno w$^\ast$ di $X^\ast$" cio\`e, usando intorni di prebase, essere assorbiti da insiemi della forma
\[\cpa{f\in X^\ast\mid \abs{f(x)}<1}\]
per $x\in X$. Notiamo che $\Gamma$ viene assorbito da $\cpa{f\in X^\ast\mid \abs{f(x)}<1}$ significa $\Gamma(x)$ limitato in $\K$. Per il corollario (\ref{CorPuntualmenteLimitatoImplicaLimitatoPerNormaOperatore}) si ha che $\Gamma$ \`e limitato in $L(X,\K)=X^\ast$.

L'altra implicazione \`e ovvia perch\'e la norma operatore gi\`a rende continui gli operatori e indebolire la topologia non pu\`o trasformare un insieme limitato in uno non limitato.
\end{proof}

\begin{remark}
Se $E\subseteq F$ \`e un sottospazio allora $\Gamma\subseteq E$ \`e limitato in $F$ se e solo se \`e limitato in $E$ per la topologia indotta.
\end{remark}

\begin{proposition}
Sia $\Gamma\subseteq X$, allora $\Gamma$ \`e w-limitato se e solo se \`e $\normd$-limitato.
\end{proposition}
\begin{proof}
Se $\Gamma$ \`e $\sigma(X,X^\ast)$-limitato allora tramite l'immersione isometrica $X\to X^{\ast\ast}$ troviamo un insieme $\sigma(X^{\ast\ast},X^\ast)$-limitato. A questo punto basta applicare la proposizione precedente (\ref{PrLimitatoInDeboleStarEquivaleLimitatoInNormaDuale}).
\end{proof}
