\documentclass[a4paper]{report}
\usepackage[utf8]{inputenc}
\usepackage{amsmath,amssymb,amsfonts,amsthm,stmaryrd}
\usepackage{mathrsfs} % per mathscr
\usepackage{dsfont} % per mathbb1
\usepackage{graphicx}% ruota freccia per le azioni
\usepackage{oldgerm} % Fractur Particolare
\usepackage{marvosym}% per il \Lightning
\usepackage{array}
\usepackage{faktor} %per gli insiemi quoziente
\usepackage{hyperref}
\usepackage{xparse} % Per nuovi comandi con tanti input opzionali
\usepackage{tikz-cd}
\usepackage{multicol}
\usepackage{multirow}
\usepackage{cancel}



\usepackage[italian]{babel}


% Ambienti per teoremi =================================
% <name> 
% <space above> 
% <space below> 
% <body font> 
% <indent amount> 
% <Theorem head font> 
% <punctuation after theorem head> 
% <space after theorem head> (default .5em) 
% <Theorem head spec>

% I nuovi ambienti sono costruiti in modo da andare alla riga successiva

\newtheoremstyle{customth}
{\topsep}{\topsep}{\itshape}{}{\bfseries}{.}{\newline}{}
\newtheoremstyle{customdef}
{\topsep}{\topsep}{\normalfont}{}{\bfseries}{.}{\newline}{}
\newtheoremstyle{customrem}
{\topsep}{\topsep}{\normalfont}{}{\itshape}{.}{\newline}{}


\theoremstyle{customth}
\newtheorem{theorem}{Teorema}[chapter]
\newtheorem{lemma}[theorem]{Lemma}
\newtheorem{corollary}[theorem]{Corollario}
\newtheorem{proposition}[theorem]{Proposizione}
\newtheorem{fact}[theorem]{Fatto}
\newtheorem{application}[theorem]{Applicazione}
\theoremstyle{customrem}
\newtheorem{remark}[theorem]{Osservazione}
\theoremstyle{customdef}
\newtheorem{definition}[theorem]{Definizione}
\newtheorem{notation}[theorem]{Notazione}
\newtheorem{example}[theorem]{Esempio}
\newtheorem{exercise}[theorem]{Esercizio}

\makeatletter
\renewenvironment{proof}[1][\proofname]
{
    \par
    \pushQED{\qed}
    \normalfont \topsep6\p@\@plus6\p@\relax
    \trivlist
    \item[\hskip\labelsep\itshape#1\@addpunct{.}]\mbox{}\\*
}
{
    \popQED\endtrivlist\@endpefalse
}
\makeatother



%========= Preambolo per quiver ================
% quiver e' uno strumento che uso spesso per
% disegnare diagrammi. L'interfaccia sul loro sito
% permette di creare in modo visivo il diagramma e
% poi esportarlo come codice LaTeX da inserire nel
% documento. Il sito e' https://q.uiver.app/ 

%-----------------------------------------------
% *** quiver ***
% A package for drawing commutative diagrams exported from https://q.uiver.app.
%
% This package is currently a wrapper around the `tikz-cd` package, importing necessary TikZ
% libraries, and defining a new TikZ style for curves of a fixed height.
%
% Version: 1.2.1
% Authors:
% - varkor (https://github.com/varkor)
% - Andr\e'C (https://tex.stackexchange.com/users/138900/andr%C3%A9c)

\NeedsTeXFormat{LaTeX2e}
%\ProvidesPackage{quiver}[2021/01/11 quiver]

% `tikz-cd` is necessary to draw commutative diagrams.
\RequirePackage{tikz-cd}
% `amssymb` is necessary for `\lrcorner` and `\ulcorner`.
\RequirePackage{amssymb}
% `calc` is necessary to draw curved arrows.
\usetikzlibrary{calc}
% `pathmorphing` is necessary to draw squiggly arrows.
\usetikzlibrary{decorations.pathmorphing}

% A TikZ style for curved arrows of a fixed height, due to Andr\e'C.
\tikzset{curve/.style={settings={#1},to path={(\tikztostart)
    .. controls ($(\tikztostart)!\pv{pos}!(\tikztotarget)!\pv{height}!270:(\tikztotarget)$)
    and ($(\tikztostart)!1-\pv{pos}!(\tikztotarget)!\pv{height}!270:(\tikztotarget)$)
    .. (\tikztotarget)\tikztonodes}},
    settings/.code={\tikzset{quiver/.cd,#1}
        \def\pv##1{\pgfkeysvalueof{/tikz/quiver/##1}}},
    quiver/.cd,pos/.initial=0.35,height/.initial=0}

% TikZ arrowhead/tail styles.
\tikzset{tail reversed/.code={\pgfsetarrowsstart{tikzcd to}}}
\tikzset{2tail/.code={\pgfsetarrowsstart{Implies[reversed]}}}
\tikzset{2tail reversed/.code={\pgfsetarrowsstart{Implies}}}
% TikZ arrow styles.
\tikzset{no body/.style={/tikz/dash pattern=on 0 off 1mm}}
%=================================================

%PER CAMBIARE I MARGINI
\usepackage[margin=4cm]{geometry}

%\usepackage{emptypage} % Pagine vuote non numerate
%\usepackage{fancyhdr} % Sistema headers
%\usepackage[Lenny]{fncychap} % Capitoli fighi
%\ChTitleVar{\Huge\bfseries}
\usepackage[hang,flushmargin]{footmisc} % Footnote non indentata


%========== Stile header e footer ==============

%\pagestyle{fancy}
% Left, Right, Even(pages), Odd(pages), Center
%\fancyhead[L]{\leftmark}
%\fancyfoot[C]{}

%\renewcommand{\chaptermark}[1]{\markboth{\textsc{#1}}{}}
%\renewcommand{\sectionmark}[1]{\markright{\textsc{#1}}}

%\renewcommand{\headrulewidth}{0.05pt}
%\renewcommand{\footnoterule}{\kern 0pt	\hrule width \textwidth height 0.5pt \kern 5pt}

%Numeri pagina in prima pagina capitoli
%\makeatletter
%\let\ps@plain\ps@empty
%\makeatother

%==== Colore dei footnote, link e citazioni ====
\definecolor{DarkBlue}{HTML}{00518B}
\definecolor{DarkRed}{HTML}{B6321C}
\hypersetup{
    colorlinks=true,
    linkcolor=DarkRed,
    filecolor=blue,
    citecolor = DarkRed,
    urlcolor=cyan,
}
\renewcommand\thefootnote{\textcolor{blue}{\arabic{footnote}}}
%============ Simboli standard =================
%----------------- Lettere ---------------------
\newcommand{\A}{\mathbb{A}}
\newcommand{\B}{\mathbb{B}}
\newcommand{\C}{\mathbb{C}}
\newcommand{\D}{\mathbb{D}}
\newcommand{\E}{\mathbb{E}}
\newcommand{\F}{\mathbb{F}}
\newcommand{\G}{\mathbb{G}}
\newcommand{\Hb}{\mathbb{H}}
\newcommand{\I}{\mathbb{I}}
\newcommand{\J}{\mathbb{J}}
\newcommand{\K}{\mathbb{K}}
\newcommand{\Lb}{\mathbb{L}}
\newcommand{\M}{\mathbb{M}}
\newcommand{\N}{\mathbb{N}}
\newcommand{\Ob}{\mathbb{O}}
\newcommand{\Pj}{\mathbb{P}}
\newcommand{\Q}{\mathbb{Q}}
\newcommand{\R}{\mathbb{R}}
\newcommand{\Sb}{\mathbb{S}}
\newcommand{\T}{\mathbb{T}}
\newcommand{\U}{\mathbb{U}}
\newcommand{\V}{\mathbb{V}}
\newcommand{\W}{\mathbb{W}}
\newcommand{\X}{\mathbb{X}}
\newcommand{\Y}{\mathbb{Y}}
\newcommand{\Z}{\mathbb{Z}}

\newcommand{\Ac}{\mathcal{A}}
\newcommand{\Bc}{\mathcal{B}}
\newcommand{\Cc}{\mathcal{C}}
\newcommand{\Dc}{\mathcal{D}}
\newcommand{\Ec}{\mathcal{E}}
\newcommand{\Fc}{\mathcal{F}}
\newcommand{\Gc}{\mathcal{G}}
\newcommand{\Hc}{\mathcal{H}}
\newcommand{\Ic}{\mathcal{I}}
\newcommand{\Jc}{\mathcal{J}}
\newcommand{\Kc}{\mathcal{K}}
\newcommand{\Lc}{\mathcal{L}}
\newcommand{\Mc}{\mathcal{M}}
\newcommand{\Nc}{\mathcal{N}}
\newcommand{\Oc}{\mathcal{O}}
\newcommand{\Pc}{\mathcal{P}}
\newcommand{\Qc}{\mathcal{Q}}
\newcommand{\Rc}{\mathcal{R}}
\newcommand{\Sc}{\mathcal{S}}
\newcommand{\Tc}{\mathcal{T}}
\newcommand{\Uc}{\mathcal{U}}
\newcommand{\Vc}{\mathcal{V}}
\newcommand{\Wc}{\mathcal{W}}
\newcommand{\Xc}{\mathcal{X}}
\newcommand{\Yc}{\mathcal{Y}}
\newcommand{\Zc}{\mathcal{Z}}

\newcommand{\Af}{\mathfrak{A}}
\newcommand{\Bf}{\mathfrak{B}}
\newcommand{\Cf}{\mathfrak{C}}
\newcommand{\Df}{\mathfrak{D}}
\newcommand{\Ef}{\mathfrak{E}}
\newcommand{\Ff}{\mathfrak{F}}
\newcommand{\Gf}{\mathfrak{G}}
\newcommand{\Hf}{\mathfrak{H}}
\newcommand{\If}{\mathfrak{I}}
\newcommand{\Jf}{\mathfrak{J}}
\newcommand{\Kf}{\mathfrak{K}}
\newcommand{\Lf}{\mathfrak{L}}
\newcommand{\Mf}{\mathfrak{M}}
\newcommand{\Nf}{\mathfrak{N}}
\newcommand{\Of}{\mathfrak{O}}
\newcommand{\Pf}{\mathfrak{P}}
\newcommand{\Qf}{\mathfrak{Q}}
\newcommand{\Rf}{\mathfrak{R}}
\newcommand{\Sf}{\mathfrak{S}}
\newcommand{\Tf}{\mathfrak{T}}
\newcommand{\Uf}{\mathfrak{U}}
\newcommand{\Vf}{\mathfrak{V}}
\newcommand{\Wf}{\mathfrak{W}}
\newcommand{\Xf}{\mathfrak{X}}
\newcommand{\Yf}{\mathfrak{Y}}
\newcommand{\Zf}{\mathfrak{Z}}

\newcommand{\af}{\mathfrak{a}}

\newcommand{\cf}{\mathfrak{c}}
\newcommand{\df}{\mathfrak{d}}
\newcommand{\ef}{\mathfrak{e}}
\newcommand{\ff}{\mathfrak{f}}
\newcommand{\gf}{\mathfrak{g}}
\newcommand{\hf}{\mathfrak{h}}

\newcommand{\jf}{\mathfrak{j}}
\newcommand{\kf}{\mathfrak{k}}
\newcommand{\lf}{\mathfrak{l}}
\newcommand{\mf}{\mathfrak{m}}
\newcommand{\nf}{\mathfrak{n}}
\newcommand{\of}{\mathfrak{o}}
\newcommand{\pf}{\mathfrak{p}}
\newcommand{\qf}{\mathfrak{q}}
\newcommand{\rf}{\mathfrak{r}}

\newcommand{\tf}{\mathfrak{t}}
\newcommand{\uf}{\mathfrak{u}}
\newcommand{\vf}{\mathfrak{v}}
\newcommand{\wf}{\mathfrak{w}}
\newcommand{\xf}{\mathfrak{x}}
\newcommand{\yf}{\mathfrak{y}}
\newcommand{\zf}{\mathfrak{z}}

\newcommand{\As}{\mathscr{A}}
\newcommand{\Bs}{\mathscr{B}}
\newcommand{\Cs}{\mathscr{C}}
\newcommand{\Ds}{\mathscr{D}}
\newcommand{\Es}{\mathscr{E}}
\newcommand{\Fs}{\mathscr{F}}
\newcommand{\Gs}{\mathscr{G}}
\newcommand{\Hs}{\mathscr{H}}
\newcommand{\Is}{\mathscr{I}}
\newcommand{\Js}{\mathscr{J}}
\newcommand{\Ks}{\mathscr{K}}
\newcommand{\Ls}{\mathscr{L}}
\newcommand{\Ms}{\mathscr{M}}
\newcommand{\Ns}{\mathscr{N}}
\newcommand{\Os}{\mathscr{O}}
\newcommand{\Ps}{\mathscr{P}}
\newcommand{\Qs}{\mathscr{Q}}
\newcommand{\Rs}{\mathscr{R}}
\newcommand{\Ss}{\mathscr{S}}
\newcommand{\Ts}{\mathscr{T}}
\newcommand{\Us}{\mathscr{U}}
\newcommand{\Vs}{\mathscr{V}}
\newcommand{\Ws}{\mathscr{W}}
\newcommand{\Xs}{\mathscr{X}}
\newcommand{\Ys}{\mathscr{Y}}
\newcommand{\Zs}{\mathscr{Z}}

\newcommand{\ula}{{\underline{a}}}
\newcommand{\ulb}{{\underline{b}}}
\newcommand{\ulc}{{\underline{c}}}
\newcommand{\uld}{{\underline{d}}}
\newcommand{\ule}{{\underline{e}}}
\newcommand{\ulf}{{\underline{f}}}
\newcommand{\ulg}{{\underline{g}}}
\newcommand{\ulh}{{\underline{h}}}
\newcommand{\uli}{{\underline{i}}}
\newcommand{\ulj}{{\underline{j}}}
\newcommand{\ulk}{{\underline{k}}}
\newcommand{\ull}{{\underline{l}}}
\newcommand{\ulm}{{\underline{m}}}
\newcommand{\uln}{{\underline{n}}}
\newcommand{\ulo}{{\underline{o}}}
\newcommand{\ulp}{{\underline{p}}}
\newcommand{\ulq}{{\underline{q}}}
\newcommand{\ulr}{{\underline{r}}}
\newcommand{\uls}{{\underline{s}}}
\newcommand{\ult}{{\underline{t}}}
\newcommand{\ulu}{{\underline{u}}}
\newcommand{\ulv}{{\underline{v}}}
\newcommand{\ulw}{{\underline{w}}}
\newcommand{\ulx}{{\underline{x}}}
\newcommand{\uly}{{\underline{y}}}
\newcommand{\ulz}{{\underline{z}}}

%---------- Funzioni standard ------------------
\DeclareMathOperator{\Adj}{Adj}
\DeclareMathOperator{\adj}{adj}
\DeclareMathOperator{\Ann}{Ann}
\DeclareMathOperator{\Arg}{Arg}
\DeclareMathOperator{\Ass}{Ass}
\DeclareMathOperator{\cha}{char}
\DeclareMathOperator{\cod}{cod}
\DeclareMathOperator{\codim}{codim}
\DeclareMathOperator{\coker}{coker}
\DeclareMathOperator{\comb}{Comb}
\DeclareMathOperator{\dom}{dom}
\DeclareMathOperator{\End}{End}
\DeclareMathOperator{\Fix}{Fix}
\DeclareMathOperator{\Hom}{Hom}
\DeclareMathOperator{\imm}{Imm}
\DeclareMathOperator{\Ind}{Ind}
\DeclareMathOperator*{\infess}{infess}
\DeclareMathOperator{\mcd}{mcd}
\DeclareMathOperator{\mcm}{mcm}
\DeclareMathOperator{\Min}{Min}
\DeclareMathOperator{\Mor}{Mor}
\DeclareMathOperator{\obj}{obj}
\DeclareMathOperator{\orb}{orb}
\DeclareMathOperator{\ord}{ord}
\DeclareMathOperator{\Proj}{Proj}
\DeclareMathOperator{\Res}{Res}
\DeclareMathOperator{\rnk}{rnk}
\DeclareMathOperator{\sgn}{sgn}
\DeclareMathOperator{\Span}{Span}
\DeclareMathOperator{\Spec}{Spec}
\DeclareMathOperator{\stab}{stab}
\DeclareMathOperator*{\supess}{supess}
\DeclareMathOperator{\Supp}{Supp}
\DeclareMathOperator{\supp}{supp}
\DeclareMathOperator{\Sym}{Sym}
\DeclareMathOperator{\tr}{tr}

\newcommand{\Real}{\,\Re\mathfrak{e}}
\newcommand{\Imag}{\,\Im\mathfrak{m}}



%-------------- Frecce -------------------------
\newcommand{\coimplies}{\Longleftrightarrow}
\newcommand{\inj}{\hookrightarrow}
\newcommand{\onto}{\twoheadrightarrow}
\newcommand{\ot}{\leftarrow}
\newcommand{\acts}{\curvearrowright}

%----------- Lettere greche -------------------
\newcommand{\al}{\alpha}
\newcommand{\de}{\delta}
\newcommand{\e}{\varepsilon}
%\newcommand{\th}{\theta}
\newcommand{\la}{\lambda}
\newcommand{\vp}{\varphi}

%-------------- Derivate ----------------------
\newcommand{\raiseargument}[1]{\raisebox{.8ex}{$#1$}}
\newcommand{\centersmallmath}[1]{\vcenter{\hbox{\scalebox{.8}{$#1$}}}}
\newcommand{\raiseargumentsmall}[1]{\raisebox{.4ex}{\scalebox{.8}{$#1$}}}
\newcommand*{\emptyfrac}[2]{\genfrac{}{}{0pt}{}{#1}{#2}}

\NewDocumentCommand{\ddxi}{O{x}mm}{
    {\frac{d^{}{#3}}{d{#1}_{#2}}}
}

\NewDocumentCommand{\dd}{O{}mm}{
    {\frac{d^{#1}{#3}}{d{#2}^{#1}}}
}

\NewDocumentCommand{\ppxi}{O{x}mm}{
    {{\frac{\partial^{}{#3}}{\partial{#1}_{#2}}}}
}

\NewDocumentCommand{\pp}{O{}mm}{
    {{\frac{\partial^{#1}{#3}}{\partial{#2}}}}
}





%========== Comandi dattilografici ============
%--------- Passaggi in derivazioni ------------
\newcommand{\pasg}[3]{\overset{\hyperref[#3]{\text{#2}}}{#1}}
\newcommand{\pasgnl}[2]{\overset{\text{#2}}{#1}}
\newcommand{\pasgnlmath}[2]{\overset{#2}{#1}}
\newcommand{\pasgmath}[3]{\overset{\hyperref[#3]{{#2}}}{#1}}

%----------- Modifica testo -------------------
\newcommand{\ul}[1]{\underline{#1}}
\newcommand{\ol}[1]{\overline{#1}}
\newcommand{\wt}[1]{\widetilde{#1}}
\newcommand{\wh}[1]{\widehat{#1}}
\newcommand{\td}[1]{\Tilde{#1}}
\newcommand{\rg}[1]{{\mathring {#1}}}
\newcommand{\under}[2]{\underset{#1}{\underbrace{#2}}}

%-------------- Parentesi ---------------------
\newcommand{\pa}[1]{\left({#1}\right)}
\newcommand{\spa}[1]{\left[{#1}\right]}
\newcommand{\cpa}[1]{\left\{{#1}\right\}}
\newcommand{\abs}[1]{\left|{#1}\right|}
\newcommand{\norm}[1]{\left\Vert{#1}\right\Vert}
\newcommand{\ps}[1]{\left\langle {#1}\right\rangle}
\newcommand{\floor}[1]{\left\lfloor {#1}\right\rfloor}
\newcommand{\ceil}[1]{\left\lceil {#1}\right\rceil}
\newcommand{\rbar}[1]{\left.{#1}\right|}

%--------------- Matrici ----------------------
\newcommand{\mat}[1]{\begin{pmatrix}#1\end{pmatrix}}
\newcommand{\emat}[1]{\begin{matrix}#1\end{matrix}}
\newcommand{\dmat}[1]{\begin{vmatrix}#1\end{vmatrix}}
\newcommand{\smat}[1]{\begin{smallmatrix}#1\end{smallmatrix}}
\newcommand{\BIG}[1]{\mathlarger{\mathlarger{\mathlarger{\mathlarger{#1}}}}}

%--------------- Funzioni ---------------------
\newcommand{\funcDef}[4]{
\begin{array}{ccc}
{#1} & \longrightarrow & {#2}\\
{#3} & \longmapsto & {#4}
\end{array}}
\newcommand{\functorDef}[6]{
\begin{array}{ccc}
{#1} & \longrightarrow & {#2}\\
{#3} & \longmapsto & {#4}\\
{#5} & \longmapsto & {#6}
\end{array}}
\newcommand{\correspDef}[6]{
\begin{array}{ccc}
{#1} & \longleftrightarrow & {#2}\\
{#3} & \longmapsto & {#4}\\
{#5} & \longmapsfrom & {#6}
\end{array}}

%---------------- Altro -----------------------
\newcommand{\bs}{\setminus}
\newcommand{\res}[1]{\raisebox{-.5ex}{$|$}_{#1}}
\newcommand{\quot}[2]{\faktor{#1}{#2}}
\newcommand{\sep}{\,\middle|\,}

\newcommand{\ii}{^{-1}}
\newcommand{\op}{^{op}}
\newcommand{\nz}{\bs\{0\}}

\newcommand{\powerset}{\mathscr{P}}
\newcommand{\del}{\partial}
\newcommand{\0}{{\underline{0}}}
\newcommand{\1}{{\vcenter{\hbox{\scalebox{1.2}{$\mathds{1}$}}}}}


\newcommand{\GL}{\mathrm{GL}}
\newcommand{\PGL}{\mathrm{PGL}}
%\NewDocumentCommand{\PGL}{o m}{
%    \IfNoValueTF{#1}
%        {{\mathbb{P}GL({#2})}}
%    {{\mathbb{P}GL_{#1}({#2})}}
%}
%\NewDocumentCommand{\GL}{o m}{
%    \IfNoValueTF{#1}
%        {{GL({#2})}}
%    {{GL_{#1}({#2})}}
%}
\newcommand{\znz}[1]{{\Z/{#1}\Z}}









% ============================================


%---------- Comandi specifici ----------------
\newcommand{\normd}{{\norm{\cdot}}}
\newcommand{\Met}{\mathrm{Met}}
\newcommand{\CMet}{\mathrm{CMet}}
\newcommand{\Ban}{\mathrm{Ban}}
\DeclareMathOperator{\assco}{assco}
\DeclareMathOperator{\co}{co}

%--------- Comandi dattilografici ------------
\newcommand{\filosofia}[1]{\begin{center}\textbf{#1}\end{center}}


% ============================================
\title{Istituzioni di Analisi Matematica\\
\large Corso del prof. Pietro Majer}

\author{Francesco Sorce}
\date{Università di Pisa\\
Dipartimento di Matematica\\
A.A. 2024/25}

\begin{document}
\maketitle

%\newpage
\tableofcontents
\newpage

\chapter{Norme e Seminorme}
Il corso si concentra sulla relazione che si crea tra la struttura lineare e la struttura topologia degli spazi normati. 

Per $\K$ intendiamo un campo tra $\R$ o $\C$.

\section{Norme e seminorme}

\begin{definition}[Seminorma]
Se $X$ \`e uno spazio vettoriale su $\K$, una \textbf{seminorma} \`e una funzione $\normd:X\to [0,+\infty)$ tale che
\begin{enumerate}
    \item $\norm{x+y}\leq \norm x+\norm y$ (\textit{Disuguaglianza triangolare})
    \item $\norm{\la x}=\la\norm x$ se $\la\in\R,\ \la>0$ (\textit{Positivamente omogenea})
    \item[2'.] $\norm{\la x}=\norm x$ se $|\la|=1$ (\textit{Isotropa})
\end{enumerate}
Se inoltre vale $\norm x=0\coimplies x=0$ allora $\norm\cdot$ \`e detta \textbf{norma}.

La coppia $(X,\norm \cdot)$ si dice \textbf{spazio (semi)normato}.
\end{definition}
\begin{remark}
Su uno spazio (semi)normato possiamo definire una (semi)distanza indotta ponendo
\[d(x,y)=\norm{x-y}.\]
\end{remark}

Diamo alcuni esempi di spazi normati e seminormati:
\begin{example}
\begin{enumerate}
    \item $X=\R^n$, $\displaystyle \norm x_\infty=\max_{i\in\cpa{1,\cdots,n}} \abs{x_i}$
    \item Per $1\leq p<\infty$, $\ell_p=\cpa{x\in\K^\N\mid \sum_{i\geq 0}\abs{x_i}^p<\infty}$ con $\norm x_p=\sum_{i\geq 0}\abs{x_i}^p$
    \item $\ell_\infty=\cpa{x\in\K^\N\mid \sup\abs{x_i}<\infty}$ con $\norm x_\infty=\sup\abs{x_i}$
    \item $\Lc^p(X,\mu)=\cpa{f:X\to\K,\text{ misurabile, }\norm f_p<\infty}$ con
    \[\norm f_p=\begin{cases}
        \pa{\int_X\abs{f(x)}^pd\mu}^{1/p} &\text{se }1\leq p<\infty\\\\
        \displaystyle \supess_{x\in X}\abs{f(x)}=\inf_{\smat{N\subseteq X,\\ \mu(N)=0}}\sup_{x\in X\bs N}\abs{f(x)} &\text{se }p=\infty
    \end{cases}\]
    \`e uno spazio seminormato ma non normato. 
    \item Spazi di Hilbert.
\end{enumerate}
\end{example}

\begin{definition}[Funzioni continue, limitate e lineari]
Siano $E,F$ spazi normati e $S$ un insieme, definiamo i seguenti spazi normati:
\begin{align*}
    \Bs(S,E)=&\cpa{f:S\to E,\text{ limitate}},\quad &&\norm{f}_{\infty,S}=\sup_{s\in S}\norm{f(s)}_E\\
    \Bs C(S,E)=&\cpa{f:S\to E,\text{ continue e limitate}},\quad &&\norm{f}_{\infty,S}=\sup_{s\in S}\norm{f(s)}_E\\
    L(E,F)=&\cpa{T:E\to F\text{ lineare, }\norm T<\infty},\quad &&\norm T=\sup_{x\in B_E(0,1)}\norm{T(x)}_F
\end{align*}
\end{definition}

\begin{definition}[Spazio duale]
Sia $V$ uno spazio vettoriale. Denotiamo con $V'$ \textbf{il duale algebrico}, cio\`e l'insieme delle mappe lineari $V\to \K$. 

Definiamo lo \textbf{spazio duale} a $V$ come $V^\ast=L(V,\K)$, cio\`e come il sottoinsieme di $V'$ dato dalle mappe continue. La norma su $V^\ast$ \`e quindi data da
\[\norm f_{V^\ast}=\sup_{\norm x\leq1}\abs{f(x)}\pasgnl={Lineare}\sup_{\norm x=1}\abs{f(x)}.\]
\end{definition}
\begin{proposition}[Per funzionale limitato equivale continuo]\label{PrPerFunzionaleLineareLimitatoEquivaleContinuo}
Per un funzionale lineare in $V^\ast$, essere limitato \`e equivalente ad essere continuo.
\end{proposition}
\begin{proof}
Se $\norm f=M\in\R_+$ allora
\[\norm{f(x)-f(y)}=\norm{f(x-y)}=\norm{f\pa{\frac{x-y}{\norm{x-y}}}}\norm{x-y}\leq\norm f\norm{x-y}=M\norm{x-y},\]
cio\`e $f$ \`e $M$-lipschitz, e quindi continua.

Sia ora $f$ lineare e continua. Per definizione di continuit\`a in $0$ esiste $\delta>0$ tale che $\norm {f(x)}=\norm{f(x)-f(0)}\leq 1$ per ogni $x\in B_V(0,\delta)$. Segue che
\[\norm{f(x)}=\norm{\frac{\norm x}\delta f\pa{\delta\frac x{\norm x}}}\leq \frac{\norm x}\delta,\]
cio\`e $\norm f_{V^\ast}\leq 1/\delta$ e quindi $f$ limitato.
\end{proof}


\begin{remark}
Se $(X,\norm\cdot)$ \`e uno spazio seminormato e $N=\ker\norm\cdot=\cpa{x\in X\mid \norm x=0}$ allora $\norm\cdot$ passa al quoziente e lo rende uno spazio normato.
\end{remark}
\begin{example}
    Considerando lo spazio seminormato $(\Lc^p(X,\mu),\normd_p)$, la costruzione sopra corrisponde a definire lo spazio normato $(L^p(X,\mu),\normd_p)$, infatti $\ker \normd_p$ sono le funzioni con supporto in un insieme trascurabile.
\end{example}

\begin{remark}
$L(E,F)\inj \Bs(B_E(0,1),F)$ mandando $T\mapsto T\res{B_E(0,1)}$. Infatti per definizione questa mappa \`e isometrica\footnote{$\norm T=\norm{T\res{B_E(0,1)}}_{\infty,B_E(0,1)}$}. Questo identifica il primo spazio con un chiuso del secondo.
\end{remark}


\subsection{Teoremini filosofici}
\begin{theorem}[Banach Mazur]\label{ThBanachMazur}
Sia $(E,\norm\cdot)$ normato, $f:E\to E$ isometria\footnote{con questo termine intendiamo che la mappa, oltre a rispettare le distanze, \`e anche bigettiva. Se non vale bigettivit\`a diremo ``inclusione isometrica"}. Allora $f$ \`e affine.
\end{theorem}
\begin{proof}[Dimostrazione. (ESERCIZIO)]
TRACCIA:
\begin{itemize}
    \item Basta provare che $\forall a,b\in E$ vale
    \[f\pa{\frac{a+b}2}=\frac{f(a)+f(b)}2\]
    (conservando questa conserva i razionali $2$-adici e quindi per continuit\`a ogni combinazione convessa)
    \item Fissati $a,b\in E$, definiamo la \emph{deficienza affine} di $f$ (rispetto ad $a$ e $b$)
    \[def(f)=\norm{\cpa{f\pa{\frac{a+b}2}-\frac{f(a)+f(b)}2}}\]
    La tesi \`e $def(f)=0$.
    \item Notiamo che 
    \[def(f)\leq \norm{f\pa{\frac{a+b}2}}+\norm{\frac{f(a)}2}+\norm{\frac{f(b)}2}=\frac12\pa{\norm{a+b}+\norm a+\norm b}\]
    \item Consideriamo l'applicazione affine che scambia $f(a)$ e $f(b)$ data da
    \[\rho(y)=f(a)+f(b)-y\]
    Poniamo $\wt f=f\ii\circ \rho\circ f$.
    \item Mostrare $def(\wt f)=2def(f)$.
    \item Se $def(f)\neq 0$, iterando otteniamo che esiste $g$ tale che $def(g)$ \`e arbitrariamente grande (raddoppio $def(f)$ tante volte), ma questo \`e assurdo perch\'e abbiamo il limite trovato prima che non dipende dalla funzione.
\end{itemize}
\end{proof}

\begin{center}
\textbf{Filosoficamente questo vuol dire che la struttura metrica in un qualche modo determina la struttura vettoriale.}
\end{center}

\begin{theorem}[Inclusione isometrica / Fr\'echet-Kuratowski]\label{ThInclusioneIsometricaFrechetKuratowski}
Sia $(M,d)$ spazio metrico. Allora esso si immerge isometricamente in uno spazio normato\footnote{addirittura di Banach.}. In particolare si immerge in $(\Bs C(M,\R),\norm\cdot_\infty)$ via l'assegnazione seguente:

Fissiamo un punto base $x_0\in M$.\footnote{saremmo tentati da $x\mapsto d(\cdot,x)$, ma la funzione in arrivo non \`e limitata e quindi non esiste una norma ben definita}
\[\funcDef{M}{\Bs C(M,\R)}{x}{d(\cdot,x)-d(\cdot,x_0)}\]
\end{theorem}
\begin{proof}
ESERCIZIO
\end{proof}

\begin{center}
\textbf{Filosoficamente questo vuol dire che studiando mappe tra spazi metrici, possiamo pensare al codomino come spazi normati.\\
Se consideriamo l'immersione di uno spazio metrico in un Banach, possiamo ``incicciottirlo" e trovare uno spazio metrico ``vicino" che \`e localmente contraibile. Queste idee a volte possono aiutare.}
\end{center}


\section{Completezza}
\begin{definition}[Successione di Cauchy]
Una successione $(x_n)$ \`e \textbf{di Cauchy} o \textbf{fondamentale} se $\forall \e>0\ \exists n\in\N$ tale che per ogni $p,\ q>n$ si ha $d(x_p,x_q)<\e$.
\end{definition}
\begin{fact}[Propriet\`a delle successioni di Cauchy]
    ~
\begin{enumerate}
    \item Ogni successione convergente \`e di Cauchy.
    \item Se $(x_n)$ \`e di Cauchy e $\wt x\in X$ \`e un punto ad essa aderente allora $\wt x$ \`e il limite.
    \item Se $(x_n)$ come sopra ha una sottosuccessione convergente, la successione converge allo stesso limite.
    \item Ogni successione di Cauchy\footnote{questa propriet\`a \`e comoda perch\'e implica $d(x_{n_k},x_{n_p})<2^{-k+1}$ per ogni $p>k$} $(x_n)$ ha una sottosuccessione $(x_{n_k})$ tale che 
    \[d(x_{n_{k+1}},x_{n_{k}})<2^{-k}.\]
\end{enumerate}
\end{fact}

\begin{definition}[Spazio completo]
Uno spazio metrico $(X,d)$ \`e \textbf{completo} se ogni successione di Cauchy in $X$ converge.

Se $(X,\norm \cdot)$ spazio normato \`e completo rispetto alla distanza indotta da $\norm\cdot$ allora si dice \textbf{di Banach}.
\end{definition}

\begin{remark}
Uno spazio normato $(X,\norm \cdot)$ \`e di Banach se e solo se ogni serie $\sum x_k$ definita a partire da una successione tale che $\norm{x_k}<2^{-k}$ \`e convergente.

Equivalentemente $X$ di Banach se ogni serie $\sum x_k$ assolutamente convergente\footnote{cio\`e $\sum \norm{x_k}$ convergente} \`e convergente.
\end{remark}
\begin{proof}
Ogni successione si pu\`o scrivere come serie, infatti $y_n=\sum_{i=0}^n x_i$ per $x_i=y_i-y_{i-1}$. Il resto segue pensando sulle definizioni.
\end{proof}

\begin{remark}
Sia $Y\subseteq X$ con $(X,d)$ metrico. 
\begin{itemize}
    \item Se $X$ \`e completo e $Y$ \`e chiuso allora $Y$ \`e completo. 
    \item Se $Y$ \`e completo allora \`e anche chiuso.
\end{itemize}
\end{remark}

\begin{proposition}[Completamento]\label{PrCompletamento}
Sia $(X,d)$ uno spazio metrico, allora
\begin{enumerate}
    \item esiste una inclusione isometrica densa di $X$ in uno spazio metrico completo
    \[j:(X,d)\inj (\wt X,\wt d)\]
    \item il completamento \`e universale, cio\`e se $j':(X,d)\to (\wt X',\wt d')$ \`e un'altra mappa come sopra allora esiste un'unica isometria $\phi:\wt X\to \wt X'$ che fa commutare il diagramma
    % https://q.uiver.app/#q=WzAsMyxbMCwwLCJYIl0sWzEsMCwiXFx3dCBYIl0sWzEsMSwiXFx3dCBYJyJdLFswLDEsImoiLDAseyJzdHlsZSI6eyJ0YWlsIjp7Im5hbWUiOiJob29rIiwic2lkZSI6InRvcCJ9fX1dLFswLDIsImonIiwyLHsic3R5bGUiOnsidGFpbCI6eyJuYW1lIjoiaG9vayIsInNpZGUiOiJ0b3AifX19XSxbMSwyLCJcXHBoaSIsMCx7InN0eWxlIjp7ImJvZHkiOnsibmFtZSI6ImRhc2hlZCJ9fX1dXQ==
    \[\begin{tikzcd}
        X & {\wt X} \\
        & {\wt X'}
        \arrow["j", hook, from=1-1, to=1-2]
        \arrow["{j'}"', hook, from=1-1, to=2-2]
        \arrow["\phi", dashed, from=1-2, to=2-2]
    \end{tikzcd}\]
\end{enumerate}
\end{proposition}
\begin{proof}
Consideriamo un paio di costruzioni
\setlength{\leftmargini}{0cm}
\begin{itemize}
\item[$\boxed{Costruzione\ 1}$] Consideriamo
\[C_X=\cpa{\xi=(x_n)_{n\in\N}\in X^\N\mid \xi\text{ di Cauchy}}\]
con una semidistanza\footnote{VERIFICARE CHE LO \`E}
\[d(\xi,\eta)=\lim_{n\to\infty}d(\xi_n,\eta_n).\]
Questo limite esiste perch\'e la successione di queste distanze \`e di Cauchy in $\R$, che \`e completo. Notiamo che
\[d(\xi,\eta)=0\coimplies d(\xi_n,\eta_n)=o(1).\]
Notiamo che $X$ ha una inclusione isometrica in $(C_X,d)$ data associando a $x$ la successione costante al valore $x$.

Consideriamo
\[\wt X=\quot{C_X}\Rs,\qquad \xi\Rs\eta\coimplies d(\xi,\eta)=0.\]
L'inclusione isometrica di prima definisce $X\inj \wt X$, ma stavolta $\wt X$ \`e uno spazio metrico per costruzione.

ESERCIZIO: VERIFICA PROPRIET\`A DI NORMA E DENSIT\`A
\item[$\boxed{Costruzione\ 2}$] Definiamo $\wt X$ come la chiusura in $(\Bs C(X),\norm\cdot_\infty)$ dell'immagine di $X$ tramite l'inclusione di Fr\'echet Kuratowski (\ref{ThInclusioneIsometricaFrechetKuratowski}).
\item[$\boxed{Costruzione\ 3}$] (Solo per $X$ spazio normato, ma per il teorema di inclusione isometrica (\ref{ThInclusioneIsometricaFrechetKuratowski}) questo \`e sufficiente) Vedremo che esiste una inclusione isometrica di $X$ nel suo biduale ($x\mapsto val_x$) e che il biduale stesso \`e completo, quindi un completamento di $X$ \`e fornito dalla chiusura di $val_\cdot(X)\subseteq X^{\ast\ast}$
\end{itemize}
\setlength{\leftmargini}{0.5cm}

\end{proof}


\begin{proposition}[Estensione per densit\`a di uniformemente continue]\label{PrEstensioneUniformementeContinue}
Siano $X$ e $Y$ spazi metrici, $Y$ completo, $D\subseteq X$ denso e $f:D\to Y$ uniformemente continua, allora esiste un'unica estensione continua $\wt f$ di $f$ a tutto $X$, inoltre $\wt f$ \`e essa stessa uniformemente continua con lo stesso modulo di continuit\`a.

% https://q.uiver.app/#q=WzAsMyxbMCwwLCJEIl0sWzEsMCwiWSJdLFswLDEsIlgiXSxbMCwyLCJcXHN1YnNldGVxIiwzLHsic3R5bGUiOnsiYm9keSI6eyJuYW1lIjoibm9uZSJ9LCJoZWFkIjp7Im5hbWUiOiJub25lIn19fV0sWzAsMSwiZiJdLFsyLDEsIlxcd3QgZiIsMix7InN0eWxlIjp7ImJvZHkiOnsibmFtZSI6ImRhc2hlZCJ9fX1dXQ==
\[\begin{tikzcd}
	D & Y \\
	X
	\arrow["f", from=1-1, to=1-2]
	\arrow["\subseteq"{marking, allow upside down}, draw=none, from=1-1, to=2-1]
	\arrow["{\wt f}"', dashed, from=2-1, to=1-2]
\end{tikzcd}\]
\end{proposition}

\begin{definition}[Categorie di spazi metrici]
Sia $\Met$ la categoria degli spazi metrici con mappe date da applicazioni uniformemente continue e $\CMet$ la sottocategoria piena dove gli oggetti sono spazi metrici completi
\end{definition}

\begin{remark}
L'operazione di completamento \`e un funtore\footnote{preserva composizione per l'unicit\`a della mappa tra estensioni} $\wt\cdot:\Met\to \CMet$. Questo funtore \`e aggiunto al funtore dimenticante / di inclusione $j:\CMet\to \Met$, infatti
\[\Hom_{\CMet}(\wt X,Y)=UC(\wt X,Y)\pasgnl\cong{(\ref{PrEstensioneUniformementeContinue})} UC(X,j(Y))=\Hom_{\Met}(X,j(Y)).\]
\end{remark}

\begin{exercise}
Verificare l'aggiunzione.
\end{exercise}


\section{Prodotto di spazi (semi)normati}
\begin{remark}
Se $Y\subseteq X$ \`e un sottospazio vettoriale e $(X,\normd)$ \`e normato allora $Y$ \`e (semi)normato con la norma indotta. La topologia indotta \`e quella di sottospazio
\end{remark}

\begin{definition}[Prodotto di spazi (semi)normati]
    Se $(X,\normd_X)$ e $(Y,\normd_Y)$ sono spazi (semi)normati, la (semi)norma prodotto \`e data da
    \[\norm{(x,y)}_{X\times Y}=\max\cpa{\norm x_X,\norm y_Y}.\]
    Questa rende $X\times Y$ uno spazio (semi)normato e
    \[B_{X\times Y}((0,0),1)=B_X(0,1)\times B_Y(0,1),\]
    cio\`e la topologia indotta \`e la topologia prodotto.
\end{definition}

\begin{definition}[Somma diretta topologica]
Due sottospazi di $(X,\normd)$ $Y$ e $Z$ sono in \textbf{somma diretta algebrica} se $+\res{Y\times Z}:Y\times Z\to X$ \`e bigettiva. Se $+\res{Y\times Z}$ \`e anche un omeomorfismo diciamo che $X$ \`e la \textbf{somma diretta topologica} di $Y$ e $Z$.
\end{definition}

\begin{remark}
$X$ \`e la somma diretta topologica di $Y$ e $Z$ se $X$ \`e isomorfo come spazio normato a $(Y\times Z,\normd_{Y\times Z})$.
\end{remark}

\begin{remark}
La mappa $+\res{Y\times Z}$ \`e sempre continua, ma in generale non \`e un omeomorfismo.
\end{remark}

\begin{definition}[Proiettore]
Un endomorfismo lineare $P:X\to X$ si dice \textbf{proiettore} se \`e idempotente, cio\`e $P^2=P$.
\end{definition}

\begin{remark}
Un proiettore definisce una decomposizione in somma diretta algebrica $X=\ker P\oplus \imm P$. Viceversa, ad ogni decomposizione in somma diretta algebrica possiamo associare un proiettore
\end{remark}

\begin{remark}
I proiettori $P_Y:X\to Y$ e $P_Z=id-P_Y:X\to Z$ sono continui se e solo se la somma \`e topologica, infatti
\[(+\res{Y\times Z})\ii=P_Y\times P_Z.\]
\end{remark}

\begin{definition}[Spazio (semi)normato quoziente]
Se $(X,\normd)$ \`e (semi)normato e $Y$ \`e un suo sottospazio allora come spazio vettoriale
\[X/Y=\cpa{x+Y\mid x\in X}.\]
Su essa definiamo la seguente norma: se $\xi\in X/Y$ allora\footnote{pensando a $\xi$ come un traslato di $Y$, la norma che stiamo definendo \`e la distanza di questo spazio affine dall'origine.}
\[\norm{\xi}_{X/Y}=\inf_{x\in\xi}\norm x.\]
\end{definition}

\begin{exercise}
$\normd_{X/Y}$ \`e una seminorma su $X/Y$ e rende la proiezione $\pi:X\to X/Y$ una applicazione aperta e continua. Pi\`u precisamente
\[\pi(B_X(0,1))=B_{X/Y}(0,1)\]
\end{exercise}
\begin{proof}
Continua perch\'e $\norm{\pi(x)}_{X/Y}\leq \norm x$ per definizione di estremo inferiore, quindi $\pi$ ha norma come operatore $\leq 1$, e quindi \`e continua.
\end{proof}

\begin{remark}
Notiamo che $X/Y$ ha effettivamente la topologia quoziente indotta da $\pi$
\end{remark}

\begin{exercise}
La (semi)norma quoziente \`e una norma se e solo se $Y$ \`e chiuso (a prescindere dal fatto che $\normd_X$ sia una norma o seminorma).
\end{exercise}

\begin{remark}
Se $Y$ e $Z$ sono seminormati allora $Y\cong \frac{Y\times Z}Z$ come spazi seminormati.
\end{remark}

\begin{remark}
Se $Y\subseteq X$ ed esiste\footnote{ci sono casi in cui non esite, come $c_0\subseteq \ell_\infty$} $Z$ tale che $X=Y\oplus Z$ allora $Z\cong X/Y$.
\end{remark}

\begin{remark}
In generale $X$ non \`e isomorfo a $Y\times X/Y$.
\end{remark}

\begin{remark}
Per quanto riguarda la completezza in queste costruzioni:
\begin{itemize}
    \item $Y$ sottospazio di $X$ con $X$ di Banach \`e un Banach se e solo se \`e chiuso
    \item $(Y\times Z,\normd_{Y\times Z})$ \`e Banach se e solo se lo sono sia $Y$ che $Z$
    \item Se $(X,\normd)$ \`e normato e $Y\subseteq X$ \`e un sottospazio chiuso allora $(X,\normd)$ \`e completo se e solo se sia $Y$ che $X/Y$ sono completi.
\end{itemize}
Notiamo che l'ultima propriet\`a implica la seconda, infatti $Y\cong \frac{Y\times Z}Z$
\end{remark}


\begin{proposition}[Duale del prodotto]\label{PrDualeProdottoEProdottoDuali}
Dati $X$ e $Y$ spazi di Banach, il duale di $X\times Y$ \`e isometricamente isomorfo a
\[(X^\ast\times Y^\ast,\normd)\]
dove $\norm{(\xi,\eta)}=\norm\xi_{X^\ast}+\norm{\eta}_{Y^\ast}$ (che \`e topologicamente equivalente a $\normd_{X^\ast\times Y^\ast}$).
\[(X^\ast\times Y^\ast,\norm{P_{X^\ast}(\cdot)}_{X^\ast}+\norm{P_{Y^\ast}(\cdot)}_{Y^\ast})\cong ((X\times Y)^\ast,\normd_{(X\times Y)^\ast}).\]
\end{proposition}


\section{Elenco di spazi completi}

\begin{proposition}
Sia $S$ insieme e $E$ Banach, allora lo spazio normato $(\Bs(S,E),\normd_{\infty,S})$ \`e completo.
\end{proposition}
\begin{proof}
~[PERSO, RIGUARDA POI]

tale che $\norm{f(s)}=\norm{\sum_kf_k(s)}\leq \sum_k\norm{f_k(s)}\leq \sum \norm f_{\infty,S}$

quindi $\norm f_{\infty,S}$
\end{proof}

\filosofia{Uno degli strumenti dell'analista: aggiungere e togliere, cio\`e
\[\pi\rho o\sigma\tau\al\vp\al\acute{\iota}\rho\e\sigma\iota\varsigma\]}

\begin{lemma}
Se $(f_k)_{k\in\N}\subseteq \Bs(S,E)$ con $f_k$ continua in $s_0$ per ogni $k$ e $f_k\to f$ uniformemente allora anche $f$ \`e continua in $s_0$.
\end{lemma}
\begin{proof}
Consideriamo
\begin{align*}
\norm{f(s)-f(s_0)}\leq&\norm{f(s)-f_k(s)}+\norm{f_k(s)-f_k(s_0)}+\norm{f_k(s_0)-f(s_0)}\leq\\
\leq&2\norm{f-f_k}_{\infty,S}+\norm{f_k(s)-f_k(s_0)}
\end{align*}
Per la convergenza uniforme di $f_k\to f$ si ha che per ogni $\e>0$ esiste $n\in\N$ tale che $\norm{f-f_n}_{\infty,S}\leq \e/3$.

Per la continuit\`a in $s_0$ di $f_n$ esiste un intorno $U$ di $s_0$ rale che $\norm{f_n(s)-f_n(s_0)}\leq \e/3$ per ogni $s\in U$. Allora per ogni $s\in U$ si ha
\[\norm{f(s)-f(s_0)}\leq2\e/3+\e/3=\e.\]
\end{proof}

\begin{proposition}
Sia $S$ spazio topologico, $E$ banach, allora $\Bs C(S,E)$ \`e completo.
\end{proposition}
\begin{proof}
Basta mostrare che $\Bs C(S,E)$ \`e chiuso in $\Bs(S,E)$. Questo segue dal fatto che la continuit\`a in un punto $s_0\in S$ si conserva per convergenza uniforme, che \`e il lemma precedente.
\end{proof}

\begin{example}
Sia $S=\N\cup\cpa{\infty}$ la compattificazione di Alexandrov di $\N$ e $E$ un banach, allora
\[c(E)\doteqdot\cpa{x:\N\to E,\text{ convergente}}\cong \Bs C(S,E)\]
Questo mostra che $c(E)$ \`e chiuso (e quindi completo) in $\ell_\infty(E)=\Bs(\N,E)$.
\end{example}


Conseguenze:
\begin{proposition}
Lo spazio $(L(X,Y),\normd)$ \`e completo
\end{proposition}
\begin{proof}
Considerando l'inclusione isometrica 
\[R:\funcDef{L(X,Y)}{\Bs(B_X(0,1),Y)}{T}{T\res{B_X(0,1)}}\]
basta vedere che $R(L(X,Y))$ \`e chiuso.

Se $(T_n)_{n\in\N}\subseteq L(X,Y)$ \`e tale che $R(T_n)\to f$ uniformemente in $\Bs(B_X(0,1),Y)$ allora mostriamo che $f$ \`e la restrizione a $B_X(0,1)$ di una qualche lineare $T$.

Mostriamo che le $T_n$ convergono puntualmente per ogni $x\in X$: se $x=0$ ok, se $x\neq 0$
\[T_n(x)=\norm x T_n(x/\norm x)=\norm x R(T_n)(x/\norm x)\to \norm x f(x/\norm x)\]

Sia $T:X\to Y$ definita da $T(x)=\norm x f(x/\norm x)$

[MOSTRARE CHE LA CONVERGENZA \`E UNIFORME, ME LO SONO PERSO]
\end{proof}

\begin{corollary}[Duale di spazio normato \`e banach]\label{CorDualeNormatoEBanach}
Il duale di uno spazio normato \`e sempre banch.
\end{corollary}

\begin{theorem}[Integrazione per serie]\label{ThIntegrazionePerSerie}
    Sia $(X,\Qs,\mu)$ \`e uno spazio di misura e sia $(f_k)_{k\in\N}\subseteq \Lc^1(X,\Qs,\mu)$ tali che
    \[\sum_{k\in \N}\norm{f_k}_1<\infty\]
Allora $\sum_{k\in\N}f_k$ converge q.o. e in norma 1.
\end{theorem}
\begin{proof}
Per ogni $n\in\N$ sia $g_n:X\to \R$ data da
\[g_n(x)=\sum_{k=0}^n\abs{f_k(x)}.\]
Notiamo che $(g_n)$ \`e una successione di funzioni misurabili non negative crescente. Inoltre $g_n\to \sum_{k\in\N}\abs{f_k(x)}$ per definizione di serie.

Per convergenza monotona
\[\sum_{k\in\N}\norm f_1\leftarrow\sum_{k=0}^n\norm{f_k}_1=\inf_X g_nd\mu\to \int_X gd\mu\]
cio\`e $\inf_X gd\mu=\sum{k\in\N}\norm f_1<\infty$, cio\`e $g\in\Lc^1$.

Inoltre $s_n=\sum_{k=0}^nf_k$ \`e una successione dominata da $g$:
\[\abs{s_n(x)}\leq \sum_{k=0}^n\abs{f_k(x)}\leq g(x).\]
Quindi la serie $\sum f_k(x)$ \`e una serie assolutamente convergente per ogni $x$ dove $g<\infty$. Poich\'e $\int g<\infty$ le eccezioni sono trascurabili, quindi quasi ovunque $\sum f_k(x)$ \`e assolutamente convergente.

Sia $f(x)=\sum f_k(x)$ dove la serie converge. Notiamo che
\[\abs{f(x)}\leq \sum_{k\in\N}\abs{f_k(x)}=g(x),\]
quindi $\norm f_1\leq \int gd\mu=\sum_{k\in\N}\norm{f_k}_1$.

Applicando come prima la stima alle code
\[\norm{f-s_n}_1=\norm{\sum_{k=n+1}^\infty f_k}_1\leq \sum_{k>n}\norm{f_k}_1=o(1)\]
dove l'ultimo termine va a 0 perch\'e $\sum\norm {f_k}_1$ \`e convergente.
\end{proof}

\begin{corollary}[Weil]\label{CorTeoremaWeil}
Siano $f_n\in \Lc^1(X,\Qs,\mu)$ convergenti in $\normd_1$. Allora esiste $n_k$ successione strettamente crescente di indici tali che $f_{n_k}$ converge quasi ovunque ed \`e dominata in $\Lc^1$.
\end{corollary}
\begin{proof}
Sia $f$ il limite in $\normd_1$. Data questa convergenza consideriamo una sottosuccessione $n_k$ tale che $\norm{f-f_{n_k}}_1<2^{-k}$. Scrivendo la successione in termini di una somma telescopica
\[f_{n_k}=f_{n_0}+\sum_{j=1}^k(f_{n_j}-f_{n_{j-1}})\]
si ha per il teorema di integrazione per serie\footnote{$\norm{f_{n_0}}_1+\sum_{j=1}^\infty \norm{f_{n_j}-f_{n_{j-1}}}_1\leq \norm{f_{n_0}}_1+\sum_{j=1}^\infty \norm{f_{n_j}-f}_1+ \sum_{j=1}^\infty \norm{f_{n_{j-1}}-f}_1<\infty$} (\ref{ThIntegrazionePerSerie}) $f_{n_k}$ converge quasi ovunque e in $Lc^1$, inoltre \`e dominata da
\[g(x)=\abs{f_{n_0}(x)}+\sum_{j=0}^\infty\abs{f_{n_j}-f_{n_{j-1}}}\geq \abs{f_{n_k}(x)}\]
con $g(x)\in \Lc^1$.
\end{proof}

\begin{proposition}[$L^1$ \`e completo]\label{PrL1Completo}
Se $(X,\Qs,\mu)$ \`e uno spazio di misura, $L^1(X,\Qs,\mu)$ \`e completo.
\end{proposition}
\begin{proof}
Segue immediatamente dal teorema di integrazione per serie (\ref{ThIntegrazionePerSerie}).
\end{proof}

\begin{remark}
La convergenza quasi ovunque di funzioni $\Lc^1(\R,dx)$ \`e \textbf{NON} \`e la convergenza rispetto a una topologia opportuna su $\Lc^1(\R,dx)$.

\begin{proposition}[Propriet\`a di Urysohn]\label{PrProprietaUrysohn}
Ogni convergenza topologica in $X$ insieme ha la seguente propriet\`a \textbf{di Urysohn}: $x_n\to x$ rispetto alla topologia se e solo se per ogni sottosuccessione $x_{n_k}$ esiste una sotto-sottosuccessione $x_{n_{k_j}}\to x$.
\end{proposition}
\begin{proof}
Se $x_n\to x$ converge ok. Se non converge allora esiste un intorno $U$ di $x$ tale che $x_n\notin U$ frequentemente, quindi troviamo una sottosuccessione $x_{n_k}$ che sta sempre fuori da $U$, quindi nessuna sua sotto-sottosuccessione pu\`o convergere a $x$.
\end{proof} 

La convergenza q.o. per successioni in $\Lc^1(\R)$ non ha la propriet\`a di Uhrisohn.
\end{remark}

\begin{definition}[Operatore di composizione]
Se $E$ \`e uno spazio di funzioni con codominio $\R$ e $f:\R\to\R$, definiamo l'operatore di composizione per $f$ come $E\ni u\mapsto f\circ u$.
\end{definition}

\begin{lemma}
Sia $u_k$ una successione che converge a $u$ in $\normd_p$. A meno di sottosuccessione $u_k\to u$ quasi ovunque e dominata in $\Lc^p$.
\end{lemma}
\begin{proof}
Teorema di Weil (\ref{CorTeoremaWeil}) in $\Lc^p$.
\end{proof}

\begin{proposition}
Lo spazio $L^p(X,\Qs,\mu)$ per $0\leq p<\infty$ \`e completo.
\end{proposition}
\begin{proof}
$L^p$ ed $L^1$ NON sono isomorfi come spazi di Banach in generale\footnote{cursiosit\`a non banale da vedere}, ma esiste un omeomorfismo localmente Lipschitz e questo basta a mostrare la completezza: se $u_k$ \`e una successione di Cauchy in $L^p$, se $\Phi$ \`e Lipschitz allora $\Phi(u_k)$ \`e ancora di Cauchy in $L^1$ e quindi converge, poi torno indietro con $\Phi\ii$, che mantiene il limite per continuit\`a.


Consideriamo
\[\Phi:\funcDef{\Lc^p}{\Lc^1}{u}{\abs{u}^p\sgn(u)}\]
Chiaramente \`e invertibile mandando $v\in L^1$ in $\abs{v}^{1/p}\sgn v$. La mappa $\Phi$ \`e l'operatore di composizione con la funzione $f(t)=\abs{t}^p\sgn t$. La continuit\`a degli operatori di composizione \`e un fatto generale. Se $u_k\to u$ converge in $\normd_p$ allora per il lemma a meno di sottosuccessione converge q.o. e dominata, quindi componendo con $f$ abbiamo ancora convergenza quasi ovunque per continuit\`a ($f(u_k)\to f(u)$ q.o.). Se $\abs{u_k}\leq g$ in $\Lc^p$ allora $\abs{u_k}^p\leq g^p$ in $\Lc^1$, similmente per $\Phi\ii$, quindi effettivamente $\Phi$ \`e un omeomorfismo.


Mostriamo ora che $\Phi$ \`e localmente lipschitz: siano $u,v\in \Lc^p(X)$
\[\abs{\Phi(u)-\Phi(v)}_1=\int_X\abs{f(u(x))-f(v(x))}d\mu(x)\]
ma se $t<s$ allora $\abs{f(t)-f(s)}\leq \sup_{t\leq \xi\leq s}\abs{f'(\xi)}\abs{t-s}$ e $\abs{f'(xi)}=p\abs{xi}^{p-1}\leq p(\max\cpa{\abs t,\abs s})^p$, quindi
\begin{align*}
    \abs{\Phi(u)-\Phi(v)}_1\leq& p\int_X\max\cpa{\abs{u(x)}^{p-1},\abs{v(x)}^{p-1}}\abs{u(x)-v(x)}d\mu\leq\\
    \leq &p\int_X\pa{\abs{u(x)}^{p-1}+\abs{v(x)}^{p-1}}\abs{u(x)-v(x)}d\mu\pasgnl\leq{H\"older}\\
    \leq&p\pa{\pa{\int_X\abs{u}^{(p-1)q}}^{1/q}+\pa{\int_X\abs{v}^{(p-1)q}}^{1/q}}\pa{\int_X\abs{u-v}^p}^{1/p}=\\
    \pasgnlmath={p-1=p/q}&p(\norm u_p^{p-1}+\norm v_p^{p-1})\norm{u-v}_p
\end{align*}
quindi $\Phi$ \`e Lipschitz di costante $2pR^{p-1}$ sulla palla $B_{L^p}(0,R)\subseteq L^p$
\end{proof}



\begin{proposition}
Lo spazio $L^\infty(X,\Qs,\mu)$ \`e completo
\end{proposition}
\begin{proof}
~[NON HO VISTO, RIGUARDA I PDF]
\end{proof}

$\norm f_{C^1}=\norm f_{\infty,\Omega}+\sum_{i=1}^n\norm{\del_i f}_{\infty,\Omega}$. Questa norma rende continua l'immaersione $C^1_b\to (C_b^0)^{n+1}$ data da $f\mapsto(f,\del_1f,\cdots,\del_n f)$

\begin{proposition}
Sia $\Omega\subseteq \R^n$ aperto. Lo spazio 
\[C^k_b(\Omega)=\cpa{f:\Omega\to\R\mid \text{di classe $C^k$ con derivate limitate su $\Omega$ fino all'ordine $k$}}\]
\`e completo.
\end{proposition}
\begin{proof}
Il caso $k=1$ \`e una conseguenza del teorema di limite sotto il segno di derivata, infatti se $f_k:\Omega\to \R$, $\del_i f_k:\Omega\to\R$ \`e tale che $\del_i f_k\to g_i$ uniformemente in $\Omega$ e $f_k\to f$ puntualmente in $\Omega$ allora esiste $\del_i f$ e vale $g_i$. Se poi $f_k\in C^1(\Omega)$ allora la $g_i$ \`e continua perch\'e limite uniforme di $\del_i f_k$ continue, quindi per il teorema del differenziale totale la $f$ \`e anche $C^1$.

Per il teorema di limite sotto il segno di derivata, l'immersione $C^1_b\to (C_b^0)^{n+1}$ ha immagine chiusa, infatti una successione $(f_k,\del_1f_k,\cdots,\del_nf_k)$ nell'immagine convergente a $(f,g_1,\cdots, g_n)$ \`e proprio una delle ipotesi del teorema di convergenza sotto segno di derivata, quindi $f_k\to f$ in $C^1$
\end{proof}
\chapter{Spazi vettoriali topologici}

\begin{definition}[Spazio vettoriale topologico]
Uno \textbf{spazio vettoriale topologico} \`e uno spazio vettoriale $X$ su $\K\in\cpa{\R,\C}$ munito di una topologia che rende continue le mappe 
\[+:X\times X\to X\quad \text{ e }\quad\cdot:\K\times X\to X.\]
\end{definition}
\begin{example}
Esempi di SVT sono
\begin{itemize}
    \item Ogni spazio normato
    \item $C(\R,\R)$ con la topologia della convergenza uniforme sui compatti.
    \item Se $X$ \`e uno spazio topologico qualunque considero $C(X,\R)$ con topologia di convergenza uniforme su compatti.
\end{itemize}
\end{example}

\begin{exercise}
La topologia della convergenza uniforme su compatti su $C(\R,\R)$ non \`e indotta da una norma.
\end{exercise}
\begin{proof}
TRACCIA
\begin{itemize}
    \item Su uno spazio normato, se $U$ e $V$ sono intorni di 0 allora esiste $\la\in\R$ tale che $\la U\supseteq V$.
    \item Mostrare che la topologia della convergenza uniforme su compatti non ha questa propriet\`a.
\end{itemize}
\end{proof}

\begin{exercise}
Ogni SVT che \`e $T_0$ \`e anche\footnote{In questo corso con $T_3$ intendiamo $T_3$ e Hausdorff} $T_3$ e\footnote{$T_{3\frac12}$ \`e $T_3$ pi\`u esiste una funzione continua che vale $1$ sul punto e $0$ sul chiuso che sto separando} $T_{3\frac12}$
\end{exercise}
\begin{exercise}[Spazi non $T_0$ non sono troppi interessanti]
Ogni SVT $X$ si decompone in somma diretta topologica $X=Y\oplus \ol{\cpa0}$ con $Y$ qualunque addendo algebrico di $\ol{\cpa{0}}$. Segue che $Y\cong X/\ol{\cpa0}$, $Y$ risulta essere $T_0$ e $\ol{\cpa0}$ ha la topologia indiscreta.
\end{exercise}

\section{Intorni dell'origine in SVT}

\begin{definition}[Filtro]
Un \textbf{filtro} $\Fc$ su un insieme $X$ \`e una famiglia non vuota di sottoinsiemi di $X$ tale che
\begin{itemize}
    \item per ogni $F\in\Fc$, $F\neq \emptyset$
    \item Se $F\in \Fc$ e $F\subseteq F'$ allora $F'\in\Fc$
    \item Se $F,F'\in\Fc$ allora $F\cap F'\in\Fc$
\end{itemize}
\end{definition}

\begin{definition}[Sottoinsieme bilanciato]
Sia $X$ un $\K$-spazio vettoriale e $A\subseteq X$. $A$ \`e \textbf{bilanciato} se per ogni $\la\in\K$ tale che $\abs\la\leq1$ si ha $a\in A\implies \la a\in A$, cio\`e 
\[B_\K(0,1)\cdot A\subseteq A.\]
\end{definition}

\begin{remark}
Se $V$ \`e bilanciato allora $0\in V$ perch\'e $0\in B_\K(0,1)$.
\end{remark}

\begin{definition}[Sottoinsieme assorbente]
Sia $X$ un $\K$-spazio vettoriale e $B\subseteq X$. $B$ \`e \textbf{assorbente} se per ogni $x\in X$ esiste $n_x\in\N$ tale che per ogni $t\geq n_x$ si ha $x\in tB$.
\end{definition}


\begin{remark}
Poich\'e in uno SVT le traslazioni $X\to X$ con $x\mapsto x+x_0$ sono omeomorfismi, per descrivere la topologia basta descrivere il filtro degli intorni di $0$.
\end{remark}



Come notazione sia $\Uc=\Uc_X$ l'insieme degli intorni di $0\in X$.

\begin{proposition}[Propriet\`a intorni di 0]\label{PrProprietaIntorni0}
$\Uc$ ha le seguenti propriet\`a
\begin{enumerate}
    \item $\Uc$ \`e un filtro
    \item Per ogni $U\in\Uc$ esiste $V\in\Uc$ tale che $V+V\subseteq U$
    \item Per ogni $U\in\Uc$ esiste $V\in\Uc$ con $V\subseteq U$ e $V$ bilanciato
    \item Ogni elemento di $\Uc$ \`e assorbente
\end{enumerate}
\end{proposition}
\begin{proof}
Dimostriamo le varie propriet\`a
\begin{enumerate}
    \item La propriet\`a $1.$ \`e vera per ogni insieme definito come ``gli intorni di $x$" per $x$ fissato in spazio topologico $X$.
    \item Segue dalla continuit\`a di $+$ in $(0,0)\in X\times X$. Basta definire $V$ in modo tale che $V\times V\subseteq +\ii(U)$.
    \item Segue dalla continuit\`a di $\cdot$ in $(0,0)$. Se $U$ intorno di $0$ in $X$, siano $\e>0$ e $V\in \Uc$ tali che $B_\K(0,\e)\times V\subseteq \cdot\ii(U)$. Allora $B_\K(0,\e)\cdot V$ \`e bilanciato e contenuto in $U$ per costruzione. Questo insieme \`e anche un intorno perch\'e si pu\`o scrivere come
    \[\bigcup_{\abs{\la}\leq \e}\la V \]
    e poich\'e $V$ \`e un intorno di $0$, ogni $\la V$ \`e un intorno di $0$, quindi anche questa unione.
    \item Segue dalla continuit\`a della mappa $\R_+\to X$ che per fissato $x_0\in X$ assegna $s\mapsto sx_0$. Infatti per ogni $U\in\Uc$ esiste $\e>0$ tale che per ogni $0\leq s\leq \e$, $sx_0\in U$ e riscrivendo questo in termini di $t=1/s$ abbiamo $x_0\in tU$ per ogni $t\geq 1/\e$. Come $n_{x_0}$ basta scegliere $\floor{\e\ii}$.
\end{enumerate}
\end{proof}

\begin{exercise}\label{ExTopologiaIndottaDaIntorniDi0}
Sia $X$ spazio vettoriale su $\K$ e $\Uc$ una famiglia si sottoinsiemi di $X$ tali che valgano le quattro propriet\`a della proposizione precedente (\ref{PrProprietaIntorni0}). Allora esiste un'unica topologia su $X$ che rende $X$ uno SVT e tale che $\Uc$ \`e il filtro degli intorni di $0$. In questa topologia $\Uc$ \`e un sistema fondamentale di intorni per $0$.
\end{exercise}
\begin{proof}
L'idea \`e che definiamo $A\subseteq X$ aperto se e solo se per ogni $a\in A$, $A-a\in\Uc$ (sto traducendo ``aperto $\coimplies$ intorno di ogni suo punto"). Si pu\`o mostrare che questa scelta definisce una topologia che rende $X$ uno SVT.
\end{proof}

\begin{exercise}
Definire analogamente una topologia di SVT su $X$ tramite degli assiomi che si basano una una base di intorni di $0$ (al posto di tutti gli intorni). Per esempio la famiglia degli intorni bilanciati di $0$.
\end{exercise}

\begin{remark}
Se uno SVT \`e $T_0$ allora \`e automaticamente $T_1$ e $T_2$, basta sfruttare propriet\`a di simmetria.
\end{remark}

\begin{remark}
Ogni SVT \`e uno spazio topologico regolare, cio\`e ogni punto ha una base di intorni chiusi. Se $X$ \`e anche $T_0$ allora $X$ \`e $T_3$.
\end{remark}
\begin{proof}
Sia $C$ un chiuso di $X$ e $x\in X$ con $x\notin C$. Sia $U\in\Uc_X$ tale che $x+U\cap C=\emptyset$, che esiste perch\'e $C$ \`e chiuso. Sia $V\in \Uc_X$ tale che $V-V\subseteq U$, allora\footnote{Un insieme come $C+V$ \`e detto intorno uniforme di $C$} $(x+V)\cap (C+V)=\emptyset$ dove $C+V$ \`e un intorno di $c$ per ogni $c\in C$ per definizione.
\end{proof}

\begin{remark}\label{PrSeparoCompattoEChiusoDisgiunti}
Se $K$ \`e compatto, $C$ chiuso con $K\cap C=\emptyset$ allora esiste $V$ tale che $(K+V)\cap (C+V)=\emptyset$.
\end{remark}
\begin{proof}
Per ogni $x\in K$ sia $V_x\in \Uc_X$ tale che $x+(V_x+V_x-V_x)$ \`e disgiunto da $C$. Abbiamo dunque un ricoprimento $\cpa{x+V_x}_{x\in K}$ di $K$, che \`e compatto, quindi estraggo un sottoricoprimento finito $\cpa{x_i+V_{x_i}}$ e definisco $V$ come l'intersezione di questi. Allora
\[(K+V)\cap (C+V)=\emptyset,\]
infatti se $x\in K+V$ allora $x=k+v$ con $k\in K$ e $v\in V$ ma $k\in x_i+V_{x_i}$ per qualche $i$, quindi $x=x_i+v_i+v$, e avendo supposto che $x_i+(V_{x_i}+V_{x_i}-V_{x_i})\cap C=\emptyset$ abbiamo che $x=x_i+v_i+v\notin C+V$.
\end{proof}


\section{SVT localmente convessi}
\begin{definition}[SVT localmente convesso]
Uno \textbf{spazio vettoriale topologico localmente convesso} (\textbf{SVTLC}) \`e uno SVT tale che $0$ ha una base di intorni convessi.
\end{definition}
\begin{example}
Diamo alcuni esempi
\begin{itemize}
    \item Ogni spazio normato
    \item $C(X)$ con $X$ spazio topologico con la topologia della convergenza uniforme sui compatti
    \item $C^\infty(\Omega)$ con $\Omega\subseteq \R^n$ aperto e topologia della convergenza uniforme sui compatti di tutte le derivate in ogni ordine
\end{itemize}
\end{example}

\begin{exercise}
Sia $\Ms=\cpa{f:[0,1]\to \R\mid \text{misurabili}}$, allora esiste una metrica su $\Ms$ che lo rende uno SVT e tale che $f_n\to f$ se e solo se $f_n\to f$ in misura, cio\`e per ogni 
\[\forall \e>0,\quad \lim_{n\to\infty}\abs{\cpa{\abs{f_n}>\e}}= 0\]
Mostrare che l'unico intorno convesso di $0$ \`e $\Ms$ stesso, da cui segue $\Ms^\ast=\cpa{0}$.
\end{exercise}

\begin{remark}
Per ci\`o che sappiamo sugli intorni di $0$ in uno SVT, se $X$ \`e SVTLC allora esiste una base $\Bc$ data dagli intorni di $0$ assorbenti, bilanciati e convessi.
\end{remark}

\begin{definition}[Disco]
Un insieme $B$ \`e detto \textbf{disco} se \`e assorbente, bilanciato e convesso.
\end{definition}

\begin{proposition}[]\label{PrTopologiaConvessaIndotta}
Sia $X$ un $\R$-SV e $\Bc$ una famiglia di sottoinsiemi di $X$ tale che
\begin{itemize}
    \item Per ogni $B\in \Bc$, $B$ \`e Assorbente, Bilanciato e Convesso
    \item Per ogni $B_1,B_2\in\Bc$ si ha $B_1\cap B_2\in \Bc$
\end{itemize}
allora $\Uc=\cpa{U\subseteq X\mid \exists r>0,\ \exists B\in\Bc\mid rB\subseteq U}$ \`e un filtro di insiemi che induce una topologia che rende $X$ uno SVT come da esercizio (\ref{ExTopologiaIndottaDaIntorniDi0}). La topologia indotta \`e anche localmente convessa.
\end{proposition}
\begin{proof}
Mostriamo le quattro propriet\`a:
\begin{itemize}
    \item Chiaramente $\Uc$ \`e un filtro.
    \item Ogni $U\in \Uc$ \`e assorbente perch\'e lo sono gli elementi di $\Bc$
    \item Per ogni $U\in\Uc$ esiste $V\in\Uc$ tale che $V+V\subseteq U$, basta scegliere $V=\frac12 B$ con $B\subseteq U$ convesso in quanto se $B$ \`e convesso $B+B=2B$
    \item Ogni $U\in\Uc$ contiene un bilanciato perch\'e contiene una versione scalata di un elemento di $\Bc$.
\end{itemize}
\end{proof}



\begin{remark}
Se $\Bc$ \`e una famiglia di dischi allora definendo $\wt \Bc=\cpa{B_1\cap B_2\mid B_1,B_2\in\Bc}$ si ha che $\wt \Bc$ rispetta gli assiomi della proposizione (\ref{PrTopologiaConvessaIndotta}) e quindi induce una topologia su $X$ che lo rende uno SVT. Questa \`e la meno fine tale che $\Bc\subseteq \Uc_X$. In particolare $\Uc_X$ ha una base data da $\cpa{rB\mid B\in\wt \Bc}$.
\end{remark}






\subsection{Funzionali di Minkowski}
\begin{definition}[Funzionale di Minkowski]
Sia $X$ un $\R$-spazio vettoriale, $C\subseteq X$ convesso, $0\in C$. Il \textbf{funzionale di Minkowski} associato a $C$ \`e dato da:
\[p_C:\funcDef{X}{[0,+\infty]}{x}{\inf\cpa{t\geq 0\mid x\in tC}}\]
dove $\inf\emptyset=\infty$ in questo formalismo.
\end{definition}

\begin{remark}
Se $B(0,1)\subseteq C\subseteq \ol{B(0,1)}$ per $X$ normato allora $p_C(x)=\norm x$.
\end{remark}

\begin{proposition}[Propriet\`a funzionali di Minkowski]\label{PrProprietaFunzionaliMinkowski}
Valgono le seguenti propriet\`a
\begin{itemize}
    \item $C$ \`e assorbente se e solo se $p_C(x)<\infty$ per ogni $x\in X$.
    \item Si ha $\cpa{p_C<1}\subseteq C\subseteq\cpa{p_C\leq 1}$
\end{itemize}
\end{proposition}
\begin{proof}
Mostriamo le varie propriet\`a
\begin{itemize}
    \item Evidente dalla definizione di assorbente.
    \item Se $p_C(x)<1$ allora esiste $0\leq t\leq 1$ tale che $x\in tC$, cio\`e $x=tc$. Poich\'e $(1-t)0=0$ si ha $x=tc+(1-t)0$ e per convessit\`a questo \`e un elemento di $C$, cio\`e $x\in C$.
    
    Se $x\in C$ allora $1\in \cpa{t\geq 0\mid x\in tC}$, quindi $p_C(x)\leq 1$.
\end{itemize}
\end{proof}


\begin{remark}[Famiglia di seminorme induce SVTLC]
    Se $\Pc$ \`e una famiglia di seminorme su $X$, possiamo definire
    \[\Bc=\cpa{B_p(0,r)\mid p\in\Pc,\ r\in \R_+},\quad B_p(0,r)=\cpa{y\in X\mid p(x-y)<r}\]
    Si pu\`o mostrare che $\Bc$ \`e un insieme di dischi e quindi induce una struttura di SVTLC su $X$.
\end{remark}

\begin{remark}
Se $\Pc$ \`e una famiglia di seminorme su $X$ e definiamo
\[\wt\Pc=\cpa{\max (p_1,\cdots, p_n)\mid p_i\in \Pc}\]
allora $\Uc=\cpa{B_p(0,r)\mid p\in\wt \Pc,r>0}$ \`e una base di intorni di $0$ che induce la topologia dell'osservazione precedente.
\end{remark}

\begin{remark}[Ogni SVTLC \`e indotto da seminorme]\label{RmOgniSVTLCDerivaDaSeminorme}
Poich\'e se $B$ \`e assorbente, bilanciato e convesso, esso produce una seminorma $p_B$ data dal funzionale di Minkowski tale che $\cpa{p_B<1}\subseteq B\subseteq \cpa{p_B\leq 1}$, ogni topologia di $X$ come SVTLC si pu\`o ottenere a partire da famiglie di seminorme.
\end{remark}

\begin{proposition}
La topologia di SVTLC indotta da $\Pc$ insieme di seminorme \`e $T_0$ se e solo se $\Pc$ \`e separante, cio\`e per ogni $x\in X\nz$ esiste $p\in\Pc$ tale che $p(x)\neq 0$.
\end{proposition}
\begin{proof}
Se $p(x)=0$ per ogni $p\in\Pc$ allora $x\in B(0,r)$ per ogni $p\in\wt \Pc$ e per ogni $r>0$, quindi $x\in U$ per ogni $U\in\Uc_X$, ovvero
\[x\in \bigcap_{U\in\Uc_X}U=\ol{\cpa 0}.\]
\end{proof}


\section{Continuit\`a di operatori lineari in SVT}
\begin{proposition}[Continuit\`a mappe lineari]\label{PrContinuitaLineariSVT}
Sia $T:X\to Y$ lineare tra SVT. Valgono le seguenti affermazioni
\begin{enumerate}
    \item $T$ \`e continua se e solo se \`e continua in $0$
    \item $T$ \`e continua se e solo se per ogni $U\in \Uc_Y$ esiste $V\in \Uc_X$ tale che $T(V)\subseteq U$
    \item Se $X$ e $Y$ sono SVTLC con topologia indotta dalle famiglie di seminorme $\Pc$ e $\Qc$ rispettivamente, $T$ \`e continua se e solo se
    \[\forall q\in \Qc,\ \exists p_1,\cdots, p_n\in \Pc,\ \exists M\geq 0\quad\text{tali che}\]
    \[\forall x\in X,\ q(Tx)\leq M\max\cpa{p_1(x),\cdots,p_n(x)}\]
    \item Se $X$ e $Y$ sono SVTLC con topologia indotta dalle famiglie di seminorme $\Pc$ e $\Qc$ rispettivamente con $\Pc$ e $\Qc$ stabili per $\max$ allora $T$ \`e continua se e solo se $\forall q\in\Qc$ esistono $p\in\Pc$ e $M\geq 0$ tali che
    \[q(Tx)\leq Mp(x)\]
\end{enumerate}
\end{proposition}
\begin{proof}
Dimostriamo le affermazioni
\begin{enumerate}
    \item Basta traslare dato che traslare \`e un omeomorfismo.
    \item Ovvio.
    \item La condizione significa che la palla di centro $0$ e raggio 1 rispettivamente alla seminorma $\max(p_1,\cdots, p_n)$ di $X$ ha immagine tramite $T$ contenuta nella palla di raggio $M$ rispetto a $q$, concludendo per il punto 2. a meno di omotetia.
    \item Caso sopra.
\end{enumerate}
\end{proof}

\begin{proposition}[Caratterizzazione funzionali continui]\label{PrCaratterizzazioneFunzionaliContinui}
Sia $f\in X'_{alg}\nz$ con $X$ un $\K$-spazio vettoriale. Le seguenti affermazioni sono equivalenti
\begin{enumerate}
    \item $f$ \`e continua
    \item $\ker f$ \`e chiuso
    \item $\ker f$ non \`e denso
    \item $f$ non \`e surgettiva su un aperto non vuoto
    \item $f$ \`e limitata su un intorno di $0$
\end{enumerate}
\end{proposition}
\begin{proof}
    Diamo le implicazioni
\setlength{\leftmargini}{0cm}
\begin{itemize}
\item[$\boxed{1.\implies2.}$] Ovvio perch\'e $\cpa{0}$ \`e chiuso in $\K$.
\item[$\boxed{2.\implies3.}$] Se $\ker f$ \`e denso allora $\ol{\ker f}=X$ e quindi ha codimensione 0, ma $\ker f$ ha codimensione 1 in quanto $f\neq0$, quindi $\ker f\neq \ol{\ker f}$, cio\`e non \`e chiuso.
\item[$\boxed{3.\implies4.}$] Se $\ker f$ non \`e denso esiste un aperto non vuoto $A$ disgiunto da $\ker f$, cio\`e $0\notin f(A)$ e in particolare $f$ non \`e surgettiva su $A$.
\item[$\boxed{4.\implies5.}$] Se $f$ non \`e surgettiva su aperto non vuoto allora non lo \`e su un intorno bilanciato $U$ di $0$ e quindi $f(U)$ \`e un insieme bilanciato di $\K$ diverso da $\K$ in quanto $f\neq 0$, dunque $f(U)$ \`e un disco e in particolare \`e limitato.
\item[$\boxed{5.\implies1.}$] Se $\abs{fx}\leq M$ per ogni $x\in U\in\Uc_X$ allora per omogeneit\`a
\[\abs{f(x)}\leq \e\quad \forall x\in \frac\e M U\in \Uc_X\]
per un qualsiasi $\e>0$, quindi $f$ \`e continua in $0$. Questo conclude perch\'e
\[f(x)=f(x_0)+f(x-x_0).\]
\end{itemize}
\setlength{\leftmargini}{0.5cm}
\end{proof}

\section{SVT I-numerabili e paranorme}
\begin{definition}[Paranorma]
Una \textbf{paranorma} sull $\K$-spazio vettoriale $X$ \`e una funzione $q:X\to[0,\infty)$ tale che
\begin{enumerate}
    \item $q(x+y)\leq q(x)+q(y)$
    \item $q(\la x)\leq q(x)$ per ogni $x\in X$ e $\la\in\K$ tale che $\abs{\la}\leq 1$
    \item Se $\la_k\to 0$ in $\K$ allora $q(\la_k x)\to 0$
\end{enumerate}
Inoltre $q$ \`e \textbf{definita} se vale
\[q(x)=0\coimplies x=0.\]
\end{definition}

\begin{remark}
Dalla propriet\`a $2.$ segue che $q(\la x)=q(x)$ se $\abs\la=1$ e che $q(\la x)\leq q(\mu x)$ se $\abs\la\leq\abs\mu$. In particolare $q(x)=q(-x)$.

Quindi $d(x,y)=q(x-y)$ \`e una (semi)distanza su $X$ (distanza se $q$ definita).
\end{remark}

\begin{exercise}
Dimostrare che $(X,d)$ \`e uno SVT per $d$ indotta da paranorma $q$.
\end{exercise}

\begin{exercise}
Sia $X$ un $\K$-SVT I-numerabile. Allora la sua topologia proviene da una paranorma (la quale \`e definita sse $X$ \`e $T_0$).
\end{exercise}
\begin{proof}
TRACCIA
\begin{itemize}
    \item Sia $\cpa{U_n}_{n\geq 0}$ base numerabile di intorni bilanciati di $0$ tali che $U_{n+1}+U_{n+1}\subseteq U_n$.
    \item Estendiamo la successione per $n<0$ ponendo $U_k=U_{k+1}+U_{k+1}$ per ogni $k<0$.
    
    Nota che $U_{k+1}+U_{k+1}\subseteq U_k$ per ogni $k\in \Z$ e gli $\cpa{U_k}_{k\in\Z}$ sono intorni bilanciati. 
    \item Poniamo
    \[q(x)=\inf\cpa{\sum_{i=1}^r2^{-ki}\mid r\in\N,(k_1,\cdots, k_r)\in\Z^r\ t.c.\ x\in U_{k_1}+U_{k_2}+\cdots+U_{k_r}}\]
    Mostra che $q$ \`e una paranorma su $X$.
    \item Nota che $\cpa{q<2^{-n-1}}\subseteq U_n\subseteq \cpa{q\leq 2^{-n}}$ e quindi $q$ induce la topologia di $X$.
\end{itemize}
\end{proof}



\section{Topologie deboli}

\begin{proposition}[Topologia iniziale nel caso SVT]\label{PrTopologiaInizialeCasoSVT}
Sia $X$ uno spazio vettoriale su $\K$ e sia $\Fc:\cpa{T_i:X\to Y_i}$ dove ogni $Y_i$ \`e SVT e $T_i$ \`e lineare, allora la topologia iniziale su $X$ indotta\footnote{vedi (\ref{PrTopologiaInizialeEsiste})} da $\Fc$ rende $X$ uno SVT.
\end{proposition}
\begin{proof}
Voglio verificare che $+$ e $\cdot$ sono mappe continue per la topologia iniziale.
% https://q.uiver.app/#q=WzAsNCxbMCwwLCJYXFx0aW1lcyBYIl0sWzEsMCwiWCJdLFswLDEsIllfaVxcdGltZXMgWV9pIl0sWzEsMSwiWV9pIl0sWzEsMywiVF9pIl0sWzAsMiwiVF9pXFx0aW1lcyBUX2kiLDJdLFswLDEsIisiXSxbMiwzLCIrX2kiLDJdXQ==
\[\begin{tikzcd}
	{X\times X} & X \\
	{Y_i\times Y_i} & {Y_i}
	\arrow["{+}", from=1-1, to=1-2]
	\arrow["{T_i\times T_i}"', from=1-1, to=2-1]
	\arrow["{T_i}", from=1-2, to=2-2]
	\arrow["{+_i}"', from=2-1, to=2-2]
\end{tikzcd}\]
% https://q.uiver.app/#q=WzAsNCxbMCwwLCJcXEtcXHRpbWVzIFgiXSxbMSwwLCJYIl0sWzAsMSwiXFxLXFx0aW1lcyBZX2kiXSxbMSwxLCJZX2kiXSxbMSwzLCJUX2kiXSxbMCwyLCJpZF9cXEtcXHRpbWVzIFRfaSIsMl0sWzAsMSwiXFxjZG90Il0sWzIsMywiXFxjZG90X2kiLDJdXQ==
\[\begin{tikzcd}
	{\K\times X} & X \\
	{\K\times Y_i} & {Y_i}
	\arrow["\cdot", from=1-1, to=1-2]
	\arrow["{id_\K\times T_i}"', from=1-1, to=2-1]
	\arrow["{T_i}", from=1-2, to=2-2]
	\arrow["{\cdot_i}"', from=2-1, to=2-2]
\end{tikzcd}\]
Per la propriet\`a universale della topologia iniziale (\ref{PrProprietaUniversaleTopologiaIniziale}), vogliamo verificare che $T_i\circ +=+_i\circ (T_i\times T_i)$ \`e continua per ogni $i$ e similmente per $T_i\circ \cdot$. Questo \`e vero perch\'e la topologia iniziale \`e rende $T_i$ continua per ogni $i$.
\end{proof}

\begin{remark}
Se ogni $Y_i$ inoltre \`e SVTLC allora anche $X$ lo \`e.
\end{remark}


\begin{definition}[Topologie deboli]
Sia $X$ un $\K$-spazio vettoriale e $\Fc\subseteq X'$ (duale algebrico). La topologia iniziale indotta da $\Fc$ viene detta la \textbf{topologia debole di $\Fc$} e si indica $\sigma(X,\Fc)$.
\end{definition}

\begin{remark}
$\sigma(X,\Fc)=\sigma(X,\Span_\K(\Fc))$ quindi senza perdita di generalit\`a possiamo sempre supporre $\Fc$ sottospazio vettoriale di $X'$.
\end{remark}

\begin{remark}
La famiglia di seminorme associata a $\Fc$ (quella che induce la stessa topologia di $SVTLC$) \`e data da
\[\Pc=\cpa{\abs{f}\mid f\in\Fc}\]
\end{remark}

\begin{remark}
La topologia debole $\sigma(X,\Fc)$ \`e $T_0$ (e quindi Hausdorff perch\'e SVT) se e solo se la famiglia $\Fc$ \`e separante ($\forall x\in X\nz,\ \exists f\in\Fc$ tale che $f(x)\neq0$).
\end{remark}


\begin{lemma}[]\label{LmDualeAlgebricoIndipendenzaEContinuita}
Siano $f_0,\cdots,f_n\in X'_{alg}$ per $X$ un $\K$-spazio vettoriale, allora sono equivalenti
\begin{enumerate}
    \item $f_0=\sum_{i=1}^n\la_if_i$
    \item $\abs{f_0}\leq M\max_{i\in\cpa{1,\cdots, n}}\abs{f_i}$ per qualche $M\geq 0$
    \item $\ker f_0\supseteq \bigcap_{i=1}^n\ker f_i$
\end{enumerate}
\end{lemma}
\begin{proof}
    Diamo le tre implicazioni
    \setlength{\leftmargini}{0cm}
    \begin{itemize}
    \item[$\boxed{1.\implies2.}$] Da $1.$ segue $\abs{f_0}\leq\sum_{i=1}^n\abs{\la_i}\abs{f_i}\leq M\max\abs{f_i}$ per $M=\sum\abs{\la_i}$.
    \item[$\boxed{2.\implies3.}$] Se $x\in \bigcap\ker f_i$, cio\`e $\ps{f_i,x}=0$ per ogni $i$, allora $\ps{f_0,x}\leq M0=0$, cio\`e $f_0(x)=0$ e abbiamo l'inclusione voluta.
    \item[$\boxed{3.\implies1.}$] Sia $F:X\to\K^n$ data da $F=(f_1,\cdots, f_n)$, allora
    \[\ker F=\bigcap\ker f_i\subseteq \ker f_0\]
    quindi abbiamo una fattorizzazione
    % https://q.uiver.app/#q=WzAsMyxbMCwwLCJYIl0sWzAsMSwiXFxLXm4iXSxbMSwwLCJcXEsiXSxbMSwyLCJMIiwyLHsic3R5bGUiOnsiYm9keSI6eyJuYW1lIjoiZGFzaGVkIn19fV0sWzAsMSwiRiIsMl0sWzAsMiwiZl8wIl1d
\[\begin{tikzcd}
	X & \K \\
	{\K^n}
	\arrow["{f_0}", from=1-1, to=1-2]
	\arrow["F"', from=1-1, to=2-1]
	\arrow["L"', dashed, from=2-1, to=1-2]
\end{tikzcd}\]
    dove $L(x_1,\cdots, x_n)=\sum \la_i x_i$ per dei $\la_i$ (in quanto \`e una forma lineare). Ma allora $f_0=L\circ F=\sum \la_i f_i$ come voluto.
    \end{itemize}
    \setlength{\leftmargini}{0.5cm}
\end{proof}

\begin{proposition}[Duale per topologia debole]\label{PrDualePerTopologiaDebole}
Dato $X$ $\K$-spazio vettoriale e $\Fc$ sottospazio di $X'_{alg}$ allora
\[(X,\sigma(X,\Fc))^\ast=\Fc\]
\end{proposition}
\begin{proof}
Sia $f_0\in (X,\sigma(X,\Fc))^\ast$, allora per la proposizione (\ref{PrContinuitaLineariSVT}) esistono $f_1,\cdots, f_n\in\Fc$ e $M\geq0$ tali che per ogni $x\in X$
\[\abs{f_0(x)}\leq M\max_{i}\abs{f_i(x)}.\]
Dunque per il lemma (\ref{LmDualeAlgebricoIndipendenzaEContinuita}) $f_0$ si scrive come combinazione lineare delle $f_i$ e quindi in particolare $f_0\in\Fc$.

L'altra inclusione \`e ovvia per definizione di topologia debole.
\end{proof}

\begin{remark}
Se $X$ ha dimensione infinita, $\sigma(X,\Fc)$ non \`e mai localmente limitata. In particolare ogni intorno di $0$ contiene uno spazio vettoriale di codimensione finita.
\end{remark}
\begin{proof}
Se $U$ intorno di $0$ per $\sigma(X,\Fc)$ allora esistono $f_1,\cdots,f_n\in\Fc$ tali che\footnote{vedi lemma (\ref{LmDualeAlgebricoIndipendenzaEContinuita})}
\[U\supseteq \bigcap_{i=1}^n\cpa{\abs{f_i}<1}\supseteq \bigcap_{i=1}^n \ker f_i\]
e l'intersezione di questi nuclei ha codimensione al massimo $n$.
\end{proof}

\begin{proposition}[Duale di lineare continua \`e debole$^\ast$-continua]\label{PrDualeOperatoreContinuoEDeboleStarContinua}
Se $T:E\to F$ \`e un operatore lineare e continuo allora $T^\ast:F^\ast\to E^\ast$ \`e debole$^\ast$-continua.
\end{proposition}
\begin{proof}
Considera le opportune composizione e la definizione di topologia debole.
\end{proof}

\subsection{Caso degli spazi normati}

\begin{definition}[Topologia debole]
Se $X$ \`e normato, la \textbf{topologia debole} su $X$ \`e la topologia debole associata a $X^\ast$, cio\`e $\sigma(X,X^\ast)$.
\end{definition}

\begin{proposition}
La topologia debole \`e localmente convessa e Hausdorff.
\end{proposition}
\begin{proof}
Per Hahn-Banach (\ref{ThHahnBanach}), il duale $X^\ast$ separa i punti
\end{proof}


\begin{definition}[Topologia debole$^\ast$]
Su $X^\ast$ possiamo considerare la topologia debole associata alle valutazioni $X\subseteq X^{\ast\ast}$, cio\`e scegliendo
\[\Fc=\cpa{val_x\in (X^\ast)'\mid x\in X}.\]
Questa \`e la \textbf{topologia debole$^\ast$} su $X^\ast$ e la indichiamo $\sigma(X^\ast,X)$.
\end{definition}

\begin{remark}
La topologia debole$^\ast$ rende $X^\ast$ uno SVTLC $T_0$ (e quindi Hausdorff), infatti se $f\in X^\ast\nz$ allora esiste $x\in X$ tale che $f(x)\neq 0$.
\end{remark}

\begin{remark}
In generale $\sigma(X^\ast,X)$ \`e meno fine di $\sigma(X^\ast,X^{\ast\ast})$. Abbiamo uguaglianza solo quando $X=X^{\ast\ast}$ in quanto se $X\neq X^{\ast\ast}$ allora dalla proposizione (\ref{PrDualePerTopologiaDebole}) ricaviamo
\[(X^\ast,\sigma(X^\ast,X))^\ast=X\neq X^{\ast\ast}=(X^\ast,\sigma(X^\ast,X^{\ast\ast}))^\ast\]
e quindi in partenza $\sigma(X^\ast,X^{\ast\ast})\neq \sigma(X^\ast,X)$
\end{remark}

\begin{remark}
Poich\'e $(X,\normd)\inj (X^{\ast\ast},\normd)$ isometricamente allora $(X,\sigma(X,X^\ast))$ ha la topologia indotta come sottospazio da\footnote{nota che $X^\ast$ lo si pu\`o pensare come immerso in $X^{\ast\ast\ast}=(X^{\ast\ast})^\ast$, quindi stiamo considerando la topologia debole$^\ast$ su $(X^{\ast})^\ast$} $(X^{\ast\ast},\sigma(X^{\ast\ast},X^\ast))$.
\end{remark}
\begin{proof}
Questo deriva dalla transitivit\`a della topologia iniziale (\ref{PrTransitivitaTopologiaIniziale}) dove la prima famiglia \`e la mappa $X\inj X^{\ast\ast}$ e l'unica altra famiglia sono gli elementi di $X^\ast$ che vanno verso $\K$.
\end{proof}


\section{Teorema di Riesz}
\begin{theorem}[Riesz]\label{ThRiesz}
Per $X$ SVT $T_0$ su $\K$ sono equivalenti
\begin{enumerate}
    \item $X$ ha dimensione finita
    \item $X\cong \K^n$ per qualche $n\in\N$
    \item $X$ \`e localmente compatto
\end{enumerate}
\end{theorem}
\begin{proof}
    Diamo le implicazioni
\setlength{\leftmargini}{0cm}
\begin{itemize}
\item[$\boxed{1.\implies2.}$] Sia $X$ SVT $T_0$ di dimensione $n$ e sia $x_1,\cdots, x_n$ una sua base di Hamel. Allora
\[\vp:\funcDef{\K^n}{X}{\la=(\la_1,\cdots,\la_n)}{\sum_{i=1}^n}\la_i x_i\]
\`e lineare, bigettiva e continua. 

Dimostriamo che \`e aperta: L'insieme $\del B(0,1)\subseteq \K^n$ visto con la norma euclidea \`e compatto, quindi $\vp(\del B(0,1))$ \`e compatto, e quindi chiuso perch\'e $X$ \`e Hausdorff. Per bigettivit\`a $0\notin \vp(\del B(0,1))$, quindi esiste un intorno $V$ di $0$ in $X$ disgiunto da $\vp(\del B(0,1))$. Senza perdita di generalit\`a $V$ bilanciato, allora $\vp\ii(V)$ \`e un insieme bilanciato di $\K^n$ disgiunto da $\del B(0,1)$, dunque $\vp\ii(V)\subseteq B(0,1)$ (se avesse un punto di modulo maggiore a $1$ allora in quanto bilanciato conterrebbe tutti i punti tra esso e $0$, intersecando il bordo). 

Questo mostra che $B(0,1)$ \`e un intorno di $0$ e quindi $\vp$ \`e aperta (per traslazione e omotetia $\vp(B(\la,r))$ \`e intorno di $\vp(\la)$ per ogni $\la\in\K^n$ e $r>0$ e concludo notando che aperti di $\K^n$ sono dati da unioni di palle).
\item[$\boxed{2.\implies3.}$] $\K^n$ \`e localmente compatto perch\'e conosciamo la topologia euclidea, quindi anche $X$ lo \`e.
\item[$\boxed{3.\implies1.}$] Sia $X$ SVT localmente compatto e $T_0$. Mostriamo che $X$ \`e I-numerabile:

Sia $V$ intorno compatto di $0$. Mostriamo che $\cpa{\frac1n V}$ \`e una base di intorni di $0$. Sia $U$ un intorno (senza perdita di generalit\`a $U$ bilanciato). Poich\'e $V$ \`e compatto e\footnote{$U$ assorbente} $V\subseteq \bigcup_{n\geq 1}nU=X$ possiamo estrarre un sottoricoprimento finito
\[V\subseteq \bigcup_{1\leq i\leq k}n_i U\pasgnl={$U$ bilanciato}\pa{\max_{1\leq i\leq k}n_i}U\]
infatti $\dfrac{n_i}{\max n_i}U\subseteq U$. Questo mostra che $\cpa{\frac1n V}$ \`e una base numerabile di intorni di $0\in X$.


Notiamo che $V$ si pu\`o coprire con un numero finito di traslati di $\frac12 V$ in quanto $V\subseteq V+\frac12V$ e applico compattezza al variare di $v+\frac12V$ per $v\in V$. Sia allora $F$ tale che $V\subseteq\bigcup_{v\in F}v+\frac12V$ con $F$ finito e poniamo $Y=\Span_\K F$. Notiamo che $Y$ ha dimensione finita.

Procedendo per induzione, per ogni $n\in\N$ si ha $V\subseteq Y+2^{-n}V$, ma $\cpa{2^{-n}V}_{n\geq0}$ \`e una base di intorni, quindi 
\[\ol Y=\bigcap_{n\geq 0}Y+2^{-n}V\supseteq V\]
e dato che $V$ \`e un intorno assorbente, $X=\bigcup_{n\geq 0}nV\subseteq \ol Y$, cio\`e $Y$ \`e denso in $X$.

Poich\'e $Y$ ha dimensione finita, per l'implicazione precendente $Y\cong \K^n$, in particolare $Y$ \`e completo. Se $x\in X=\ol Y$, poich\'e $X$ \`e I-numerabile, si ha che esiste $y_k\to x$ in $X$ con $y_k\in Y$ con $(y_k)$ di Cauchy in $X$ e quindi anche in $Y$, che per\`o \`e completo, quindi $y_k\to y$ per $y\in Y$, ma $X$ \`e Hausdorff, quindi $y=x$.
\end{itemize}
\setlength{\leftmargini}{0.5cm}
\end{proof}

\begin{remark}
Se non avessimo supposto $T_0$ potremmo considerare $\quot X{\ol{\cpa0}}$ e troveremmo $X\cong \K^n\oplus \ol{\cpa0}$.
\end{remark}


\section{Successioni generalizzate (nets)}
\begin{definition}[Net]
Un \textbf{net} su un insieme $X$ \`e una funzione $f:D\to X$ su $(D,\geq)$ poset diretto\footnote{diretto nel senso che per ogni $i,j\in D$ esiste $k\in D$ tale che $i\leq k$ e $j\leq k$.}.
\end{definition}

\begin{example}[Somme di Riemann]
Sia $u:[a,b]\to X$ una funzione con $X$ SVT. La \textbf{somma di Riemann} per $u$ relativa ad una suddivisione $P=\cpa{a=t_0<t_1<\cdots<t_n=b}$ e una scelta di punti $\Xi=\cpa{\xi_1,\cdots, \xi_n}$ con $\xi_i\in [t_{i-1},t_i]$ \`e
\[S(u;P,\Xi)=\sum_{i=1}^nu(\xi_i)(t_i-t_{i-1}).\]
Possiamo prendere $D=\cpa{(P,\Xi)}$ l'insieme delle possibili partizioni e scelte di punti. $D$ \`e un poset: $(P,\Xi)\geq (P',\Xi')$ se $P\supseteq P'$.

In questo contesto l'integrale di Riemann sarebbe il limite rispetto al net $D\to X$ dato da $(P,\Xi)\mapsto S(u;P,\Xi)$.
\end{example}

\begin{example}[Somme infinite]
Data $\cpa{x_i}_{i\in I}\subseteq X$ con $X$ SVT consideriamo
\[S:\funcDef{\Ps_{fin}(I)}{X}{F}{\sum_{i\in F}x_i}\]
$\Ps_{fin}(I)$ \`e parzialmente ordinato per inclusione e la somma sarebbe il limite.
\end{example}

\begin{definition}[Definitivamente e frequentemente]
Diciamo che se $\cpa{P_\al}_{\al\in D}$ sono propriet\`a indicizzate su $D$ insieme diretto allora \textbf{$P_\al$ vale definivamente} (risp. \textbf{frequentemente}) se esiste $\al\in D$ tale che per ogni $\beta\geq \al$ in $D$ vale $P_\beta$ (risp. per ogni $\al\in D$ esiste $\beta\in D$ tale che vale $P_\beta$).
\end{definition}
\begin{remark}
Se $D\neq \N$ allora pu\`o succedere che ``frequentemente"$\neq$``infinite volte".
\end{remark}

\begin{definition}[Convergenza per net]
Se $f:D\to X$ \`e un net su $X$ spazio topologico si ha che $f$ \textbf{converge a $x\in X$} se per ogni $U$ intorno di $x$ si ha che $f(i)\in U$ definitivamente.
\end{definition}

\begin{definition}[Punti di accumulazione per net]
Se $f:D\to X$ \`e un net su $X$ spazio topologico si ha che $x$ \`e un \textbf{punto di accumulazione} di $f$ se per ogni $U$ intorno di $x$, $f(i)\in U$ frequentemente.
\end{definition}

\begin{definition}[Sottonet]
Una $\vp:D'\to D$ con $D,D'$ insiemi diretti tale che per ogni $i\in D$ esiste $i'\in D'$ tale che $\vp(j)\geq i$ per ogni $j\geq i'$ \`e detta \textbf{cofinale}.

Sia $f:D\to X$ un net, allora $f\circ\vp:D'\to X$ per $\vp$ cofinale \`e un \textbf{sottonet} di $f$.
\end{definition}
\begin{remark}
Una successione \`e un net su $\N$, una sottosuccessione \`e quindi in particolare un sottonet, ma non tutti i sottonet di una successione sono sottosuccessioni.
\end{remark}

\begin{exercise}
Se $f:D\to X$ spazio topologico e $x\in X$ allora $x$ \`e aderente a $f$ se e solo se $x$ \`e limite di qualche sottonet di $f$.
\end{exercise}

\begin{remark}
Dato $f:D\to X$ net, l'insieme $A$ dei punti aderenti a $f$ \`e
\[A=\bigcap_{j\in D} \ol{\cpa{f(i)\mid i\geq j}}\]
infatti $x$ \`e aderente se e sono se per ogni intorno $U$ e ogni $j\in D$ esiste $i\geq j$ tale che $f(i)\in U$, cio\`e per ogni $j\in D$ $U\cap\cpa{f(i)\mid i\geq j}\neq \emptyset$, ovvero per ogni $j\in D$ si ha $x\in \ol{f(i)\mid i\geq j}$.
\end{remark}

\begin{exercise}
$X$ spazio topologico \`e compatto per ricoprimenti se e solo se ogni net in $X$ ha punti aderenti, cio\`e se e solo se per ogni net su $X$ esiste un sottonet convergente.
\end{exercise}

\begin{exercise}
Usare l'esercizio sopra per dimostrare Tychonoff.
\end{exercise}
\begin{proof}
IDEA:
\begin{itemize}
	\item Sia $f:D\to \prod_{\la\in \Lambda}X_\la$ un net, vogliamo trovare dei punti aderenti.
	\item Consideriamo l'insieme
	\[S=\cpa{(N,x)\mid x\in \prod_{\la\in N}X_\la,\ N\subseteq \Lambda,\ x\text{ aderente per }P_N\circ f:D\to \prod_{\la\in N}X_\la}\]
	esso \`e non vuoto perch\'e se $N$ \`e un singoletto allora $P_N\circ f$ \`e un net verso uno spazio compatto, quindi ha un punto aderente. Ordiniamo $S$ ponendo $(N,x)\leq (N',x')$ se $N\subseteq N'$ e $P_N(x')=x$.

	Vale la condizione della catena, infatti se $\cpa{(N_\al,x_\al)}$ \`e una catena ascendente in $S$ allora basta considerare $N=\bigcup N_\al$ e $x\in \prod_{\la\in N} X_\la$ dato da $x(\la)=x_\al(\la)$ per un qualche $\al$ tale che $\la\in N_\al$. Notiamo che $x$ cos\`i definito \`e aderente a $P_N\circ f$ perch\'e gli $x_\al$ sono aderenti e questo basta per la definizione di topologia prodotto.

	Dunque per il lemma di Zorn esiste un dominio massimale $(N,x)$
	\item Se per assurdo $N\neq \Lambda$ allora esiste $\la\in \Lambda\bs N$, ma allora possiamo estendere $(N,x)$ a $(N\cup \cpa\la,\wt x)$ per $\wt x=x$ fuori $\la$ e uguale a un qualche aderente a $P_{\cpa{\la}}\circ f$ in $\la$. Questo nega la massimalit\`a.
\end{itemize}
\end{proof}

\begin{exercise}
Per $X$ spazio topologico e $S\subseteq X$ si ha $x\in \ol S$ se e solo se esiste $f:D\to S$ net convergente a $x$.
\end{exercise}

\begin{definition}[Net di Cauchy]
Sia $X$ SVT. Un net $f:D\to X$ \`e \textbf{di Cauchy} se per ogni $U\in\Uc_X$ esiste $i\in D$ tale che per ogni $p\geq i$ e $q\geq i$ vale $f(p)-f(q)\in U$.

Equivalentemente il net $\wt f:D\times D\to X$ definito da $\wt f(i,j)=f(i)-f(j)$ con $(i,j)\geq(i',j')\coimplies i\geq i'\wedge j\geq j'$ converge a $0$.
\end{definition}

\begin{definition}[Completo per nets]
Uno SVT \`e \textbf{completo per nets} se ogni net di Cauchy converge.
\end{definition}

\begin{exercise}
Uno SVT I-numerabile \`e completo per nets se e solo se \`e completo per successioni.
\end{exercise}
\chapter{Limitatezza e Banach-Steinhaus}

\section{Limitatezza}
\begin{definition}[Insieme limitato]
Un sottoinsieme $S$ di uno SVT $X$ con $\Uc$ intorni di $0$ \`e \textbf{limitato} se \`e assorbito da ogni elemento di $\Uc$, cio\`e\footnote{questa condizione \`e equivalente a chiedere $tU\supseteq S$ per ogni $t$ con $\abs t\geq n$ o a chiedere che l'assorbimento valga per elementi di una pre-base di intorni di $0$ al posto di tutti gli elementi di $\Uc$.} per ogni $U\in\Uc$ esiste $n\in\N$ tale che $nU\supseteq S$.
\end{definition}
\begin{remark}
Valgono le seguenti propriet\`a
\begin{enumerate}
    \item Se $S$ \`e limitato allora anche $\ol S$ lo \`e, basta considerare intorni chiusi.
    \item Se $S$ e $S'$ sono limitati, $S\cup S'$ lo \`e.
    \item Ogni compatto \`e limitato, basta scegliere un intorno limitato di $x$ per ogni $x\in K$ e poi estrarre un sottoricoprimento finito. Un tale intorno esiste scalando intorni di $0$ bilanciati.
    \item Ogni $T:X\to Y$ lineare e continua tra SVT \`e limitata, cio\`e per ogni $S\subseteq X$ limitato, $T(S)$ \`e limitato. In generale non vale il viceversa ma vale se $X$ e $Y$ sono normati.
\end{enumerate}
\end{remark}

\begin{proposition}[Limitatezza in SVTLC]\label{PrLimitatezzaInSVTLC}
Se $(X,\Pc)$ \`e SVTLC allora $S\subseteq X$ \`e limitato se e solo se per ogni seminorma $p\in \Pc$, $p$ \`e limitata su $S$.
\end{proposition}
\begin{proof}
$p$ limitata su $S$ significa che 
\[S\subseteq B_p(0,R_p)=\frac{R_p}\e B_p(0,\e)\]
e le palle $\cpa{B_p(0,\e)}_{p\in\Pc, \e>0}$ sono una prebase di intorni di $0\in X$.
\end{proof}
\begin{corollary}
Se $(X,\normd)$ \`e normato allora $S$ \`e limitato se e solo se $\exists R>0$ tale che $S\subseteq B(0,R)$.
\end{corollary}

\begin{exercise}
Se $X$ \`e I-numerabile e $T:X\to Y$ lineare tale che per ogni $x_k\to 0$ in $X$ esiste $x_{k_j}$ tale che $T(x_{k_j})$ limitata allora $T$ \`e continua.
\end{exercise}

\begin{proposition}[Caratterizzazione sequenziale della limitatezza]\label{PrCaratterizzazioneSequenzialeLimitatezza}
Se $X$ SVT e $S\subseteq X$, $S$ \`e limitato se e solo se per ogni $(s_k)$ successione in $S$ e per ogni $(\al_k)$ successione in $\K$ infinitesima, si ha $\al_k s_k\to 0$.
\end{proposition}
\begin{proof}
Sia $S$ limitato, $(s_k)$ successione in $S$ e $(\al_k)$ successione infinitesima in $\K$. Sia $U$ intorno bilanciato di $0$ e sia $n$ tale che $S\subseteq nU$. Notiamo che definitivamente $\abs{\al_k}<\frac1n$, quindi
\[\al_k s_k\in \al_k S\subseteq \al_k nU\pasgnl\subseteq{k grande} U.\]

Supponiamo ora $S$ non limitato, allora esiste $U\in\Uc_X$ che non assorbe $S$, cio\`e per ogni $n\in\N$ esiste $s_n\in S\bs nU$. Dunque $(s_n)$ \`e una successione in $S$ tale che $\frac1n s_n\notin U$ per costruzione, dunque $\frac1n s_n$ non tende a $0\in X$ nonostante $\frac1n$ sia infinitesima.
\end{proof}

\begin{proposition}\label{PrSuccessioniCauchyLimitate}
Le successioni di Cauchy sono limitate.
\end{proposition}
\begin{proof}
Sia $(x_k)$ una successione di Cauchy in $X$, cio\`e per ogni $U\in \Uc_X$ esiste $n\in\N$ tale che per ogni $p,q\geq n$ si ha $x_p-x_q\in U$.

Fissiamo $U\in \Uc_X$ e sia $V$ bilanciato tale che $V+V\subseteq U$. Per la definizione di successione di Cauchy esiste $n_0$ tale che $x_k-x_{n_0}\in V$ per ogni $k\geq n_0$, cio\`e $x_k\in x_{n_0}+V$.

Inoltre, esiste $m$ tale che $x_k\in mV$ per ogni $k\leq n_0$ dato che un insieme finito \`e limitato. Allora per ogni $k\in\N$ si ha $x_k\in mV+V$, infatti se $k\leq n_0$ allora abbiamo $mV$, se $k>n_0$ allora $x_{n_0}\in mV$ e $x_k\in x_{n_0}+V\subseteq mV+V$.

Poich\'e $V$ \`e bilanciato, $mV+V\subseteq mV+mV=m(V+V)\subseteq mU$.
\end{proof}

\section{Spazi di Baire e II-categoria}
\begin{theorem}[Baire]\label{ThBaire}
Se $\cpa{A_k}_{k\geq 0}$ \`e una famiglia numerabile di aperti densi di uno spazio metrico completo allora $\bigcap A_k$ \`e denso.
\end{theorem}
\begin{proof}
Per induzione si definisce una successione di palle chiuse di $X$ dove $B_0$ \`e arbitraria e
\[B_k=\ol{B(x_k,r_k)}\text{ tali che }B_{k+1}\subseteq B_k\cap A_k\text{ e }r_k=o(1)\]
che possiamo fare perch\'e $A_k$ \`e un aperto denso.

Allora la successione dei centri \`e una successione di Cauchy, infatti se $p,q\geq n$ si ha $x_p,x_q\in B_n$ e quindi $d(x_p,x_q)\leq 2r_n$. Dunque $x_n\to x^\ast$ in $X$ per completezza. Inoltre, poich\'e $x_k\in B_n$ definitivamente, $x^\ast=\lim x_k\in B_n$ per ogni $n$ (dato che $B_n$ \`e chiuso). In particolare $x^\ast\in B_{n+1}\subseteq A_n$ per ogni $n$ e quindi $x^\ast\in \bigcap A_n$. Per costruzione $x^\ast\in B_0$, quindi per ogni palla $B_0$ abbiamo mostrato che $B_0\cap \bigcap A_n\neq \emptyset$, cio\`e $\bigcap A_n$ \`e denso.
\end{proof}

\begin{exercise}
La stessa conclusione vale se $X$ \`e localmente compatto al posto di metrico completo.
\end{exercise}

\begin{definition}[Spazio di Baire]
Uno spazio topologico \`e \textbf{di Baire} se ogni intersezione numerabile di aperti densi \`e densa.
\end{definition}

\begin{remark}
Ogni aperto non vuoto di $X$ di Baire \`e ancora di Baire. Basta verificare che ogni aperto denso di $A$ \`e della forma $A\cap U$ con $U$ aperto denso di $X$.
\end{remark}


\begin{definition}[Sottoinsieme di I- e II-categoria]
Un sottoinsieme $S$ di $X$ \`e di \textbf{I-categoria (di Baire) in $X$} se \`e unione numerabile di insiemi $(E_i)_{i\in\N}$ con $int(\ol{E_i})=\emptyset$.

Inoltre $S$ \`e di \textbf{II-categoria (di Baire) in $X$} se non \`e di I-categoria.
\end{definition}

\begin{remark}
Se $X$ \`e di Baire e $S\subseteq X$ \`e di I-categoria allora $X\bs S$ \`e di II-categoria in quanto $X$ stesso \`e di II-categoria (se $X=\bigcup E_i$ con $E_i$ chiusi a parte interna vuota allora $\emptyset=\bigcap E_i^c$ con $E_i^c$ aperti densi, ma questo \`e assurdo perch\'e $X$ di Baire).
\end{remark}

\section{Teorema di Banach-Steinhaus}

\begin{definition}[Famiglia equicontinua]
Una famiglia $\Gamma$ di operatori lineari continui fra SVT $X$ e $Y$ \`e \textbf{equicontinua} se per ogni $U\in\Uc_Y$ esiste $V\in \Uc_X$ tale che per ogni $T\in\Gamma$, $T(V)\subseteq U$.
\end{definition}

\begin{remark}
Possiamo riformulare la condizione nei seguenti modi: per ogni $U\in\Uc_Y$ esiste $V\in\Uc_X$ tale che
\[\forall T\in\Gamma,\ V\subseteq T\ii(U)\coimplies V\subseteq \bigcap_{T\in \Gamma}T\ii(U)\doteqdot \Gamma\ii(U).\]
Equivalentemente la condizione predica che per ogni $V\in\Uc_Y$ si abbia $\Gamma\ii(V)\in\Uc_X$.
\end{remark}

\begin{remark}
Se $T:X\to Y$ fra spazi normati, la norma degli operatori
\[\norm T=\norm T_{\infty,B(0,1)}=\text{migliore costante di Lipschitz per }T.\]
\end{remark}

\begin{example}
Se $X$ e $Y$ sono normati, $\Gamma$ \`e equicontinua se e solo se $\Gamma$ \`e limitato in $L(X,Y)$ rispetto alla norma degli operatori.
\end{example}

\begin{theorem}[Banach-Steinhaus / Uniforme limitatezza]\label{ThBanachSteinhausUniformeLimitatezza}
Siano $X,Y$ SVT, $S\subseteq X$ di seconda categoria e $\Gamma\subseteq L(X,Y)$ con $\Gamma$ puntualmente limitata su $S\subseteq X$, cio\`e per ogni $s\in S$, $\Gamma(s)=\bigcup_{T\in \Gamma} T(s)$ \`e limitato in $Y$.

Allora $\Gamma$ \`e equicontinua.
\end{theorem}
\begin{proof}
Sia $U\in\Uc_Y$ e consideriamo $V\in\Uc_Y$ chiuso tale che $V-V\subseteq U$. Per ipotesi, per ogni $x\in S$ si ha che $\Gamma(x)$ \`e limitato in $Y$, quindi viene assorbito da $V$, cio\`e esiste $n_x\in\N$ tale che per ogni $T\in\Gamma$ si ha $T(x)\in n_x V$, cio\`e tale che
\[x\in\bigcap_{T\in \Gamma}n_xT\ii(V)=n_x\bigcap_{T\in \Gamma}T\ii(V)=n_x\Gamma\ii(V).\]
Dunque $S\subseteq \bigcup_{n\in\N}n\Gamma\ii(V)$. Notiamo che poich\'e $V$ \`e chiuso, $T\ii(V)$ \`e chiuso e quindi anche $\Gamma\ii(V)$ lo \`e perch\'e intersezione di chiusi. Poich\'e $S$ \`e di seconda categoria anche l'unione delle versioni riscalate di $\Gamma\ii(V)$ lo \`e, dunque questo insieme non \`e unione numerabile di chiusi con parte interna vuota, quindi almeno uno tra gli $n\Gamma\ii(V)$ ha parte interna non vuota, quindi anche $\Gamma\ii(V)$ ha parte interna non vuota scalando per $\frac1n$.

Quindi $\Gamma\ii(V)$ \`e intorno di qualche suo punto, dunque\footnote{se $a_0\in int(A)$ allora $A-a_0\subseteq A-A$ \`e un intorno di $0$.} $\Gamma\ii(V)-\Gamma\ii(V)$ \`e un intorno di $0$. 

Ricordando che $V-V\subseteq U$ si ha
\[T\ii(U\supseteq T\ii(V-V)=T\ii(V)-T\ii(V))\supseteq \Gamma\ii(V)-\Gamma\ii(V)\]
quindi passando all'intersezione su $T\in\Gamma$ si ha
\[\Gamma\ii(U)\supseteq \Gamma\ii(V)-\Gamma\ii(V)\in\Uc_X,\]
cio\`e abbiamo mostrato che per ogni $U\in\Uc_Y$ si ha $\Gamma\ii(U)\in\Uc_X$, che \`e equivalente all'equicontinuit\`a di $\Gamma$.
\end{proof}

\begin{corollary}[Sottoinsiemi limitati di operatori]\label{CorPuntualmenteLimitatoImplicaLimitatoPerNormaOperatore}
    Se $X$ e $Y$ sono Banach e $\Gamma\subseteq L(X,Y)$ \`e puntualmente limitata in $X$ (o volendo anche un sottoinseme di $X$ di II-categoria) allora $\Gamma$ \`e un insieme limitato in $L(X,Y)$.
\end{corollary}
\begin{proof}
Diretta applicazione di Banach-Stenhaus (\ref{ThBanachSteinhausUniformeLimitatezza}) notando che spazi di Banach sono in particolare SVT e che equicontinuit\`a per la norma su $L(X,Y)$ significa limitatezza. **************
\end{proof}

\begin{exercise}
Siano $X,Y$ SVT. Trovare la topologia meno fine $\tau$ di SVT su $L(X,Y)$ per la quale
\[\Gamma\text{ puntualmente limitato in $L(X,Y)$}\coimplies \Gamma\text{ limitato nella topologia $\tau$}.\]
\end{exercise}


\begin{corollary}\label{CorConvergenzaLineariSePuntualmenteConvergente}
Siano $X$ e $Y$ Banach e sia $(T_n)\subseteq L(X,Y)$ puntualmente convergente. Allora il limite $T$ \`e ancora lineare, continuo e con norma
\[\norm T\leq \liminf_{n\to\infty}\norm{T_n}.\]
\end{corollary}
\begin{proof}
Per il corollario precedente (\ref{CorPuntualmenteLimitatoImplicaLimitatoPerNormaOperatore}) si ha che $(T_n)$ sono limitati in $\normd$ e il limite puntuale \`e lineare in quanto
\[T_n(\al x+\beta y)=\al T_n(x)+\beta T_n(y)\to \al T(x)+\beta T(y).\]
Questo mostra che $T$ \`e limitato e lineare, quindi $T\in L(X,Y)$.

Inoltre per ogni $x\in X$ si ha
\[\norm{T(x)}=\lim_{n}\norm{T_n(x)}\leq \pa{\sup_n\norm{T_n}}\norm x,\]
quindi $\norm T\leq \sup_n\norm{T_n}$. Ragionando analogamente per una sottosuccessione di $(T_n)$ che in norma converge a $\liminf_n\norm{T_n}$ ricaviamo
\[\norm T\leq \liminf_n\norm{T_n}.\]
\end{proof}

\begin{remark}
In generale NON vale $T_n\to T$ in $\normd$.
\end{remark}

\begin{proposition}[Bilineare separatamente continua \`e continua]\label{PrBilineareSeparatamenteContinuaEContinua}
Sia $b:X\times Y\to Z$ bilineare e separatamente continua, cio\`e per ogni $x\in X, y\in Y$ si ha che $b(x,\cdot):Y\to Z$ e $b(\cdot, y):X\to Z$ sono lineari e continue. Allora $b$ \`e continua, cio\`e
\[\sup_{\norm x\leq 1, \norm y\leq 1}\norm{b(x,y)}<\infty.\]
\end{proposition}
\begin{proof}
Consideriamo la famiglia
\[\Gamma=\cpa{b(x,\cdot):Y\to Z}_{x\in X,\ \norm x\leq 1}\subseteq L(Y,Z).\]
Per ipotesi $\Gamma$ \`e puntualmente limitata in $Y$, infatti per ogni $y\in Y$
\[\sup_{b(x,\cdot)\in\Gamma} \norm{b(x,\cdot)}_{L(Y,Z)}=\sup_{\norm x\leq 1}\norm{b(x,y)}_Z= \norm{b(\cdot,y)}_{L(X,Z)}\norm y<\infty\]
Allora $\Gamma$ \`e limitata in $\normd_{L(Y,Z)}$, cio\`e per ogni $x\in X$ tale che $\norm x\leq 1$ si ha
\[\norm{b(x,y)}_Z\leq M\norm y\]
e quindi al variare di $y$ con $\norm y\leq 1$ troviamo $\norm b_{L(X\times Y,Z)}\leq M$.
\end{proof}

\begin{exercise}
Esiste una isometria lineare
\[\funcDef{L(X,L(Y,Z))}{L^2(X\times Y,Z)}{T}{(x,y)\mapsto T(x)(y)}\]
dove $L^2(X\times Y,Z)$ sono le bilineari.
\end{exercise}

\begin{proposition}[w$^\ast$-limitato vs limitato in $\normd_{X^\ast}$]\label{PrLimitatoInDeboleStarEquivaleLimitatoInNormaDuale}
Sia $Y=\K$ e $X$ Banach. Sia $\Gamma\subseteq X^\ast$, allora $\Gamma$ \`e w$^\ast$-limitato se e solo se \`e limitato in $\normd_{X^\ast}$.
\end{proposition}
\begin{proof}
Essere limitato nella topologia debole$^\ast$ significa ``essere assorbito da ogni intorno w$^\ast$ di $X^\ast$" cio\`e, usando intorni di prebase, essere assorbiti da insiemi della forma
\[\cpa{f\in X^\ast\mid \abs{f(x)}<1}\]
per $x\in X$. Notiamo che $\Gamma$ viene assorbito da $\cpa{f\in X^\ast\mid \abs{f(x)}<1}$ significa $\Gamma(x)$ limitato in $\K$. Per il corollario (\ref{CorPuntualmenteLimitatoImplicaLimitatoPerNormaOperatore}) si ha che $\Gamma$ \`e limitato in $L(X,\K)=X^\ast$.

L'altra implicazione \`e ovvia perch\'e la norma operatore gi\`a rende continui gli operatori e indebolire la topologia non pu\`o trasformare un insieme limitato in uno non limitato.
\end{proof}

\begin{remark}
Se $E\subseteq F$ \`e un sottospazio allora $\Gamma\subseteq E$ \`e limitato in $F$ se e solo se \`e limitato in $E$ per la topologia indotta.
\end{remark}

\begin{proposition}
Sia $\Gamma\subseteq X$, allora $\Gamma$ \`e w-limitato se e solo se \`e $\normd$-limitato.
\end{proposition}
\begin{proof}
Se $\Gamma$ \`e $\sigma(X,X^\ast)$-limitato allora tramite l'immersione isometrica $X\to X^{\ast\ast}$ troviamo un insieme $\sigma(X^{\ast\ast},X^\ast)$-limitato. A questo punto basta applicare la proposizione precedente (\ref{PrLimitatoInDeboleStarEquivaleLimitatoInNormaDuale}).
\end{proof}

\subsection{Teorema della mappa aperta}
\begin{theorem}[Mappa aperta]\label{ThMappaAperta}
Siano $X,Y$ Banach e $T:X\to Y$ lineare continuo e tale che $T(X)$ \`e di II-categoria in $Y$ (per esempio $T$ surgettivo). Allora $T$ \`e una mappa aperta.
\end{theorem}
\begin{proof}
Sia $B$ la palla unitaria chiusa di $X$. Basta mostrare che $T(B)$ \`e un intorno di $0$ in $Y$ (per omotetia e traslazione seguir\`a che $T$ manda intorni di $x$ in intorni di $T(x)$, cio\`e \`e aperta). Notiamo che 
\[X=\bigcup_n nB\implies T(X)=\bigcup_n nT(B)\]
Per ipotesi $T(X)$ \`e di II-categoria in $Y$, quindi per qualche $n$ si ha che $\ol{n T(B)}$ ha parte interna non vuota e quindi $\ol{T(B)}$ stesso ha parte interna non vuota. Poich\'e\footnote{ricorda che in generale se $f$ \`e continua allora $f(\ol A)\subseteq \ol{f(A)}$. 

In questo caso la mappa \`e $(x,y)\mapsto x-y$ e usiamo il fatto che $T$ \`e lineare e \[\ol{T(B)\times T(B)}=\ol{T(B)}\times \ol{T(B)}.\]} 
\[\ol{T(B)}-\ol{T(B)}\subseteq \ol{T(B-B)}=\ol{T(2B)}=2\ol{T(B)}\]
si ha che $\ol{T(B)}$ \`e un intorno di $0\in Y$.


Mostriamo ora che $T(B)$ stesso \`e un intorno di $0$. Poich\'e la chiusura \`e l'intersezione degli aperti che contengono $T(B)$ si ha in particolare che
\[\ol{T(B)}=T(B)+\frac12\ol{T(B)}.\]
Se $y_0\in\ol{T(B)}$ allora esistono $x_0\in B$ e $y_1\in \frac12\ol{T(B)}$ tali che $y_0=T(x_0)+\frac12 y_1$.

Iterando troviamo $y_2\in \ol{T(B)}$ tale che $y_1=T(x_1)+\frac12 y_1$ per $x_1\in B$. Questo definisce due successioni $(x_n)\subseteq B$ e $(y_n)\subseteq \ol{T(B)}$ tali che $y_n=T(x_n)+\frac12 y_{n+1}$, quindi
\[y_0=T(x_0)+\frac12 T(x_1)+\cdots+2^{-n}T(x_n)+2^{-n-1}y_{n+1}=T\pa{\sum_{i=0}^n2^{-i}x_i}+2^{-n-1}y_{n+1}.\]
Poich\'e $X$ \`e completo la serie $\sum_{i=0}^n2^{-i}x_i$ converge as un punto $x^\ast\in 2B$ (perch\'e assolutamente convergente e $\sum_{n\geq 0} 2^{-n}=2$). 

Siccome $T$ \`e continua, $T(B)$ \`e limitato e quindi $\ol{T(B)}$ \`e limitato, quindi 
\[\norm{2^{-n-1}y_{n+1}}\leq 2^{-n-1}\norm T\to 0,\]
quindi (per continuit\`a di $T$) si ha $y_0=T(x^\ast)$, cio\`e
\[\ol{T(B)}\subseteq 2T(B),\]
in particolare $T(B)$ \`e un intorno di $0$ per omotetia.
\end{proof}

\begin{remark}[Lineare continuo allora omeo se e solo se bigettivo]
Un operatore lineare continuo \`e un omeomorfismo se e solo se \`e bigettivo. Questo \`e immediato da mappa aperta (\ref{ThMappaAperta}).
\end{remark}

\begin{remark}
Se $T:X\to Y$ lineare continuo allora induce
% https://q.uiver.app/#q=WzAsMyxbMCwwLCJYIl0sWzEsMCwiWSJdLFswLDEsIlgvXFxrZXIgVCJdLFswLDIsIlxccGkiLDJdLFswLDEsIlQiXSxbMiwxLCJcXHd0IFQiLDJdXQ==
\[\begin{tikzcd}
	X & Y \\
	{X/\ker T}
	\arrow["T", from=1-1, to=1-2]
	\arrow["\pi"', from=1-1, to=2-1]
	\arrow["{\wt T}"', from=2-1, to=1-2]
\end{tikzcd}\]
con $\wt T$ lineare continua iniettiva. Se $T$ \`e surgettiva allora per il teorema della mappa aperta (\ref{ThMappaAperta}) $\wt T$ \`e un omeomorfismo lineare.
\end{remark}

\begin{remark}
Se $T:X\to Y$ \`e lineare e continua allora
\[\text{aperta $\coimplies$ surgettiva $\coimplies$ identificazione.}\]
\end{remark}

\begin{theorem}[Grafico chiuso]\label{ThGraficoChiuso}
Siano $X,Y$ Banach, $T:X\to Y$ lineare. Allora $T$ \`e continua se e solo se
\[\Gamma=\cpa{(x,T(x))\in X\times Y\mid x\in X}\]
\`e chiuso.
\end{theorem}
\begin{proof}
Data una qualsiasi mappa continua $f:X\to Y$ con $Y$ Hausdorff si ha che $\Gamma$ \`e la preimmagine della diagonale di $Y\times Y$ rispetto alla mappa $id_Y\times f$. Poich\'e $Y$ \`e un Banach (e quindi metrico e quindi $T_2$) effettivamente abbiamo la prima implicazione.
\smallskip

Supponiamo ora che $\Gamma$ sia chiuso. Poich\'e $X\times Y$ \`e prodotto di Banach esso stesso \`e banach e quindi $\Gamma$ \`e banach perch\'e chiuso di un Banach. Osserviamo ora che
\[T(x)=P_Y((x,T(x)))=P_Y((P_X\res{\Gamma})\ii(x))\implies T=P_Y\circ(P_X\res{\Gamma})\ii.\]
Poich\'e $P_X\res{\Gamma}$ \`e bigettiva, continua e lineare, per il teorema della mappa aperta (\ref{ThMappaAperta}) essa \`e un omeomorfismo, quindi $T$ \`e continuo in quanto composizione di $P_Y$ e $P_X\res{\Gamma}\ii$ continue.
\end{proof}

\begin{exercise}
Sia $T:X\to Y$ lineare fra Banach. Controntare la continuit\`a di $T$ con le topologie forti e deboli di $X$ e $Y$
\begin{align*}
(X,w)&\to (Y,w)\\
(X,w)&\to (Y,s)\\
(X,s)&\to (Y,w)\\
(X,s)&\to (Y,s)
\end{align*}
\end{exercise}
\begin{proof}
Hint: usare grafico chiuso (\ref{ThGraficoChiuso}) ricordando che sottospazi vettoriali di Banach sono chiusi forti se e solo se sono chiusi deboli e osservando chi \`e la topologia debole di $X\times Y$ (topologia prodotto)

Tre di queste nozioni sono equivalenti e una no. Quella diversa \`e pi\`u forte? Pi\`u debole?
\end{proof}


\begin{definition}[Forte iniettivit\`a]
Una mappa $T:X\to Y$ lineare continua \`e \textbf{fortemente iniettiva} se esiste $c>0$ tale che 
\[\forall x\in X\qquad \norm{T(x)}\geq c\norm x.\]
\end{definition}

\begin{proposition}
Se $X$ e $Y$ sono banach e $T:X\to Y$ lineare continua, $T$ \`e fortemente iniettiva se e solo se $T$ \`e iniettiva e $\imm T$ \`e chiuso.
\end{proposition}
\begin{proof}
Diamo le implicazioni
\setlength{\leftmargini}{0cm}
\begin{itemize}
\item[$\boxed{\implies}$] Iniettiva ok. Sia $T':X\to \imm T\subseteq Y$ la stessa mappa di $T$ ma con codominio ristretto. Notiamo che $T'$ \`e invertibile perch\'e iniettiva e surgettiva per costruzione e che ha inversa continua per la disuguaglianza in ipotesi, quindi $\imm(T)$ \`e Banach perch\'e $X$ \`e Banach e quindi $\imm T$ \`e chiuso in $Y$.
\item[$\boxed{\impliedby}$] Se $T$ \`e iniettiva con immagine chiusa allora $T':X\to \imm T$ \`e invertibile. Inoltre, poich\'e $\imm T$ \`e Banach perch\'e chiuso di $Y$, si ha che per mappa aperta (\ref{ThMappaAperta}) vale $(T')\ii$ continua, cio\`e $T$ fortemente iniettiva.
\end{itemize}
\setlength{\leftmargini}{0.5cm}
\end{proof}

\begin{proposition}[Retrazioni e sezioni per lineari continue]\label{PrRetrazioniSezioniPerOperatoriLineariContinui}
Sia $T\in L(X,Y)$ con $X,Y$ Banach. Allora $T$ \`e una\footnote{cio\`e esiste $S:Y\to X$ tale che $T$ \`e l'inversa destra / sinistra di $S$.}
\begin{itemize}
    \item inversa destra $\coimplies$ iniettiva e $\imm T$ \`e complementato\footnote{cio\`e esiste $V\subseteq Y$ tale che $Y=\imm T\oplus V$.}
    \item inversa sinistra $\coimplies$ surgettivo e $\ker T$ \`e complementato.
\end{itemize}
\end{proposition}
\begin{proof}
Se $T:X\to Y$ e $S:Y\to X$ sono una coppia tale che $S\circ T=id_X$ allora $T\circ S=P$ \`e un proiettore lineare continuo, infatti
\[P^2=(T\circ S)\circ (T\circ S)=T\circ id_X \circ S=T\circ S.\]
Quindi $Y=\ker P\oplus \imm P$ e $\ker P=\ker S$, $\imm P=\imm T$, ovvero
\[Y=\ker S\oplus \imm T\]
come volevamo.
\medskip

Viceversa, se $T$ \`e iniettivo e $\imm T$ \`e complementata (rispettivamente $S$ \`e surgettivo e $\ker S$ complementato) allora considero un proiettore $P_{\imm T}$ (ok per la decomposizione in somma diretta) e definisco $S=(T')\ii\circ P_{\imm T}$ che \`e inversa sinistra di $T$ (rispettivamente definisco un proiettore $Q$ su $\ker S$ con $id_Y-Q$ proiettore sul supplementare $V$ fissato di $\ker S$, a questo punto considero $S\res V\ii$, che diventa inversa destra).
\end{proof}

\begin{proposition}[Norme confrontabili su Banach sono equivalenti]\label{PrNormeConfrontabiliSuBanachSonoEquivalenti}
Due norme si Banach confrontabili sullo stesso $\K$-spazio vettoriale sono equivalenti.
\end{proposition}
\begin{proof}
Caso particolare di ``$T$ bigettiva continua" scegliendo $T=id_X$.
\end{proof}

\begin{exercise}
Su uno spazio normato $X$ di dimensione infinita esistono sempre forme lineari non continue. 
\end{exercise}

\begin{remark}
Esistono $L:X\to X$ lineari bigettive non continue
\end{remark}
\begin{proof}
Fisso $f$ forma discontinua e fisso $u\in X$, definiamo
\[L(x)=x+f(x)u\]
e notiamo che
\begin{align*}
L^2(x)=&L(x+f(x)u)=L(x)+f(x)L(u)=\\
=&x+f(x)u +f(x)(u+f(u)u)=\\
=&x+(2f(x)+f(x)f(u))u.
\end{align*}
Se $u$ \`e tale che $f(u)=-2$ allora $L^2=id_X$, cio\`e $L$ involuzione. In particolare $L$ \`e bigettiva ma continua se e solo se $f$ lo \`e, e non lo \`e quindi $L$ non continua su $(X,\normd_1)$ Banach.

Poniamo $\norm x_2=\norm{L(x)}_2$. Notiamo che $\normd_2$ rende $X$ Banach in quanto $L:(X,\normd_1)\to (X,\normd_2)$ \`e una isometria. Notiamo dunque che $\normd_1$ e $\normd_2$ sono norme che rendono $X$ banach e che non sono equivalenti ($L$ \`e discontinua per $\normd_1$ ma continua per $\normd_2$).
\end{proof}

\begin{exercise}
Siano $\normd_1,\ \normd_2$ norme sullo stesso $X$ e sia $\normd_3=\normd_1+\normd_2$. Allora
\begin{enumerate}
    \item Una successione $(x_n)$ converge a $x\in X$ in $\normd_3$ se e solo se converge a $x$ in $\normd_1$ e $|normd_2$.
    \item $(x_n)$ \`e di Cauchy in $X$ se e solo se \`e di Cauchy sia per $\normd_1$ che per $\normd_2$.
\end{enumerate}
\end{exercise}

\begin{exercise}
TROVA L'IMBROGLIO:

\noindent
``\textbf{Proposizione.}" Tutte le norme di Banach sullo stesso $X$ sono equivalenti.
\begin{proof}[``Dimostrazione"]
    Siano $\normd_1$ e $\normd_2$ di Banach. Notiamo che $\normd_3$ \`e pi\`u fine della altre due e che $(x_n)$ \`e di Cauchy per $\normd_3$ se e solo se lo \`e per le altre due, quindi per il punto 1. della proposizione precedente la successione converge in $\normd_3$. Segue dunque che, poich\'e $\normd_3$ \`e pi\`u fine allora \`e confrontabile con le altre due, quindi le tre norme sono equivalenti.
\end{proof}
\end{exercise}


\section{SVT I-numerabili e paranorme}
\begin{definition}[Paranorma]
Una \textbf{paranorma} sull $\K$-spazio vettoriale $X$ \`e una funzione $q:X\to[0,\infty)$ tale che
\begin{enumerate}
    \item $q(x+y)\leq q(x)+q(y)$
    \item $q(\la x)\leq q(x)$ per ogni $x\in X$ e $\la\in\K$ tale che $\abs{\la}\leq 1$
    \item Se $\la_k\to 0$ in $\K$ allora $q(\la_k x)\to 0$
\end{enumerate}
Inoltre $q$ \`e \textbf{definita} se vale
\[q(x)=0\coimplies x=0.\]
\end{definition}

\begin{remark}
Dalla propriet\`a $2.$ segue che $q(\la x)=q(x)$ se $\abs\la=1$ e che $q(\la x)\leq q(\mu x)$ se $\abs\la\leq\abs\mu$. In particolare $q(x)=q(-x)$.

Quindi $d(x,y)=q(x-y)$ \`e una (semi)distanza su $X$ (distanza se $q$ definita).
\end{remark}

\begin{exercise}
Dimostrare che $(X,d)$ \`e uno SVT per $d$ indotta da paranorma $q$.
\end{exercise}

\begin{exercise}
Sia $X$ un $K$-SVT I-numerabile. Allora la sua topologia proviene da una paranorma (la quale \`e definita sse $X$ \`e $T_0$).
\end{exercise}
\begin{proof}
TRACCIA
\begin{itemize}
    \item Sia $\cpa{U_n}_{n\geq 0}$ base numerabile di intorni bilanciati di $0$ tali che $U_{n+1}+U_{n+1}\subseteq U_n$.
    \item Estendiamo la successione per $n<0$ ponendo $U_k=U_{k+1}+U_{k+1}$ per ogni $k<0$.
    
    Nota che $U_{k+1}+U_{k+1}\subseteq U_k$ per ogni $k\in \Z$ e gli $\cpa{U_k}_{k\in\Z}$ sono intorni bilanciati. 
    \item Poniamo
    \[q(x)=\inf\cpa{\sum_{i=1}^r2^{-ki}\mid r\in\N,(k_1,\cdots, k_r)\in\Z^r\ t.c.\ x\in U_{k_1}+U_{k_2}+\cdots+U_{k_r}}\]
    Mostra che $q$ \`e una paranorma su $X$.
    \item Nota che $\cpa{q<2^{-n-1}}\subseteq U_n\subseteq \cpa{q\leq 2^{-n}}$ e quindi $q$ induce la topologia di $X$.
\end{itemize}
\end{proof}
\chapter{Topologie deboli, Limitatezza e Banach-Steinhaus}

\section{Topologie deboli}

\begin{proposition}[Topologia iniziale nel caso SVT]\label{PrTopologiaInizialeCasoSVT}
Sia $X$ uno spazio vettoriale su $\K$ e sia $\Fc:\cpa{T_i:X\to Y_i}$ dove ogni $Y_i$ \`e SVT e $T_i$ \`e lineare, allora la topologia iniziale su $X$ indotta\footnote{vedi (\ref{PrTopologiaInizialeEsiste})} da $\Fc$ rende $X$ uno SVT.
\end{proposition}
\begin{proof}
Voglio verificare che $+$ e $\cdot$ sono mappe continue per la topologia iniziale.
% https://q.uiver.app/#q=WzAsNCxbMCwwLCJYXFx0aW1lcyBYIl0sWzEsMCwiWCJdLFswLDEsIllfaVxcdGltZXMgWV9pIl0sWzEsMSwiWV9pIl0sWzEsMywiVF9pIl0sWzAsMiwiVF9pXFx0aW1lcyBUX2kiLDJdLFswLDEsIisiXSxbMiwzLCIrX2kiLDJdXQ==
\[\begin{tikzcd}
	{X\times X} & X \\
	{Y_i\times Y_i} & {Y_i}
	\arrow["{+}", from=1-1, to=1-2]
	\arrow["{T_i\times T_i}"', from=1-1, to=2-1]
	\arrow["{T_i}", from=1-2, to=2-2]
	\arrow["{+_i}"', from=2-1, to=2-2]
\end{tikzcd}\]
% https://q.uiver.app/#q=WzAsNCxbMCwwLCJcXEtcXHRpbWVzIFgiXSxbMSwwLCJYIl0sWzAsMSwiXFxLXFx0aW1lcyBZX2kiXSxbMSwxLCJZX2kiXSxbMSwzLCJUX2kiXSxbMCwyLCJpZF9cXEtcXHRpbWVzIFRfaSIsMl0sWzAsMSwiXFxjZG90Il0sWzIsMywiXFxjZG90X2kiLDJdXQ==
\[\begin{tikzcd}
	{\K\times X} & X \\
	{\K\times Y_i} & {Y_i}
	\arrow["\cdot", from=1-1, to=1-2]
	\arrow["{id_\K\times T_i}"', from=1-1, to=2-1]
	\arrow["{T_i}", from=1-2, to=2-2]
	\arrow["{\cdot_i}"', from=2-1, to=2-2]
\end{tikzcd}\]
Per la propriet\`a universale della topologia iniziale (\ref{PrProprietaUniversaleTopologiaIniziale}), vogliamo verificare che $T_i\circ +=+_i\circ (T_i\times T_i)$ \`e continua per ogni $i$ e similmente per $T_i\circ \cdot$. Questo \`e vero perch\'e la topologia iniziale \`e rende $T_i$ continua per ogni $i$.
\end{proof}

\begin{remark}
Se ogni $Y_i$ inoltre \`e SVTLC allora anche $X$ lo \`e.
\end{remark}


\begin{definition}[Topologie deboli]
Sia $X$ un $\K$-spazio vettoriale e $\Fc\subseteq X'$ (duale algebrico). La topologia iniziale indotta da $\Fc$ viene detta la \textbf{topologia debole di $\Fc$} e si indica $\sigma(X,\Fc)$.
\end{definition}

\begin{remark}
$\sigma(X,\Fc)=\sigma(X,\Span_\K(\Fc))$ quindi senza perdita di generalit\`a possiamo sempre supporre $\Fc$ sottospazio vettoriale di $X'$.
\end{remark}

\begin{remark}
La famiglia di seminorme associata a $\Fc$ (quella che induce la stessa topologia di $SVTLC$) \`e data da
\[\Pc=\cpa{\abs{f}\mid f\in\Fc}\]
\end{remark}

\begin{remark}
La topologia debole $\sigma(X,\Fc)$ \`e $T_0$ (e quindi Hausdorff perch\'e SVT) se e solo se la famiglia $\Fc$ \`e separante ($\forall x\in X\nz,\ \exists f\in\Fc$ tale che $f(x)\neq0$).
\end{remark}


\begin{lemma}[]\label{LmDualeAlgebricoIndipendenzaEContinuita}
Siano $f_0,\cdots,f_n\in X'_{alg}$ per $X$ un $\K$-spazio vettoriale, allora sono equivalenti
\begin{enumerate}
    \item $f_0=\sum_{i=1}^n\la_if_i$
    \item $\abs{f_0}\leq M\max_{i\in\cpa{1,\cdots, n}}\abs{f_i}$ per qualche $M\geq 0$
    \item $\ker f_0\supseteq \bigcap_{i=1}^n\ker f_i$
\end{enumerate}
\end{lemma}
\begin{proof}
    Diamo le tre implicazioni
    \setlength{\leftmargini}{0cm}
    \begin{itemize}
    \item[$\boxed{1.\implies2.}$] Da $1.$ segue $\abs{f_0}\leq\sum_{i=1}^n\abs{\la_i}\abs{f_i}\leq M\max\abs{f_i}$ per $M=\sum\abs{\la_i}$.
    \item[$\boxed{2.\implies3.}$] Se $x\in \bigcap\ker f_i$, cio\`e $\ps{f_i,x}=0$ per ogni $i$, allora $\ps{f_0,x}\leq M0=0$, cio\`e $f_0(x)=0$ e abbiamo l'inclusione voluta.
    \item[$\boxed{3.\implies1.}$] Sia $F:X\to\K^n$ data da $F=(f_1,\cdots, f_n)$, allora
    \[\ker F=\bigcap\ker f_i\subseteq \ker f_0\]
    quindi abbiamo una fattorizzazione
    % https://q.uiver.app/#q=WzAsMyxbMCwwLCJYIl0sWzAsMSwiXFxLXm4iXSxbMSwwLCJcXEsiXSxbMSwyLCJMIiwyLHsic3R5bGUiOnsiYm9keSI6eyJuYW1lIjoiZGFzaGVkIn19fV0sWzAsMSwiRiIsMl0sWzAsMiwiZl8wIl1d
\[\begin{tikzcd}
	X & \K \\
	{\K^n}
	\arrow["{f_0}", from=1-1, to=1-2]
	\arrow["F"', from=1-1, to=2-1]
	\arrow["L"', dashed, from=2-1, to=1-2]
\end{tikzcd}\]
    dove $L(x_1,\cdots, x_n)=\sum \la_i x_i$ per dei $\la_i$ (in quanto \`e una forma lineare). Ma allora $f_0=L\circ F=\sum \la_i f_i$ come voluto.
    \end{itemize}
    \setlength{\leftmargini}{0.5cm}
\end{proof}

\begin{proposition}[Duale per topologia debole]\label{PrDualePerTopologiaDebole}
Dato $X$ $\K$-spazio vettoriale e $\Fc$ sottospazio di $X'_{alg}$ allora
\[(X,\sigma(X,\Fc))^\ast=\Fc\]
\end{proposition}
\begin{proof}
Sia $f_0\in (X,\sigma(X,\Fc))^\ast$, allora per la proposizione (\ref{PrContinuitaLineariSVT}) esistono $f_1,\cdots, f_n\in\Fc$ e $M\geq0$ tali che per ogni $x\in X$
\[\abs{f_0(x)}\leq M\max_{i}\abs{f_i(x)}.\]
Dunque per il lemma (\ref{LmDualeAlgebricoIndipendenzaEContinuita}) $f_0$ si scrive come combinazione lineare delle $f_i$ e quindi in particolare $f_0\in\Fc$.

L'altra inclusione \`e ovvia per definizione di topologia debole.
\end{proof}

\begin{remark}
Se $X$ ha dimensione infinita, $\sigma(X,\Fc)$ non \`e mai localmente limitata. In particolare ogni intorno di $0$ contiene uno spazio vettoriale di codimensione finita.
\end{remark}
\begin{proof}
Se $U$ intorno di $0$ per $\sigma(X,\Fc)$ allora esistono $f_1,\cdots,f_n\in\Fc$ tali che\footnote{vedi lemma (\ref{LmDualeAlgebricoIndipendenzaEContinuita})}
\[U\supseteq \bigcap_{i=1}^n\cpa{\abs{f_i}<1}\supseteq \bigcap_{i=1}^n \ker f_i\]
e l'intersezione di questi nuclei ha codimensione al massimo $n$.
\end{proof}

\begin{proposition}[Duale di lineare continua \`e debole$^\ast$-continua]\label{PrDualeOperatoreContinuoEDeboleStarContinua}
Se $T:E\to F$ \`e un operatore lineare e continuo allora $T^\ast:F^\ast\to E^\ast$ \`e debole$^\ast$-continua.
\end{proposition}
\begin{proof}
Considera le opportune composizione e la definizione di topologia debole.
\end{proof}

\subsection{Caso degli spazi normati}

\begin{definition}[Topologia debole]
Se $X$ \`e normato, la \textbf{topologia debole} su $X$ \`e la topologia debole associata a $X^\ast$, cio\`e $\sigma(X,X^\ast)$.
\end{definition}

\begin{proposition}
La topologia debole \`e localmente convessa e Hausdorff.
\end{proposition}
\begin{proof}
Per Hahn-Banach (\ref{ThHahnBanach}), il duale $X^\ast$ separa i punti
\end{proof}


\begin{definition}[Topologia debole$^\ast$]
Su $X^\ast$ possiamo considerare la topologia debole associata alle valutazioni $X\subseteq X^{\ast\ast}$, cio\`e scegliendo
\[\Fc=\cpa{val_x\in (X^\ast)'\mid x\in X}.\]
Questa \`e la \textbf{topologia debole$^\ast$} su $X^\ast$ e la indichiamo $\sigma(X^\ast,X)$.
\end{definition}

\begin{remark}
La topologia debole$^\ast$ rende $X^\ast$ uno SVTLC $T_0$ (e quindi Hausdorff), infatti se $f\in X^\ast\nz$ allora esiste $x\in X$ tale che $f(x)\neq 0$.
\end{remark}

\begin{remark}
In generale $\sigma(X^\ast,X)$ \`e meno fine di $\sigma(X^\ast,X^{\ast\ast})$. Abbiamo uguaglianza solo quando $X=X^{\ast\ast}$ in quanto se $X\neq X^{\ast\ast}$ allora dalla proposizione (\ref{PrDualePerTopologiaDebole}) ricaviamo
\[(X^\ast,\sigma(X^\ast,X))^\ast=X\neq X^{\ast\ast}=(X^\ast,\sigma(X^\ast,X^{\ast\ast}))^\ast\]
e quindi in partenza $\sigma(X^\ast,X^{\ast\ast})\neq \sigma(X^\ast,X)$
\end{remark}

\begin{remark}
Poich\'e $(X,\normd)\inj (X^{\ast\ast},\normd)$ isometricamente allora $(X,\sigma(X,X^\ast))$ ha la topologia indotta come sottospazio da\footnote{nota che $X^\ast$ lo si pu\`o pensare come immerso in $X^{\ast\ast\ast}=(X^{\ast\ast})^\ast$, quindi stiamo considerando la topologia debole$^\ast$ su $(X^{\ast})^\ast$} $(X^{\ast\ast},\sigma(X^{\ast\ast},X^\ast))$.
\end{remark}
\begin{proof}
Questo deriva dalla transitivit\`a della topologia iniziale (\ref{PrTransitivitaTopologiaIniziale}) dove la prima famiglia \`e la mappa $X\inj X^{\ast\ast}$ e l'unica altra famiglia sono gli elementi di $X^\ast$ che vanno verso $\K$.
\end{proof}

\section{Spazi di Baire e II-categoria}
\begin{theorem}[Baire]\label{ThBaire}
Se $\cpa{A_k}_{k\in\N}$ \`e una famiglia numerabile di aperti densi di uno spazio metrico completo allora $\bigcap A_k$ \`e denso.
\end{theorem}
\begin{proof}
Per induzione si definisce una successione di palle chiuse di $X$ dove $B_0$ \`e arbitraria e
\[B_k=\ol{B(x_k,r_k)}\text{ tali che }B_{k+1}\subseteq B_k\cap A_k\text{ e }r_k=o(1)\]
che possiamo fare perch\'e $A_k$ \`e un aperto denso.

Allora la successione dei centri \`e una successione di Cauchy, infatti se $p,q\geq n$ si ha $x_p,x_q\in B_n$ e quindi $d(x_p,x_q)\leq 2r_n$. Dunque $x_n\to x^\ast$ in $X$ per completezza. Inoltre, poich\'e $x_k\in B_n$ definitivamente, $x^\ast=\lim x_k\in B_n$ per ogni $n$ (dato che $B_n$ \`e chiuso). In particolare $x^\ast\in B_{n+1}\subseteq A_n$ per ogni $n$ e quindi $x^\ast\in \bigcap A_n$. Per costruzione $x^\ast\in B_0$, quindi per ogni palla $B_0$ abbiamo mostrato che $B_0\cap \bigcap A_n\neq \emptyset$, cio\`e $\bigcap A_n$ \`e denso.
\end{proof}

\begin{exercise}
La stessa conclusione vale se $X$ \`e localmente compatto al posto di metrico completo.
\end{exercise}

\begin{definition}[Spazio di Baire]
Uno spazio topologico \`e \textbf{di Baire} se ogni intersezione numerabile di aperti densi \`e densa.
\end{definition}

\begin{remark}
Ogni aperto non vuoto di $X$ di Baire \`e ancora di Baire. Basta verificare che ogni aperto denso di $A$ \`e della forma $A\cap U$ con $U$ aperto denso di $X$.
\end{remark}


\begin{definition}[Sottoinsieme di I- e II-categoria]
Un sottoinsieme $S$ di $X$ \`e di \textbf{I-categoria (di Baire) in $X$} se \`e unione numerabile di insiemi $(E_i)_{i\in\N}$ con $int(\ol{E_i})=\emptyset$.

Inoltre $S$ \`e di \textbf{II-categoria (di Baire) in $X$} se non \`e di I-categoria.
\end{definition}

\begin{remark}
Se $X$ \`e di Baire e $S\subseteq X$ \`e di I-categoria allora $X\bs S$ \`e di II-categoria in quanto $X$ stesso \`e di II-categoria (se $X=\bigcup E_i$ con $E_i$ chiusi a parte interna vuota allora $\emptyset=\bigcap E_i^c$ con $E_i^c$ aperti densi, ma questo \`e assurdo perch\'e $X$ di Baire).
\end{remark}

\section{Teorema di Banach-Steinhaus}

\begin{definition}[Famiglia equicontinua]
Una famiglia $\Gamma$ di operatori lineari continui fra SVT $X$ e $Y$ \`e \textbf{equicontinua} se per ogni $U\in\Uc_Y$ esiste $V\in \Uc_X$ tale che per ogni $T\in\Gamma$, $T(V)\subseteq U$.
\end{definition}

\begin{remark}
Possiamo riformulare la condizione nei seguenti modi: per ogni $U\in\Uc_Y$ esiste $V\in\Uc_X$ tale che
\[\forall T\in\Gamma,\ V\subseteq T\ii(U)\coimplies V\subseteq \bigcap_{T\in \Gamma}T\ii(U)\doteqdot \Gamma\ii(U).\]
Equivalentemente la condizione predica che per ogni $V\in\Uc_Y$ si abbia $\Gamma\ii(V)\in\Uc_X$.
\end{remark}

\begin{remark}
Se $T:X\to Y$ fra spazi normati, la norma degli operatori
\[\norm T=\norm T_{\infty,B(0,1)}=\text{migliore costante di Lipschitz per }T.\]
\end{remark}

\begin{example}
Se $X$ e $Y$ sono normati, $\Gamma$ \`e equicontinua se e solo se $\Gamma$ \`e limitato in $L(X,Y)$ rispetto alla norma degli operatori.
\end{example}

\begin{theorem}[Banach-Steinhaus / Uniforme limitatezza]\label{ThBanachSteinhausUniformeLimitatezza}
Siano $X,Y$ SVT, $S\subseteq X$ di seconda categoria e $\Gamma\subseteq L(X,Y)$ con $\Gamma$ puntualmente limitata su $S\subseteq X$, cio\`e per ogni $s\in S$, $\Gamma(s)=\bigcup_{T\in \Gamma} T(s)$ \`e limitato in $Y$.

Allora $\Gamma$ \`e equicontinua.
\end{theorem}
\begin{proof}
Sia $U\in\Uc_Y$ e consideriamo $V\in\Uc_Y$ chiuso tale che $V-V\subseteq U$. Per ipotesi, per ogni $x\in S$ si ha che $\Gamma(x)$ \`e limitato in $Y$, quindi viene assorbito da $V$, cio\`e esiste $n_x\in\N$ tale che per ogni $T\in\Gamma$ si ha $T(x)\in n_x V$, cio\`e tale che
\[x\in\bigcap_{T\in \Gamma}n_xT\ii(V)=n_x\bigcap_{T\in \Gamma}T\ii(V)=n_x\Gamma\ii(V).\]
Dunque $S\subseteq \bigcup_{n\in\N}n\Gamma\ii(V)$. Notiamo che poich\'e $V$ \`e chiuso, $T\ii(V)$ \`e chiuso e quindi anche $\Gamma\ii(V)$ lo \`e perch\'e intersezione di chiusi. Poich\'e $S$ \`e di seconda categoria anche l'unione delle versioni riscalate di $\Gamma\ii(V)$ lo \`e, dunque questo insieme non \`e unione numerabile di chiusi con parte interna vuota, quindi almeno uno tra gli $n\Gamma\ii(V)$ ha parte interna non vuota, quindi anche $\Gamma\ii(V)$ ha parte interna non vuota scalando per $\frac1n$.

Quindi $\Gamma\ii(V)$ \`e intorno di qualche suo punto, dunque\footnote{se $a_0\in int(A)$ allora $A-a_0\subseteq A-A$ \`e un intorno di $0$.} $\Gamma\ii(V)-\Gamma\ii(V)$ \`e un intorno di $0$. 

Ricordando che $V-V\subseteq U$ si ha
\[T\ii(U\supseteq T\ii(V-V)=T\ii(V)-T\ii(V))\supseteq \Gamma\ii(V)-\Gamma\ii(V)\]
quindi passando all'intersezione su $T\in\Gamma$ si ha
\[\Gamma\ii(U)\supseteq \Gamma\ii(V)-\Gamma\ii(V)\in\Uc_X,\]
cio\`e abbiamo mostrato che per ogni $U\in\Uc_Y$ si ha $\Gamma\ii(U)\in\Uc_X$, che \`e equivalente all'equicontinuit\`a di $\Gamma$.
\end{proof}

\begin{corollary}[Sottoinsiemi limitati di operatori]\label{CorPuntualmenteLimitatoImplicaLimitatoPerNormaOperatore}
    Se $X$ e $Y$ sono Banach e $\Gamma\subseteq L(X,Y)$ \`e puntualmente limitata in $X$ (o volendo anche un sottoinseme di $X$ di II-categoria) allora $\Gamma$ \`e un insieme limitato in $L(X,Y)$.
\end{corollary}
\begin{proof}
Diretta applicazione di Banach-Stenhaus (\ref{ThBanachSteinhausUniformeLimitatezza}) notando che spazi di Banach sono in particolare SVT e che equicontinuit\`a per la norma su $L(X,Y)$ significa limitatezza. **************
\end{proof}

\begin{exercise}
Siano $X,Y$ SVT. Trovare la topologia meno fine $\tau$ di SVT su $L(X,Y)$ per la quale
\[\Gamma\text{ puntualmente limitato in $L(X,Y)$}\coimplies \Gamma\text{ limitato nella topologia $\tau$}.\]
\end{exercise}


\begin{corollary}\label{CorConvergenzaLineariSePuntualmenteConvergente}
Siano $X$ e $Y$ Banach e sia $(T_n)\subseteq L(X,Y)$ puntualmente convergente. Allora il limite $T$ \`e ancora lineare, continuo e con norma
\[\norm T\leq \liminf_{n\to\infty}\norm{T_n}.\]
\end{corollary}
\begin{proof}
Per il corollario precedente (\ref{CorPuntualmenteLimitatoImplicaLimitatoPerNormaOperatore}) si ha che $(T_n)$ sono limitati in $\normd$ e il limite puntuale \`e lineare in quanto
\[T_n(\al x+\beta y)=\al T_n(x)+\beta T_n(y)\to \al T(x)+\beta T(y).\]
Questo mostra che $T$ \`e limitato e lineare, quindi $T\in L(X,Y)$.

Inoltre per ogni $x\in X$ si ha
\[\norm{T(x)}=\lim_{n}\norm{T_n(x)}\leq \pa{\sup_n\norm{T_n}}\norm x,\]
quindi $\norm T\leq \sup_n\norm{T_n}$. Ragionando analogamente per una sottosuccessione di $(T_n)$ che in norma converge a $\liminf_n\norm{T_n}$ ricaviamo
\[\norm T\leq \liminf_n\norm{T_n}.\]
\end{proof}

\begin{remark}
In generale NON vale $T_n\to T$ in $\normd$.
\end{remark}

\begin{proposition}[Bilineare separatamente continua \`e continua]\label{PrBilineareSeparatamenteContinuaEContinua}
Sia $b:X\times Y\to Z$ bilineare e separatamente continua, cio\`e per ogni $x\in X, y\in Y$ si ha che $b(x,\cdot):Y\to Z$ e $b(\cdot, y):X\to Z$ sono lineari e continue. Allora $b$ \`e continua, cio\`e
\[\sup_{\norm x\leq 1, \norm y\leq 1}\norm{b(x,y)}<\infty.\]
\end{proposition}
\begin{proof}
Consideriamo la famiglia
\[\Gamma=\cpa{b(x,\cdot):Y\to Z}_{x\in X,\ \norm x\leq 1}\subseteq L(Y,Z).\]
Per ipotesi $\Gamma$ \`e puntualmente limitata in $Y$, infatti per ogni $y\in Y$
\[\sup_{b(x,\cdot)\in\Gamma} \norm{b(x,\cdot)}_{L(Y,Z)}=\sup_{\norm x\leq 1}\norm{b(x,y)}_Z= \norm{b(\cdot,y)}_{L(X,Z)}\norm y<\infty\]
Allora $\Gamma$ \`e limitata in $\normd_{L(Y,Z)}$, cio\`e per ogni $x\in X$ tale che $\norm x\leq 1$ si ha
\[\norm{b(x,y)}_Z\leq M\norm y\]
e quindi al variare di $y$ con $\norm y\leq 1$ troviamo $\norm b_{L(X\times Y,Z)}\leq M$.
\end{proof}

\begin{exercise}
Esiste una isometria lineare
\[\funcDef{L(X,L(Y,Z))}{L^2(X\times Y,Z)}{T}{(x,y)\mapsto T(x)(y)}\]
dove $L^2(X\times Y,Z)$ sono le bilineari.
\end{exercise}

\begin{proposition}[w$^\ast$-limitato vs limitato in $\normd_{X^\ast}$]\label{PrLimitatoInDeboleStarEquivaleLimitatoInNormaDuale}
Sia $Y=\K$ e $X$ Banach. Sia $\Gamma\subseteq X^\ast$, allora $\Gamma$ \`e w$^\ast$-limitato se e solo se \`e limitato in $\normd_{X^\ast}$.
\end{proposition}
\begin{proof}
Essere limitato nella topologia debole$^\ast$ significa ``essere assorbito da ogni intorno w$^\ast$ di $X^\ast$" cio\`e, usando intorni di prebase, essere assorbiti da insiemi della forma
\[\cpa{f\in X^\ast\mid \abs{f(x)}<1}\]
per $x\in X$. Notiamo che $\Gamma$ viene assorbito da $\cpa{f\in X^\ast\mid \abs{f(x)}<1}$ significa $\Gamma(x)$ limitato in $\K$. Per il corollario (\ref{CorPuntualmenteLimitatoImplicaLimitatoPerNormaOperatore}) si ha che $\Gamma$ \`e limitato in $L(X,\K)=X^\ast$.

L'altra implicazione \`e ovvia perch\'e la norma operatore gi\`a rende continui gli operatori e indebolire la topologia non pu\`o trasformare un insieme limitato in uno non limitato.
\end{proof}

\begin{remark}
Se $E\subseteq F$ \`e un sottospazio allora $\Gamma\subseteq E$ \`e limitato in $F$ se e solo se \`e limitato in $E$ per la topologia indotta.
\end{remark}

\begin{proposition}
Sia $\Gamma\subseteq X$, allora $\Gamma$ \`e w-limitato se e solo se \`e $\normd$-limitato.
\end{proposition}
\begin{proof}
Se $\Gamma$ \`e $\sigma(X,X^\ast)$-limitato allora tramite l'immersione isometrica $X\to X^{\ast\ast}$ troviamo un insieme $\sigma(X^{\ast\ast},X^\ast)$-limitato. A questo punto basta applicare la proposizione precedente (\ref{PrLimitatoInDeboleStarEquivaleLimitatoInNormaDuale}).
\end{proof}

\chapter{Lemma di iterazione e Iniettivit\`a / Surgettivit\`a di mappe lineari}



\section{Lemma di iterazione}

\begin{lemma}[di iterazione]\label{LmDiIterazione}
Siano $X$ e $Y$ spazi di Banach, $B$ palla unitaria chiusa di $X$, $T\in L(X,Y)$, $U$ limitato, $U\subseteq Y$ tali che se $0<t<1$ allora
\[U\subseteq TB+tU.\]
Allora si ha $(1-t)U\subseteq TB$.
\end{lemma}
\begin{proof}
Sia $u_0\in U$, allora esistono $x_0\in B$ e $u_1\in U$ tali che 
\[u_0=T(x_0)+tu_1\]
Iterando troviamo $u_2\in U$ e $x_1\in B$ tali che $u_1=T(x_1)+tu_2$ e cos\`i via. Questo definisce quindi due successioni $(u_n)\subseteq U$ e $(x_n)\subseteq B$. Notiamo che per ogni $n\in\N$
\[u_0=t^{n+1}y_{n+1}+\sum_{i=0}^nt^iT(x_i)=T\pa{\sum_{i=0}^nt^ix_i}+t^{n+1}y_{n+1}.\]
Poich\'e $X$ \`e completo, la serie $\sum_{i=0}^\infty t^i x_i$ converge ad un punto $x^\ast\in \frac1{1-t}B$ in quanto $\sum_{i=0}^\infty t^i=\frac1{1-t}$.

Poich\'e $U$ \`e limitato esiste $M>0$ tale che $U\subseteq B(0,M)$, quindi $\norm{t^{n+1} y_{n+1}}\leq t^{n+1}M$ e questa successione converge a $0$ quindi $t^{n+1} y_{n+1}$ converge a $0$. Segue che
\[y_0=\lim_{n\to\infty} T\pa{\sum_{i=0}^nt^i x_i}+\under{=o(1)}{t^{n+1}y_{n+1}}\pasgnl={$T$ continua}T\pa{\lim_{n\to\infty}\sum_{i=0}^nt^ix_i}=T(x^\ast)\]
quindi $y_0\in \frac1{1-t}T B$, cio\`e
\[U\subseteq \frac1{1-t}TB\coimplies (1-t)U\subseteq TB.\]
\end{proof}

\begin{remark}
Se $U$ \`e un intorno di $0$ limitato in $Y$, o anche $U$ assorbente, allora $T$ \`e surgettivo.
\end{remark}

\begin{theorem}[Lemma di Urysohn]\label{ThLemmaUrysohn}
Se $X$ \`e normale e $F_0$,$F_1$ sono chiusi disgiunti di $X$ allora esiste $f$ tale che $F_0=\cpa{f=0}$ e $F_1=\cpa{f=1}$.
\end{theorem}

\begin{theorem}[Teorema di estensione di Tietze]\label{ThEstensioneTietze}
Se $X$ \`e T4, $Y\subseteq X$ chiuso, $f\in C^0(Y,\R)$, allora $f$ si estende ad una continua su $X$.
\end{theorem}
\begin{proof}
Basta il caso di $f$ limitata tanto la continuit\`a \`e una condizione che \`e invariante componendo per un omeomorfismo e $\R\cong (0,1)$.

La tesi \`e che l'operatore di restrizione (il quale \`e lineare e continuo)
\[R:C_b^0(X)\to C_b^0(Y)\]
\`e surgettivo. Basta applicare il lemma (\ref{LmDiIterazione}) come segue:
\[3B_{C_b(Y)}\subseteq R\pa{B_{C_b(X)}}+2B_{C_b(Y)}\]
e chiamiamo $U=3B_{C_b(Y)}$, $T=(2/3)\cdot$. Sia $f\in 3B_{C_b(Y)}$. Per il lemma di Urysohn esiste $g:X\to[-1,1]$ continua tale che $g=-1$ su $\cpa{x\in Y\mid -3\leq f\leq -1}$ e $g=1$ su $\cpa{x\in Y\mid 3\geq f\geq 1}$ (i due insiemi sono chiusi perch\'e $Y$ \`e chiuso e $f$ \`e continua).
\[f=g\res Y+(f-g\res Y)\]
ma notiamo allora che $g\in B_{C_b(X)}$ e quindi $g\res Y\in R(B_{C_b(X)})$, mentre $f-g\res Y\in 2B_{C_b(Y)}$, infatti su $\cpa{x\in Y\mid -3\leq f\leq -1}$ abbiamo $g=-1$ e quindi $\norm{f-g}_{\infty,Y}\leq 2$, su $\cpa{x\in Y\mid 3\geq f\geq 1}$ abbiamo $g=1$ e quindi di nuovo $\norm{f-g}_{\infty,Y}\leq 2$, e infine sui punti rimanenti, siccome $f\in 3B_{C_b(Y)}$, si ha $\norm f\leq 1$ e stesso per $g$, quindi $\norm{f-g\res Y}_{\infty,Y}\leq 2$ di nuovo.

Questo verifica le ipotesi del lemma di iterazione (\ref{LmDiIterazione}), quindi 
\[(1-2/3)B_{C_b(Y)}\subseteq R(B_{C_b(X)}).\]
\end{proof}

\begin{theorem}[Dugundji]\label{ThEstensioneDugundji}
Sia $(M,d)$ spazio metrico, $A\subseteq M$ chiuso, $E$ banach e $f:A\to E$ continua (basta limitata) allora esiste una estensione di $f$ continua a tutto $M$ con la stessa norma.
\end{theorem}

\begin{remark}
In realt\`a l'estensione di $f$ a $M$ si pu\`o dare come un operatore di estensione
\[\Ec:C_b(A,E)\to C_b(M,E).\]
Questo operatore \`e inverso destro dell'operatore di restrizione $R:C_b(M,E)\to C_b(A,E)$ che abbiamo usato nel teorema di Tietze (\ref{ThEstensioneTietze}).
\end{remark}


\begin{theorem}[Sollevamento per operatori lineari / Bartles-Groves]\label{ThSollevamentoOperatoriLineariBartles-Groves}
Sia $L:E\to F$ lineare continuo surgettivo con $E,F$ banach. $M$ spazio metrico e $f:M\to F$ continua, allora $f$ si pu\`o sollevare a $E$, cio\`e esiste $\wt f:M\to E$ continua tale che $f=L\circ \wt f$.
% https://q.uiver.app/#q=WzAsMyxbMSwwLCJFIl0sWzEsMSwiRiJdLFswLDEsIk0iXSxbMCwxLCJMIl0sWzIsMSwiZiIsMl0sWzIsMCwiXFx3dCBmIiwwLHsic3R5bGUiOnsiYm9keSI6eyJuYW1lIjoiZGFzaGVkIn19fV1d
\[\begin{tikzcd}
	& E \\
	M & F
	\arrow["L", from=1-2, to=2-2]
	\arrow["{\wt f}", dashed, from=2-1, to=1-2]
	\arrow["f"', from=2-1, to=2-2]
\end{tikzcd}\]
In altre parole, \`e surgettivo l'operatore (lineare e continuo)
\[L_\ast:\funcDef{C_b(M,E)}{C_b(M,F)}{g}{L\circ g}.\]
\end{theorem}
\begin{proof}
Applichiamo il lemma di surgettivit\`a lineare come segue: sia $f\in C_b(M,E)$ e consideriamo un sollevamento approssimato $g$ costruito con partizioni dell'unit\`a a partire da sollevamenti approssimati locali che sono costanti. 
\end{proof}
\begin{remark}
Se $M=F$ e $f=id_F$ allora questo restituisce una inversa destra continua  (ma possibilmente non lineare) $\sigma$ di $L$. Quindi ogni operatore lineare surgettivo ammette una inversa destra continua. Inoltre se $L$ non ammette inversa destra lineare allora $\sigma$ non \`e neanche differenziabile in alcun punto (se fosse differenziabile $L\circ \sigma=id\implies L\circ \Dc\sigma=id$)
\end{remark}

\begin{remark}
Se $X,Y$ banach, l'insieme degli operatori surgettivi $\Sc\Uc=\cpa{L\in L(X,Y,L\text{ surg})}$ \`e aperto in $L(X,Y)$. 
\end{remark}
\begin{proof}
Se $T\in \Sc\Uc(X,Y)$ allora esso induce
% https://q.uiver.app/#q=WzAsMyxbMCwwLCJYIl0sWzEsMCwiWSJdLFswLDEsIlgvXFxrZXIgVCJdLFswLDEsIlQiXSxbMCwyXSxbMiwxLCJcXHd0IFQiLDJdXQ==
\[\begin{tikzcd}
	X & Y \\
	{X/\ker T}
	\arrow["T", from=1-1, to=1-2]
	\arrow[from=1-1, to=2-1]
	\arrow["{\wt T}"', from=2-1, to=1-2]
\end{tikzcd}\]
Sia $k=(\norm{\wt T\ii})\ii\in \R$, allora per ogni $H\in L(X,Y)$ con $\norm H<k$ abbiamo $T+H\in \Sc\Uc(X,Y)$: per definizione di $k$ vale $\wt T\ii(B_Y)\subseteq \frac1k B_X$ perch\'e $\frac1k=\norm{\wt T\ii}$ e quindi
\[kB_Y\subseteq TB_X=\wt T(\pi B_X)=\wt T(B_{X/\ker T})\]
Dunque 
\[kB_Y\subseteq TB_X\subseteq (T+H)B_X+HB_X\subseteq (T+H)B_X+\frac{\norm H}k(kB_Y)\]
quindi $T+H$ verifica le ipotesi del lemma di iterazione (\ref{LmDiIterazione}) con $t=\frac{\norm H}k<1$.
\end{proof}


\subsection{Teorema della mappa aperta}

\begin{theorem}[Mappa aperta]\label{ThMappaAperta}
Siano $X,Y$ Banach e $T:X\to Y$ lineare continuo e tale che $T(X)$ \`e di II-categoria in $Y$ (per esempio $T$ surgettivo). Allora $T$ \`e una mappa aperta.
\end{theorem}
\begin{proof}
Sia $B$ la palla unitaria chiusa di $X$. Basta mostrare che $T(B)$ \`e un intorno di $0$ in $Y$ (per omotetia e traslazione seguir\`a che $T$ manda intorni di $x$ in intorni di $T(x)$, cio\`e \`e aperta). Notiamo che 
\[X=\bigcup_n nB\implies T(X)=\bigcup_n nT(B)\]
Per ipotesi $T(X)$ \`e di II-categoria in $Y$, quindi per qualche $n$ si ha che $\ol{n T(B)}$ ha parte interna non vuota e quindi $\ol{T(B)}$ stesso ha parte interna non vuota. Poich\'e\footnote{ricorda che in generale se $f$ \`e continua allora $f(\ol A)\subseteq \ol{f(A)}$. 

In questo caso la mappa \`e $(x,y)\mapsto x-y$ e usiamo il fatto che $T$ \`e lineare e \[\ol{T(B)\times T(B)}=\ol{T(B)}\times \ol{T(B)}.\]} 
\[\ol{T(B)}-\ol{T(B)}\subseteq \ol{T(B-B)}=\ol{T(2B)}=2\ol{T(B)}\]
si ha che $\ol{T(B)}$ \`e un intorno di $0\in Y$.


Mostriamo ora che $T(B)$ stesso \`e un intorno di $0$. Poich\'e la chiusura \`e l'intersezione degli aperti che contengono $T(B)$ si ha in particolare che
\[\ol{T(B)}=T(B)+\frac12\ol{T(B)}.\]
Siccome $T$ \`e continua, $T(B)$ \`e limitato e quindi $\ol{T(B)}$ \`e limitato, quindi per il lemma di iterazione (\ref{LmDiIterazione}) di ha
\[\pa{1-\frac12}\ol{T(B)}\subseteq T(B)\coimplies \ol{T(B)}\subseteq 2T(B),\]
in particolare $T(B)$ \`e un intorno di $0$ per omotetia.
\end{proof}

\begin{remark}[Lineare continuo allora omeo se e solo se bigettivo]
Un operatore lineare continuo \`e un omeomorfismo se e solo se \`e bigettivo. Questo \`e immediato da mappa aperta (\ref{ThMappaAperta}).
\end{remark}

\begin{remark}
Se $T:X\to Y$ lineare continuo allora induce
% https://q.uiver.app/#q=WzAsMyxbMCwwLCJYIl0sWzEsMCwiWSJdLFswLDEsIlgvXFxrZXIgVCJdLFswLDIsIlxccGkiLDJdLFswLDEsIlQiXSxbMiwxLCJcXHd0IFQiLDJdXQ==
\[\begin{tikzcd}
	X & Y \\
	{X/\ker T}
	\arrow["T", from=1-1, to=1-2]
	\arrow["\pi"', from=1-1, to=2-1]
	\arrow["{\wt T}"', from=2-1, to=1-2]
\end{tikzcd}\]
con $\wt T$ lineare continua iniettiva. Se $T$ \`e surgettiva allora per il teorema della mappa aperta (\ref{ThMappaAperta}) $\wt T$ \`e un omeomorfismo lineare.
\end{remark}

\begin{remark}
Se $T:X\to Y$ \`e lineare e continua allora
\[\text{aperta $\coimplies$ surgettiva $\coimplies$ identificazione.}\]
\end{remark}

\begin{theorem}[Grafico chiuso]\label{ThGraficoChiuso}
Siano $X,Y$ Banach, $T:X\to Y$ lineare. Allora $T$ \`e continua se e solo se
\[\Gamma=\cpa{(x,T(x))\in X\times Y\mid x\in X}\]
\`e chiuso.
\end{theorem}
\begin{proof}
Data una qualsiasi mappa continua $f:X\to Y$ con $Y$ Hausdorff si ha che $\Gamma$ \`e la preimmagine della diagonale di $Y\times Y$ rispetto alla mappa $id_Y\times f$. Poich\'e $Y$ \`e un Banach (e quindi metrico e quindi $T_2$) effettivamente abbiamo la prima implicazione.
\smallskip

Supponiamo ora che $\Gamma$ sia chiuso. Poich\'e $X\times Y$ \`e prodotto di Banach esso stesso \`e banach e quindi $\Gamma$ \`e banach perch\'e chiuso di un Banach. Osserviamo ora che
\[T(x)=P_Y((x,T(x)))=P_Y((P_X\res{\Gamma})\ii(x))\implies T=P_Y\circ(P_X\res{\Gamma})\ii.\]
Poich\'e $P_X\res{\Gamma}$ \`e bigettiva, continua e lineare, per il teorema della mappa aperta (\ref{ThMappaAperta}) essa \`e un omeomorfismo, quindi $T$ \`e continuo in quanto composizione di $P_Y$ e $P_X\res{\Gamma}\ii$ continue.
\end{proof}

\begin{exercise}
Sia $T:X\to Y$ lineare fra Banach. Controntare la continuit\`a di $T$ con le topologie forti e deboli di $X$ e $Y$
\begin{align*}
(X,w)&\to (Y,w)\\
(X,w)&\to (Y,s)\\
(X,s)&\to (Y,w)\\
(X,s)&\to (Y,s)
\end{align*}
\end{exercise}
\begin{proof}
Hint: usare grafico chiuso (\ref{ThGraficoChiuso}) ricordando che sottospazi vettoriali di Banach sono chiusi forti se e solo se sono chiusi deboli e osservando chi \`e la topologia debole di $X\times Y$ (topologia prodotto)

Tre di queste nozioni sono equivalenti e una no. Quella diversa \`e pi\`u forte? Pi\`u debole?
\end{proof}


\subsubsection{Norme confrontabili}

\begin{proposition}[Norme confrontabili su Banach sono equivalenti]\label{PrNormeConfrontabiliSuBanachSonoEquivalenti}
Due norme su Banach confrontabili sullo stesso $\K$-spazio vettoriale sono equivalenti.
\end{proposition}
\begin{proof}
Se le norme sono confrontabili, $id_X$ \`e continua se sul dominio consideriamo la topologia pi\`u fine. Chiaramente $id_X$ \`e lineare, quindi per il teorema della mappa aperta (\ref{ThMappaAperta}) si ha che $id_X$ \`e aperta. Poich\'e $id_X$ \`e bigettiva questo mostra che $id_X$ \`e un omeomorfismo.
\end{proof}

\begin{exercise}
Su uno spazio normato $X$ di dimensione infinita esistono sempre forme lineari non continue. 
\end{exercise}


\begin{remark}
Esistono $L:X\to X$ lineari bigettive non continue
\end{remark}
\begin{proof}
Fisso $f$ forma discontinua e fisso $u\in X$, definiamo
\[L(x)=x+f(x)u\]
e notiamo che
\begin{align*}
L^2(x)=&L(x+f(x)u)=L(x)+f(x)L(u)=\\
=&x+f(x)u +f(x)(u+f(u)u)=\\
=&x+(2f(x)+f(x)f(u))u.
\end{align*}
Se $u$ \`e tale che $f(u)=-2$ allora $L^2=id_X$, cio\`e $L$ involuzione. In particolare $L$ \`e bigettiva ma continua se e solo se $f$ lo \`e, e non lo \`e quindi $L$ non continua su $(X,\normd_1)$ Banach.

Poniamo $\norm x_2=\norm{L(x)}_2$. Notiamo che $\normd_2$ rende $X$ Banach in quanto $L:(X,\normd_1)\to (X,\normd_2)$ \`e una isometria. Notiamo dunque che $\normd_1$ e $\normd_2$ sono norme che rendono $X$ banach e che non sono equivalenti ($L$ \`e discontinua per $\normd_1$ ma continua per $\normd_2$).
\end{proof}

\begin{exercise}
Siano $\normd_1,\ \normd_2$ norme sullo stesso $X$ e sia $\normd_3=\normd_1+\normd_2$. Allora
\begin{enumerate}
    \item Una successione $(x_n)$ converge a $x\in X$ in $\normd_3$ se e solo se converge a $x$ in $\normd_1$ e $\normd_2$.
    \item $(x_n)$ \`e di Cauchy in $X$ se e solo se \`e di Cauchy sia per $\normd_1$ che per $\normd_2$.
\end{enumerate}
\end{exercise}

\begin{exercise}
TROVA L'IMBROGLIO:

\noindent
``\textbf{Proposizione.}" Tutte le norme di Banach sullo stesso $X$ sono equivalenti.
\begin{proof}[``Dimostrazione"]
    Siano $\normd_1$ e $\normd_2$ di Banach. Notiamo che $\normd_3$ \`e pi\`u fine della altre due e che $(x_n)$ \`e di Cauchy per $\normd_3$ se e solo se lo \`e per le altre due, quindi per il punto 1. della proposizione precedente la successione converge in $\normd_3$. Segue dunque che, poich\'e $\normd_3$ \`e pi\`u fine allora \`e confrontabile con le altre due, quindi le tre norme sono equivalenti.
\end{proof}
\end{exercise}



\section{Iniettivit\`a e surgettivit\`a di mappe lineari}

Cerchiamo di capire che relazione c'\`e tra iniettivit\`a e surgettivit\`a delle mappe $T$ e $T^\ast$ per $T:X\to Y$ lineare continua.

\subsection{Forte iniettivit\`a}

\begin{definition}[Forte iniettivit\`a]
Una mappa $T:X\to Y$ lineare continua \`e \textbf{fortemente iniettiva} se esiste $c>0$ tale che 
\[\forall x\in X\qquad \norm{T(x)}\geq c\norm x.\]
\end{definition}

\begin{proposition}
Se $X$ e $Y$ sono banach e $T:X\to Y$ lineare continua, $T$ \`e fortemente iniettiva se e solo se $T$ \`e iniettiva e $\imm T$ \`e chiuso.
\end{proposition}
\begin{proof}
Diamo le implicazioni
\setlength{\leftmargini}{0cm}
\begin{itemize}
\item[$\boxed{\implies}$] Iniettiva ok. Sia $T':X\to \imm T\subseteq Y$ la stessa mappa di $T$ ma con codominio ristretto. Notiamo che $T'$ \`e invertibile perch\'e iniettiva e surgettiva per costruzione e che ha inversa continua per la disuguaglianza in ipotesi, quindi $\imm(T)$ \`e Banach perch\'e $X$ \`e Banach e quindi $\imm T$ \`e chiuso in $Y$.
\item[$\boxed{\impliedby}$] Se $T$ \`e iniettiva con immagine chiusa allora $T':X\to \imm T$ \`e invertibile. Inoltre, poich\'e $\imm T$ \`e Banach perch\'e chiuso di $Y$, si ha che per mappa aperta (\ref{ThMappaAperta}) vale $(T')\ii$ continua, cio\`e $T$ fortemente iniettiva.
\end{itemize}
\setlength{\leftmargini}{0.5cm}
\end{proof}

\begin{proposition}[Retrazioni e sezioni per lineari continue]\label{PrRetrazioniSezioniPerOperatoriLineariContinui}
Sia $T\in L(X,Y)$ con $X,Y$ Banach. Allora $T$ \`e una\footnote{cio\`e esiste $S:Y\to X$ tale che $T$ \`e l'inversa destra / sinistra di $S$.}
\begin{itemize}
    \item inversa destra $\coimplies$ iniettiva e $\imm T$ \`e complementato\footnote{cio\`e esiste $V\subseteq Y$ tale che $Y=\imm T\oplus V$.}
    \item inversa sinistra $\coimplies$ surgettivo e $\ker T$ \`e complementato.
\end{itemize}
\end{proposition}
\begin{proof}
Se $T:X\to Y$ e $S:Y\to X$ sono una coppia tale che $S\circ T=id_X$ allora $T\circ S=P$ \`e un proiettore lineare continuo, infatti
\[P^2=(T\circ S)\circ (T\circ S)=T\circ id_X \circ S=T\circ S.\]
Quindi $Y=\ker P\oplus \imm P$ e $\ker P=\ker S$, $\imm P=\imm T$, ovvero
\[Y=\ker S\oplus \imm T\]
come volevamo.
\medskip

Viceversa, se $T$ \`e iniettivo e $\imm T$ \`e complementata (rispettivamente $S$ \`e surgettivo e $\ker S$ complementato) allora considero un proiettore $P_{\imm T}$ (ok per la decomposizione in somma diretta) e definisco $S=(T')\ii\circ P_{\imm T}$ che \`e inversa sinistra di $T$ (rispettivamente definisco un proiettore $Q$ su $\ker S$ con $id_Y-Q$ proiettore sul supplementare $V$ fissato di $\ker S$, a questo punto considero $S\res V\ii$, che diventa inversa destra).
\end{proof}

\begin{theorem}[Surgettivit\`a e aggiunti]\label{ThSurgettivitaEAggiunti}
Sia $T\in L(X,Y)$ con $X,Y$ banach e tale che $T^\ast$ fortemente iniettivo (iniettivo pi\`u immagine chiusa). Allora $T$ \`e surgettivo.
\end{theorem}
\begin{proof}
Senza perdita di generalit\`a supponiamo $\norm{T^\ast y}\geq \norm y$ (ricordiamo che fortemente iniettivo significa $\norm{T^\ast y}\geq k\norm y$ per qualche $k<1$, ma a meno di riscalare $T$ supponiamo $k=1$). Sia $0<t<1$ e mostriamo che $B_Y\subseteq TB_X+tB_Y$, cos\`i facendo possiamo invocare il lemma di iterazione (\ref{LmDiIterazione}) e mostrare $B_Y\subseteq T(B_X)$, in particolare $T$ \`e surgettivo.

Supponiamo per assurdo che non valga $B_Y\subseteq TB_X+tB_Y$, allora esiste $y_0\in B_Y$ tale che $y_0\notin TB_X+tB_Y$ (convesso aperto). Per il teorema di Hahn-Banach in forma di separazione di aperti convessi (\ref{ThSeparazioneDiConvessi}) esiste $y^\ast_0\in Y^\ast\nz$ tale che $\forall x\in B_X$ e $\forall y\in B_y$ si ha
\[\ps{y^\ast_0,Tx+ty}\leq \ps{y^\ast_0,y_0}\leq \norm{y^\ast_0}\]
Allora
\[\ps{T^\ast y_0^\ast,x}+t\ps{y_0^\ast,y}\leq \norm{y_0^\ast}\]
passando all'estremo superiore per $x\in B_X$ e $y\in B_Y$ troviamo
\[\norm{T^\ast y_0}+t\norm{y^\ast_0}\leq\norm{y_0^\ast}\]
cio\`e $\norm{T^\ast y_0}\leq (1-t)\norm{y^\ast_0}$, contraddicendo l'ipotesi di forte iniettivit\`a ($\norm{T^\ast y}\geq \norm y$ per ogni $y$)
\end{proof}

\begin{remark}
In realt\`a vale anche $T^\ast$ surgettivo se e solo se $T$ fortemente iniettivo.
\end{remark}


\subsection{Polare, prepolare, annullatore, preannullatore}
\begin{definition}[Assolutamente convesso]
Un insieme bilanciato e convesso si dice \textbf{assolutamente convesso}. Per un insieme $S$ ha senso l'\textbf{inviluppo assolutamente convesso}
\begin{align*}
    \assco(S)=&\bigcap_{\smat{C\text{ ass.conv.}\\ C\supseteq S}}C=\cpa{\sum_{i=1}^n\la_i a_i\mid a_1,\cdots, a_n\in S,\ \la_i\in \K,\ \sum_{i=1}^n\abs{\la_i}\leq1}=\\
    =&\ol{B_\K(0,1)}\co(S)
\end{align*}
\end{definition}

\begin{definition}[Polare e prepolare]
Sia $X$ SVT, $A\subseteq X$, $B\subseteq X^\ast$. Definiamo la \textbf{polare di $A$} come
\[A^0=\cpa{x^\ast\in X^\ast\mid \abs{\ps{x^\ast,x}}\leq 1,\ \forall x\in A}=\bigcap_{x\in A}\cpa{x}^0\]
Definiamo $x^0=\cpa{x^\ast\in X^\ast\mid \abs{\ps{x^\ast,x}}\leq 1}\supseteq \ker(\iota_{X}(x))$.

Definiamo il \textbf{prepolare} di $B$ come
\[B_0=\cpa{x\in X\mid \abs{\ps{x^\ast,x}}\leq 1,\forall x^\ast\in B}=\bigcap_{x^\ast\in B}\cpa{x^\ast}_0\]
\end{definition}

\begin{remark}
La polare di un qualche insieme \`e assolutamente convessa e w$^\ast$-chiusa. La prepolare \`e assolutamente convessa e chiusa in $X$ (anche forte).
\end{remark}

\begin{definition}[Annullatore e preannullatore]
Sia $X$ SVT, $A\subseteq X$ e $B\subseteq X^\ast$. Definiamo l'\textbf{annullatore} di $A$ come
\[A^\perp=\cpa{x^\ast\in X^\ast\mid \ps{x^\ast,x}=0,\ \forall x\in A}=\bigcap_{x\in A}\Ann(x)=\bigcap_{x\in A}\cpa{x}^\perp\]
e il \textbf{preannullatore} di $B$ come
\[B_\perp=\cpa{x\in X\mid \forall \ps{x^\ast,x}=0,\ \forall x^\ast\in B}=\bigcap_{x^\ast\in B}\ker x^\ast=\bigcap_{x^\ast\in B}(x^\ast)_\perp.\]
\end{definition}

\begin{remark}
Se $A$ e $B$ sono sottospazi vettoriali o coni in generale allora
\[A^0=A^\perp,\quad B_0=B_\perp.\]
\end{remark}

Da ora in poi supponiamo $(X,\normd)$ normato e sia $i_X:X\inj X^{\ast\ast}$. 

\begin{proposition}[Polare e prepolare in normato]\label{PrPolareEPrepolareInNormato}
Della definizione si ha
\begin{itemize}
    \item $A^0=(i_X(A))_0$
    \item $B_0=i_X\ii(B^0)=B^0\cap X$
\end{itemize}
\end{proposition}
\begin{proof}
Segue dal fatto che $\ps{x^\ast,x}=\ps{i_X(x),x^\ast}$. Per esempio
\begin{align*}
    A^0=&\cpa{x^\ast\in X^\ast\mid \abs{\ps{x^\ast,x}}\leq 1, \forall x\in A}=\\
    =&\cpa{x^\ast\in X^\ast\mid \abs{\ps{i_X(x),x^\ast}}\leq 1, \forall x\in A}=\\
    =&\cpa{x^\ast\in X^\ast\mid \abs{\ps{y,x^\ast}}\leq 1, \forall y\in i_X(A)\subseteq X^{\ast\ast}}=(i_X(A))_0.
\end{align*}
\end{proof}

\begin{remark}
Per le palle unitarie chiuse vale
\[(B_X)^0=B_{X^\ast},\qquad (B_{X^\ast})_0=B_X\]
dove per la seconda uguaglianza usiamo Hahn-Banach per dire $(B_{X^\ast})^0\cap X=B_X$.
\end{remark}

\begin{proposition}
Siano $A\subseteq X$ e $B\subseteq X^\ast$, allora
\[(A^0)_0=\ol{\assco(A)},\qquad (B_0)^0=\ol{\assco(B)}^{w^\ast}.\]
\end{proposition}
\begin{proof}
Dalla definizione \`e chiaro che $A\subseteq (A^0)_0$ e $B\subseteq (B_0)^0$. Poich\'e $(A^0)_0$ \`e assolutamente convesso e chiuso vale
\[(A_0)^0\supseteq\ol{\assco(A)}\]
e per lo stesso motivo $(B_0)^0\supseteq \ol{\assco(B)}^{w^\ast}$.

Sia $a\notin \ol{\assco(A)}$. Per Hahn-Banach (\ref{ThSeparazioneDiConvessi}) esiste\footnote{$X_\R$ \`e $X$ visto come $\R$-spazio vettoriale.} $f_0\in X_\R^\ast$ tale che $\ps{f_0,a}>\gamma\geq \ps{f_0,x}$ per ogni $x\in \ol{\assco(A)}$. A meno di riscalare $f_0$ supponiamo $\gamma=1$. Allora $\abs{\ps{f_0,x}}\leq 1$ per ogni $x\in \ol{\assco(A)}$. 

Se $\K=\R$ poniamo $f=f_0$, se $\K=\C$ allora poniamo $\ps{f,x}=\ps{f_0,x}-i\ps{f_0,ix}$ e notiamo che
\[\sup_{x\in\ol{\assco(A)}}\abs{\ps{f,x}}=\sup_{x\in\ol{\assco(A)}}\abs{\ps{f_0,x}}\]

Dunque $f\in A^0$, ma $\abs{\ps{f,a}}\geq \ps{f_0,a}>1$, quindi $a\notin (A^0)_0$. Questo mostra l'inclusione $(A^0)_0\subseteq \ol{\assco(A)}$.
\end{proof}

\begin{remark}
$(A^\perp)_\perp=\ol{\Span(A)}$ e $(B_\perp)^\perp=\ol{\Span(B)}^{w^\ast}$, infatti polare e prepolare coincidono con annullatore e preannullatore per coni e chiaramente 
\[(A^\perp)_\perp=(\Span(A)^\perp)_\perp,\qquad (B_\perp)^\perp=(\Span(B)_\perp)^\perp.\]
\end{remark}

\begin{remark}
Se $A\subseteq X$ allora $A$ \`e denso se e solo se $A^\perp=(0)$ e $B\subseteq X^\ast$ \`e $w^\ast$-denso se e solo se $B_\perp=(0)$.
\end{remark}

\begin{proposition}[Relazione tra nucleo e immagine tra $T$ e $T^\ast$]\label{PrNucleiImmaginiTrasposteAnnullatoriEPreannullatori}
Se $T\in L(X,Y)$ allora
\begin{itemize}
    \item $\ker T=(\imm T^\ast)_\perp$
    \item $\ker T^\ast=(\imm T)^\perp$
    \item $(\ker T)^\perp =\ol{\imm T^\ast}^{w^\ast}$
    \item $(\ker T^\ast)_\perp=\ol{\imm T}$.
\end{itemize}
\end{proposition}
\begin{proof}
Abbiamo una catena di equivalenze
\begin{gather*}
    x\in \ker T\\
    Tx=0\\
    \ps{y^\ast,Tx}=0\quad \forall y^\ast\in Y^\ast\\
    \ps{T^\ast y^\ast,x}=0\quad \forall y^\ast\in Y^\ast\\
    x\in (\imm T^\ast)_\perp.
\end{gather*}
dove la seconda equivalenza \`e data da Hahn-Banach (\ref{CorHahnBanachPerSpaziNormati}).
Segue che $(\ker T)^\perp=((\imm T^\ast)_\perp)^\perp=\ol{\imm T^\ast}^{w^\ast}$.

L'altro caso si fa allo stesso modo.
\end{proof}

\begin{corollary}[Iniettivit\`a e aggiunti]\label{CorIniettivitaEAggiunti}
Sia $T\in L(X,Y)$, allora
\begin{align*}
    T\text{ iniettivo }\coimplies&\imm T^\ast\text{ \`e w$^{\ast}$-denso in }X^\ast\\
    T^\ast\text{ iniettivo }\coimplies&\imm T\text{ \`e denso in }Y
\end{align*}
\end{corollary}
\begin{proof}
Segue da (\ref{PrNucleiImmaginiTrasposteAnnullatoriEPreannullatori}), dove per\`o per dire che $(\ker T)^\perp=X^\ast\implies \ker T=(0)$ stiamo usando Hahn-Banach (\ref{CorHahnBanachPerSpaziNormati}) (se $\ker T$ contiene un vettore non nullo allora possiamo costruire un elemento di $X^\ast$ che non si annulla su quel vettore, e quindi che non si annulla su $\ker T$).
\end{proof}

\begin{exercise}
Scrivere un criterio per ``essere inverso sinistro lineare" per $T\in L(X,Y)$ deducendolo dal lemma di iterazione.
\end{exercise}


\begin{theorem}[Goldstine]\label{ThGoldstine}
Sia $(X,\normd)$ normato e $B_X=\ol{B_X(0,1)}$, allora
\[\ol{i_X(B_X)}^{\sigma(X^{\ast\ast},X^\ast)}=B_{X^{\ast\ast}}\]
e quindi $\ol{X}^{w^\ast}=X^{\ast\ast}$.
\end{theorem}
\begin{proof}
Calcoliamo (la topologia debole$^\ast$ su $X^{\ast\ast}$ \`e $\sigma(X^{\ast\ast},X^\ast)$):
\begin{align*}
    \ol{i_X(B_X)}^{\sigma(X^{\ast\ast},X^\ast)}=(i_X(B_X)_0)^0\pasgnl={(\ref{PrPolareEPrepolareInNormato})}(B_X^0)^0=(B_{X^\ast})^0=B_{X^{\ast\ast}}
\end{align*}
\end{proof}


\subsection{Caso dei Banach}
\begin{proposition}[Duale di sottospazi e di un quoziente]\label{PrDualeDiSottospaziEDualeQuoziente}
Dato $Y$ sottospazio chiuso di $X$ Banach abbiamo le seguenti isometrie lineari:
\begin{enumerate}
    \item $Y^\ast\cong X^\ast/Y^\perp$
    \item $(X/Y)^\ast\cong Y^\perp\subseteq X^\ast$
\end{enumerate}
\end{proposition}
\begin{proof}
Data l'inclusione $j_Y:Y\to X$ otteniamo $j_Y^\ast:X^\ast\to Y^\ast$. Il nucleo di $j_Y^\ast$ sono i funzionali in $X^\ast$ che si restringono al funzionale nullo su $Y^\ast$, cio\`e gli $f\in X^\ast$ tali che \[j_Y^\ast(f)=f\circ j_Y=f\res Y=0\] e quindi $\ker j_Y^\ast=Y^\perp$. Per il teorema di Hahn-Banach (\ref{ThHahnBanach}), $j_Y^\ast$ \`e surgettiva in quanto ogni funzionale su $Y$ si estende ad uno su $X$ perch\'e $X$ Banach e $Y$ chiuso. Per il teorema di isomorfismo esiste un'unica mappa $\phi$ che fa commutare
% https://q.uiver.app/#q=WzAsMyxbMCwwLCJYXlxcYXN0Il0sWzAsMSwiWF5cXGFzdC9cXEFubihZKSJdLFsxLDAsIlleXFxhc3QiXSxbMCwyLCJqX1leXFxhc3QiXSxbMCwxLCJcXHBpIiwyXSxbMSwyLCJcXHBoaSIsMix7InN0eWxlIjp7ImJvZHkiOnsibmFtZSI6ImRhc2hlZCJ9fX1dXQ==
\[\begin{tikzcd}
	{X^\ast} & {Y^\ast} \\
	{X^\ast/Y^\perp}
	\arrow["{j_Y^\ast}", from=1-1, to=1-2]
	\arrow["\pi"', from=1-1, to=2-1]
	\arrow["\phi"', dashed, from=2-1, to=1-2]
\end{tikzcd}\]
Per questioni di algebra $\phi$ \`e lineare e poich\'e $j_Y^\ast$ \`e continua e $\pi$ induce la topologia quoziente, $\phi$ \`e continua. Verifichiamo che \`e una isometria.
\begin{gather*}
    B_{X^\ast/Y^\perp}(0,1)=\pi(B_{X^\ast}(0,1))\\
    \phi(B_{X^\ast/Y^\perp}(0,1))=\phi(\pi(B_{X^\ast}(0,1)))=j_Y^\ast(B_{X^\ast}(0,1))\pasgnl={(\ref{ThHahnBanach})}B_{Y^\ast}(0,1).
\end{gather*}
dove nell'ultimo passaggio abbiamo usato il fatto che l'estensione data da Hahn-Banach mantiene la norma.

\bigskip
\noindent
Data la proiezione $\pi:X\to X/Y$ otteniamo $\pi^\ast:(X/Y)^\ast\to X^\ast$.
Sia $\vp\in(X/Y)^\ast$ e $f=\pi^\ast(\vp)=\vp\circ \pi$. Si ha che
\[\vp(B_{X/Y})=\vp(\pi(B_X))=f(B_X),\]
quindi $\norm\vp_{(X/Y)^\ast}=\norm f_{X^\ast}$, cio\`e $\pi^\ast$ \`e una immersione isometrica.

Sia $f\in X^\ast$, si ha che $f\in Y^\perp$ se e solo se $Y\subseteq \ker f$ che succede se e solo se $f$ si fattorizza tramite $\pi$ per propriet\`a universale. Quindi $f\in Y^\perp$ se e solo se $f=\vp\circ \pi=\pi^\ast(\vp)$ per qualche $\vp:X/Y\to \R$, cio\`e se e solo se $f\in \pi^\ast((X/Y)^\ast)$. Quindi $\imm \pi^\ast=Y^\perp$.

Restringendo il codominio all'immagine troviamo quanto voluto.
\end{proof}


\begin{proposition}[Banach riflessivi]\label{PrBanachERiflessivoSSEIlDualeERiflessivo}
Sia $X$ banach\footnote{banach serve perch\'e $X^\ast$ \`e isometrico a $\wh X^\ast$ dove $\wh X$ \`e il completamento di $X$.}, allora $X$ \`e riflessivo se e solo se $X^\ast$ \`e riflessivo.
\end{proposition}
\begin{proof}
Ricordiamo che
% https://q.uiver.app/#q=WzAsMyxbMCwxLCJYXlxcYXN0Il0sWzEsMCwiWF57XFxhc3RcXGFzdFxcYXN0fSJdLFsyLDEsIlheXFxhc3QiXSxbMCwxLCJpX3tYXlxcYXN0fSJdLFsxLDIsIihpX1gpXlxcYXN0Il0sWzAsMiwiaWRfe1heXFxhc3R9IiwyXV0=
\[\begin{tikzcd}
	& {X^{\ast\ast\ast}} \\
	{X^\ast} && {X^\ast}
	\arrow["{(i_X)^\ast}", from=1-2, to=2-3]
	\arrow["{i_{X^\ast}}", from=2-1, to=1-2]
	\arrow["{id_{X^\ast}}"', from=2-1, to=2-3]
\end{tikzcd}\]
Se $X$ \`e riflessivo, cio\`e $i_X$ \`e isomorfismo, allora $(i_X)^\ast$ \`e un isomorfismo per funtorialit\`a. Dal diagramma allora segue che $i_{X^\ast}$ \`e un isomorfismo, infatti
\[(i_X)^\ast\circ i_{X^\ast}=id_{X^\ast}\implies i_{X^\ast}=((i_X)^\ast)\ii\circ (i_X)^\ast\circ i_{X^\ast} = ((i_X)^\ast)\ii.\]
Quindi $i_{X^\ast}$ \`e un isomorfismo, cio\`e $X^\ast$ \`e riflessivo.


Viceversa, se $i_{X^\ast}$ \`e un isomorfismo allora $(i_X)^\ast$ \`e un isomorfismo per motivi analoghi a prima, quindi $i_X$ ha immagine densa (iniettivit\`a di $(i_X)^\ast$ e (\ref{PrNucleiImmaginiTrasposteAnnullatoriEPreannullatori})), ma l'immagine di $i_X$ \`e sempre chiusa, quindi $X$ \`e riflessivo (iniettivit\`a di $i_X$ vale sempre perch\'e immersione isometrica). 
\end{proof}

\begin{remark}
Se $X$ non \`e riflessivo allora nessun duale successivo pu\`o essere riflessivo.
\end{remark}



\begin{theorem}[Immagine chiusa]\label{ThImmagineChiusa}
Siano $X,Y$ banach, $T\in L(X,Y)$, allora sono equivalenti
\begin{enumerate}
    \item $\imm T$ \`e $\normd$-chiuso
    \item $\imm T$ \`e $w$-chiuso
    \item $\imm T^\ast=(\ker T^\ast)_\perp$
    \item $\imm T^\ast$ \`e $\normd$-chiuso
    \item $\imm T^\ast$ \`e $w^\ast$-chiuso
    \item $\imm T^\ast=(\ker T)^\perp$
\end{enumerate}
\end{theorem}
\begin{proof}
1. e 2. sono sempre equivalenti per sottospazi vettoriali.
\[\ol{\imm T}\pasgnl={(\ref{PrNucleiImmaginiTrasposteAnnullatoriEPreannullatori})}((\ker T^\ast)_\perp)\]
quindi 2. \`e equivalente a 3. Similmente 5. e 6. sono equivalenti per $\ol{\imm T^\ast}^{w^\ast}=(\ker T)^\perp$. Poich\'e la topologia debole$^\ast$ \`e meno fine della topologia forte, 5. implica 4.

Resta da mostrare solo 4.$\implies$1. e 1.$\implies$6.
\setlength{\leftmargini}{0cm}
\begin{itemize}
\item[$\boxed{4.\implies1.}$] Supponiamo $\imm T^\ast$ chiuso forte. Siano $Z=\ol{\imm T}$ e $S:X\to Z$ la mappa ottenuta da $T$ restringendo il codominio, che possiamo fare perch\'e $\imm T\subseteq Z$.

Per costruzione $\imm S$ \`e densa in $Z$ e la tesi \`e $S$ surgettiva. Dualizzando la successione
% https://q.uiver.app/#q=WzAsMyxbMCwxLCJYIl0sWzEsMCwiWiJdLFsyLDEsIlkiXSxbMCwxLCJTIl0sWzEsMiwiIiwwLHsic3R5bGUiOnsidGFpbCI6eyJuYW1lIjoiaG9vayIsInNpZGUiOiJ0b3AifX19XSxbMCwyLCJUIiwyXV0=
\[\begin{tikzcd}
	& Z \\
	X && Y
	\arrow[hook, from=1-2, to=2-3]
	\arrow["S", from=2-1, to=1-2]
	\arrow["T"', from=2-1, to=2-3]
\end{tikzcd}\]
troviamo
% https://q.uiver.app/#q=WzAsMyxbMiwxLCJYXlxcYXN0Il0sWzEsMCwiWl5cXGFzdCJdLFswLDEsIlleXFxhc3QiXSxbMiwwLCJUXlxcYXN0IiwyXSxbMSwwLCJTXlxcYXN0Il0sWzIsMV1d
\[\begin{tikzcd}
	& {Z^\ast} \\
	{Y^\ast} && {X^\ast}
	\arrow["{S^\ast}", from=1-2, to=2-3]
	\arrow[from=2-1, to=1-2]
	\arrow["{T^\ast}"', from=2-1, to=2-3]
\end{tikzcd}\]
dove la mappa $Y^\ast\to Z^\ast$ \`e la restrizione del dominio, che \`e surgettiva per il teorema di Hahn-Banach (\ref{CorHahnBanachPerSpaziNormati}), dunque $S^\ast(Z)=T^\ast(Y)$. Notiamo che $T^\ast(Y)$ \`e chiuso in norma, quindi anche $S^\ast(Z)$ lo \`e. Poich\'e $\imm S$ \`e densa, $S^\ast$ \`e iniettiva, quindi per la caratterizzazione (\ref{ThSurgettivitaEAggiunti}) $S^\ast$ \`e fortemente iniettivo e perci\`o $S$ \`e surgettivo.
\item[$\boxed{1.\implies6.}$] \`E sempre vero che $\imm T^\ast\subseteq (\ker T)^\perp$ in quanto $(\ker T)^\perp$ \`e la chiusura di $\imm T$ per la topologia debole$^\ast$ (\ref{PrNucleiImmaginiTrasposteAnnullatoriEPreannullatori}).

Sia $x^\ast\in (\ker T)^\perp$, cio\`e $\ker T\subseteq \ker x^\ast$. Consideriamo la mappa lineare (a priori non continua)
\[\xi:\funcDef{\imm T}{\K}{T(x)}{x^\ast(x)}\]
che \`e ben definita perch\'e se $T(x)=T(x')$ allora $x-x'\in \ker T\subseteq \ker x^\ast$. Notiamo che $x^\ast=\xi\circ T$.

Poich\'e $\imm T$ \`e chiuso (stiamo assumendo 1.) esso \`e banach, quindi si ha che $T$ \`e aperta come mappa $X\to \imm T$ per il teorema della mappa aperta (\ref{ThMappaAperta}), quindi induce la topologia quoziente, il che significa che $\xi$ era un funzionale lineare CONTINUO.

Per il teorema di Hahn-Banach (\ref{CorHahnBanachPerSpaziNormati}) $\xi$ si estende a $y^\ast\in Y^\ast$ e poich\'e $x^\ast=\xi\circ T$ si ha $x^\ast=y^\ast\circ T=T^\ast(y^\ast)$, cio\`e $x^\ast\in \imm T^\ast$.
\end{itemize}
\setlength{\leftmargini}{0.5cm}
\end{proof}

Abbiamo dunque
\[T\text{surg.}\coimplies \begin{cases}
    \imm T&\text{chiusa}\\
    \imm T&\text{densa}
\end{cases} \coimplies \begin{cases}
    \imm T^\ast&\text{chiusa}\\
    T^\ast&\text{iniettiva}
\end{cases}\coimplies T^\ast\text{ fortemente iniettiva}\]



\chapter{Separabilit\`a e Spazi uniformemente convessi}

\section{Separabilit\`a vs Metrizzabilit\`a}

[RISCRIVERE POI PERCH\'E NON SI CAPISCE NIENTE]

\begin{lemma}\label{LmFunzionaleAssumeDistanzaPerFissatoElemento}
Se $Y$ \`e normato, $Z\subseteq Y$ e $g\in Y$ allora esiste $\vp\in Y^\ast$ tale che $\norm\vp=1$, $\ps{\vp,g}=dist(g,Z)$ e $\vp\in Z^\perp$.
\end{lemma}
\begin{proof}
	Sia $\pi:Y\to Y/Z$ la mappa quoziente. Applichiamo Hahn-Banach (\ref{ThHahnBanach}) a $Y/Z$: esiste $\psi\in (Y/Z)^\ast$ tale che \[\ps{\psi,\pi(g)}=\norm{\pi g}=dist(g,Z)\]
	di norma 1. Poniamo $\vp=\pi^\ast\psi$.
	\[\ps{\vp,g}=\ps{\pi^\ast\psi,g}=\ps{\psi,\pi g}=dist(g,Z)\]
	e $\norm{\vp}=\norm\psi=1$ perch\'e $\pi^\ast:(Y/Z)^\ast\to Z^\perp\subseteq Y^\ast$ \`e una isometria.
\end{proof}

\begin{theorem}[Separabilit\`a in termini di metrizzabilit\`a di palle]\label{ThSeparabilitaInTerminiDiMetrizzabilitaDiPalle}
Sia $X$ spazio normato. Siano $B_X$ e $B_{X^\ast}$ palle unitarie chiuse.
\begin{enumerate}
	\item se $X^\ast$ \`e $\normd$-separabile allora anche $X$ lo \`e.
	\item $X$ \`e $\normd$-separabile se e solo se $(B_{X^\ast},\sigma(X^\ast,X))$ \`e metrizzabile
	\item $X^\ast$ \`e $\normd$-separabile se e solo se $(B_X,\sigma(X,X^\ast))$ \`e metrizzabile
\end{enumerate}
\end{theorem}
\begin{proof}
Mostriamo le proposizioni:
\setlength{\leftmargini}{0cm}
\begin{enumerate}
	\item Sia $X^\ast$ separabile e sia $\cpa{f_k}$ numerabile denso. Per ogni $k\in\N$ sia $x_k\in X$ tale che
	\[\begin{cases}
		\norm{x_k}=1\\
		\abs{\ps{f_k,x_k}}\geq \frac12\norm{f_k}
	\end{cases}\]
	Affermiamo che $Y=\Span(\cpa{x_k}_{k\in\N})$ \`e denso in $X$: basta verificare che $\cpa{x_k}_{k\in\N}^\perp=(0)$ per (\ref{PrNucleiImmaginiTrasposteAnnullatoriEPreannullatori}). Sia $f\in X^\ast$ tale che $\ps{f,x_k}=0$ per ogni $k$ e sia $f_{k_j}$ una sottosuccessione di $\cpa{f_k}$ che converge a $f$ in norma. Allora
	\begin{align*}
		\frac12\norm{f_{k_j}}\leq& \abs{\ps{f_{k_j},x_{k_j}}}\leq\abs{\ps{f_{k_j}-f,x_{k_j}}}+\abs{\ps{f,x_{k_j}}}=\\
		=&\abs{\ps{f_{k_j}-f,x_{k_j}}}\leq \norm{f_{k_j}-f}\under{=1}{\norm{x_{k_j}}}=o_{j\to\infty}(1)
	\end{align*}
	dove quella norma \`e un $o(1)$ perch\'e $f_{k_j}\to f$.
	\item Diamo le due implicazioni
	\setlength{\leftmargini}{0cm}
	\begin{itemize}
	\item[$\boxed{\implies}$] Sia $X$ separabile e $\cpa{x_{k}}_{k\geq1}$ numerabile denso in $B_X$. Definiamo una norma su $X^\ast$ ponendo
	\[\abs{\norm{f}}=\sum_{n\geq 1}2^{-n}\abs{\ps{f,x_n}}\overset{\forall m}\geq 2^{-m}\ps{f,x_m}.\]
	Per costruzione
	\[\abs{\norm f}\leq \sum_{n\geq 1}2^{-n}\abs f\norm{x_k}\leq \pa{\sum_{n\geq 1}2^{-n}}\norm f=\norm f,\]
	cio\`e $\abs{\normd}$ \`e meno fine di $\normd$. 

	Affermiamo che $id:(B_{X^\ast},\abs\normd)\to (B_{X^\ast},\sigma(X^\ast,X))$ \`e un omeomorfismo\footnote{nota che $id:({X^\ast},\abs\normd)\to ({X^\ast},\sigma(X^\ast,X))$ non potrebbe esserlo se $\dim X\geq \aleph_0$.}. 
	
	Poich\'e il dominio \`e metrico basta mostrare la continuit\`a sequenziale. Sia allora $(f_k)_{k\in \N}$ una successione in $B_{X^\ast}$ con $f_k\to f\in B_{X^\ast}$ convergente per $\abs\normd$. Vogliamo mostrare che $f_k\to f$ rispetto alla norma debole$^\ast$. Senza perdita di generalit\`a supponiamo $f=0$ (altrimenti basta considerare $\frac12(f_k-f)$).

	Se $\abs{\norm{f_k}}\to 0$ allora $\abs{\norm{f_k}}\geq 2^{-n}\abs{\ps{f_k,x_n}}=o_k(1)$ per ogni $n$, quindi $f_k$ converge puntualmente a $0$ su $(x_n)$. Inoltre le $f_k$ sono funzioni $1$-Lipschitz $B_X\to \K$ e l'insieme di convergenza di una successione di funzioni equicontinue (a valori in spazio metrico completo) \`e sempre un chiuso per Ascoli Arzel\`a. Dunque le successioni convergono puntualmente dappertutto per densit\`a su $B_X$ (quindi anche su $X$ per omogeneit\`a).

	Il limite \`e 0 perch\'e \`e sono funzioni $1$-Lipschitz.

	Allora $f_k\to 0$ nella topologia debole$^\ast$ perch\'e questa \`e esattamente la topologia indotta dalla topologia prodotto.


	Questo mostra la continuit\`a di $id:(B_{X^\ast},\abs\normd)\to (B_{X^\ast},\sigma(X^\ast,X))$. Se il dominio \`e compatto allora abbiamo una mappa bigettiva, continua da compatto in Hausdorff, dunque \`e un omeomorfismo. $(B_{X^\ast},\abs\normd)$ \`e compatto sequenzialmente perch\'e, se $\cpa{f_k}$ \`e una successione in $B_{X^\ast}$ allora le $f_k$ sono $1$-Lipschitz e limitate come funzioni su $B_X$, quindi per argomento diagonale (vedi Ascoli-Arzel\`a) esiste una sottosuccessione $f_{k_j}$ convergente su ogni $x_n$. Essendo questa sottosuccessione equicontinua essa converge su tutto $X$ puntualmente. Il limite \`e $f$ lineare su $X$ e $1$-lipschitz e quindi $f\in X^\ast$. Infine $\abs{\norm {f_{k_j}-f}}\to 0$ perch\'e
	\[\abs{\norm {f_{k_j}-f}}=\sum_{n\geq 1}2^{-n}\abs{\ps{f_{k_j}-f,x_n}}=o_j(1)\]
	dove l'ultima ugualgianza vale perch\'e ogni termine \`e infinitesimo ed \`e dominata dalla serie geometrica di fattore $1/2$.

	Questo mostra che $(B_{X^\ast},\abs\normd)$ \`e sequenzialmente compatto e questo conclude.
	\item[$\boxed{\impliedby}$] Supponiamo $(B_{X^\ast},\sigma(X^\ast,X))$ metrizzabile. Osserviamo che per $F\in \Ps_{fin}(X)$ si ha che
	\[F^0=\cpa{x^\ast\in X^\ast\mid \abs{\ps{x^\ast,x}}\leq 1\ \forall x\in F}=\bigcap_{x\in F}\cpa{x}^0\]
	\`e un intorno di $0$ ($F$ \`e \emph{finito}) in $(X^\ast,\sigma(X^\ast,X))$, in realt\`a questi sono una base di intorni per la topologia $\sigma(X^\ast,X)$.

	Se $(B_{X^\ast},\sigma(X^\ast,X))$ \`e metrizzabile allora in particolare \`e I-numerabile, quindi esiste una successione $(F_n)_{n\geq 0}\subseteq \Ps_{fin}(X)$ tale che $(F_n^0\cap B_{X^\ast})_{n\geq 0}$ \`e una base di intorni di $0$. Notiamo in particolare che $\bigcap_{n\geq 0}F_n^0\cap B_{X^\ast}=(0)$.

	Senza perdita di generalit\`a supponiamo anche $F_{n+1}\supseteq 2F_n$ (se avevamo una successione valida basta aggiungere la riscalatura del termine prima e l'insieme resta finito). Ricordiamo che se $A\subseteq B$ allora $B^0\subseteq A^0$, quindi gli intorni che prendiamo diventano inscatolati.
	\begin{align*}
		(0)=&\bigcap_{n\geq 0}(F_n^0\cap B_{X^\ast})=\pa{\bigcap_{n\geq 0}F_n^0}\cap B_{X^\ast}=\pa{\bigcup_{n\geq 0} F_n}^0\cap B_{X^\ast}=\\
		=&\pa{\assco\pa{\bigcup_{n\geq 0} F_n}}^0\cap B_{X^\ast}\overset{(\star)}=\pa{\Span\pa{\bigcup_{n\geq 0}F_n}}^0\cap B_{X^\ast}=\\
		=&\pa{\Span\pa{\bigcup_{n\geq0}F_n}}^\perp\cap B_{X^\ast}
	\end{align*}
	dove l'uguaglianza $(\star)$ vale perch\'e per ipotesi $2\bigcup F_n\subseteq \bigcup F_n$, quindi prendendo l'inviluppo assolutamente convesso troviamo esattamente lo $\Span$ lineare:
	se $\sum \la_i s_i\in \Span\pa{\bigcup_{n\geq0}F_n}$ allora
	\[\sum \la_i s_i=\sum \frac{\la_i}{2^N} (2^N s_i)\in \assco\pa{\bigcup_{n\geq 0} F_n}\quad\text{ per $N$ tale che }\sum \frac{\abs{\la_i}}{2^N}\leq 1.\]
	Dunque, poich\'e $\pa{\Span\pa{\bigcup_{n\geq0}F_n}}^\perp\cap B_{X^\ast}=(0)$ e $B_{X^\ast}$ \`e una palla, 
	\[\pa{\Span\pa{\bigcup_{n\geq0}F_n}}^\perp=(0),\]
	cio\`e $\Span\pa{\bigcup_{n\geq0}F_n}$ \`e denso in $X$ (per $\normd$). Quindi $X$ \`e $\normd$-separabile se consideriamo $\Span_\Q\pa{\bigcup_{n\geq0}F_n}$ ($\bigcup_{n\geq0}F_n$ \`e numerabile perch\'e unione numerabile di finiti).
	\end{itemize}
	\setlength{\leftmargini}{0.5cm}
	\item Diamo le due implicazioni
	\setlength{\leftmargini}{0cm}
	\begin{itemize}
	\item[$\boxed{\implies}$] Segue dalla stessa freccia nel caso 2. notando che $(B_{X^{\ast\ast}},\sigma(X^{\ast\ast},X^\ast))$ \`e metrizzabile e quindi anche $(B_{X},\sigma(X^{\ast\ast},X^\ast))=(B_{X},\sigma(X,X^\ast))$ lo \`e.
	\item[$\boxed{\impliedby}$] Sia $(B_X,\sigma(X,X^\ast))$ metrizzabile. Come per il punto 2. si ha che ogni $F\in \Ps_{fin}(X^\ast)$ definisce
	\[F_0=\cpa{x\in X\mid \abs{\ps{x^\ast,x}}\leq 1\ \forall x^\ast\in F}=\bigcap_{x^\ast\in F}\cpa{x^\ast}_0\]
	intorno di $0$ in $(X,\sigma(X,X^\ast))$. La famiglia $\cpa{F_0}_{F\in\Ps_{fin}(X^\ast)}$ \`e quindi una base di intorni di $0$ rispetto a $\sigma(X,X^\ast)$. Poich\'e $B_X$ \`e $w$-metrizzabile essa \`e I-numerabile quindi esiste una successione $(F_n)_{n\geq 0}\subseteq \Ps_{fin}(X^\ast)$ tale che $(F_n)_0\cap B_X=i_X(F_n^0)\cap B_X\doteqdot F_n^0\cap B_X$ sono una base di $\sigma(X,X^\ast)$ ristretta a $B_X$.

	In particolare $\bigcap_{n\geq 0}(F_n^0\cap B_X)=(0)$. Assumiamo inoltre $F_{n+1}\supseteq 2F_n$ come prima.\footnote{
	Potremmo provare a ragionare come per il punto 2.:
	\[(0)=\pa{\Span\pa{\bigcup_{n\geq 0}F_n}}^\perp\cap B_X,\]
	quindi $\Span\pa{\bigcup_{n\geq 0}F_n}_\perp=(0)$, cio\`e $\Span\pa{\bigcup_{n\geq 0}F_n}$ \`e $w^\ast$-denso. Questo non basta.}
	Supponiamo per assurdo che $\Span\pa{\bigcup_{n\geq 0}F_n}$ non sia $\normd$-denso, cio\`e 
	\[Z=\ol{\Span\pa{\bigcup_{n\geq 0}F_n}}^{\normd}\neq X^\ast,\] cio\`e esiste $g\in X^\ast\bs Z$.

	Per il lemma (\ref{LmFunzionaleAssumeDistanzaPerFissatoElemento}) esiste $\vp\in X^{\ast\ast}$ tale che $\norm \vp=1$, $Z\subseteq \ker \vp$ e $\ps{\vp,g}=dist(g,Z)$. A meno di cambiare $g$ supponiamo $dist(g,Z)=1$. 
	
	Notiamo che $\cpa{x\in B_X\mid \ps{g,x}<\frac12}$ \`e un intorno di $0\in B_X$ nella topologia $\sigma(X,X^\ast)$, quindi contiene un intorno di base $F_m^0\cap B_X$. Poniamo
	\[A=\cpa{\eta\in X^{\ast\ast}\mid \ps{\eta,g}>\frac12,\ \abs{\ps{\eta,f}}<1\ \forall f\in F_m}.\]
	$A$ \`e aperto in $\sigma(X^{\ast\ast},X^\ast)$ perch\'e intersezione finita di aperti (la condizione su $\frac12$ e una per ogni elemento di $F_m$). Notiamo che $\vp\in A$ perch\'e $\ps{\vp,g}=1$ e $\ps{\vp,f}=0$ per ogni $f\in Z\supseteq F_m$.

	Per Goldstine (\ref{ThGoldstine}) $\ol{B_X}^{w^\ast}=B_{X^{\ast\ast}}$ ma si ha che $A\cap B_X\neq \emptyset$ perch\'e $\vp\in A\cap B_{X^{\ast\ast}}=A\cap \ol{B_X}^{w^\ast}$.

	Quindi esiste $\wt x\in B_X$ tale che $i_X(\wt x)\in A$, cio\`e $\ps{g,\wt x}>\frac12$ e $\abs{\ps{f,\wt x}}<1$ per ogni $f\in F_m$, cio\`e dalla seconda condizione $\wt x\in F_m^0\cap B_X$ ma questo era esattamente l'intorno che avevamo scelto dentro $\cpa{g<\frac12}$, quindi $g(\wt x)>\frac12$ e $g(\wt x)<\frac12$ assurdo.
	\end{itemize}
	\setlength{\leftmargini}{0.5cm}
\end{enumerate}
\setlength{\leftmargini}{0.5cm}
\end{proof}

\begin{exercise}
Sia $X$ spazio vettoriale con due norme $\normd_1$ e $\normd_2$ tali che $\normd_2$ \`e pi\`u fine di $\normd_1$ ($\norm x_1\leq \norm x_2$ per ogni $x\in X$)\footnote{Come notazione $(X_1,\normd_1)=(X,\normd_1)$ e $(X_2,\normd_2)=(X,\normd_2)$.}.
\begin{itemize}
	\item $B_{X_2}$ \`e $\sigma(X_1,X_1^\ast)$-metrizzabile se e solo se $X_1^\ast$ \`e separabile rispetto a $\normd_{X_2^\ast}$.
\end{itemize}
\end{exercise}


\section{Spazi uniformemente convessi}
\begin{definition}[Norma uniformemente convessa]
Per uno spazio normato $(X,\normd)$, la norma si dice \textbf{uniformemente convessa} se per ogni $\e>0$ esiste $\delta>0$ tale che per ogni $x,y\in B_X$ si ha
\[\norm{\frac{x+y}2}>1-\delta\implies \norm{x-y}<\e.\]
\end{definition}
\begin{example}
Se $H$ \`e uno spazio di Hilbert allora \`e uniformemente convesso e questo \`e testimoniato dalla identit\`a del parallelogramma: 
\[\norm{x+y}^2+\norm{x-y}^2=2(\norm x^2+\norm y^2)\]
e quindi se $\norm x,\norm y\leq 1$ allora
\[\norm{x-y}\leq\sqrt{4-\norm{x+y}^2}= 2\pa{1-\pa{\frac{\norm{x+y}}2}^2}^{1/2}\]
\end{example}

\begin{example}
La norma $\normd_p$ su $\R^2$ per $1<p<\infty$ \`e uniformemente convessa, anche $\normd_p$ su $L^p$.
\end{example}

\begin{theorem}[Milman-Pettis]\label{ThMilmanPettis}
Spazi di Banach uniformemente convessi sono riflessivi.
\end{theorem}
\begin{proof}[Dimostrazione (di Kakutani).]
Sia $(X,\normd)$ banach U.C. e sia $\eta\in X^{\ast\ast}$. Vogliamo mostrare che $\eta$ \`e una valutazione $val_{\wt x}$ per qualche $\wt x\in X$.

Per ogni $k\geq 1$ siano $\delta_k>0$ come nella definizione di U.C. per $\e=1/k$, cio\`e per ogni $x,y\in B_X$ vale
\[\norm{\frac{x+y}2}>1-\delta_k\implies \norm{x-y}<\frac1k.\]
Senza perdita di generalit\`a supponiamo $\delta_k\to 0$. Sia $(f_k)$ una successione in $B_{X^\ast}$ massimizzante per $\norm\eta$, cio\`e:
\[\norm{\eta}\doteqdot \sup_{\norm f=1}\abs{\ps{\eta,f}}\overset{S.P.G.}=1,\]
allora $\norm{f_k}=1$ e $\ps{\eta,f_k}>1-\delta_k$ per ogni $k\geq 1$.
Sia $f_0\in X^\ast$ qualsiasi e $\delta_0=+\infty$. Definiamo
\[A_n=\cpa{\theta\in X^{\ast\ast}\mid \abs{\ps{\theta,f_k}-\ps{\eta,f_k}}<\frac1n\ \text{e }\ps{\theta,f_k}>1-\delta_k\ \forall k\in\cpa{0,\cdots,k}}\]
Notiamo che $A_n$ \`e un intorno aperto di $\eta$ per la topologia $\sigma(X^{\ast\ast},X^\ast)$, quindi per Goldstine (\ref{ThGoldstine}) si ha $A_n\cap i_X B_X\neq \emptyset$, dunque esiste $x_n\in B_X$ tale che $i_X(x_n)\in A_n$, ovvero (ricorda che $i_X(x_n)=val_{x_n}$)
\[\begin{cases}
\abs{\ps{f_k,x_n}-\ps{\eta,f_k}}<\frac1n&\\
\ps{f_k,x_n}>1-\delta_k &\forall k\leq n
\end{cases}\]
Per $1\leq p<q<\infty$ si ha
\[\norm{\frac{x_p+x_q}2}\geq \ps{f_p,\frac{x_p+x_q}2}=\frac12\ps{f_p,x_p}+\frac12\ps{f_p,x_q}\geq 1-\delta_k\]
e quindi $\norm{x_p-x_q}\leq \frac1p$, cio\`e $(x_n)$ \`e una successione di Cauchy. Poich\'e $X$ \`e un Banach e questi punti stanno in $B_X$ si ha che la successione converge a $\wt x\in B_X$. Prendendo il limite in $n$ del sistema sopra troviamo 
\[\begin{cases}
\ps{f_k,\wt x}=\ps{\eta,f_k}&\forall k\\
\ps{f_k,\wt x}\geq 1-\delta_k &\forall k
\end{cases}\]
Notiamo che il sistema di equazioni $\ps{f_k,x}=\ps{\eta,f_k}$ al variare di $k$ ha una unica soluzione in $B_X$, ovvero $\wt x$: se $\ps{f_k,\wt y}=\ps{\eta,f_k}$ allora per ogni $k$
\[1\geq \norm{\frac{\wt x+\wt y}2}\geq \ps{f_k,\frac{\wt x+\wt y}2}=\ps{\eta,f_k}\geq 1-\delta_k\]
e quindi $\norm{\frac{\wt x+\wt y}2}=1$, ma allora per uniforme convessit\`a $\wt x=\wt y$.

Quindi, a prescindere dalla scelta di $f_0$ troviamo sempre lo stesso $\wt x$, dunque per ogni $f\in X^\ast$ vale $\ps{f,\wt x}=\ps{\eta,f}$ perch\'e $0$ era incluso nel sistema che ci stavamo portando dietro. Abbiamo quindi mostrato che $val_{\wt x}(f)=\eta(f)$ per ogni $f$, cio\`e $\eta=val_{\wt x}$.
\end{proof}
\begin{proof}[Dimostrazione (via nets)]
Sia $\eta\in X^{\ast\ast}$ con $\norm \eta=1$. Per Goldstine (\ref{ThGoldstine}) si ha $\ol{B_X}^{\sigma(X^{\ast\ast},X^\ast)}=B_{X^{\ast\ast}}$ quindi esiste un net $x:D\to B_X$ convergente a $\eta$ in $\sigma(X^{\ast\ast},X^\ast)$.

Consideriamo ora il nuovo net $x_\al+x_\beta:D\times D\to X$ e notiamo che $x_\al+x_\beta\to 2\eta$. Siano $\e>0$ e $\delta>0$ come nella definizione di uniforme convessit\`a e sia $f\in X^\ast$ tale che $\norm f=1$ e $\ps{\eta,f}> 1-\delta$ (ok perch\'e $\norm \eta=1$).

Allora $\ps{f,\frac{x_\al+x_\beta}2}=\ps{\frac{val_{x_\al+x_\beta}}2,f}\to \ps{\eta, f}>1-\delta$, quindi
\[\norm{\frac{x_\al+x_\beta}2}\geq \ps{f,\frac{x_\al+x_\beta}2}\geq 1-\delta\]
definitivamente e quindi
\[\norm{x_\al-x_\beta}\leq \e\]
definitivamente, quindi $x_\al$ \`e un net di Cauchy e quindi converge a $\wt x\in X$ perch\'e $X$ \`e Banach e quindi \`e completo anche per nets. Concludiamo notando che $\wt x=\eta$ per unicit\`a del limite. 
\end{proof}

\begin{example}
Per $1<p<\infty$ gli spazi $(L^p(X,\mu),\normd_p)$ sono uniformemente convessi e quindi riflessivi per Milman Pettis (\ref{ThMilmanPettis}).
\end{example}

\begin{exercise}
Isomorfismo tra $L^q$ e $(L^p)^\ast$.
\end{exercise}
\begin{proof}
Considerare per $p,q$ coniugati
\[T_{p,g}:\funcDef{L^q}{(L^p)^\ast}{g}{f\mapsto \int_X fg d\mu}.\]
Questa mappa \`e lineare e isometrica per H\"older, infatti
\[\abs{\int_X fg d\mu}\leq \norm f_p\norm g_q\]
e quindi $\norm{T_{p,q}g}\leq \norm g_q$, cio\`e $T_{p,g}$ \`e continuo con norma degli operatori $\leq 1$. In realt\`a \`e una isometria perch\'e possiamo scegliere una $f$ opportuna tale che $\norm f_p=1$ e $T_{p,g}(g)(f)=\norm g_q$.

Per provare che $T_{p,q}$ sono surgettive l'idea \`e considerare $\al$ come sotto
% https://q.uiver.app/#q=WzAsMyxbMCwwLCJMXnAiXSxbMSwwLCIoTF5wKV57XFxhc3RcXGFzdH0iXSxbMiwwLCIoTF5xKV5cXGFzdCJdLFsxLDIsIlRfe3AscX1eXFxhc3QiXSxbMCwxLCJpX3tMXnB9Il0sWzAsMiwiXFxhbHBoYSIsMix7ImN1cnZlIjozfV1d
\[\begin{tikzcd}
	{L^p} & {(L^p)^{\ast\ast}} & {(L^q)^\ast}
	\arrow["{i_{L^p}}", from=1-1, to=1-2]
	\arrow["\alpha"', curve={height=18pt}, from=1-1, to=1-3]
	\arrow["{T_{p,q}^\ast}", from=1-2, to=1-3]
\end{tikzcd}\]
e notare che $\al=T_{q,p}$.

Per Milman-Pettis (\ref{ThMilmanPettis}) la $i_{L^p}$ \`e isometrica, quindi si ha che $T_{p,q}$ \`e surgettivo se e solo se $T_{q,p}^\ast$ \`e surgettivo, ma $T_{q,p}^\ast$ \`e surgettivo se e solo se (\ref{PrNucleiImmaginiTrasposteAnnullatoriEPreannullatori}) $T_{q,p}$ \`e fortemente iniettivo e questo \`e vero.
\end{proof}

\chapter{Compattezza nei Banach}

\section{Compattezza dei polari: Banach-Alaoglu}
\begin{theorem}[Banach-Alaoglu-Bourbaki]\label{ThBanachAlaogluBourbaki}
Sia $X$ SVT e $V\in \Uc_X$. Allora il polare di $V$
\[V^0=\cpa{f\in X^\ast\mid \abs{\ps{f,x}}\leq 1\ \forall x\in V}\]
\`e compatto nella topologia $\sigma(X^\ast,X)$, cio\`e\footnote{propriet\`a universale} quella indotta su $X^\ast$ dalla topologia prodotto su $\K^X$.
\end{theorem}
\begin{proof}
Senza perdita di generalit\`a supponiamo $V$ assolutamente convesso e chiuso:
\[V^0\pasgnl={{(\ref{PrPolarePrepolareIteratiDannoChiusuraAssolutamenteConvessa})}}\ol{\assco(V)}^0.\]
Sia allora $V$ intorno assolutamente convesso chiuso di $0$ in $X$. Sia $p$ il funzionale di Minkowski di $V$. Notiamo che $p$ \`e una seminorma su $X$ e (\ref{PrProprietaFunzionaliMinkowski}) $V=\ol{B_p(0,1)}$. Notiamo che $f\in V^0$ se e solo se
\[\abs{\ps{f,x}}\leq p(x)\quad \forall x\in X\]
infatti se $\abs{\ps{f,x}}\leq 1$ per ogni $x\in V$ allora per $x\in X$ con $p(x)\neq 0$ si ha $p(x/p(x))=1$ e quindi $x/p(x)\in V=\ol{B_p(0,1)}$, ma allora $\abs{\ps{f,x/p(x)}}\leq 1$, cio\`e $\abs{\ps{f,x}}\leq p(x)$. Se in vece $p(x)=0$ allora $\Span(x)\in V$ per definizione di $p$, quindi $\ps{f,x}=0$ e vale comunque $\abs{\ps{f,x}}\leq p(x)$.

Viceversa, se $\abs{\ps{f,x}}\leq p(x)$ per ogni $X$ in particolare per $x\in V$, poich\'e l\`i abbiamo $p(x)\leq 1$ abbiamo $\abs{\ps{f,x}}\leq p(x)\leq 1$ per $x\in V$.

Notiamo che la condizione $\abs{\ps{f,x}}\leq 1$ su $V$ assicura che $f$ sia continua (perch\'e limitata in intorno di 0 (\ref{PrCaratterizzazioneFunzionaliContinui})), quindi possiamo scrivere
\[V^0=\cpa{f\in X'_{alg}\mid \abs{\ps{f,x}}\leq p(x)\ \forall x\in X}=X'_{alg}\cap\under{\text{compatto per Tychonoff}}{\prod_{x\in X}\ol{B_\K(0,p(x))}}\subseteq X^\ast\subseteq \K^X.\]

Osserviamo che $X'_{alg}$ \`e chiuso in $\K^X$ perch\'e si scrive come intersezione di chiusi per la topologia prodotto di $\K^X$
\begin{align*}
    X'_{alg}=&\bigcap_{\smat{\al,\beta\in\K\\x,y\in X}}\cpa{f\in\K^X\mid P_{\al x+\beta y}(f)-\al P_x(f)-\beta P_y(f)=0}=\\
    =&\bigcap_{\smat{\al,\beta\in\K\\x,y\in X}}\ker\pa{P_{\al x+\beta y}-\al P_x-\beta P_y}.
\end{align*}
Quindi $V^0$ si identifica con un chiuso in un compatto per la topologia prodotto, e quindi \`e compatto per la topologia prodotto su $\K^X$ in quanto \`e uno spazio Hausdorff.
\end{proof}


\begin{corollary}
Se $X$ \`e Banach allora la palla duale chiusa $\ol{B_{X^\ast}(0,1)}$ \`e compatta per la topologia $w^\ast$ su $X^\ast$.
\end{corollary}

\begin{remark}
Da questo corollario scendono varie applicazioni, per esempio al calcolo delle variazioni ma non solo.
\end{remark}



\begin{theorem}[Kakutani]\label{ThKakutani}
Uno spazio $X$ di Banach \`e riflessivo se e solo se $B_X$ (palla unitaria chiusa) \`e $w$-compatta.
\end{theorem}
\begin{proof}
Se $X$ \`e riflessivo allora $i_X: (B_{X}, w)\to (B_{X^{\ast\ast}},w^\ast)$ \`e un omeomorfismo e quindi $(B_X,w)$ \`e compatta per Banach-Alaoglu (\ref{ThBanachAlaogluBourbaki}).

Supponiamo dunque $B_X$ compatta in $\sigma(X,X^\ast)$, allora anche $i_X(B_X)$ \`e compatta in $X^{\ast\ast}$ per $\sigma(X^{\ast\ast},X^\ast)$, in particolare \`e chiusa. Per il teorema di Goldstine (\ref{ThGoldstine}) $i_X(B_X)$ \`e anche densa in $B_{X^{\ast\ast}}$. Mettendo tutto insieme abbiamo $i_X(B_X)=B_{X^{\ast\ast}}$, quindi $i_X$ \`e bigettiva e quindi $X$ \`e riflessivo.
\end{proof}

\begin{remark}
ATTENZIONE: queste compattezze sono per ricoprimenti, non per successioni!!!
\end{remark}

\begin{proposition}[Banach si immergono in continue su compatto]\label{PrBanchSiImmergonoInContinueSuCompatto}
Se $X$ banach allora $X$ si immerge isometricamente in $(C(K),\normd_\infty)$ per qualche $K$ compatto Hausdorff.
\end{proposition}
\begin{proof}
Sia $K=(\ol{B_{X^\ast}},\sigma(X^\ast,X))$. $K$ \`e T2 compatto per Banach-Alaoglu (\ref{ThBanachAlaogluBourbaki}), inoltre abbiamo una inclusione
% https://q.uiver.app/#q=WzAsNSxbMCwwLCJYIl0sWzEsMCwiWF57XFxhc3RcXGFzdH0iXSxbMiwwLCJDKEspIl0sWzEsMSwiZiJdLFsyLDEsImZcXHJlcyBLIl0sWzAsMSwiaV9YIiwwLHsic3R5bGUiOnsidGFpbCI6eyJuYW1lIjoiaG9vayIsInNpZGUiOiJ0b3AifX19XSxbMSwyXSxbMyw0LCIiLDAseyJzdHlsZSI6eyJ0YWlsIjp7Im5hbWUiOiJtYXBzIHRvIn19fV1d
\[\begin{tikzcd}
	X & {X^{\ast\ast}} & {C(K)} \\
	& f & {f\res K}
	\arrow["{i_X}", hook, from=1-1, to=1-2]
	\arrow[from=1-2, to=1-3]
	\arrow[maps to, from=2-2, to=2-3]
\end{tikzcd}\]
che \`e isometrica perhc\'e $\norm{x}_X=\norm{val_x}_{X^{\ast\ast}}$, da cui $\norm{f}_{X^{\ast\ast}}=\norm{f}_{\infty,K}$.
\end{proof}


\begin{remark}
Questa proposizione possiamo rappresentare isometricamente $X^\ast$ come $C(K)^\ast/X^\perp$ (\ref{PrDualeDiSottospaziEDualeQuoziente}) e il duale di $C(K)$ si rappresenta via misure di Baire finite.
\end{remark}


\section{Compattezza in Banach per la norma}
\begin{theorem}[Mazur]\label{ThMazur}
Sia $(X,\normd)$ banach, $K\subseteq X$ compatto, allora $\ol{\co(K)}$ \`e compatto.
\end{theorem}
\begin{proof}
Sia $B=B_X(0,1)$. Proviamo che $\co(K)$ \`e totalmente limitato (quindi relativamente compatto in $X$ che \`e completo). Sia $\e>0$. Siccome $K$ \`e compatto, esiste $F\in \Ps_{fin}(X)$ tale che 
\[K\subseteq F+\frac\e2B=\bigcup_{x\in F}B(x,\e/2).\]
Quindi $\co(K)\subseteq \co(F)+\frac\e2 B$ (perch\'e convesso che contiene $K$). Se $F=\cpa{f_1,\cdots, f_m}$ allora $\co(F)$ \`e compatto, infatti \`e immagine continua del simplesso standard 
\[\Delta^{m-1}=\cpa{(\la_1,\cdots, \la_m)\in\R^m\mid \la_i\geq 0,\ \sum\la_i=1}\]
tramite la mappa ovvia $\Phi:\Delta^{m-1}\to X$ data da $e_i\mapsto f_i$.

Quindi esiste un insieme finito $G\in\Ps_{fin}(X)$ tale che
\[\co(F)\subseteq G+\frac\e2B\]
e quindi 
\[\co(K)\subseteq \co(F)+\frac\e2B\subseteq G+\frac\e2B+\frac\e2 B=G+\e B,\]
cio\`e $\co(K)$ \`e totalmente limitato.
\end{proof}

\begin{theorem}[Dieudonn\'e]\label{ThDieudonne}
Sia $(X,\normd)$ banach, $K\subseteq X$ compatto, allora esiste una successione $(x_n)_{n\in\N}\subseteq X$ tale che $x_n\to 0$ e $K\subseteq \ol{\co(\cpa{x_n}_{n\in\N})}$.
\end{theorem}
\begin{proof}
Senza perdita di generalit\`a supponiamo $K\subseteq B=B_X(0,1)$.

Per ogni $n\in\N$ esiste $F_n\in\Ps_{fin}(K)$ tale che 
\[K\subseteq F_n+4^{-n}B\]
cio\`e ogni $x\in K$ dista meno di $4^{-n}$ da qualche $x'\in F_n$. Per comodit\`a $F_0=\cpa{0}$ (ok perch\'e abbiamo supposto $K\subseteq B$).

Quindi $D=\bigcup_{n\geq 0}F_n$ \`e un sottoinsieme denso di $K$. Sia $y\in D\nz$, allora $y\in F_n$ per qualche $n\in\N_+$. Siccome $F_{n-1}$ \`e una $4^{-n+1}$-rete di $K$ esiste $y_{n-1}\in F_{n-1}$ tale che $\norm{y_n-y_{n-1}}<4^{-n+1}$. Iterando troviamo $y_n,y_{n-1},\cdots, y_1,y_0$ con $y_i\in F_i$ e $\norm{y_i-y_{i-1}}<4^{-i+1}$ per ogni $i\leq n$. Notiamo che
\[y=y_n=\sum_{k=1}^n y_k-y_{k-1} +\under{=0}{y_0}=\sum_{k=1}^n 2^{-k}\pa{2^k(y_k-y_{k-1})}+\under{=0}{2^{-n}y_0}\]
\`e una combinazione convessa di $2^k(y_k-y_{k-1})$ per $k=1,\cdots, n$ e $y_0=0$.

Inoltre, siccome $\norm{y_k-y_{k-1}}<4^{-k+1}$, si ha $\norm{2^k(y_k-y_{k-1})}<2^{-k+2}$.

Notiamo che per ogni $k\geq 1$ si ha
\[2^k(y_k-y_{k-1})\in A_k=2^k(F_k-F_{k-1})\text{ insieme finito}\]
Inoltre $A_k\subseteq 2^{-k+2}B$ per quanto detto. Ponendo
\[A=\bigcup_{k\geq1} A_k \cup \cpa{0}\]
si ha che ogni $y\in D$ si scrive come combinazione convessa di elementi di $A$. 

Per concludere basta mostrare che $A$ \`e il supporto di una successione infinitesima:
per ogni $\e>0$, $A\bs \e B$ \`e finito in quanto 
\[A\bs \e B\subseteq \bigcup_{2^{-k+2}>\e}A_k=\bigcup_{k<2-\log_2\e}A_k.\]
Dunque una qualsiasi enumerazione $(x_n)_{n\in\N}$ di $A$ definisce una successione infinitesima tale che $D\subseteq \co(\cpa{x_n}_{n\in \N})$ e quindi $K=\ol D\subseteq \ol{\co(\cpa{x_n}_{n\in \N})}$.
\end{proof}

\section{Topologie polari}
\begin{definition}[Topologie polari]
Sia $X$ banach e fissiamo $\As\subseteq \Ps(X)$ dove ogni insieme \`e limitato. La \textbf{topologia polare} su $X^\ast$ associata a $\As$ \`e la topologia di SVTLC associate alle (semi)norme uniformi $\cpa{\normd_{\infty,A}\mid A\in\As}$. A volte indichiamo la topologia associata a $\As$ con $\tau_\As$.
\end{definition}
\begin{example}
Se $\As=\Ps_{fin}(X)$ allora la topologia polare associata \`e la debole$^\ast$ $\sigma(X^\ast, X)$
\end{example}
\begin{example}
Se $\As=\Ks$ \`e l'insieme dei compatti di $X$ allora la topologia polare associata \`e la topologia di convergenza uniforme sui compatti.
\end{example}
\begin{remark}
Per il teorema di Dieudonne (\ref{ThDieudonne}) la topologia di convergenza uniforme sui compatti \`e anche la topologia polare associata a 
\[\Ks_0=\cpa{K\subseteq X\mid \forall \e>0\ K\bs \e B\in\Ps_{fin}(X)},\]
cio\`e gli insiemi che si accumulano al pi\`u in $0$.
\end{remark}
\begin{example}
Se $\As=\Bs$ \`e l'insieme dei sottoinsiemi limitati troviamo la norma duale ($\normd_{\infty,B_X}=\normd_{B_X^\ast}$)
\end{example}

\begin{remark}
Si pu\`o sempre assumere che $\As$ sia una famiglia di insiemi assolutamente convessi in quanto
\[\norm{f}_{\infty,A}=\norm{f}_{\infty,\assco(A)}.\]
\end{remark}

\begin{remark}[Perch\'e si chiama topologia polare?]
Fissiamo una famiglia $\As$. Ricordiamo che per ogni $A\in \As$ si ha 
\[A^0=\cpa{f\in X^\ast\mid \abs{\ps{f,x}}\leq 1\ \forall x\in A}=\ol B(0,1,\normd_{\infty,A}).\]
Senza perdita di generalit\`a supponiamo che $\As$ soddisfi
\begin{enumerate}
	\item $\forall A\in\As$ e $\forall t>0$, $tA\in\As$
	\item $\forall A,B\in \As,\ \exists C\in\As$ tale che $C\supseteq A\cup B$
\end{enumerate}
Allora la famiglia $\cpa{A^0}_{A\in\As}$, cio\`e le palle unitarie delle norme $\normd_{\infty,A}$ \`e una base di intorni di $0$ per la topologia polare associata a $\As$.
\end{remark}

\subsection{Topologia bounded-weak-star e Krein-\v Smulian}
\begin{definition}[Topologia limitata-debole$^\ast$]
Se $(X,\normd)$ banach, la topologia \textbf{bounded weak$^\ast$} (abbreviata $bw^\ast$) su $X^\ast$ \`e la topologia limite topologico di $X_n=(nB_{X^\ast},w^\ast)$. Cio\`e, un insieme $A\subseteq X^\ast$ \`e aperto in questa topologia se e solo se per ogni $n$, $A_n\cap n B_{X^\ast}$ \`e aperto nella topologia $w^\ast$.
\end{definition}
\begin{remark}
Prendere palle chiuse o aperte, cambiare successione di raggi (purch\'e tenda a $+\infty$) o cambiare il centro delle palle non cambia la topologia $bw^\ast$.
\end{remark}
\begin{remark}
$bw^\ast$ \`e invariante per traslazioni, cio\`e $A\in bw^\ast\coimplies A+v_0\in bw^\ast$ per un qualsiasi $v_0\in X$.
\end{remark}

\begin{theorem}\label{ThBoundedWeakStarELaTopologiaDiConvergenzaUniformeSuCompatti}
La topologia $bw^\ast$ \`e la topologia della convergenza uniforme su compatti $\tau_\Ks$.
\end{theorem}
\begin{proof}
Abbiamo notato che $bw^\ast$ \`e invariante per traslazioni, quindi basta mostrare che le due topologie hanno gli stessi intorni di $0$.
\setlength{\leftmargini}{0cm}
\begin{itemize}
\item[$\boxed{\tau_{\Ks_0}\subseteq bw^\ast}$] Una base di intorni di $0$ per $\tau_\Ks$ \`e
\[\cpa{A^0\mid A\in\Ks_0}\quad \text{dove }\Ks_0=\cpa{K\subseteq X\mid \forall \e>0\ K\bs \e B\in\Ps_{fin}(X)}.\]
Per ogni $A\in\Ks_0$ vogliamo mostrare che $A^0$ \`e aperto per $bw^\ast$, cio\`e per ogni $n\geq 1$ chiediamo che sia aperta l'intersezione
\begin{align*}
	A^0\cap nB_{X^\ast}=&A^0\cap n B_X^0=A^0\cap\pa{\frac1n B_X}^0=\pa{A\cup \frac1nB_X}^0=\\
	=&\pa{\pa{A\bs \frac1nB_X}\cup \frac1nB_X}^0=\\
	=&\pa{A\bs \frac1nB_X}^0\cap nB_{X^\ast}
\end{align*}
Poich\'e $A\in\Ks_0$ si ha che $A\bs \frac1nB_X$ \`e finito, quindi $\pa{A\bs \frac1nB_X}^0$ \`e un intorno di $0$ in $w^\ast$ e quindi $A^0\cap nB_{X^\ast}$ \`e effettivamente $w^\ast$-aperto.
\item[$\boxed{bw^\ast\subseteq\tau_{\Ks_0}}$] Sia $U$ un intorno aperto di $0$ per $bw^\ast$. Vogliamo costruire un insieme $A\in\Ks_0$ tale che $A^0\subseteq U$.

Costruiamo per induzione una successione $(A_n)$ di insiemi finiti tali che
\begin{enumerate}
	\item $(A_n)^0\cap nB_{X^\ast}\subseteq U$
	\item $A_{n+1}\subseteq A_n\cup \frac1n B_X$
\end{enumerate}
\setlength{\leftmargini}{0cm}
\begin{itemize}
\item[$\boxed{n=1}$] Poich\'e $U$ \`e aperto in $bw^\ast$ esiste $A_1$ finito tale che $A_1^0\cap B_{X^\ast}\subseteq U\cap B_{X^\ast}\subseteq U$, infatti gli insiemi $A_1^0\cap B_{X^\ast}$ sono base di intorni nella topologia indotta dalla $bw^\ast$ su $B_{X^\ast}$ e chiaramente $U\cap B_{X^\ast}$ \`e un aperto per questa topologia.
\item[$\boxed{n+1}$] Supponiamo di aver costruito $A_1,\cdots, A_n$ finiti con le due propriet\`a. Costruiamo $A_{n+1}$:
\begin{align*}
	\emptyset=&A_n^0\cap nB_{X^\ast}\cap U^c\cap\under{\text{tecnicamente superflua}}{(n+1)B_{X^\ast}}=\\
	=&A_n^0\cap n\pa{\bigcup_{x\in B_X}\cpa{x}}^0\cap U^c\cap (n+1)B_{X^\ast}=\\
	=&A_n^0\cap \pa{\bigcap_{x\in B_X}\cpa{\frac xn}^0}\cap U^c\cap (n+1)B_{X^\ast}=\\
	=&\bigcap_{x\in B_X}\pa{A_n\cup \cpa{\frac xn}}^0\cap \pa{U^c\cap (n+1)B_{X^\ast}}.
\end{align*}
Questa \`e una intersezione di insiemi $w^\ast$ chiusi e limitati: $U^c$ \`e $bw^\ast$ chiuso perch\'e $U$ aperto in $bw^\ast$, $B_{X^\ast}$ \`e $w^\ast$-chiuso perch\'e \`e la palla chiusa, quindi l'intersezione \`e $w^\ast$ chiusa perch\'e $U\cap B_{X^\ast}$ \`e un aperto $w^\ast$ in $B_{X^\ast}$. Ogni $\pa{A_n\cup \cpa{\frac xn}}^0$ \`e $w^\ast$ chiuso per (\ref{PrPolarePrepolareIteratiDannoChiusuraAssolutamenteConvessa}).

Per Banach-Alaoglu (\ref{ThBanachAlaogluBourbaki}) questa intersezione \`e $w^\ast$-compatta e quindi esiste $J_n\subseteq B_X$ finito tale che
\begin{align*}
	\emptyset=&\bigcap_{x\in J_n}\pa{A_n\cup\cpa{\frac xn}}^0\cap U^c\cap (n+1)B_{X^\ast}=\\
	=&\pa{A_n\cup \frac1n J_n}^0\cap U^c\cap (n+1)B_{X^\ast}
\end{align*}
Poniamo $A_{n+1}=A_n\cup \frac1n J_n$. Verifichiamo le due condizioni
\begin{enumerate}
	\item $\emptyset=A_{n+1}^0\cap U^c\cap (n+1)B_{X^\ast}\implies A_{n+1}^0\cap (n+1)B_{X^\ast}\subseteq U$
	\item $A_{n+1}\subseteq A_n\cup \frac1n B_X$ perch\'e $J_n\subseteq B_X$
\end{enumerate}
\end{itemize}
\setlength{\leftmargini}{0.5cm}
Sia $A=\bigcup A_n$. La condizione 2. garantisce che $A$ si pu\`o accumulare solo in $0$, inoltre per ogni $n$
\[A^0\cap n B_{X^\ast}\overset{A^0\subseteq A_n^0}\subseteq A_n^0\cap n B_{X^\ast}\subseteq U\]
quindi prendendo l'unione al variare di $n$, $A^0\subseteq U$.
\end{itemize}
\setlength{\leftmargini}{0.5cm}
\end{proof}

\begin{remark}
$bw^\ast$ \`e una topologia di SVT
\end{remark}


\begin{theorem}\label{ThBoundedWeakStarEWeakStarInduconoLaStessaTopologiaSuXNelBiduale}
Si ha che $(X^\ast,\tau_\Ks)^\ast=(X^\ast,w^\ast)^\ast$.
\end{theorem}
\begin{proof}
Poich\'e $\tau_\Ks=bw^\ast$ \`e pi\`u fine di $w^\ast$ abbiamo immediatamente $(X^\ast,w^\ast)^\ast\subseteq (X^\ast,\tau_\Ks)^\ast$.

Sia $\vp:X^\ast\to \K$ lineare e $\tau_\Ks$-continua. Vogliamo mostrare che sia una valutazione. La continuit\`a per $\tau_\Ks$ significa:
\[\exists K\subseteq X\text{ compatto t.c. }\abs{\ps{\vp,f}}\leq \norm{f}_{\infty,K}\]
in quanto $\Ks$ \`e gi\`a chiuso per omotetie, intersezioni e unioni finite.

Inoltre senza perdita di generalit\`a possiamo considerare $K\in \Ks_0$, cio\`e $K=\cpa{x_n}_{n\geq 0}$ con $x_n\to 0$. Dunque
\[\abs{\ps{\vp,f}}\leq \max_{n\geq 0}\abs{\ps{f,x_n}}\]
dove al posto di $\sup$ usiamo $\max$ perch\'e $\abs{\ps{f,x_n}}$ \`e una successione infinitesima di reali non negativi.

\`E quindi ben definito un operatore lineare e continuo
\[T:\funcDef{X^\ast}{c_0}{f}{(\ps{f,x_n})_{n\geq 0}}\]
la continuit\`a vale perch\'e $\norm{Tf}_\infty=\max_{n\geq 0}\abs{\ps{f,x_n}}\leq \pa{\max \norm {x_n}}\norm f$ dove $\max \norm {x_n}$ \`e ben definito perch\'e $x_n\to 0$.

Inoltre la disuguaglianza $\abs{\ps{\vp,f}}\leq \max_{n\geq 0}\abs{\ps{f,x_n}}$ garantisce che $\ker T\subseteq \ker \vp$, quindi abbiamo una fattorizzazione
% https://q.uiver.app/#q=WzAsNCxbMCwwLCJYXlxcYXN0Il0sWzIsMCwiXFxLIl0sWzAsMSwiVChYKSJdLFsxLDEsImNfMCJdLFswLDEsIlxcdnAiXSxbMCwyLCJUIiwyXSxbMiwxLCJcXHd0IFxcdnAiXSxbMiwzLCJcXHN1YnNldGVxIiwzLHsic3R5bGUiOnsiYm9keSI6eyJuYW1lIjoibm9uZSJ9LCJoZWFkIjp7Im5hbWUiOiJub25lIn19fV1d
\[\begin{tikzcd}
	{X^\ast} && \K \\
	{T(X)} & {c_0}
	\arrow["\vp", from=1-1, to=1-3]
	\arrow["T"', from=1-1, to=2-1]
	\arrow["{\wt \vp}", from=2-1, to=1-3]
	\arrow["\subseteq"{marking, allow upside down}, draw=none, from=2-1, to=2-2]
\end{tikzcd}\]
Notiamo che $\wt \vp$ \`e continua perch\'e se $y=Tf$ allora 
\[\abs{\ps{\wt \vp,y}}=\abs{\ps{\vp,f}}\leq \max_{n\geq 0}\abs{\ps{f,x_n}}=\norm{y}_{c_0}\implies \norm{\wt \vp}\leq 1.\]
Per Hahn-Banach (\ref{CorHahnBanachPerSpaziNormati}) $\wt \vp$ si estende a tutto $c_0$ con la stessa norma, ma i funzionali continui su $c_0$ sono quelli della forma $(x_i)_{i\geq 0}\mapsto \sum_{i\geq 0} \la_i x_i$ per $(\la_i)_{i\geq 0}\in \ell_1$.

Quindi esiste $\la\in \ell_1$ tale che per ogni $f\in X^\ast$ si ha
\[\ps{\vp,f}=\ps{\wt \vp,Tf}=\sum_{n\geq 0} \la_n\ps{f,x_n}=\ps{f,\sum_{n\geq 0}\la_nx_n}\]
dove l'ultimo passaggio \`e valido perch\'e la serie \`e assolutamente convergente e $f$ \`e continua.

In conclusione, $u=\sum_{n\geq 0}\la_n x_n\in X$ rappresenta $\vp$, cio\`e $\ps{\vp,f}=\ps{f,u}$ e questo conclude.
\end{proof}

\begin{theorem}[Krein-\v Smulian]\label{ThKreinSmulian}
Sia $(X,\normd)$ spazio di Banach, $C\subseteq X^\ast$ convesso, allora $C$ \`e $w^\ast$-chiuso se e solo se per ogni $n\in\N$ si ha $C\cap nB_{X^\ast}$ \`e $w^\ast$-chiuso.
\end{theorem}
\begin{proof}
La seconda condizione \`e equivalente a $C$ chiuso in $bw^\ast=\tau_{\Ks}$ (\ref{ThBoundedWeakStarELaTopologiaDiConvergenzaUniformeSuCompatti}) e questa topologia ha lo stesso duale della $w^\ast$ (\ref{ThBoundedWeakStarEWeakStarInduconoLaStessaTopologiaSuXNelBiduale}) e questo conclude per il teorema di Hanh-Banach/separazione dei convessi (\ref{ThSeparazioneDiConvessi}).
\end{proof}






\section{Compattezza per la topologia debole}
\subsection{Varie nozioni di compattezza}
\begin{definition}[Numerabile compattezza]
$X$ spazio topologico \`e \textbf{numerabilmente compatto} (abbreviato \textbf{NC}) se vale una delle sequenti equivalenti condizioni:
\begin{itemize}
	\item per ogni $S\subseteq X$ infinito ha punti di $\omega$-accumulazione, cio\`e esiste $x\in X$ tale che per ogni $U$ intorno di $x$ si ha $\abs{U\cap S}\geq \aleph_0$, ovvero
\[\bigcap_{F\in\Ps_{fin}(S)}\ol{S\bs F}\neq \emptyset\]
\item Per ogni $(F_n)$ successione di chiusi in $X$ non vuoti decrescenti per inclusione si ha $\bigcap F_n\neq\emptyset$.
\item Per ogni ricoprimento aperto $\cpa{U_n}$ numerabile di $X$ esiste un sottoricoprimento finito.
\end{itemize}
$A\subseteq X$ \`e \textbf{relativamente numebrabilmente compatto} (abbreviato \textbf{RNC}) se vale una delle seguenti
\begin{itemize}
	\item Ogni $S\subseteq A$ infinito ha punti di $\omega$-accumulazione in $X$
	\item Ogni $(a_n)\subseteq A$ successione ha punti di accumulazione in $X$.
\end{itemize}
\end{definition}
\begin{definition}[Sequenzialmente compatto]
$X$ spazio topologico \`e \textbf{sequenzialmente compatto} (abbreviato \textbf{SC}) se per ogni $(x_n)$ successione in $X$ esiste una sottosuccessione convergente.
\smallskip

\noindent
$A\subseteq X$ \`e \textbf{relativamente sequenzialmente compatto} (abbreviato \textbf{RSC}) se ogni successione in $A$ ha una sottosuccessione convergente in $X$.
\end{definition}

\begin{proposition}
Se $A\subseteq X$ spazi topologici allora valgono le seguenti implicazioni:
% https://q.uiver.app/#q=WzAsNSxbMCwwLCJcXG9se0N9Il0sWzIsMCwiXFxvbHtTQ30iXSxbMSwxLCJcXG9se05DfSJdLFszLDEsIlJTQyJdLFsyLDIsIlJOQyJdLFswLDIsIiIsMCx7ImxldmVsIjoyfV0sWzEsMiwiIiwyLHsibGV2ZWwiOjJ9XSxbMSwzLCIiLDAseyJsZXZlbCI6Mn1dLFszLDQsIiIsMCx7ImxldmVsIjoyfV0sWzIsNCwiIiwwLHsibGV2ZWwiOjJ9XV0=
\[\begin{tikzcd}
	{\ol{C}} && {\ol{SC}} \\
	& {\ol{NC}} && RSC \\
	&& RNC
	\arrow[Rightarrow, from=1-1, to=2-2]
	\arrow[Rightarrow, from=1-3, to=2-2]
	\arrow[Rightarrow, from=1-3, to=2-4]
	\arrow[Rightarrow, from=2-2, to=3-3]
	\arrow[Rightarrow, from=2-4, to=3-3]
\end{tikzcd}\]
dove la barra sopra la sigla significa che chiediamo che $\ol A$ in $X$ abbia la propriet\`a.
\end{proposition}
\begin{example}[Compatto $T_2$ non implica sequenzialmente compatto]
Sia $2=\cpa{0,1}$ spazio topologico discreto, $X=2^{2^\N}=\cpa{f:2^\N\to\cpa{0,1}}=\Ps(\Ps(\N))$.
La mappa di valutazione
\[\funcDef{2^\N\times \N}{2}{(f,n)}{f(n)}\]
definisce in modo canonico una successione $val:\N\to 2^{2^\N}$. Questa successione non ha estratte convergenti, infatti convergenza in uno spazio con la topologia prodotto significa convergenza puntuale, quindi se $n_k$ \`e una ipotetica successione crescente di naturali che definisce la sottosuccessione allora per ogni $f\in 2^{\N}$ si dovrebbe avere $val_{n_k}(f)=f(n_k)$ convergente (in $2=\cpa{0,1}$ con la topologia discreta), cio\`e $f(n_k)$ definitivamente costante, ma questo non \`e possibile perch\'e per ogni fissata sottosuccessione $val_{n_k}$ possiamo considerare una funzione tale che $f(n_k)=k\mod2$.
\end{example}
\begin{exercise}[Sequenzialmente compatto non implica compatto]
Sia $X=\omega_1=[0,\omega_1)=\cpa{\text{ordinali numerabili}}$ con la topologia dell'ordine (quella che ha per base gli intervalli aperti).

Notiamo che $\omega_1$ \`e SC, infatti ogni successione ha una sottosuccessione monotona (vero in ogni insieme totalmente ordinato) e questa successione converge: se \`e decrescente \`e stazionaria per definizione di buon ordine, se \`e crescente allora converge al suo estremo superiore, che sta in $\omega_1$.


Eppure $X$ non \`e compatto perch\'e \`e unione degli intervalli aperti $\bigcup_{\al\in X}[0,\al)$, che non ha sottoricoprimenti finiti.
\end{exercise}

\begin{exercise}[$SC\nRightarrow\ol{NC}$, e quindi in particolare $RSC\nRightarrow\ol{NC}$]
Sia $X=(\omega+1)\times (\omega_1+1)\bs \cpa{(\omega,\omega_1)}=[0,\omega_1]\times[0,\omega_1]\bs\cpa{(\omega,\omega_1)}$ e sia $A=(\omega+1)\times \omega_1=[0,\omega]\times[0,\omega_1)$

$A$ \`e SC perch\'e lo sono $\omega+1$ e $\omega_1$, inoltre $\ol{A}=X$ perch\'e i punti $(\al,\omega_1)$ sono di accumulazione. Notiamo per\`o che $X$ non \`e NC infatti l'insieme $B\subseteq X$ dato da $B=\omega\times\cpa{\omega_1}$ non ha punti di accumulazione in $X$ (\`e isomorfo a $\omega$ e l'unico punto di accumulazione sarebbe l'angolino $(\omega,\omega_1)$ che $X$ non ha per costruzione).
\end{exercise}

Questi esempi mostrano che in generale
% https://q.uiver.app/#q=WzAsNSxbMCwwLCJcXG9se0N9Il0sWzIsMCwiXFxvbHtTQ30iXSxbMSwxLCJcXG9se05DfSJdLFszLDEsIlJTQyJdLFsyLDIsIlJOQyJdLFswLDIsIiIsMCx7ImxldmVsIjoyfV0sWzEsMiwiIiwyLHsibGV2ZWwiOjJ9XSxbMSwzLCIiLDAseyJsZXZlbCI6Mn1dLFszLDQsIiIsMCx7ImxldmVsIjoyfV0sWzIsNCwiIiwwLHsibGV2ZWwiOjJ9XSxbMSwwLCJOTyIsMSx7ImNvbG91ciI6WzEsMTAwLDQ1XSwic3R5bGUiOnsidGFpbCI6eyJuYW1lIjoiYXJyb3doZWFkIn19fSxbMSwxMDAsNDUsMV1dLFszLDIsIk5PIiwxLHsiY29sb3VyIjpbMSwxMDAsNDVdfSxbMSwxMDAsNDUsMV1dLFsyLDAsIiIsMSx7ImN1cnZlIjotMSwiY29sb3VyIjpbMSwxMDAsNDVdfV0sWzIsMSwiIiwxLHsiY3VydmUiOi0xLCJjb2xvdXIiOlsxLDEwMCw0NV19XSxbMywxLCIiLDEseyJjdXJ2ZSI6MSwiY29sb3VyIjpbMSwxMDAsNDVdfV0sWzQsMiwiIiwxLHsiY3VydmUiOi0xLCJjb2xvdXIiOlsxLDEwMCw0NV19XSxbNCwzLCIiLDEseyJjdXJ2ZSI6MSwiY29sb3VyIjpbMSwxMDAsNDVdfV1d
\[\begin{tikzcd}
	{\ol{C}} && {\ol{SC}} \\
	& {\ol{NC}} && RSC \\
	&& RNC
	\arrow[Rightarrow, from=1-1, to=2-2]
	\arrow["NO"{description}, color={rgb,255:red,230;green,4;blue,0}, tail reversed, from=1-3, to=1-1]
	\arrow[Rightarrow, from=1-3, to=2-2]
	\arrow[Rightarrow, from=1-3, to=2-4]
	\arrow[color={rgb,255:red,230;green,4;blue,0}, curve={height=-6pt}, from=2-2, to=1-1]
	\arrow[color={rgb,255:red,230;green,4;blue,0}, curve={height=-6pt}, from=2-2, to=1-3]
	\arrow[Rightarrow, from=2-2, to=3-3]
	\arrow[color={rgb,255:red,230;green,4;blue,0}, curve={height=6pt}, from=2-4, to=1-3]
	\arrow["NO"{description}, color={rgb,255:red,230;green,4;blue,0}, from=2-4, to=2-2]
	\arrow[Rightarrow, from=2-4, to=3-3]
	\arrow[color={rgb,255:red,230;green,4;blue,0}, curve={height=-6pt}, from=3-3, to=2-2]
	\arrow[color={rgb,255:red,230;green,4;blue,0}, curve={height=6pt}, from=3-3, to=2-4]
\end{tikzcd}\]

\begin{remark}
Se $A\subseteq X$, $f:X\to Y$ continua e $A$ \`e RNC allora $f(A)\subseteq Y$ \`e RNC.
\end{remark}




\subsection{Eberlein-\v Smulian}
\begin{remark}
Se $A$ \`e RNC in $(X,w)$ allora \`e limitato, infatti basta mostrare che per ogni $f\in X^\ast$ si ha $f(A)$ limitato, che \`e vero perch\'e $f(A)\subseteq \K$ \`e RNC ma in $\R^n$ questo implica limitato.
\end{remark}

\begin{theorem}[Eberlein-\v Smulian]\label{ThEberleinSmulian}
Sia $E$ spazio di Banach e $A\subseteq E$. Rispetto alla \ul{topologia debole} di $E$ sono equivalenti
\begin{enumerate}
	\item $\ol A^w$ \`e numerabilmente compatta
	\item $A$ \`e relativamente numerabilmente compatto
	\item $\ol A^w$ \`e sequenzialmente compatta
	\item $A$ \`e relativamente sequenzialmente compatto
	\item $\ol A^w$ \`e compatto
\end{enumerate}
\end{theorem}
\begin{proof}
Basta mostrare le implicazioni in blu
% https://q.uiver.app/#q=WzAsNSxbMCwwLCJcXG9se0N9Il0sWzIsMCwiXFxvbHtTQ30iXSxbMSwxLCJcXG9se05DfSJdLFszLDEsIlJTQyJdLFsyLDIsIlJOQyJdLFswLDIsIiIsMCx7ImxldmVsIjoyfV0sWzEsMiwiIiwyLHsibGV2ZWwiOjJ9XSxbMSwzLCIiLDAseyJsZXZlbCI6Mn1dLFszLDQsIiIsMCx7ImxldmVsIjoyfV0sWzIsNCwiIiwwLHsibGV2ZWwiOjJ9XSxbNCwzLCIiLDEseyJjdXJ2ZSI6MiwibGV2ZWwiOjIsImNvbG91ciI6WzIzNSwxMDAsNDRdfV0sWzMsMCwiIiwxLHsibGV2ZWwiOjIsImNvbG91ciI6WzIzNSwxMDAsNDRdfV0sWzIsMSwiIiwxLHsiY3VydmUiOjIsImxldmVsIjoyLCJjb2xvdXIiOlsyMzUsMTAwLDQ0XX1dXQ==
\[\begin{tikzcd}
	{\ol{C}} && {\ol{SC}} \\
	& {\ol{NC}} && RSC \\
	&& RNC
	\arrow[Rightarrow, from=1-1, to=2-2]
	\arrow[Rightarrow, from=1-3, to=2-2]
	\arrow[Rightarrow, from=1-3, to=2-4]
	\arrow[color={rgb,255:red,0;green,19;blue,224}, curve={height=12pt}, Rightarrow, from=2-2, to=1-3]
	\arrow[Rightarrow, from=2-2, to=3-3]
	\arrow[color={rgb,255:red,0;green,19;blue,224}, Rightarrow, from=2-4, to=1-1]
	\arrow[Rightarrow, from=2-4, to=3-3]
	\arrow[color={rgb,255:red,0;green,19;blue,224}, curve={height=12pt}, Rightarrow, from=3-3, to=2-4]
\end{tikzcd}\]
\setlength{\leftmargini}{0cm}
\begin{itemize}
\item[$\boxed{RNC \implies RSC}$] Sia $(a_n)\subseteq A$, dobbiamo mostrare che $(a_n)$ ha una sottosuccessione $w$-convergente in $E$. Sia
\[V=\ol{\Span(\cpa{a_n}_{n\in\N})}\subseteq E,\]
in particolare $V$ \`e un sottospazio vettoriale chiuso e separabile, quindi $V^\ast$ ha una palla unitaria $w^\ast$-separabile\footnote{CREDO che il motivo sia perch\'e se $B\subseteq V$ denso allora i funzionali continui nella palla di possono stanno nella chiusura di un insieme di sollevamenti dei funzionali duali ai punti di $B$, dove per ottenere i sollevamenti usiamo Hahn-Banach (\ref{CorHahnBanachPerSpaziNormati})}. Sia $D\subseteq V^\ast$ numerabile e denso. Con argomento diagonale troviamo una sottosuccessione di $(a_n)$ (che continuiamo a chiare $(a_n)$) tale che $\ps{f,a_n}$ converge per ogni $f\in D$.

Sia $a_\infty\in E$ un punto di accumulazione di $(a_n)\subseteq A$ (stiamo assumendo $A$ RNC). Allora per ogni $f\in D$, $\ps{f,a_\infty}$ \`e punto di accumulazione della successione convergente $\ps{f,a_n}$, quindi $\ps{f,a_\infty}$ \`e il limite ($\R$ \`e Hausdorff).

Affermo che ci\`o vale per ogni $f\in V^\ast$: se non fosse cos\`i esisterebbe $g\in V^\ast$ tale che $\ps{g,a_n}\not\to\ps{g,a_\infty}$, ma allora estraendo una sottosuccessione esisterebbe una sottosuccessione tale che $\ps{g,a_{n_k}}$ converge ad un limite diverso da $\ps{g,a_{\infty}}$. Se $b_\infty\in E$ \`e di $w$-accumulazione per $(a_{n_k})$ si trova come prima che per ogni $f\in D$, $\ps{f,a_{n_k}}\to \ps{f,b_\infty}$, ma essendo $(a_{n_k})$ una sottosuccessione di quella di prima $\ps{f,a_{n_k}}\to \ps{f,a_\infty}$. Eppure $\ps{g,a_{n_k}}\to \ps{g,b_\infty}\neq \ps{g,a_\infty}$ e questo \`e assurdo perch\'e $D$ \`e $w^\ast$-denso.

Dunque $\ps{f,a_n}\to \ps{f,a_\infty}$ per ogni $f\in V^\ast$, quindi $\ps{f,a_n}\to \ps{f,a_\infty}$ per ogni $f\in E^\ast$ in quanto $f\res V\in V^\ast$. Questo significa esattamente che $a_n\to a_\infty$ nella topologia debole, come volevamo.
\item[$\boxed{RSC \implies \ol C}$] Mostriamo che la chiusura $\sigma(E^{\ast\ast},E^\ast)$ di $A$ in $E^{\ast\ast}$ \`e in realt\`a contenuta in $E$. Se questo \`e vero allora questa \`e anche la chiusura in $\sigma(E,E^\ast)$ e quindi \`e compatta per Banach-Alaoglu (\ref{ThBanachAlaogluBourbaki}) infatti
\[A\subseteq E\subseteq E^{\ast\ast}\leadsto \ol A^E=\ol A^{E^{\ast\ast}}\cap E.\]
Sia $\eta\in \ol A^{\sigma(E^{\ast\ast},E^\ast)}$ e mostriamo che $\eta\in E$. Quello che faremo \`e mostrare che $\ker \eta\subseteq E^\ast$ \`e $\sigma(E^\ast,E)$-chiuso\footnote{forma lineare \`e continua se e solo se nucleo \`e chiuso (\ref{PrCaratterizzazioneFunzionaliContinui}) e mostrare che $\eta$ \`e $w^\ast$-continua \`e la stessa cosa di dire che $\eta$ \`e una valutazione.}.

Per Krein-\v Smulian (\ref{ThKreinSmulian}) basta vedere che $\ker\eta\cap \ol{B_{E^\ast}(0,1)}$ \`e $w^\ast$-chiuso (e quindi per omotetia $\ker\eta \cap\ol{B(0,R)}$ chiuso e per Krein-\v Smulian questo mostra che $\ker \eta$ stesso \`e chiuso).

Sia $g_0\in \ol{\ker\eta\cap B_{E^\ast}}^{w^\ast}$ e mostriamo che $g_0\in\ker\eta$, cio\`e $\ps{\eta,g_0}=0$ (chiaramente $g_0$ sta nella palla). Partendo da $g_0$ costruiamo due successioni $a_n\in A$ e $g_n\in\ker\eta\cap B_{E^\ast}$ in modo che
\[\begin{cases}
	\ps{g_i,a_n}-\ps{\eta,g_i}<\frac1n &\forall 0\leq i\leq n-1\\
	\abs{\ps{g_n,a_i}-\ps{g_0,a_i}}<\frac1n &\forall 1\leq i\leq n
\end{cases}\]
Questo si pu\`o fare per induzione: definiti $g_1,\cdots, g_{n-1}$ esiste $a_n$ verificante la prima condizione perch\'e quella condizione definisce un intorno di $\eta$ per la topologia $\sigma(E^{\ast\ast},E^\ast)$, che quindi interseca $A$ in quanto $\eta$ appartiene alla chiusura di $A$. Definiti $a_1,\cdots, a_{n}$ esiste $g_n$ che verifica la seconda condizione perch\'e quelle disuguaglianze definiscono un intorno di $g_0$ nella topologia $\sigma(E,E^\ast)$ e questo interseca $\ker\eta\cap B_{E^\ast}$.


Poich\'e $\ps{\eta,g_n}=0$ per ogni $n\geq 1$ vale per $i$ fissato
\[\begin{cases}
	\ps{g_0,a_n}-\ps{\eta,g_0}=o(1)\\
	\ps{g_i,a_n}=o(1)\\
	\ps{g_n,a_i}-\ps{g_0,a_i}=o(1)
\end{cases}\]
Poich\'e $A$ \`e RSC, a meno di sottosuccessione $a_n\to a_\infty\in E$ debolmente.

[FINIAMO LA PROSSIMA VOLTA]
\end{itemize}
\setlength{\leftmargini}{0.5cm}
\end{proof}

























\newpage
Riassumendo:
\begin{itemize}
	\item Su $X^\ast$
	\begin{itemize}
		\item Banach-Alaoglu: $B_{X^\ast}$ \`e $w^\ast$-compatta
	\item Se $X$ \`e separabile allora $(B_{X^\ast},w^\ast)$ \`e metrizzabile, quindi \`e anche sequenzialmente compatta. (Mostrato indipendentemente tramite Ascoli-Arzel\'a).
	\end{itemize}
	\item Su $X$
	\begin{enumerate}
		\item Dieudonn\'e: i compatti (norma) $K$ sono contenuti in $\ol{\co(x_n)}$ per $x_n\to 0$
		\item Eberlein-\v Smulian: compatti per debole.
	\end{enumerate}
\end{itemize}
\chapter{Funzioni regolari e funzioni a supporto compatto}

Sia $\Omega$ aperto di $\R^n$ non vuoto.
\begin{notation}
Sia $k(\Omega)=\cpa{K\subseteq \Omega\mid K\text{ compatto}}$. 
\end{notation}

\begin{definition}[Spazio di Fr\'echet]
Uno spazio topologico \`e di \textbf{Fr\'echet} se \`e SVTLC, metrizzabile e completo.
\end{definition}


\section{Funzioni regolari}
\begin{definition}[Funzioni continue]
Definiamo l'insieme delle funzioni continue su $\Omega$ come
\[C^0(\Omega)=\cpa{f:\Omega\to\R\mid \text{continue}}\]
\end{definition}
\begin{proposition}\label{PrContinueSonoSpazioFrechet}
L'insieme $C^0(\Omega)$ munito della topologia indotta dalle seminorme uniformi
\[\cpa{\normd_{\infty,K}}_{K\in k(\Omega)}\]
\`e uno spazio di Fr\'echet.
\end{proposition}
\begin{proof}
In quanto topologia indotta da seminorme abbiamo che $C^0(\Omega)$ \`e uno SVTLC.

\setlength{\leftmargini}{0cm}
\begin{itemize}
\item[$\boxed{\text{metrizzabile}}$] Se $(K_j)_{j\in\N}$ \`e una successione di compatti tale che $K_i\subseteq int(K_{i+1})$ e $\Omega=\bigcup_{j\geq 0}K_j$ allora le seminorme $\cpa{\normd_{\infty,K_j}}$ topologizzano $C^0(\Omega)$. Quindi per esempio possiamo considerare $K_j=\cpa{x\in\Omega\mid dist(x,\Omega^c)\leq 2^{-j}}\cap \ol B(0,j)$ e definire la distanza come
\[d(f,g)=\sum_{j\geq 0}2^{-j}\arctan(\norm{f-g}_{\infty,K_j}).\]
\item[$\boxed{\text{completo}}$] $(f_n)\subseteq C^0(\Omega)$ \`e di Cauchy se per ogni $j$ si ha $(f_n\res{K_j})_n$ di Cauchy in $C^0(K_j)$, quindi $f_n$ converge uniformemente su $K_j$ e il limite \`e una funzione $f\in C^0(\Omega)$ (definiamo puntualmente a priori ma \`e una convergenza uniforme su compatti quindi il limite \`e una funzione continua).
\end{itemize}
\setlength{\leftmargini}{0.5cm}
\end{proof}



\begin{notation}
Sia $\al=(\al_1,\cdots, \al_n)\in \N^n$ e $f\in C^m(\Omega)$, allora
\[\del^\al f=\pp[\al_n]{x_n^{\al_n}}{}\cdots \pp[\al_1]{x_1^{\al_1}}{} f.\]
Chiamiamo $n$ la \textbf{lunghezza} di $\al$ e $\abs{\al}=\sum_{i=1}^n\al_i$ il \textbf{peso} o \textbf{grado} di $\al$.
\end{notation}

\begin{remark}
Per il teorema di Schwarz non importa l'ordine delle derivate sopra.
\end{remark}

\begin{definition}[Spazio $C^m(\Omega)$]
Poniamo
\[C^m(\Omega)=\cpa{f:\Omega\to \R\mid\forall \al\ t.c. \abs{\al}\leq m,\  \del^\al f\in C^0(\Omega)}.\]
\end{definition}

\begin{proposition}\label{Pr-mRegolariSonoSpazioFrechet}
L'insieme $C^m(\Omega)$ con la topologia indotta dalle seminorme
\[\norm{f}_{\al,\infty,K}=\norm{\del^\al f}_{\infty,K}\]
considerate al variare di $\abs\al\leq m$ e $K\in k(\Omega)$ \`e uno spazio di Fr\'echet.
\end{proposition}
\begin{proof}
Equivalente possiamo considerare le norme $\cpa{p_{m,K}}_{K\in k(\Omega)}$ date da
\[p_{m,K}(f)=\max_{\abs{\al}\leq m}\norm{\del^\al f}_{\infty,K}.\]
La metrizzabilit\`a segue come prima.
\medskip

Per la complettezza basta usare il teorema di limite sotto il segno di derivata: Se $(f_n)\subseteq C^m(\Omega)$ \`e di Cauchy, cio\`e $\abs\al\leq m$ e $\forall K\in k(\Omega)$ si ha $\del^\al f_j$ di Cauchy in $C^0(\Omega)$, allora per ogni $\abs{\al}\leq m$ si ha convergenza uniforme sui compatti
\[\del^\al f_j\to g_\al\]
per qualche $g_\al\in C^0(\Omega)$. Si conclude (per induzione su $m$) che $f=\lim f_j$ \`e di classe $C^m$ e che $\del^\al f=g_\al$.
\end{proof}

\begin{remark}
$C^m(\Omega)$ ha la topologia iniziale data dalle mappe
\[\funcDef{C^m(\Omega)}{C^0(K)}{f}{\del^\al f\res K}\]
al variare di $K\in k(\Omega)$ e $\abs\al\leq m$.
\end{remark}

\begin{definition}[Spazio $C^\infty(\Omega)$]
Definiamo 
\[C^\infty(\Omega)=\bigcap_{m\geq 0}C^m(\Omega).\]
\end{definition}
\begin{remark}
Anche $C^\infty(\Omega)$ \`e di Fr\'echet.
\end{remark}


\begin{remark}
I limitati di $C^\infty(\Omega)$ sono relativamente compatti.
\end{remark}
\begin{proof}
Se $A\subseteq C^\infty(\Omega)$ \`e limitato allora per ogni $K\in k(\Omega)$ e per ogni $m\in \N$ si ha che 
\[\sup_{f\in A}p_{m+1,K}(f)\leq C(m,K)\in\R,\]
quindi le derivate delle $\del^\al f$ per $f\in A$ sono limitate uniformemente su $K$. Questo in particolare vale per $K$ compatto convesso, quindi le $\del^\al f$ sono equilipschitz (teorema del valor medio).

Allora per Ascoli-Arzel\'a abbiamo che $\cpa{\del^\al f}$ \`e un compatto in $C^0(K)$. Dunque (argomento diagonale) ogni successione $(f_j)\subseteq A$ ha una sottosuccessione convergente uniformemente sui compatti e quindi in $C^\infty(\Omega)$.
\end{proof}

\begin{remark}
Nel caso di $C^m(\Omega)$ si ha che i limitati di $C^{m+1}(\Omega)\subseteq C^m(\Omega)$ sono relativamente compatti in $C^m(\Omega)$.
\end{remark}

\section{Funzioni a supporto compatto}

\begin{definition}[Funzioni a supporto compatto]
Definiamo
\[C^0_C(\Omega)=\bigcup_{K\in k(\Omega)}C_K,\qquad C_K=\cpa{f\in C^0(\R^n)\mid \supp f\subseteq K}\]
Analogamente (anche per $m=\infty$)
\[C_C^m(\Omega)=C^m(\Omega)\cap C^0_C(\Omega)=\bigcup_{K\in k(\Omega)}C^m_K,\qquad C_K^m=C^m(\Omega)\cap C_K.\]
\end{definition}

\begin{remark}
$C^0_C(\Omega)$ \`e denso in $C^0(\Omega)$ e similmente per ordini pi\`u alti.
\end{remark}

Poich\'e questi spazi sono definiti in modo naturale come unione, la topologia naturale su $C_C^m(\Omega)$ \`e la pi\`u fine topologia di SVT che renda continue le inclusioni $C_K^m\inj C^m_C(\Omega)$, dove $C^m_K$ ha la topologia indotta da $C^m(\Omega)$.






\subsection{Lo spazio \texorpdfstring{$C_C$}{CC}}
Sia $X_n=\cpa{x\in\R^n\mid x_i=0\ \forall i\geq n}\cong \R^n$ e consideriamo questo spazio con la (unica\footnote{equivalenza delle norme}) topologia di SVT $T_0$, cio\`e la topologia euclidea. Sia $X_n\inj X_{n+1}$ l'inclusione.

Poniamo
\[C_C=\R^\omega=\bigcup_{n\geq 0}X_n,\qquad \R^n=\cpa{x\in\R^\N\mid x_i=0\ \forall i\geq n}.\]

\begin{remark}
Qualunque topologia di SVT su $C_C$ rende continue le inclusioni, perch\'e induce su $X_n$ una topologia che non \`e pi\`u fine di quella euclidea. Quindi la topologia di limite induttivo su $C_C$ \`e la pi\`u fine topologia di SVT.
\end{remark}


\begin{remark}
Questa topologia \`e localmente convessa perch\'e lo sono le topologie sugli $X_n$.
\end{remark}

\begin{remark}
La topologia limite su $C_C$ deve essere quella indotta da TUTTE le seminorme su $C_C$
\end{remark}

\begin{notation}
$e_i\in C_C$ \`e la successione identicamente nulla eccetto nell'indice $i$ dove vale 1.
\end{notation}
\begin{remark}
Se $p:C_C\to [0,\infty)$ \`e una seminorma e $x\in C_C$ (che scriviamo $x=\sum_{i=0}^n x_i e_i$) allora
\[p(x)=p\pa{\sum_{i=0}^n x_i e_i}\leq \sum_{i=0}^n \abs{x_i} p(e_i)\]
dunque ogni seminorma \`e maggiorata da una seminorma della forma
\[p_\la(x)=\sum_{i\geq 0}\la_i\abs{x_i}\]
per qualche $\la\in [0,\infty)^\N$.
\end{remark}
\begin{corollary}
La famiglia $\cpa{p_\la}_{\la\in [0,\infty)^\N}$ \`e una famiglia di seminorme che topologizza $C_C$.
\end{corollary}


\begin{remark}
$C_C$ \`e completo sequenzialmente.
\end{remark}
\begin{proof}
Ogni successione di Cauchy \`e limitata quindi, poich\'e gli $X_n$ sono chiusi negli $X_{n+k}$, si ha per (\ref{PrProprietaLimitiInduttiviStretti}) che la successione \`e contenuta in qualche $X_n$ e gli $X_n$ sono completi. 
\end{proof}

\begin{remark}
$C_C$ non \`e metrizzabile, quindi in particolare non \`e di Fr\'echet.
\end{remark}
\begin{proof}
Supponiamo per assurdo che $C_C$ sia metrizzabile.
La famiglia $\cpa{C_C\bs X_n}_{n\in\N}$ \`e numebrabile e di aperti densi in spazio metrico completo $C_C$ (densi perch\'e sottospazi hanno parte interna vuota), quindi  per il teorema di Baire (\ref{ThBaire}) 
\[C_C=\ol{\bigcap_n C_C\bs X_n}=\ol{\emptyset}=\emptyset,\]
che \`e assurdo.
\end{proof}


\begin{remark}
Ogni forma lineare su $C_C$ \`e continua (perch\'e continua quando ristretta a $\R^n$), quindi
\[(C_C)^\ast=\R^\N\]
in quanto una forma lineare \`e identificata dai valori che assume su una base.
\end{remark}

\begin{exercise}
Il duale di $\R^\N$ con la topologia prodotto delle topologie di seminorme $\cpa{\normd_{\infty,[0,n]}}_{n\geq 0}$ \`e $C_C$, infatti la topologia prodotto \`e $\sigma(\R^\N,C_C)$.
\end{exercise}

\begin{exercise}
Proviamo che la topologia di $C_C$ come limite induttivo stretto di SVT coincide con la topologia limite topologico $\tau_\infty$.
\end{exercise}
\begin{proof}
In questa topologia $A\subseteq C_C$ \`e aperto se e solo se per ogni $n$ si ha $A\cap X_n$ aperto di $X_n$. In particolare tale $A$ \`e aperto nella topologia $LF$ di $C_C$.

Dalla definizione \`e evidente che $LF$ \`e invariante per traslazioni quindi per vedere che le due topologie coincidono basta vedere che ogni intorno di $0$ in $\tau_\infty$ contiene un intorno di $0$ di $LF$.

Vogliamo\footnote{questo basta perch\'e $\cpa{p_\la\leq 1}$ contiene un aperto.} definire una successione $(\la_i)\subseteq\R_+$ tale che $\cpa{p_\la\leq 1}\subseteq V$, cio\`e per ogni $n\in\N$ e ogni $x=\sum_{i=0}^{n-1} x_ie_i\in X_n\subseteq C_C$, se $\sum_{i=0}^{n-1}\la_i\abs{x_i}\leq 1$ allora $x\in V$.

Definiti $\la_0,\cdots,\la_{n-1}$ con questa propriet\`a allora per ogni $s\geq 0$ consideriamo 
\[K_s=\cpa{x\in X_n\mid \sum_{i=0}^{n-1}\la_i\abs{x_i}+s\abs{x_n}\leq 1}\bs V.\]
Ogni $K_s$ \`e un compatto di $X_n$, inoltre $s\mapsto X_s$ \`e decrescente per inclusione ($s$ pi\`u grade \`e un vincolo pi\`u forte).

Notiamo che $\bigcap_{s>0}K_s=\emptyset$, infatti se esistesse un elemento di questa intersezione allora la coordinata $x_n$ sarebbe nulla, cio\`e $x\in X_{n-1}$, ma per il passo induttivo abbiamo vuoto.

Dunque per compattezza esiste $\wt s>0$ tale che $K_{\wt s}=\emptyset$. Definiamo $\la_n=\wt s$.
\end{proof}

\begin{corollary}
Per ogni spazio topologico $X$, $f:C_C\to X$ \`e continua se e solo se $f\res{X_n}$ continua. Poich\'e su $X_n$ continua equivale a sequenzialmente continua, $f$ \`e continua se e solo se \`e sequenzialmente continua.
\end{corollary}



\subsection{Lo spazio \texorpdfstring{$C_C^0(\Omega)$}{CC0Omega}}

\begin{definition}[Continue a supporto compatto]
Fissato $\Omega\subseteq\R^n$ aperto consideriamo lo spazio $C_C^0(\Omega)$ come limite induttivo stretto degli spazi di Banach
\[C^0_K=\cpa{f:\R^n\to \R\mid \supp f\subseteq K}\]
al variare di $K\in k(\Omega)$.
\end{definition}

\begin{remark}
Possiamo equivalentemente considerare il limite di $C_{K_j}^0$ con $K_j$ compatti, $K_{j}\subseteq int(K_{j+1})$ e $\bigcup K_j=\Omega$.
\end{remark}

\begin{remark}
$C_{K_j}^0$ \`e chiuso in $C^0_{K_{j+1}}$ quindi per propriet\`a generali dei limiti induttivi stretti (\ref{PrProprietaLimitiInduttiviStretti})
\begin{enumerate}
    \item $A$ limitato in $C^0_C(\Omega)$ se e solo se $A$ \`e contenuto e limitato in qualche $C^0_K$
    \item $C^0_K$ sono sottospazi chiusi di $C_C^0(\Omega)$. 
    \item $C_C^0(\Omega)$ \`e localmente convesso, sequenzialmente completo ma non metrizzabile (di nuovo per Baire (\ref{ThBaire})).
\end{enumerate}
\end{remark}


\begin{exercise}
La topologia $LF$ di $C_C^0(\Omega)$ \`e STRETTAMENTE meno fine della topologia di limite induttivo topologico.
\end{exercise}

\begin{notation}
Poniamo $C^0(\Omega)_+=\cpa{\sigma:\Omega\to\R_+\text{ continue}}$.
\end{notation}

\begin{proposition}[Costruzione di una famiglia di seminorme]\label{PrTopologizzazioneInNormeDiLimiteInduttivoContinueASupportoCompatto}
La famiglia $\cpa{p_\sigma}_{\sigma\in C^0(\Omega)_+}$ di seminorme date da
\[p_\sigma(u)=\norm{\sigma u}_\infty\quad \forall u\in C_C^0(\Omega)\]
topologizza lo spazio $C_C^0(\Omega)$.
\end{proposition}
\begin{proof}
Diamo alcune definizioni:
\begin{itemize}
    \item Sia $\displaystyle \vp(x)=\frac1{dist(x,\Omega^c)}+\norm x$. Nota che $\cpa{\vp\leq c}$ \`e compatto perch\'e $\vp$ tende a $\infty$ vicino ai bordi.
    \item Posto $K_i=\cpa{x\in\Omega\mid \abs{\vp(x)-i}\leq 1}\in k(\Omega)$ si ha $\Omega=\bigcup_{i\geq 0}int(K_i)$ e $K_i\cap K_j=\emptyset$ se $\abs{i-j}\geq 2$.
    \item Poniamo $\eta_j=(1-\abs{\vp(x)-j})_+$, segue che $\eta_j\in C_C^0(\Omega)$, $0\leq \eta_j\leq 1$, $\supp \eta_j\subseteq K_j$ e $\sum_{j\geq 0}\eta_j=1$ in quanto, per ogni $t$,
    \[\sum_{j\geq 0}(1-\abs{t-j})_+=1.\]
\end{itemize}
Sia $U$ aperto convesso di $O$ in $C^0_C(\Omega)$. Vogliamo trovare $\sigma\in C^0(\Omega)$ tale che $\cpa{p_\sigma\leq 1}\subseteq U$ (cio\`e la topologia indotta da $\cpa{p_\sigma}$ \`e pi\`u fine della $LF$).

Per $j\geq 0$ sia $\delta_j=\inf\cpa{\norm u_\infty\mid u\in C^0_{K_j}\bs U}$, che \`e strettamente positiva (in quanto $U\cap C^0_K$ \`e intorno di $0$ in $C^0_{K_j}$) e contiene la palla
\[\cpa{\norm u_\infty<\delta,\ u\in C^0_{K_j}}.\]
Definiamo $\rho\in C^0(\Omega)_+$ come segue:
sia $\e_j$ il minimo di $\cpa{2^{-j-2}\delta_{j-1},2^{-j-1}\delta_{j},2^{-j}\delta_{j+1}}$ e consideriamo la funzione
\[\rho=\sum_{j\geq0}\e_j\eta_j\]
questa funzione \`e positiva, \`e una combinazione convessa di tre degli $\e_j$. Sia $\sigma=\frac1\rho$.

Notiamo ora che per ogni $u\in C^0_C(\Omega)$ tale che
\[\abs{u(x)}\leq \rho(x)\]
si ha $u\in U$ (cio\`e $\cpa{\norm{\sigma(u)}_{\infty}\leq 1}\subseteq U$), infatti se $\abs{u}\leq \rho$ allora per ogni $i$ si ha $u\eta_i\in C^0_{K_i}$ e
\[\abs{u\eta_i}\leq \rho\eta_i=\sum_{i\geq 0}\e_j\eta_j\eta_i=\sum_{i-1\leq j\leq i+1}\e_j\eta_j\leq \max_{i-1\leq j\leq i+1}\cpa{\e_j}\leq 2^{-i-1}\delta_i\]
quindi $\abs{2^{i+1}u\eta_i}_{\infty}\leq \delta_i$ e siccome $2^{i+1}u\eta_i\in C^0_{K_i}$ allora per la scelta di $K_i$ si ha $2^{i+1}u\eta_i\in U$. Allora
\[u=\sum_i u\eta_i=\sum_i 2^{-i-1} (2^{i+1}u\eta_i)\]
e dato che $U$ \`e convesso questo mostra $u\in U$.





L'altra inclusione delle topologie deriva da: per ogni $j$, $C^0_{K_j}\inj C^0_C$ \`e continua rispetto alla famiglia di seminorme $\cpa{p_\sigma}$ e quindi questa topologia \`e meno fine della topologia limite. La continuit\`a segue perch\'e per ogni $u\in C^0_K$ e ogni $\sigma\in C^0(\Omega)_+$ si ha
\[p_\sigma(u)=\norm{\sigma u}_\infty\leq \norm{\sigma}_{\infty,K}\norm u_\infty.\]
\end{proof}

\begin{remark}
Data $f\in C^0(\Omega)$ possiamo considerare su $C^0_C(\Omega)$ l'operatore di moltiplicazione per $f$:
\[M_f:\funcDef{C^0_C(\Omega)}{(C^0_b(\Omega),\normd_\infty)}{u}{fu}.\]
La topologia di $C^0_C(\Omega)$ ($LF$) coincide con la topologia debole della famiglia $\cpa{M_f}$, cio\`e \`e la topologia iniziale associata a questa famiglia.
\end{remark}




\subsection{Lo spazio \texorpdfstring{$\Dc(\Omega)$}{DOmega}}
\begin{definition}
Poniamo
\[\Dc(\Omega)=\cpa{f\in C^\infty(\Omega)\mid \exists K\in k(\Omega)\ t.c.\ \supp f\subseteq K}.\]
\end{definition}
\begin{remark}
Possiamo dare a $\Dc(\Omega)$ la topologia di limite induttivo stretto degli $C^\infty_K$.
\end{remark}

\begin{remark}
Come prima, su $C^\infty_K$ questa \`e la topologia indotta dalle seminorme $\cpa{p_m}_{m\geq0}$ con
\[p_m(f)=\max_{\abs{\al}\leq m}\norm{\del^\al f}_{\infty}\]
che inducono topologia di SVTLC, metrico completo (cio\`e di Fr\'echet).
\end{remark}

\begin{remark}
$A\subseteq \Dc(\Omega)$ \`e limitato se e solo se \`e contenuto e limitato in $C^\infty_K$.
\end{remark}


Diamo una seconda descrizione della topologia di $\Dc(\Omega)$ in termini di seminorme.
\begin{definition}
Dati $\sigma,\mu:\Omega\to \R_+$ continue definiamo la seminorma $p_{\sigma,\mu}$ su $\Dc(\Omega)$ come
\[p_{\sigma,\mu}(u)=\max_{\smat{x\in\Omega\\\al\in\N^n\\\abs{\al}\leq \mu(x)}}\abs{\sigma(x)\del^\al u(x)}\qquad \forall u\in \Dc(\Omega)=\bigcup_{K\in k(\Omega)}C^\infty_K.\]
Abbiamo buona definizione perch\'e per ogni $u$ in $\Dc(\Omega)$ esiste $K$ compatto tale che $u\in C^\infty_K$, quindi il massimo ha senso in quanto basta considerare $x\in K$ al posto di $x\in \Omega$.
\end{definition}

\begin{definition}[Funzione propria]
$f:X\to \R$ funzione continua \`e \textbf{propria} se $\cpa{f\leq c}$ \`e compatto in $X$.
\end{definition}

\begin{remark}[Formula di Newton per derivate]
Vale l'identit\`a
\[\del^\beta(u\cdot v)=\sum_{\al\leq \beta}\binom{\beta}\al\del^\al u\del^{\beta-\al}v\]
dove
\[\binom\beta\al=\prod_{1\leq i\leq n}\binom{\beta_i}{\al_i}.\]
\end{remark}

\begin{proposition}\label{PrTopologizzazioneInNormeDiLimiteInduttivoDOmega}
La topologia $LF$ di $\Dc(\Omega)$ \`e indotta dalle seminorme $\cpa{p_{\sigma,\mu}}$.
\end{proposition}
\begin{proof}
Mostriamo che $LF$ \`e pi\`u fine:

Se $\cpa{p_m}$ sono le seminorme date prima che topologizzano $C^\infty_K$ e $u\in C^\infty_K$ allora
\[p_{\sigma,\mu}(u)\leq\norm{\sigma}_{\infty,K}\cdot p_{\norm{\mu}_{\infty,K}}(u)\]
dunque le inclusioni $C^\infty_K$ in $\Dc(\Omega)$ sono continue per le seminorme $\cpa{p_{\sigma,\mu}}$, ovvero per ogni $K$ \`e continua
\[(C^\infty_K,\cpa{p_m})\inj(\Dc(\Omega),\cpa{p_{\sigma,\mu}})\]
e quindi \`e continua anche la mappa identit\`a
\[(\Dc(\Omega),LF)\to(\Dc(\Omega),\cpa{p_{\sigma,\mu}})\]
per definizione di topologia limite induttivo.

\bigskip

\noindent Mostriamo ora che $LF$ \`e meno fine:

Sia $\vp:\Omega\to\R$ di classe $C^\infty$ con $\vp(x)>0$ per ogni $x\in \Omega$ e propria.

Definiamo a mano una partizione dell'unit\`a:
\begin{itemize}
    \item Definiamo $K_i=\cpa{x\in\Omega\mid \abs{\vp(x)-i}\leq 1}\in k(\Omega)$
    \item Sia $g:\R\to\R$ ci classe $C^\infty$ con $0\leq g\leq 1$, $\supp g\subseteq [-1,1]$, $g(t)=g(-t)$, $g(t)+g(1-t)=1$, cio\`e
\[\sum_{j\geq 0}g(t-j)=1\quad \forall t\geq 0\]
\item Sia $\eta_i(x)=g(\vp(x)-i)$.
\end{itemize}
Nota che $0\leq \eta_i\leq 1$, $\eta_i\in C^\infty$, $\sum_{i\geq 1}\eta_i(x)=1$ per ogni $x\in \Omega$, $\supp \eta_i\subseteq K_i$ e $\eta_i\eta_j=0$ se $\abs{i-j}\geq 2$.


Sia $U$ un intorno di $0$ convesso\footnote{sappiamo che $LF$ \`e una topologia di SVTLC} in $(\Dc(\Omega),LF)$, allora per ogni $i\geq 0$ si ha $U\cap C_{K_i}^\infty$ \`e un intorno di $0$ in $C^\infty_{K_i}$ per definizione, quindi esistono $m_i\in\N$ e $\delta_i>0$ tali che
\[\cpa{f\in C^\infty_{K_i}\mid p_{m_i}(f)\leq \delta_i}\subseteq U.\]
Definiamo $\sigma,\mu\in C^0(\Omega)_+$ incollando i numeri $\delta_i$ e $m_i$ con la partizione di unit\`a $\cpa{\eta_i}$:
\begin{itemize}
    \item $\ell_i\doteqdot \max_{\abs{i-j}\leq 1}m_j=\max\cpa{m_{i-1},m_i,m_{i+1}}$
    \item $\mu(x)=\sum_{j\geq 0}\ell_i\eta_j$
    \item quindi per ogni $i\geq 0$ si ha $m_i\leq \min_{\abs{1-j}\leq 1}\ell_i=\min\cpa{\ell_{i-1},\ell_i,\ell_{i+1}}$ e per ogni $x\in K_i$
\[\mu(x)=\sum_{\abs{i-j}\leq 1}\ell_j\eta_j\geq \min_{\abs{i-j}\leq 1}\ell_i\pa{\sum_i\eta_i}=\min_{\abs{i-j}\leq 1}\ell_i\geq m_i.\]
\end{itemize}
\begin{itemize}
    \item $n_i=2^{-i-1-m_i}p_{m_i}(\eta_i)\ii\delta_i$
    \item $\e_i=\min_{\abs{i-j}}n_j$
    \item Per ogni $i\geq 0$
\[n_i\geq \max\cpa{\e_{i-1},\e_i,\e_{i+1}}.\]
\item Definiamo $\sigma(x)=\pa{\sum_{i\geq 0}\e_i\eta_i}\ii$.
\item Per ogni $x\in K_i$
\[\sigma(x)\ii=\sum_{j\geq 0}\e_j\eta_j=\sum_{\abs{i-j}\leq 1}\e_j\eta_j\leq\max_{\abs{j-i}\leq 1}\e_j\leq n_i\]
\end{itemize}
Dunque per ogni $i\geq 0$ e $x\in K_i$
\[\mu(x)\geq m_i,\qquad \sigma(x)\ii\leq n_i\]
quindi 
\[\cpa{f\in\Dc(\Omega)\mid p_{\sigma,\mu}(f)<1}\subseteq U\]
infatti se $f$ appartiene a questo insieme allora per ogni $i\geq 0$ la funzione $2^{i+1}\eta_i f$ appartiene a $C^\infty_{K_i}$ e ha seminorma $p_{m_i}$ minore di $\delta_i$, cio\`e per ogni $\beta$ tale che $\abs\beta\leq m_i$ si ha
\begin{align*}
    \abs{\del^\beta(2^{i+1}\eta_i f)}=&2^{i+1}\abs{\del^\beta(\eta_i f)}=2^{i+1}\abs{\sum_{\al\leq \beta}\binom\beta\al\del^{\beta-\al}\eta_i\del^\al f}\leq\\
    \leq&2^{i+1}\pa{\sum_{\al\leq \beta}\binom\beta\al}p_{m_i}(\eta_i)\pa{\sigma(x)\max_{\abs\al\leq m_i}\abs{\del^\al f} }\sigma(x)\ii\pasgnlmath\leq{\smat{\abs{\beta}\leq m_i\\\abs\al\leq m_i\leq \mu(x)}}\\
    \leq&2^{i+1+m_i}p_{m_i}(\eta_i)\cdot p_{\sigma,\mu}(f)\cdot \sigma(x)\ii\pasgnlmath\leq{\sigma(x)\ii\leq n_i}\\
    \leq&2^{i+1+m_i}p_{m_i}(\eta_i)\cdot p_{\sigma,\mu}(f)\cdot2^{-i-1-m_i}p_{m_i}(\eta_i)\ii\delta_i=\\
    =&p_{\sigma,\mu(f)}\delta_i<1\cdot \delta_i.
\end{align*}
Dunque se $p_{\sigma,\mu}(f)<1$, la funzione $2^{i+1}\eta_i f$ \`e tale che
\[p_{m_i}(2^{i+1}\eta_i f)\leq 1\]
quindi, poich\'e $C^\infty_{K_i}\cap\cpa{p_{m_i}\leq \delta_i}\subseteq U$, questo mostra che $2^{i+1}\eta_i f\in U$ e quindi
\[f=\sum_{i\geq 0}2^{-i-1}(2^{i+1}\eta_i f)\]
\`e combinazione convessa finita di elementi di $U$ e poich\'e abbiamo preso $U$ convesso questo conclude.
\end{proof}

\begin{exercise}
La moltiplicazione (puntuale)
\[\funcDef{\Dc(\Omega)\times\Dc(\Omega)}{\Dc(\Omega)}{(f,g)}{fg}\]
\`e continua? S\`i.

Sono continue la moltiplicazioni
\[\funcDef{C^0_C(\Omega)\times C^0_C(\Omega)}{C^0_C(\Omega)}{(f,g)}{fg}\text{ e }\funcDef{C_C\times C_C}{C_C}{(f,g)}{fg}\text{ ?}\]
\end{exercise}
\begin{solution}
Usare le seminorme e la caratterizzazione di continuit\`a per le bilineari (\ref{PrBilineareSeparatamenteContinuaEContinua}).
\end{solution}


\section{Altre propriet\`a di \texorpdfstring{$\Dc(\Omega)$}{DOmega}}
\subsection{Spazi barilati}
\begin{definition}[Botte e spazi barilati]
Una \textbf{botte} o \textbf{barile} in $X$ SVTLC \`e un insieme
\begin{itemize}
    \item assorbente
    \item assolutamente convesso
    \item chiuso.
\end{itemize}
Affermiamo che $X$ \`e uno \textbf{spazio botte} / \textbf{spazio barilato} (\textbf{barreled space}) se ogni barile \`e un intorno di $0$.
\end{definition}
\begin{remark}
Ogni spazio di Fr\'echet \`e uno spazio botte.
\end{remark}
\begin{proof}
Riadatta dimostrazione di Banach-Steinhaus (\ref{ThBanachSteinhausUniformeLimitatezza}): 

\noindent
Se $B\subseteq X$ botte allora $X=\bigcup nB$ in quanto assorbente. Per Baire (\ref{ThBaire}) uno degli $nB$ (e quindi $B$) ha parte interna non vuota. Poich\'e 
\[\frac12 (int(B)-int(B))\subseteq \frac12(B-B)=B\]
si ha che $B$ \`e intorno di $0$.
\end{proof}

\begin{remark}
Limiti induttivi di spazi barilati sono barilati.
\end{remark}
\begin{proof}
Sia $X_\infty=\varinjlim X_n$ con $X_n$ barilati. Sia $B\subseteq X_\infty$ botte. Allora per ogni $n$ si ha $B\cap X_n$ botte in $X_n$ (assorbente, assolutamente convesso perch\'e $X_n$ sottospazio vettoriale, chiuso perch\'e $X_n\inj X_\infty$ continua). Allora $B\cap X_n$ \`e intorno di $0$ per ogni $n$ ed \`e convesso, quindi $B$ \`e un intorno di $0$ in $X_\infty$.
\end{proof}
\begin{corollary}
    Ogni $LF$-spazio (limite induttivo di Fr\'echet) \`e barilato. In particolare anche $\Dc(\Omega)$.
\end{corollary}




\subsection{Spazi Bornologici}

\begin{definition}[Spazi Bornologici]
Un insieme $B\subseteq X$ SVT si dice \textbf{Bornofago} se assorbe ogni insieme limitato.
\smallskip

\noindent
$X$ SVTLC \`e \textbf{Bornologico} se ogni sottoinsieme (assolutamente)convesso\footnote{chiedere assolutamente convesso o convesso \`e equivalente} e bornofago \`e un intorno di $0$.
\end{definition}

\begin{proposition}\label{PrInInumerabileAssorbenteImplicaIntornoDi0}
    Se $X$ \`e SVT I-numerabile e $C\subseteq X$ non \`e un intorno di $0$ allora $C$ non \`e assorbente.
\end{proposition}
\begin{proof}
Se $C$ non \`e intorno di $0$ allora esiste una successione $(x_n)$ con $x_n\notin C$ per ogni $n$ e tale che $x_n\to 0$.

Se $p$ \`e una paranorma per $X$ (\ref{SVTINumerabileVieneDaParanorma}) allora $p(x_n)\to 0$ a meno di estrarre una sottosuccessione. Si pu\`o quindi assumere $p(x_n)=o(1/n)$. Sia $y_n=nx_n\notin nC$. Nota che
\[p(y_n)=p(nx_n)\leq np(x_n)=o(1)\]
quindi $\cpa{y_n}$ \`e limitato in quanto $y_n\to 0$ ma non \`e assorbito da $C$ per costruzione. 
\end{proof}
\begin{corollary}
Ogni SVTLC I-numerabile \`e bornologico.
\end{corollary}

\begin{fact}
    Ogni limite induttivo di spazi bornologici \`e bornologico
\end{fact}
\begin{proof}
Se $X_n$ bornologici con limite $X_\infty$ allora sia $B$ convesso e bornofago in $X_\infty$, allora $B\cap X_n$ \`e ancora convesso. $B\cap X_n$ \`e ancora bornofago perch\'e ogni limitato in $X_n$ \`e limitato in $X_\infty$ per continuit\`a delle inclusioni. Quindi $B\cap X_n$ \`e intorno di $0$ convesso in $X_n$ e quindi $B$ stesso \`e intorno di $0$ convesso di $X_\infty$.
\end{proof}
\begin{corollary}
    Ogni spazio $LF$ \`e bornologico e quindi in particolare anche $\Dc(\Omega)$.
\end{corollary}
\chapter{Distribuzioni}
\begin{definition}[Distribuzione]
Fissato $\Omega\subseteq \R^n$ aperto, una \textbf{distribuzione} \`e una forma lineare e continua su $\Dc(\Omega)$. Lo spazio delle distribuzioni \`e dunque il duale topologico di $\Dc(\Omega)$, cio\`e $\Dc(\Omega)^\ast$, che per\`o in questo contesto viene spesso indicato $\Dc'(\Omega)$ per ragioni storiche.
\end{definition}

\begin{remark}
$u:\Dc(\Omega)\to \R$ lineare \`e una distribuzione SE \`e continua, cio\`e se per ogni $K\in k(\Omega)$ si ha che
\[u\res{C^\infty_K}:C_K^\infty\to \R\]
\`e continua, o equivalentemente se per ogni $K\in k(\Omega)$ esistono $m\in\N$ e $C>0$ tali che
\[\abs{\ps{u,f}}\leq C p_m(f)\quad \forall f\in C_K^\infty.\]
\end{remark}

\begin{definition}[Ordine di una distribuzione]
Se $m\in\N$ \`e tale che per ogni $K\in k(\Omega)$ esiste $C>0$ tale che
\[\abs{\ps{u,f}}\leq C p_m(f)\quad \forall f\in C_K^\infty.\]
si dice che $u$ ha \textbf{ordine minore o uguale a $m$}. Se non esiste un tale $m$ diciamo che $u$ ha ordine $\infty$, mentre se esite allora l'\textbf{ordine di $u$} \`e il minimo tale $m$. Indichiamo l'ordine di $u$ con $\ord(u)$.
\end{definition}

\begin{remark}
Intuitivamente l'ordine \`e ``il massimo ordine di derivate" che pu\`o apparire scrivendo $u$ esplicitamente.
\end{remark}

\begin{example}
La valutazione in un punto \`e una distribuzione di ordine $0$. La valutazione della prima derivata in un punto \`e una distribuzione di ordine $1$.
\end{example}


\begin{remark}
Valgono:
\begin{itemize}
\item Per ogni $K\in k(\Omega)$ e per ogni $f_j\to 0$ in $C^\infty_K$ vale $\ps{u,f_j}\to0$.
\item Per ogni $f_j\to 0$ in $\Dc(\Omega)$, $\ps{u,f_j}\to 0$.
\item Per (\ref{PrTopologizzazioneInNormeDiLimiteInduttivoDOmega}) esistono $\sigma,\mu$ tali che
\[\abs{\ps{u,f}}\leq p_{\sigma,\mu}(f)\qquad \forall f\in \Dc(\Omega)\]
\end{itemize}
\end{remark}

\begin{definition}[Integrabili su compatti]
Definiamo le funzioni \textbf{integrabili su compatti} $L^1_{loc}(\Omega)$ come le funzioni $u$ su $\Omega$ tali che per ogni $K\in k(\Omega)$ si ha $u\res K\in L^1(K)$.
\end{definition}


\begin{definition}[Inclusione delle localmente integrabili nelle distribuzioni]
Definiamo
\[T:\funcDef{L^1_{loc}(\Omega)}{\Dc'(\Omega)}{u}{T_u:\funcDef{\Dc(\Omega)}{\R}{f}{\int_\Omega fu\,dx}}\]
\end{definition}

\begin{proposition}\label{PrInclusioneLocalmenteIntegrabiliInDistribuzioni}
La mappa $T$ \`e ben definita e iniettiva.
\end{proposition}
\begin{proof}
L'integrale $\int_\Omega fu\,dx$ \`e ben definito perch\'e $f$ ha supporto compatto ed \`e continua (quindi $uf=u\res K f$ \`e integrabile).
$T_u$ \`e continua su ogni $C^\infty_K$ perch\'e
\[\abs{\ps{T_u,f}}\leq\norm{\mu_K}_1\norm{f}_{\infty}=\norm{\mu\res K}p_0(f).\]
L'iniettivit\`a \`e evidente perch\'e se $u\neq v$ in $L^1_{loc}$ allora esiste un insieme di misura non negativa dove non coincidono, opportune mollificazioni della caratteristica di questo insieme mostrano che $T_u\neq T_v$.
\end{proof}

\begin{example}
Le seguenti sono distribuzioni:
\begin{enumerate}
\item \textbf{Valutazioni di derivate}: Sia $x_0\in\Omega$ allora le seguenti mappe sono distribuzioni per ogni $\al$
\[\funcDef{\Dc(\Omega)}{\R}{f}{\del^\al f(x_0)}\]
in quanto $\abs{\del^\al f(x_0)}\leq p_m(f)$ per $m=\abs{\al}$.
\item \textbf{Integrale contro $u\in L_{loc}^1$ fissata}: l'immagine di $T$, cio\`e le mappe della forma
\[T_u:\funcDef{\Dc(\Omega)}{\R}{f}{\int_\Omega fu\,dx}\]
sono distribuzioni.
\item Se $(x_j)\subseteq\Omega$ con $x_j$ che esce da ogni compatto definitivamente (va verso il bordo) allora
\[\funcDef{\Dc(\Omega)}{\R}{f}{\sum_{i=0}^\infty\del^{\al_i} f(x_i)}\]
\`e una distribuzione (qualunque sia la successione degli $\al_i\in\N^n$ che consideriamo, tanto su ogni compatto la somma \`e finita).
\end{enumerate}
\end{example}

\begin{definition}[Bracket di Iverson]
Il \textbf{Bracket di Iverson} per una condizione booleana $\vp$ su un insieme $A$ \`e la funzione caratteristica di quella condizione, cio\`e
\[[x]=\chi_{\cpa{x\in A\mid \vp(x)}}.\]
\end{definition}

\begin{definition}[]
Dato $\Omega\subseteq\R^n$ aperto definiamo
\[\Theta=\cpa{\theta:\Omega\times \N^n\to[0,\infty), \text{``localmente finita"}}\]
cio\`e per ogni $x\in\Omega$ esiste $U$ intorno di $x$ tale che per ogni $y\in U$
\[\cpa{\al\in\N^n\mid \theta(y,\al)\neq 0}\quad\text{\`e finito}.\]
\end{definition}

\begin{remark}
Se $\theta\in\Theta$ allora per ogni $K\in k(\Omega)$ si ha che esiste $N\in\N$ tale che per ogni $x\in K$ e per ogni $\abs{\al}\geq N$ vale $\theta(x,\al)=0$.
\end{remark}
\begin{remark}
Ogni $\theta\in\Theta$ \`e maggiorato da una $\wt \theta$ della forma
\[\wt \theta(x,\al)=\sigma(x)[\abs{\al}\leq \mu(x)]\]
dove $\sigma,\mu\in C^0(\Omega)_+$ e $[\cdot]$ \`e il Bracket di Iverson,
\[\mu(x)=\max\cpa{\abs{\al}\mid \theta(x,\al)\neq 0}\]
eccetera (vedi capitolo precedente).
\end{remark}

\begin{notation}
Per ogni $\theta\in \Theta$ e $u\in \Dc(\Omega)$ poniamo
\[\norm u_\theta=\norm{\theta\cdot \del^\bullet u(\bullet)}_{\infty,\Omega\times \N^n}\]
\end{notation}

\section{Estensioni e operazioni sulle distribuzioni}
\subsection{Estensioni}
Possiamo considerare estensioni di operatori su $\Dc(\Omega)$ a operatori su $\Dc'(\Omega)$ tramite le inclusioni
\[\Dc(\Omega)\subseteq C^1(\Omega)\subseteq L^1_{loc}\overset T\subseteq \Dc'(\Omega)\]
dove come prima
\[T:\funcDef{L^1_{loc}(\Omega)}{\Dc'(\Omega)}{u}{T_u:\funcDef{\Dc(\Omega)}{\R}{f}{\int_\Omega fu\,dx}}\]
\begin{remark}
Ponendo $\ps{T_u,\vp}=\int_\Omega u\vp dx$, per ogni $K\in k(\Omega)$ e $\vp\in C^\infty_K$ si ha
\[\abs{\ps{T_u,\vp}}\leq C_K\norm\vp_\infty\quad\text{dove }C_K=\int_K\abs udx=\norm u_{1,K}\text{ e }\norm\vp_\infty=p_0(\vp).\]
\end{remark}

\begin{exercise}
La mappa lineare $T:L^1_{loc}(\Omega)\to \Dc'(\Omega)$ \`e continua rispetto alle topologie
\begin{itemize}
    \item Su $L^1_{loc}(\Omega)$ consideriamo la topologia di spazio di Fr\'echet indotta dalle norme $\cpa{\normd_{1,K}}_{K\in k(\Omega)}$.
    \item Su $\Dc'(\Omega)$ consideriamo la topologia debole $\sigma(\Dc'(\Omega),\Dc(\Omega))$.
\end{itemize}
\end{exercise}
\begin{proof}
$T$ \`e continua per queste topologie se e solo se per ogni $\vp\in \Dc(\Omega)$ si ha che
\[\funcDef{L^1_{loc}(\Omega)}{\R}{u}{\ps{T_u,\vp}}\]
\`e continua e questo \`e vero se e solo se per ogni $K\in k(\Omega)$ esiste $C$ tale che per $u\in C^\infty_K$
\[\abs{\ps{T_u,\vp}}\leq C \norm{u}_{1,K}=C\int_K \abs{u}dx\]
ma questo \`e vero perch\'e
\[\abs{\ps{T_u,\vp}}=\abs{\int_\Omega u\vp dx}\leq \int_K\abs udx \norm\vp_{\infty}=\norm{u}_{1,K}\norm{\vp}_\infty.\]
\end{proof}

\begin{notation}
Se non c'\`e pericolo di confusione consideriamo $T$ come una inclusione e scriviamo $L^1_{loc}\subseteq \Dc'(\Omega)$ e $T_u=u$. Spesso si usa anche $u(\vp)$ al posto di $\ps{T_u,\vp}=\ps{u,\vp}$.
\end{notation}


\begin{definition}[Funzioni nulle al bordo]
Definiamo $C_0(X)$ come
\[C_0(X)=\cpa{f\in C(X)\mid \lim_{x\to\infty}f(x)= 0\text{ in }\wt X}\]
dove $\wt X=X\cup \cpa\infty$ \`e la compattificazione di Alexandroff.
\end{definition}

\begin{proposition}[Distribuzioni di ordine limitato si estendono a $(C_0^m)^\ast$]\label{PrDistribuzioniDiOrdineFinitoSiEstendonoAFunzionaliContinuiSuRegolariNulleAlBordo}
Le distribuzioni di ordine minore o uguale a $m$ si estendono a funzionali lineari continui su tutto 
\[C^m_0(\Omega)=\ol{C^\infty_C(\Omega)}^{C^m(\Omega)}=\cpa{f\in C^m(\Omega)\mid \forall \abs\al\leq m,\ \del^\al f\in C^0_0(\Omega)}.\]
\end{proposition}
\begin{proof}
Se $u\in \Dc'(\Omega)$ ha ordine $\leq m$ allora \`e continua per la topologia indotta da $C^m(\Omega)$ (infatti per ogni $K\in k(\Omega)$ esiste $C_K$ tale che $\abs{u(\vp)}\leq C_Kp_m(\vp)$). Quindi si estende per continuit\`a in modo unico a una forma lineare continua sulla chiusura (per la topologia di $C^m$), cio\`e $C^m_0(\Omega)\subseteq C^m(\R^n)$:

Fissiamo $K\in k(\Omega)$ e sia $dist(K,\Omega^c)=\e$. Sia $\vp$ funzione $C^\infty$ non negativa con supporto contenuto in $B(0,\e/4)$ e tale che $\int \vp=1$. Sia 
\[\eta=\vp\ast\chi_{K+\frac\e4B}.\]
Per costruzione $\eta=1$ su $K$ e $\eta=0$ in $\Omega\bs K+\frac\e2 B$.
Quindi dato $K\in k(\Omega)$ esiste $\eta\in C^\infty_C(\Omega)$ con $0\leq \eta\leq 1$ e $\eta\res K=1$.


Concludere mostrando che la chiusura in $C^m(\Omega)$ di $C_C^\infty(\Omega)$ \`e $C^m_0(\Omega)$ usando l'approssimazione via convoluzioni e la moltiplicazione per $\eta$.
\end{proof}

\subsection{Derivazione}
\begin{proposition}
L'operatore $\del_i$ di derivazione su $\Dc(\Omega)$ si estende ad un operatore su $\Dc'(\Omega)$ nel senso di sopra ponendo
\[\del_i=-\del_i^\ast:\funcDef{\Dc'(\Omega)}{\Dc'(\Omega)}{u}{-u\circ \del_i}\]
\end{proposition}
\begin{proof}
Per $u\in C^1(\Omega)$ e $\del_i u\in C^0(\Omega)$ si ha che $u$ e $\del_i u$ appartengono a $L^1_{loc(\Omega)}$. Le distribuzioni $T_u$ e $T_{\del_i u}$ sono legate dalla relazione data dall'integrazione per parti\footnote{i termini al bordo spariscono perch\'e tutto ha supporto compatto.}:
\[\ps{T_u,\del_i\vp}=\int_\Omega u\del_i\vp dx=-\int_\Omega\del_i u\vp dx=-T_{\del_i u} \vp\]
cio\`e $T_{\del_iu}=-T_u\circ \del_i$ (qu\`i $\del_i$ \`e inteso in senso classico).
\smallskip

\noindent
Definendo quindi
\[\del_i:\funcDef{\Dc'(\Omega)}{\Dc'(\Omega)}{u}{-u\circ \del_i}\]
abbiamo una estensione di $\del_i$ a $\Dc'(\Omega)$. Il codominio \`e effettivamente $\Dc'(\Omega)$ perch\'e $\del_i:\Dc(\Omega)\to\Dc(\Omega)$ \`e lineare e continua, quindi $-u\circ \del_i$ \`e composizione di due mappe lineari e continue.
\end{proof}



\begin{remark}
Pi\`u in generale \`e definito $\del^\al$ su $\Dc'(\Omega)$ e vale $\ps{\del^\al u,\vp}=(-1)^{\abs\al}\ps{u,\del^\al\vp}$.
\end{remark}



\begin{remark}
$\ord(\del_i u)\leq\ord(u)+1$.
\end{remark}



\subsection{Moltiplicazione per funzione liscia}
\begin{notation}
Definiamo $\Ec(\Omega)=C^\infty(\Omega)$.
\end{notation}

\begin{remark}
Se $f\in \Ec(\Omega)$ \`e definito un operatore lineare
\[M_f:\funcDef{\Dc(\Omega)}{\Dc(\Omega)}{\vp}{f\vp}\]
\end{remark}

\begin{remark}
$M_f$ \`e continuo.
\end{remark}
\begin{proof}
A livello dei $C^\infty_K$ il supporto resta contenuto in $K$ dopo la moltiplicazione e
\begin{align*}
    p_{m,K}(M_f(\vp))=&p_{m,K}(f\vp)=\max_{\abs\al\leq m}\norm{\del^\al (f\vp)}_\infty=\\
    =&\max_{\abs\al\leq m}\norm{\sum_{\beta\leq \al}\binom\al\beta\del^\beta f\del^{\al-\beta}\vp}_\infty\leq\\
    \leq& (2^m p_{m,K}(f))p_m(\vp).
\end{align*}
\end{proof}

\begin{remark}
Applicando $T$ \`e definito un operatore di moltiplicazione sulle distribuzioni
\[M_f:\funcDef{\Dc'(\Omega)}{\Dc'(\Omega)}{u}{fu}\]
dove
\[(fu)(\vp)=u(\vp f).\]
\end{remark}




\section{Distribuzioni di ordine limitato come misure}
\begin{remark}
C'\`e una immersione isometrica
\[\funcDef{C^m_0(\Omega)}{C^0_0(\wt \Omega)^N}{\vp}{(\del^\al\vp)_{\abs\al\leq m}}\]
dove $N=\#\cpa{\al\in\N^n\mid \abs\al\leq m}$ e $\wt \Omega$ \`e la compattificazione di $\Omega$ a un punto.
\smallskip

\noindent
Segue che $C^m_0(\Omega)^\ast\inj (C^0_0(\Omega)^\ast)^N$ per Hahn-Banach (\ref{ThHahnBanach}).
\end{remark}

\begin{fact}
Se $u\in \Dc'(\Omega)$ con $\ord(u)\leq m$, quindi tale che si estende a $u\in C^m_0(\Omega)^\ast$ (\ref{PrDistribuzioniDiOrdineFinitoSiEstendonoAFunzionaliContinuiSuRegolariNulleAlBordo}), allora grazie a $C^m_0(\Omega)^\ast\inj (C^0_0(\Omega)^\ast)^N$ otteniamo che esistono $N$ misure di Radon\footnote{boreliane, finite sui compatti, regolari da fuori sui Boreliani e regolari da dentro per gli aperti.} $\cpa{\mu_\al}_{\abs\al\leq m}$ tali che
\[\ps{u,\vp}=\sum_{\abs\al\leq m}\int_\Omega \vp d\mu_\al=\sum_{\abs\al\leq m}\vp\rho_\al d\nu_\al\]
con $\nu_\al\geq 0$ e $\abs\rho\leq 1$.
\end{fact}

\begin{remark}
Queste misure non sono uniche perch\'e abbiamo usato Hahn-Banach (\ref{ThHahnBanach}) per estendere $u$ da $C^m_0(\Omega)$ a $C^0_0(\wt \Omega)^N$.
\end{remark}


\begin{remark}
Le distribuzioni di ordine $0$ sono misure di Radon, cio\`e
\[\ord(u)=0\implies u(\vp)=\int_\Omega\vp d\mu\quad \mu\text{ misura relativa finita sui compatti.}\]
\end{remark}

\begin{fact}
    Ogni distribuzione positiva $u\in \Dc'(\Omega)$, cio\`e tale che $u(\vp)\geq 0$ per ogni $\vp\in \Dc(\Omega)$ tale che $\vp\geq 0$ ha ordine 0.
\end{fact}
\begin{proof}
Per ogni $K\in k(\Omega)$ sia $\eta\in C^\infty(\Omega)$ con $0\leq \eta\leq 1$ con $\supp\eta\subseteq \Omega$ e $\eta=1$ su $K$. Allora per ogni $\vp\in C^\infty_K$ vale
\[\norm\vp_\infty\eta\pm \vp\geq 0\]
infatti su un punto di $K$ $\eta=1$ e quindi applico la definizione di $\normd_\infty$, mentre su un punto che non appartiene a $K$ abbiamo $\vp=0$ e quindi la disuguaglianza continua a valere.
Dunque 
\[u(\norm\vp_\infty\eta\pm\vp)=\norm\vp_\infty u(\eta)\pm u(\vp)\geq 0\]
e quindi $\abs{u(\vp)}\leq u(\eta)\norm\vp_\infty=C_Kp_0(\vp)$, cio\`e $u$ ha ordine $0$.
\end{proof}


\section{Successioni di distribuzioni}
\begin{proposition}[]\label{PrSuccessioniDiDistribuzioni}
Sia $(u_j)\subseteq \Dc'(\Omega)$ una successione convergente puntualmente, cio\`e $(u_j(\vp))$ converge in $\R$ per ogni $\vp\in \Dc(\Omega)$. Allora il limite
\[u(\vp)=\lim_{j}u_j(\vp)\]
definisce una distribuzione $u$. Inoltre per ogni $K\in k(\Omega)$ esiste $C_K\geq 0$ e $m\in\N$ tale che per ogni $j\geq 0$
\[\abs{u_j(\vp)}\leq C_Kp_m(\vp)\quad\forall \vp\in C^\infty_K.\]
\end{proposition}
\begin{proof}
Sia $K\in k(\Omega)$. Notiamo che $u_j\res{C^\infty_K}$ \`e una successione puntualmente limitata su $C^\infty_K$ (per ogni $\vp\in C^\infty_K$ si ha $\abs{u_j(\vp)}\leq C_\vp$). Siccome $C^\infty_K$ \`e uno spazio di Fr\'echet (e quindi di Baire) per Banach-Steinhaus (\ref{ThBanachSteinhausUniformeLimitatezza}) la successione $(u_j\res{C^\infty_K})$ \`e limitata in $C^\infty_K$, cio\`e vale la disuguaglianza affermata.

Allora questa stima vale anche per $u$ limite puntuale, il quale \`e anche ovviamente lineare, quindi $u\in \Dc'(\Omega)$.
\end{proof}

\begin{corollary}
    Se $(\vp_j)\subseteq\Dc(\Omega)$ \`e una successione convergente in $\Dc(\Omega)$ a $\vp$ allora
    \[u_j(\vp_j)\to u(\vp)\]
\end{corollary}
\begin{proof}
$\vp_j\to \vp$ implica che esiste $K\in k(\Omega)$ tale che $\vp_j\in C^\infty_K$ e $\vp_j\to \vp$ in $C^\infty_K$ e ora che sappiamo che tutte le funzioni hanno supporto nello stesso $K$ possiamo usare la disguguaglianza
\[\abs{u_j(\vp_j)}\leq C_Kp_m(\vp_j)\to 0\]
dove per l'ultimo limite ho supposto senza perdita di generalit\`a $\vp=0$ (altrimenti sostituisco $\vp_j$ con $\vp_j-\vp$).
\end{proof}


\section{Distribuzioni sono un fascio}
\begin{proposition}\label{PrDistribuzioniSonoUnFascio}
Il funtore $\Dc':(\text{Aperti di }\R^n)^{op}\to (SVTLC)$ definisce un fascio, cio\`e:
\begin{enumerate}
    \item Per ogni aperto $\Omega$ di $\R^n$ \`e ben definito $\Dc'(\Omega)$.
    \item Per ogni contenimento $U\subseteq V\subseteq \R^n$ di aperti abbiamo una funzione di restrizione
    \[\rho^V_U:\Dc'(V)\to\Dc'(U)\]
    tale che $\rho^U_U=id_{\Dc'(U)}$ e se $U\subseteq V\subseteq W$ allora 
    \[\rho^V_U\circ\rho^W_V=\rho^W_U.\]
    \item Se $\Omega$ aperto ammette un ricoprimento aperto $\cpa{\Omega_i}$ e $u\in \Dc'(\Omega)$ allora $\rho^\Omega_{\Omega_i}(u)\doteqdot u_i=0$ per ogni $i$ implica che $u=0$.
    \item Se $\Omega$ aperto ammette un ricoprimento aperto $\cpa{\Omega_i}$ e per ogni $i$ abbiamo $u_i\in \Dc'(\Omega_i)$ tali che 
    \[\rho^{\Omega_i}_{\Omega_{i}\cap \Omega_j}(u_i)=\rho^{\Omega_j}_{\Omega_{i}\cap \Omega_j}(u_j)\]
    per ogni coppia $i,j$ allora esiste $u\in \Dc'(\Omega)$ tale che $\rho^\Omega_{\Omega_i}(u)= u_i$.
\end{enumerate}
\end{proposition}
\begin{proof}
Mostriamo le varie propriet\`a:
\setlength{\leftmargini}{0cm}
\begin{enumerate}
    \item Ovvio.
    \item Dati due aperti $U\subseteq V\subseteq \R^n$ esiste una inclusione
    \[\Dc(U)=\bigcup_{K\in k(U)}C^\infty_K\inj \Dc(V)=\bigcup_{K\in k(V)}C^\infty_K\]
    e quindi un operatore di restrizione
    \[\rho^V_U:\Dc'(V)\to \Dc'(U).\]
    Questo operatore chiaramente rispetta le due propriet\`a.
    \item Sia $u\in \Dc(\Omega)$ tale che
\[u\res{\Omega_j}=\rho^\Omega_{\Omega_j}(u)=0.\]
Per ogni $\vp$ in $\Dc(\Omega)$ esiste $F\subseteq I$ finito tale che $K=\supp \vp\subseteq \bigcup_{j\in F}\Omega_j$. Esiste inoltre una partizione di unit\`a $\cpa{\eta_j}_{j\in F}\subseteq \Dc(\Omega)$ tale che
\[\eta_j\in \Dc(\Omega_j)\quad\text e\quad \sum_{j\in F}\eta_j=1\text{ su }K.\]
Allora $f=\sum_{j\in F} f\eta_j$ e $u(f)=\sum_{j\in F}u(f\eta_j)=0$ in quanto $u\res{\Omega_j}=0$.
\item Sia $\cpa{\Omega_i}$ un ricoprimento aperto di $\Omega$ e per ogni $i$ abbiamo $u_i\in \Dc'(\Omega_i)$ tali che 
\[\rho^{\Omega_i}_{\Omega_{i}\cap \Omega_j}(u_i)=\rho^{\Omega_j}_{\Omega_{i}\cap \Omega_j}(u_j)\]
per ogni coppia $i,j$. Definiamo $u(\vp)$ per $\vp\in\Dc(\Omega)$ come segue:

Sia $K\in k(\Omega)$ tale che $\vp\in C^\infty_K$ e sia $F\subseteq I$ finito tale che $K\subseteq \bigcup_{i\in F}\Omega_i$. Siano $\cpa{\eta_j}_{j\in F}\subseteq C^\infty(\Omega)$ tali che $\supp\eta_j\subseteq \Omega_j$, $0\leq \eta_j\leq 1$ e $\sum_{j\in F}\eta_j=1$ su $K$.
Poniamo
\[u(\vp)=\sum_{j\in F}u_j(\vp\eta_j)\]
Notiamo che $\vp\eta_j\in\Dc(\Omega_j)$ quindi ha senso valutare $u_j$ nel prodotto.
La definizione non dipende dalla famiglia $\cpa{\eta_j}$ in quanto se $\eta_j'$ ha le stesse propriet\`a allora
\begin{align*}
    \sum_{j\in F}u_j(\vp\eta_j)=&\sum_{j\in F}u_j\pa{\sum_{i\in F}\vp\eta_j\eta_i'}=\sum_{i,j\in F}u_j(\vp\eta_j\eta_i')\pasgnl={ipotesi}\\
    =&\sum_{i,j\in F}u_i(\vp\eta_j\eta_i')=\sum_{i\in F}u_i(\vp\eta_i').
\end{align*}
Per costruzione $u$ eredita la linearit\`a e la continuit\`a delle $u_i$, quindi \`e un elemento di $\Dc'(\Omega)$.
\end{enumerate}
\setlength{\leftmargini}{0.5cm}
\end{proof}

\begin{remark}
Si pu\`o considerare pi\`u in generale il fascio delle distribuzioni su una variet\`a $C^\infty$ di dimensione $n$, basta incollare i fasci di distribuzioni su un ricoprimento di aperti omeomorfi a $\R^n$.
\end{remark}

\subsection{Distribuzioni a supporto compatto}
\begin{definition}[Supporto di una distribuzione]
Fissiamo una distribuzione $u\in \Dc'(\Omega)$. Sia $\Omega_0$ il pi\`u grande aperto tale che $u\res{\Omega_0}=0$. Il chiuso $\Omega\bs \Omega_0$ si dice \textbf{supporto} di $u$ e si indica $\supp(u)$.
\end{definition}
\begin{remark}
$\Omega_0$ \`e ben definito in quanto \`e l'unione di tutti gli aperti dove $u$ si restringe alla mappa nulla: Poich\'e $\Dc'$ \`e un fascio (\ref{PrDistribuzioniSonoUnFascio}) e per definizione $u\res{\Omega_0}$ ha tutte le restizioni a $U\subseteq \Omega_0$ aperto banali, $u\res{\Omega_0}=0$.
\end{remark}

\begin{definition}[Distribuzione a supporto compatto]
Se $\supp(u)$ \`e compatto, $u$ si dice \textbf{a supporto compatto}. Scriviamo l'insieme delle distribuzioni a supporto compatto con $\Dc'_C(\Omega)$.
\end{definition}

\begin{proposition}\label{PrSupportoCompattoEContinuita}
Se $u\in \Dc_C'(\Omega)$ e $K\in k(\Omega)$ allora valgono le implicazioni dall'alto verso il basso
\begin{enumerate}
    \item $\supp(u)\subseteq int(K)$.
    \item Esistono $C\geq 0$ e $m\in\N$ tali che per ogni $\vp\in\Dc(\Omega)$ si ha 
    \[\abs{u(\vp)}\leq Cp_{m,K}(\vp),\]
    cio\`e $u$ \`e continua.
    \item $\supp(u)\subseteq K$.
\end{enumerate}
\end{proposition}
\begin{proof}
    Mostriamo le due implicazioni
\setlength{\leftmargini}{0cm}
\begin{itemize}
\item[$\boxed{1.\implies2.}$] Siano $\supp(u)\subseteq int(K)$ e $\psi\in C^\infty(\Omega)$ con $\supp(\psi)\subseteq K$ e $\psi=1$ su un intorno $U$ di $\supp(u)$.

Allora per ogni $\vp\in \Dc(\Omega)$ si ha che $(1-\psi)\vp$ \` e nulla su $U$, quindi
\[\cpa{(1-\psi)\vp\neq 0}\subseteq \Omega\bs U\implies \supp((1-\psi)\vp)=\ol{\cpa{(1-\psi)\vp\neq 0}}\subseteq \Omega\bs U\]
Segue che $(1-\psi)\vp$ e $u$ hanno supporto disgiunto, dunque
\[0=u((1-\psi)\vp)=u(\vp)-u(\psi\vp),\]
cio\`e per ogni $\vp\in \Dc(\Omega)$ si ha
\[u(\vp)=u(\psi\vp).\]
Per continuit\`a di $u$ come elemento di $\Dc'(\Omega)$, poich\'e $\psi\vp\in C^\infty_K$, esistono $m\in \N$ e $C\geq 0$ tali che
\[\abs{u(\vp)}=\abs{u(\psi\vp)}\leq Cp_{m,K}(\psi\vp)\leq C'p_{m,K}(\vp)\]
dove l'ultima stima \`e un conto gi\`a visto che usa la formula di Leibnitz\footnote{\[p_{m,K}(\psi\vp)=\max_{\abs\al\leq m}\norm{\del^\al(\psi\vp)}_{\infty,K}=\max_{\abs\al\leq m}\norm{\sum_{\beta\leq \al}\binom \al\beta\del^{\al-\beta}\psi\del^{\beta} vp}_{\infty,K}\leq 2^mp_m(\psi)p_{m,K}(\vp).\]}, quindi $u$ \`e continua per la topologia indotta\footnote{e quindi potremmo estendere $u$ con Hahn-Banach (\ref{ThHahnBanach}).} da $\Ec(\Omega)$ su $\Dc(\Omega)$.
\item[$\boxed{2.\implies3.}$] Se per ogni $\vp\in \Dc(\Omega)$ vale $\abs{u(\vp)}\leq Cp_{m,K}(\vp)$ allora in particolare vale se $\vp$ ha supporto in $\Omega\bs K$, ma in tal caso $p_{m,K}(\vp)=0$, cio\`e $u$ \`e nulla su $\Dc(\Omega\bs K)$ e quindi il supporto \`e contenuto in $K$.
\end{itemize}
\setlength{\leftmargini}{0.5cm}
\end{proof}
\begin{corollary}
Le distribuzioni a supporto compatto in $\Omega$ si possono identificare con gli elementi di\footnote{la topologia su $\Ec(\Omega)$ \`e quella indotta dalle seminorme $\cpa{p_{m,K}}_{m\in\N,\ K\in k(\Omega)}$.} $\Ec'(\Omega)=(C^\infty(\Omega))^\ast$.
\end{corollary}
\begin{proof}
Se $u$ ha supporto compatto \`e continua per la topologia indotta da $\Ec(\Omega)$ su $\Dc(\Omega)$ e quindi per Hahn-Banach (\ref{ThHahnBanach}) si estende ad una forma lineare continua su tutto $\Ec(\Omega)$. Questa estensione \`e in realt\`a unica perch\'e $\Dc(\Omega)$ \`e denso in $\Ec(\Omega)$. In questo senso possiamo identificare $\Ec'(\Omega)$ con $\Dc_C'(\Omega)$ come spazi vettoriali.
\end{proof}

\begin{remark}
Se $u\in \Dc_C'(\Omega)$ allora ha anche ordine finito per il punto 1. della proposizione sopra (\ref{PrSupportoCompattoEContinuita}).
\end{remark}

\begin{example}[Non vale 3.$\implies$2. di (\ref{PrSupportoCompattoEContinuita})]
Sia $n=1$, $\Omega=\R$ e consideriamo $u\in\Dc'(\R)$ tale che
\[u(\vp)=\sum_{k\geq 1}\frac1k\pa{\vp\pa{\frac1k}-\vp(0)},\quad \forall \vp\in \Dc(\R).\]
La serie \`e assolutamente convergente perch\'e $\abs{\vp(\frac1k)-\vp(0)}\leq\norm{\dot\vp}_\infty \frac1k$ per Lagrange e 
\[\sum_{k\geq 1}\frac1k\abs{\vp\pa{\frac1k}-\vp(0)}\leq \pa{\sum_{k\geq 1}\frac1{k^2}}\norm{\dot\vp}_\infty.\]
Da questa scrittura si vede anche che $u$ dipende da $\vp$ con continuit\`a rispetto alla norma $\norm{\del\bullet}_{\infty}$.

Se $\vp$ ha supporto disgiunto da $K=\cpa0\cup\cpa{\frac1k}_{k\geq0}$ allora $\supp(u)\subseteq K$ (cio\`e vale la condizione 3.).

Eppure non vale la condizione 2. per $K$ infatti per $m\in \N$ sia $\vp_m$ una funzione $\Dc(\Omega)$ tale che $\vp_m=0$ su un intorno di $[0,\frac1{m+1}]$ e $\vp_m=1$ su un intorno di $[\frac1m,1]$, allora
\[\vp_m\pa{\frac1k}=\chi_{\cpa{k\leq m}},\quad \vp_m^{(j)}(x)=0\ \forall x\in K,\ \forall j\geq 1\]
perci\`o $p_{m,K}(\vp_m)=\norm{\vp_m}_{\infty}=1$ quindi
\[u(\vp_m)=\sum_{k=1}^m\frac1k\]
cio\`e $u$ non \`e limitata e quindi non esistono $m,C$ tali che 
\[\abs{u(\vp)}\leq C p_{m,K}(\vp)\quad \forall \vp\in \Dc(\R)\]
\end{example}


\begin{example}
Se $K$ \`e un singoletto $\cpa{x_0}$ per $x_0\in \Omega$ allora le implicazioni di (\ref{PrSupportoCompattoEContinuita}) si possono invertire: Se $\supp(u)=\cpa{x_0}$ allora esistono $m\in \N$ e costanti $\cpa{c_\al}_{\abs\al\leq m}$ tali che per ogni $\vp\in \Dc(\Omega)$
\[u(\vp)=\sum_{\abs{\al}\leq m}c_\al\del^\al \vp(x_0).\]
Risulta $c_\al=u\pa{\frac{(x-x_0)^\al}{\al!}}$:
\[u\pa{\frac{(x-x_0)^\al}{\al!}}=\sum_{\abs\beta\leq m}\frac{c_\beta}{\al!}\del^\beta((x-x_0)^\al)=\frac{c_\al}{\al!} \del^\al(x-x_0)^\al=c_\al\cdot 1.\]
\end{example}
\begin{proof}
Senza perdita di generalit\`a sia $x_0=0$ e $u\in\Dc'(\Omega)$ con $\supp(u)=\cpa{0}$. Scegliamo $\eta\in C^\infty(\R^n)$ con $\supp(\eta)\subseteq B(0,2)$ e $\eta\res{B(0,1)}=1$.

Definiamo
\[\eta_\e(x)=\eta\pa{\frac x\e}\quad\leadsto\quad \supp(\eta_\e)\subseteq B(0.2\e),\ \eta_\e\res{B(0,\e)}=1\]
Quindi 
\[\del^\al\eta_\e(x)=\e^{-\abs\al}\del^\al\eta\pa{\frac x\e}.\]
Per ogni $\vp\in \Dc(\Omega)$, $\eta_\e\vp\in \Dc(\Omega)$ e $(1-\eta_\e)\vp$ ha supporto su $\Omega\bs B(0,\e)$, quindi la $u$ su annulla su questa funzione. Dunque per ogni $\e>0$
\[u(\vp)=u(\eta_\e\vp)\]
Mostriamo il seguente caso particolare: se $\vp\in\Dc(\Omega)$ \`e tale che $\del^\al\vp(0)=0$ per ogni $\abs\al\leq m$ per un qualche $m$ allora $u(\vp)=0$. In queste ipotesi si ha grazie alla continuit\`a di $u$ su $C^\infty_{\ol{B(0,\e_0)}}$ e al fatto che $\eta_\e\vp\in \Dc(B(0,\e_0))$ per ogni $\e<\e_0$ che
\[\abs{u(\vp)}=\abs{u(\eta_\e\vp)}\leq C_0 p_{m,B(0,\e_0)}(\eta_\e\vp)\]
Si conclude che $u(\vp)=0$ osservando che 
\[p_{m,B(0,\e_0)}(\eta_\e\vp)\overset{\supp(\eta_\e)\subseteq B(0,\e)}=p_{m}(\eta_\e\vp)=o(1)\] 
per $\e\to0$, cio\`e $\eta_\e\vp\to 0$ in $C^m(\R^n)$: poich\'e $\del^\al\eta_\e(x)=\e^{-\abs\al}\del^\al \eta\pa{\frac x\e}$ si ha $p_m(\eta_\e)=\e^{-m}p_m(\eta)$. D'altra parte dalla formula di Taylor, poich\'e $\del^\al\vp(0)=0$ per ogni $\abs\al\leq m$, si ha
\[\abs{\vp(x)}=O(\abs{x}^{m+1}),\qquad \abs{\del^\al\vp(x)}=O(\abs{x}^{m-\abs\al+1})\]
per $x\to 0$. Allora
\[p_m(\eta_\e\vp)\leq 2^m \max_{\smat{\abs\la\leq m\\ \beta\leq\al\\x\in B(0,2\e)}}\abs{\del^{\al-\beta }\eta_\e\del^\beta\vp}=O(\e^{-m+\abs\beta}\e^{m-\abs\beta+1})=O(\e)\]
Quindi $u(\vp)=0$ per ogni $\vp\in \Dc(\Omega)$ con $\del^\al\vp(0)=0$ per $\abs\al\leq m$.

\bigskip

Ora consideriamo $\vp$ qualunque. Dalla formula di Taylor
\[\vp(x)=\sum_{\abs\al\leq m}\frac1{\al!}\del^\al\vp(0)x^\al + \rho(x)\]
con $\rho\in \Dc(\Omega)$ tale che $\del^\al\rho(0)=0$ per ogni $\abs\al\leq m$. Mettendo tutto insieme abbiamo finito perch\'e $u(\rho)=0$ e
\[u(\vp)=\sum_{\abs\al\leq m}\frac1{\al!}c_\al\del^\al\vp(0)\]
con $c_\al=u(x^\al)$.
\end{proof}


















\begin{exercise}
[Convoluzione di una funzione $C^\infty$ a supp.cpt. e una distribuzione.]
Per ogni $f\in \Dc(\R^n)$ \`e definita la mappa
\[\funcDef{\Ec(\R^n)}{\Ec(\R^n)}{\vp}{f\ast\vp}\]
Per ogni $f\in \Ec(\R^n)$, la convoluzione induce una mappa.
\[\Dc(\R^n)\to \Ec(\R^n)\]
Queste mappe sono continue e lineari. Restano definite le trasposte
\[\Ec'(\R^n)\to\Ec'(\R^n),\qquad \Ec'(\R^n)\to \Dc'(\R^n)\]
tali che $u\mapsto \wt u=u\circ (f\ast\bullet)$, cio\`e\footnote{dove $g$ appartiene a $\Ec(\R^n)$ nel primo caso e a $\Dc(\R^n)$ nel secondo.} $\wt u(g)=u(f\ast g)$.

Tenendo presente le propriet\`a della convoluzione questo fornisce una estensione dell'operazione di convoluzione alle distribuzioni.

Cosa si pu\`o dire sulla continuit\`a dell'operazione (per esempio con $(\Ec',\sigma(\Ec',\Ec))$ e $(\Dc',\sigma(\Dc',\Dc))$)?
\end{exercise}














\chapter{Operatori compatti fra Banach}

\section{Definizioni}
\begin{definition}[Mappa compatta]
Una mappa $T:X\to Y$ con $X,Y$ spazi di Banach \`e \textbf{compatta} se \`e continua e per ogni $S\subseteq X$ limitato, $T(S)$ \`e relativamente compatto in $Y$, cio\`e $\ol{T(S)}$ \`e compatto.
\end{definition}
\begin{remark}
Siccome $Y$ \`e completo basta chiedere che $T$ mandi limitati in totalmente limitati.
\end{remark}

\begin{remark}
Se $T$ \`e lineare allora non serve imporre continuit\`a in quanto un insieme totalmente limitato \`e in particolare limitato. Inoltre basta controllare solo $S=B(0,1)$ palla chiusa.
\end{remark}

\begin{remark}
$T\in L(X,Y)$ \`e compatto se e solo se per ogni $(x_n)$ successione limitata in $X$, $(Tx_n)$ ha una sottosuccessione convergente.
\end{remark}

\begin{proposition}\label{PrOperatoreCompattoDominioRiflessivo}
Se $X$ \`e riflessivo allora $T$ \`e compatto se e solo se per ogni successione $(x_n)$ debolmente convergente a $0$ vale $\norm{Tx_n}\to 0$ in $Y$, cio\`e $T$ \`e sequenzialmente continuo per le topologie $(X,w)\to (Y,s)$.
\end{proposition}
\begin{proof}
Consideriamo prima il caso di $X$ riflessivo e separabile. 

Se $T$ \`e compatto, $(x_n)\xrightarrow{w}0\implies (Tx_n)\xrightarrow{w}0$ e $(Tx_n)$ ha sottosuccessione convergente a $0$ in quanto separabile (\ref{ThSeparabilitaInTerminiDiMetrizzabilitaDiPalle}), ma allora $(Tx_n)$ stessa converge a $0$ per la propriet\`a di Urysohn (\ref{PrProprietaUrysohn}).

Viceversa se $T$ \`e sequenzialmente continuo da debole a forte e $(x_n)$ \`e una successione limitata. Per il teorema di Kakutani (\ref{ThKakutani}) $X$ riflessivo implica $B_X$ $w$-compatta e per Eberlein-\v Smulian (\ref{ThEberleinSmulian}) questo \`e equivalente a $B_X$ $w$-sequenzialmente compatta. Poich\'e $(x_n)$ \`e limitata essa \`e contenuta in qualche $nB_X$ e per quanto detto questo insieme \`e $w$-sequenzialmente compatto, quindi $(x_n)$ ammette una estratta $w$-convergente. Per ipotesi su $T$, l'immagine di questa sottosuccessione \`e una sottosuccessione di $(Tx_n)$ convergente.

\medskip

Se $X$ \`e riflessivo (potenzialmente non separabile) allora posso considerare il sottospazio chiuso generato dalla successione $(x_n)$ e questo \`e riflessivo separabile quindi la tesi passa.
\end{proof}


\begin{exercise}
Se $X$ e $Y$ sono entrambi riflessivi, $T$ \`e compatto se e solo se per ogni $(x_n)\subseteq X$ con $x_n\xrightarrow{w}0$ e per ogni $(y_n^\ast)\subseteq Y^\ast$ con $y_n^\ast\xrightarrow{w^\ast}0$ vale $\ps{y_n^\ast,T x_n}\to 0$.
\end{exercise}
\begin{proof}
ESERCIZIO, caso particolare di quella sopra.
\end{proof}

\begin{remark}
Nota che le ipotesi di riflessivit\`a sono necessarie, per $Y=\ell_\infty$ e la successione data da $(e_n)$ la tesi fallisce.
\end{remark}


\begin{proposition}
Se $H$ \`e spazio di Hilbert e $(x_n)\subseteq H$ allora essa converge $\normd$ a $x\in H$ se e solo se $x_n\xrightarrow{w} x$ e $\norm{x_n}\to \norm x$.
\end{proposition}
\begin{proof}
Sviluppiamo
\[\norm{x_n-x}^2=\norm{x_n}^2-2\Real(\ps{x,x_n})+\norm x^2\to \norm x^2-2\under{=\ps{x,x}}{\Real(\ps{x,x})}+\norm x^2=0.\]
\end{proof}

\begin{exercise}
Esprimere la compattezza di $T\in L(X,Y)$ tra $X,Y$ Hilbert usando la propriet\`a sopra.
\end{exercise}


\begin{remark}\label{ReImmagineOperatoreCompattoESeparabile}
L'immagine di un operatore compatto \`e separabile.
\end{remark}
\begin{proof}
$\imm(T)=\bigcup_{n\geq 0}nT(B)$ e $T(B)$ separabile perch\'e relativamente compatto in metrico\footnote{Per ogni $n\in\N\nz$ possiamo costruire il ricoprimento $\cpa{B(x,n\ii)}_{x\in \ol{T(B)}}$ ed estrarre un numero finito di centri di queste palle. Unendo questi insiemi di centri abbiamo una unione numerabile di insiemi finiti, quindi numerabile, e la chiusura di questo insieme \`e tutto $\ol{T(B)}$ perch\'e se una palla ha raggio $\e>n\ii$ deve contenere uno dei punti definiti al livello $n$.}.
\end{proof}



\section{Propriet\`a di \texorpdfstring{$L_C(X,Y)$}{LC(X,Y)}}

\begin{definition}[]
Sia $L_C(X,Y)$ lo \textbf{spazio degli operatori compatti} tra $X,Y$ Banach.
\end{definition}
\begin{remark}
$L_C(X,Y)$ \`e un sottospazio vettoriale chiuso di $L(X,Y)$.
\end{remark}
\begin{proposition}
Se $X,Y,Z$ Banach e $T\in L(X,Y),\ S\in L(Y,Z)$ allora $ST\in L_C(X,Z)$ se almeno uno tra $T$ e $S$ \`e compatto.

In particolare se $X=Y=Z$ allora $L_C(X)=L_C(X,X)$ \`e un ideale bilatero chiuso dell'algebra di Banach\footnote{Spazio di Banach che \`e un'algebra tale che $\norm{xy}\leq\norm x\norm y$ e $\norm 1=1$.} $L(X)$ degli operatori limitati su $X$.
\end{proposition}
\begin{proof}
Mostriamo che $L_C(X,Y)$ \`e uno spazio vettoriale chiuso con la propriet\`a di assorbimento data:
\setlength{\leftmargini}{0cm}
\begin{itemize}
\item[$\boxed{\text{sp.vett.}}$] Siano $T,S\in L_C(X,Y)$. Allora 
\[(T+S)(B_X)\subseteq T(B_X)+S(B_X)\subseteq +(\ol{T B_X}\times\ol{S B_X})\]
poich\'e $\ol{T B_X}$ e $\ol{S B_X}$ sono compatti anche il loro prodotto lo \`e, e quindi anche l'immagine sotto $+:Y\times Y\to Y$. Dunque $(T+S)(B_X)$ \`e relativamente compatto in $Y$.

$\la T$ \`e compatto perch\'e $\la T(B_X)=T(\la B_X)$.
\item[$\boxed{\text{chiuso}}$] Sia $T\in \ol{L_C(X,Y)}$. Per $S\in L_C(X,Y)$ si ha
\[T B_X=(S+(T-S))B_X\subseteq S B_X+(T-S)B_X\subseteq S(B_X)+\norm{T-S}_{L(X,Y)} B_Y\]
dunque $T(B_X)$ \`e totalmente limitato in quanto per ogni $\e>0$ scegliamo $S\in L_C(X,Y)$ con $\norm{S-T}_{L(X,Y)}<\e/2$. Poich\'e $S(B_X)$ \`e totalmente limitato esiste $F\subseteq S(B_X)$ finito tale che $S(B_X)\subseteq F+\e B_Y$, allora
\[T(B_X)\subseteq F+\frac\e2 B_Y+\frac\e2 B_Y=F+\e B_Y\]
cio\`e \`e totalmente limitato per arbitrareit\`a di $\e$.
\item[$\boxed{\text{assorb.}}$] Notiamo che
\[B_X\xrightarrow{T}T(B_X)\xrightarrow{S}S(T(B_X))\]
e $ST(B_X)$ \`e compatto perch\'e ogni opertatore limitato manda limitati in limitati e (relativamente) compatti in (relativamente) compatti.
\end{itemize}
\setlength{\leftmargini}{0.5cm}

\end{proof}

\begin{definition}[Algebra di Calkin]
L'\textbf{algebra di Calkin} di $X$ spazio di Banach \`e l'algebra quoziente
\[c(X)=\quot{L(X)}{L_C(X)}\]
\end{definition}


\section{Operatori compatti di rango finito}

\begin{definition}[]
Dati $X,Y$ Banach definiamo
\[L_f(X,Y)=\cpa{T\in L(X,Y)\mid \rnk(T)\in\N}\]
\end{definition}
\begin{remark}
Notiamo che $L_f(X,Y)$ \`e un sottospazio vettoriale di $L_C(X,Y)$.
\end{remark}
\begin{proof}
Se $T\in L_f(X,Y)$ allora manda limitati di $X$ in limitati di $\imm(T)\cong \R^n$ e i limitati di $\R^n$ sono anche totalmente limitati.
\end{proof}

\begin{remark}
In generale
\[L_f(X,Y)\subsetneq\ol{L_f(X,Y)}\subsetneq L_C(X,Y)\]
\end{remark}

\begin{exercise}
$L_f(X,Y)=\ol{L_f(X,Y)}$ se e solo se almeno uno tra $X$ e $Y$ ha dimensione finita.
\end{exercise}


\begin{proposition}\label{PrFormaOperatoriRangoFinito}
    Gli operatori $T\in L_f(X,Y)$ sono quelli della forma
    \[Tx=\sum_{k=1}^n\ps{\al_k,x}y_k\]
    con $\al_1,\cdots,\al_n\in X^\ast$, $y_1,\cdots, y_n\in Y$.
\end{proposition}

\begin{fact}
Pu\`o accadere che
\[L(X)=\R id_X+\ol{L_f(X)}\]
\end{fact}


\begin{proposition}[]\label{PrInHilbertOperatoriRangoFinitoDenseInOperatoriCompatti}
Se $H$ \`e di Hilbert allora $\ol{L_f(H)}=L_C(H)$.
\end{proposition}
\begin{proof}
Sia $T\in L_C(H)$. Sia $(P_n)$ una successione di proiettori ortogonali di rango finito tale che
\[\ol{\bigcup_{n\geq 0}P_n(H)}\pasgnl={(\ref{ReImmagineOperatoreCompattoESeparabile})}\ol{T(H)}\]
Per costruzione $T_n=P_nT\in L_f(H)$, mostriamo che $T_n\xrightarrow{\normd}T$. Poich\'e
\[\norm{T_n-T}=\sup_{\norm x\leq 1}\norm{P_nT x-Tx}=\sup_{y\in T(B_H)}\norm{P_n y-y}=\norm{P_n-id_H\res{\ol{T(B_H)}}}_{\infty,\ol{T(B)H}}\]
si ha che questa convergenza in norma significa che $P_n\res{\ol{T(B_H)}}:\ol{TB_H}\to H$ convege a $I:=id_H\res{\ol{T(B_H)}}$ uniformemente su $\ol{T(B_H)}$.

La successione $\pa{P_n T\res{\ol{T(B_H)}}}$ \`e una successione di mappe equilipschitz che converge puntualmente all'identit\`a:
\[\norm{P_nT}\leq \norm{P_n}\norm T\leq \under{\text{indip. da }n}{\norm T},\]
quindi per Ascoli-Arzel\'a abbiamo convergenza uniforme sui compatti e quindi in particolare sul compatto $\ol{T(B_H)}$.
\end{proof}

\begin{lemma}[]\label{LmOperatoreNonChiusoSiRestringeAInvertibileSuSottospazioDiDimensioneInfinita}
Se $T\in L(X)$ per $X$ Banach NON \`e compatto allora esiste $Y\subseteq X$ sottospazio chiuso di dimensione infinita tale che $T:Y\to TY$ \`e invertibile.
\end{lemma}

\begin{exercise}
Se $H$ \`e di Hilbert
\begin{enumerate}
    \item $L_C(H)$ \`e il pi\`u piccolo ideale bilatero chiuso non nullo di $L(H)$
    \item Se $H$ \`e separabile e infinito dimesionale (isomorfo $\ell_2$) allora $L_C(H)$ \`e l'unico ideale bilatero chiuso proprio
    \[(0)\subsetneq L_C(H)\subsetneq L(H).\]
\end{enumerate}
\end{exercise}
\begin{proof}
Mostriamo i due fatti
\begin{enumerate}
    \item Sia $I$ un ideale bilatero chiuso non nullo di $L(H)$. Allora, scegliendo opportuni elementi di $L(H)$ con cui comporre un funzionale non nullo di $I$, $I$ contiene ogni operatore di rango 1
    \[x\mapsto \ps{\al,x}y\]
    e quindi (\ref{PrFormaOperatoriRangoFinito}) anche ogni operatore di rango finito, ma allora in quanto chiuso contiene $\ol{L_f(H)}=L_C(H)$.
    \item Poich\'e $H$ \`e Hilbert separabile, ogni sottospazio di $H$ chiuso di dimensione infinita \`e isomorfo a $H$. Se $T\in I\bs L_C(H)$ allora per il lemma (\ref{LmOperatoreNonChiusoSiRestringeAInvertibileSuSottospazioDiDimensioneInfinita}) esiste $Y$ di dimensione infinita tale che $T\res Y$ invertibile, cio\`e abbiamo isometrie $U,V$ tali che
    \[H\xrightarrow{U} Y\xrightarrow{T}  TY \xrightarrow{V} H\]
    restituendo $VTU\in \GL(H)$ e quindi $I=(1)$.
\end{enumerate}
\end{proof}

\begin{example}[Operatore integrale con nucleo $k(x,y)$]
Sia $(M,d)$ metrico compatto e $\mu$ misura di borel finita su $M$. Sia $k\in C^0(M\times M)$ e definiamo
\[T_k:\funcDef{C^0(M)}{C^0(M)}{u}{x\mapsto\int_M k(x,y)u(y)d\mu(y)}\]
Allora $T_k$ \`e lineare e continua: per ogni $u\in C^0(M)$ e per ogni $x\in M$
\begin{align*}
\abs{T_ku(x)}\leq&\int_M\abs{k(x,y)}\abs{u(y)}d\mu(y)\leq \mu(M)\norm{k}_{\infty,M\times M}\norm{u}_{\infty,M}
\end{align*}
quindi $\norm{T_k u}_{\infty,M}\leq \mu(M)\norm{k}_{\infty,M\times M}\norm u_{\infty,M}$ da cui
\[\norm{T_k}\leq \mu(M)\norm k_{\infty,M\times M}.\]
Mostriamo ora che $T_k$ \`e compatto. Sia $\omega$ un modulo di continuit\`a\footnote{cio\`e $\abs{k(x,y)-k(x',y')}\leq \omega(\abs{x-x'}+\abs{y-y'})$. Esiste perch\'e $M$ \`e compatto e quindi $k$ continua implica $k$ uniformemente continua per Heine-Cantor.} per $k$, allora per $u\in C^0(M)$
\[\abs{T_ku(x)-T_ku(x')}\leq \int_M\abs{k(x,y)-k(x',y)}\abs{u(y)}d\mu(y)\leq \omega(\abs{x-x'})\mu(M)\norm u_{\infty,M}\]
dunque $T_k(B_M(0,1))$ \`e una famiglia di funzioni equicontinue (con modulo di continuit\`a $\mu(M)\omega$) ed equilimitate (da $\mu(M)\norm k_{\infty,M\times M}$), quindi \`e compatto per Ascoli-Arzel\'a.
\end{example}

\begin{exercise}
Sia $(M,d)$ metrico compatto e $\mu$ misura di borel finita su $M$. Sia $k\in L^2(M\times M,\mu)$ e definiamo
\[T_k:\funcDef{L^2(M)}{L^2(M)}{u}{x\mapsto\int_M k(x,y)u(y)d\mu(y)}\]
che \`e ben definito per Fubini. $T_k$ \`e un operatore compatto.
\end{exercise}
\begin{proof}
TRACCIA:
Vediamo $T_k$ come limite di operatori di rango finito $T_{k_n}$. Precisamente se $\cpa{E_j}_{1\leq j\leq n}$ \`e una partizione misurabile di $M$ e $k_n$ \`e della forma 
\[k_n=\sum_{1\leq i,j\leq n}c_{i,j}\chi_{E_i\times E_j}\]
allora $k_n\in L_f(L^2(M))$ e per scelte opportune delle partizioni e dei coefficienti delle $k_n$ si ha che $k_n\to k$ in $L^2$. Conseguentemente $T_{k_n}\to T_k$ in norma degli operatori.

La scelta ottimale per $c_{i,j}$ fissato $\cpa{E_j}$ \`e la proiezione ortogonale di $k$ sullo spazio generato dalle $\chi_{E_i\times E_j}$, cio\`e
\[c_{i,j}=\frac1{\mu(E_i)\mu(E_j)}\int_{E_i\times E_j}k(x,y)d(\mu\otimes\mu).\]
Per ogni $u\in L^2(M)$ si ha
\begin{align*}
    \norm{T_k u}_2^2=&\int_M\abs{\int_M k(x,y)u(y)d\mu(y)}^2d\mu(x)\overset{\text{Cauchy-Schwarz}}\leq\\
    \leq&\int_M\pa{\int_M \abs{k(x,y)}^2d\mu(y)}\pa{\int_M \abs{u(y)}^2d\mu(y)}d\mu(x)=\\
    =&\norm k_{2,M\times M}^2\norm u_{2,M}^2
\end{align*}
dunque $T_k$ ha norma degli operatori limitata da $\norm k_{2,M\times M}$, quindi anche la corrispondenza
\[\funcDef{L^2(M\times M)}{L(L^2(M))}{k}{T_k}\]
\`e lineare e continua.
\end{proof}


\begin{example}[Operatori diagonali su $\ell_2$]
Sia $u\in \ell_\infty$ e definiamo un operatore ``{diagonale}" $T_u$ su $\ell_2$ ponendo per ogni $x\in \ell_2$
\[T_u(x)=(u(i)x(i))_{i}\]
(cio\`e moltipliplichiamo le entrate corrispondenti tra $x$ e $u$). Notiamo che
\[\norm{T_u x}_2\leq \norm{u}_\infty\norm x_2\]
dunque $T_u$ ha la norma degli operatori maggiorata da $\norm u_\infty$. In realt\`a $\norm{T_u}=\norm u_\infty$ prendendo opportune approssimazioni.

Quindi abbiamo una inclusione isometrica
\[\ell_\infty\inj L(\ell_2)\]
\ul{Quali $u\in \ell_\infty$ danno luogo a $T_u\in L_C(\ell_2)$? }\smallskip

Se $u$ ha supporto finito allora $T_u$ ha rango finito e quindi in particolare \`e un operatore compatto. Essendo $u\to T_u$ isometrica, considerando le chiusure si ha che le $u\in c_0$ producono $T_u\in L_C(\ell_2)$. Questi sono tutti perch\'e $\ell_2$ \`e uno spazio di Hilbert:
\[\cpa{T_u\mid u\in c_0}=L_C(\ell_2)\cap\ell_\infty.\]
\end{example}



\begin{theorem}[Shauder]\label{ThShauder}
Se $T\in L(X,Y)$ allora $T$ \`e compatto se e solo se $T^\ast$ \`e compatto.
\end{theorem}
\begin{proof}
Diamo le due implicazioni
\setlength{\leftmargini}{0cm}
\begin{itemize}
\item[$\boxed{\implies}$] Sia $T\in L_C(X,Y)$ e sia $(x_n^\ast)\subseteq T^\ast(B_{Y^\ast})$. Vogliamo mostrare che questa successione ha estratte convergenti. Si ha che $x^\ast_n=T^\ast y_n^\ast=y_n^\ast\circ T$ per qualche successione $(y_n^\ast)\subseteq B_{Y^\ast}$. Come funzioni $Y\to \K$ si ha che le $y_n^\ast$ sono $1$-Lipschitz quindi per Ascoli-Arzel\'a hanno una sottosuccessione uniformemente convergente sul compatto $\ol{T(B_X)}$ (e quindi di Cauchy), cio\`e $x_n^\ast =y_n^\ast\circ T$ \`e di Cauchy in $X^\ast$ e dunque converge.
\item[$\boxed{\impliedby}$] Sia $T^\ast:Y^\ast\to X^\ast$ compatto. Allora per la prima parte $T^{\ast\ast}:X^{\ast\ast}\to Y^{\ast\ast}$ \`e compatto e quindi anche $T=T^{\ast\ast}\res{X}$ lo \`e ($X\inj X^{\ast\ast}$ \`e una immersione isometrica).
% https://q.uiver.app/#q=WzAsNCxbMCwwLCJYIl0sWzEsMCwiWSJdLFswLDEsIlhee1xcYXN0XFxhc3R9Il0sWzEsMSwiWV57XFxhc3RcXGFzdH0iXSxbMCwxLCJUIl0sWzIsMywiVF57XFxhc3RcXGFzdH0iLDJdLFswLDIsIiIsMix7InN0eWxlIjp7InRhaWwiOnsibmFtZSI6Imhvb2siLCJzaWRlIjoidG9wIn19fV0sWzEsMywiIiwwLHsic3R5bGUiOnsidGFpbCI6eyJuYW1lIjoiaG9vayIsInNpZGUiOiJ0b3AifX19XV0=
\[\begin{tikzcd}
	X & Y \\
	{X^{\ast\ast}} & {Y^{\ast\ast}}
	\arrow["T", from=1-1, to=1-2]
	\arrow[hook, from=1-1, to=2-1]
	\arrow[hook, from=1-2, to=2-2]
	\arrow["{T^{\ast\ast}}"', from=2-1, to=2-2]
\end{tikzcd}\]
\end{itemize}
\setlength{\leftmargini}{0.5cm}
\end{proof}








\chapter{Teoria spettrale per operatori limitati su Banach}
\section{Spettro e operatori risolventi}

\begin{definition}[Spettro di un operatore]
Per $X$ spazio di Banach e $A\in L(X)$, lo \textbf{spettro} di $A$ \`e
\[\sigma(A)=\cpa{\la\in\C\mid \la-A\notin \GL(X)}\]
L'insieme $\rho(A)=\C\bs \sigma(A)$ \`e detto \textbf{insieme risolvente}.
\end{definition}

\begin{remark}[Spettro \`e chiuso]
Notiamo che $\la\mapsto \la-A$ \`e continua e $\GL(X)$ \`e un aperto di $L(X)$, quindi $\sigma(A)=\rho(A)^c=(\la\mapsto\la-A)\ii(\GL(X))^c$ \`e chiuso.
\end{remark}

\begin{remark}
Nel caso di $X$ Banach su $\R$ si considera la sua complessificazione $X_\C=\C\otimes_\R X=X\times X$ dove il prodotto \`e inteso munito della struttura complessa indotta da
\[J=\mat{0&-id_X\\id_X&0}\]
cio\`e $(a+bi)\ulx=a\ulx +bJ\ulx$ per ogni $\ulx\in X\times X$.
\end{remark}

\begin{remark}
Anche se $\la\in \sigma(A)$ non necessariamente $\la-A$ non \`e iniettiva. 
\end{remark}

\begin{definition}[Spettro puntuale]
Definiamo lo \textbf{spettro puntuale} come
\[\sigma_{pt}(A)=\cpa{\la\in \C\mid \ker(\la-A)\neq 0}.\]
Gli elementi dello spettro puntuale sono detti \textbf{autovalori}.
\end{definition}

\begin{proposition}[]\label{PrProprietaSpettro}
Sia $A\in L(X)$, allora
\begin{itemize}
    \item $\sigma(A)$ \`e contenuto in $\ol B(0,\norm A)$ e quindi compatto\footnote{abbiamo gi\`a visto che \`e chiuso}.
    \item L'applicazione
    \[\funcDef{\rho(A)}{L(X)}{\la}{(\la-A)\ii}\]
    \`e analitica e infinitesima per $\la\to \infty$. Pi\`u precisamente: Per ogni $\la$ tale che $\abs\la\geq \norm A$ si ha
    \[(\la-A)\ii=\sum_{n=0}^\infty \la^{-n-1}A^n\]
    e per ogni $\la_0\in \rho(A)$ e ogni $\la\in B(\la_0,\norm{(\la_0-A)\ii})$ si ha
    \[(\la-A)\ii=\sum_{n=0}^\infty (\la_0-\la)^n((\la_0-A)\ii)^{n+1}.\]
\end{itemize}
\end{proposition}
\begin{proof}
Se mostriamo il secondo punto abbiamo il primo in quanto se per ogni $\abs{\la}\geq \norm A$ abbiamo questo sviluppo, in particolare $(\la-A)\ii$ \`e ben definita per $\abs{\la}\geq \norm A$, quindi l'inversa pu\`o venire a mancare solo per $\la$ contenuti in $\ol{B}(0,\norm A)$.

Se vale il primo sviluppo in serie allora
\[\norm{(\la-A)\ii}\leq \sum_{n=0}^\infty \norm{\la^{-n-1}A^n}=\frac1{\abs\la-\norm A}\]
e quindi la mappa in esame \`e infinitesima per $\la\to \infty$.
\medskip

Tutto segue se mostriamo che per un operatore $H$ di norma $\norm H<1$ in uno spazio di Banach vale lo sviluppo in serie di $(1-H)\ii$, infatti la seconda espansione in serie \`e una caso particolare dello sviluppo
\[(K-H)\ii=K\ii+K\ii H K\ii+K\ii HK\ii H K\ii+\cdots\]
per $K\in \GL(X)$ e $\norm H<\norm{K\ii}\ii$, che segue dal caso $K=1$ notando
\[(K-H)\ii=K\ii(I-HK\ii)\ii,\qquad \norm{H K\ii}<1.\]
Notiamo che se $\norm H<1$ allora la serie $\sum_{n=0}^\infty H^n$ converge assolutamente ad un operatore lineare continuo e per questioni algebriche questa espansione \`e l'inversa di $(1-H)$.
\end{proof}


\begin{definition}[Operatore risolvete]
Se $\la\in \rho(A)$, l'\textbf{operatore risolvente relativo a $\la$} \`e $(\la-A)\ii$.
\end{definition}

\begin{remark}
Vale l'\textbf{identit\`a risolvente}
\[(\la-A)\ii-(\mu-A)\ii=(\mu-\la)(\la-A)\ii(\mu-A)\ii.\]
\end{remark}
\begin{proof}
Basta calcolare:
\begin{align*}
    (\mu-\la)(\la-A)\ii(\mu-A)\ii+(\mu-A)\ii=&((\mu-\la)(\la-A)\ii+1)(\mu-A)\ii=\\
    =&(\la-A)\ii(\mu-\la+(\la-A))(\mu-A)\ii=\\
    =&(\la-A)\ii(\mu-A)(\mu-A)\ii=\\
    =&(\la-A)\ii.
\end{align*}
\end{proof}

\begin{proposition}[]\label{PrSpettroENonVuoto}
Se $A\in L(X)$ e $X\neq 0$ allora $\sigma(A)$ \`e non vuoto.
\end{proposition}
\begin{proof}
Segue dal teorema di Liouville applicato a
\[\funcDef{\rho(A)}{\C}{\la}{\ps{x^\ast,(\la-A)\ii x}}\]
con $x\in X$ e $x^\ast\in X^\ast$ variabili. Infatti queste funzioni sono olomorfe su $\rho(A)$ e infinitesime per $\la\to\infty$ (\ref{PrProprietaSpettro}), quindi se avessimo $\sigma(A)=\emptyset$ allora le mappe sarebbero olomorfe definite su tutto $\C$ e infinitesime all'infinito (in particolare limitate), quindi costanti (al valore $0$ in quanto infinitesime) per ogni $x$ e $x^\ast$, quindi $(\la-A)\ii=0$ come mappa, che \`e assurdo perch\'e per definizione di $\rho(A)$ \`e invertibile ma $X\neq 0$.
\end{proof}

\subsection{Raggio spettrale e Cauchy-Hadamard-Gelfand}
\begin{definition}[Raggio spettrale]
Sia $A\in L(X)$. Il suo \textbf{raggio spettrale} \`e
\[r_A=\max_{\la\in \sigma(A)}\abs{\la}\in [0,\norm A]\]
Notiamo che questo massimo esiste perch\'e $\sigma(A)$ \`e compatto (\ref{PrProprietaSpettro}).
\end{definition}




\begin{lemma}[]\label{LmSuccessioneSubadditivaConvergeSeDivisaPerIndice}
Sia $(a_n)\subseteq\R$ una successione subadditiva\footnote{$a_{n+m}\leq a_n+a_m$} allora esiste il limite
\[\lim_{n\to\infty}\frac{a_n}n=\inf_n\frac{a_n}n.\] 
\end{lemma}
\begin{proof}
Dato $d\geq 1$, ogni $n\in \N$ si scrive $n=p_nd+k_n$ con $0\leq k_n<d$ e $p_n=\floor{\frac nd}$. Allora per ogni $n\geq 1$
\begin{align*}
    \inf_{m\geq 1}\frac{a_m}m\leq& \frac{a_n}n=\frac{a_{p_n d+k_n}}{n}\leq \frac1n\pa{p_n a_d+a_{k_n}}\leq\\
    \leq&\frac{dp_n}n \frac{a_d}d+\frac1n\max_{1\leq k<d}a_k=\\
    =&(1+o(1))\frac{a_d}d+o(1)
\end{align*}
quindi, prendendo il $\limsup_{n}$ e poi $\inf_{d\geq 1}$
\[\inf_{m\geq 1}\frac{a_m}m\leq\liminf_{n}\frac{a_n}n\leq \limsup_{n}\frac{a_n}n\leq \inf_{d\geq 1}\frac{a_d}d\]
dunque esiste il limite ed \`e pari all'estremo inferiore.
\end{proof}


\begin{proposition}[Formula di Cauchy-Hadamard-Gelfand]\label{FormulaCauchyHadamardGelfand}
Vale la seguente identit\`a
\[r_A=\lim_{n\to\infty}\norm{A^n}^{1/n}=\inf_{n\geq 1}\norm{A^n}^{1/n}.\]
\end{proposition}
\begin{proof}
Applichiamo il lemma (\ref{LmSuccessioneSubadditivaConvergeSeDivisaPerIndice}) alla successione $a_n=\log\norm{A^n}$, che \`e subadditiva perch\'e $\norm{A^{n+m}}\leq \norm{A^n}\norm{A^m}$. Per continuit\`a dell'esponenziale questo mostra che il limite nel testo esiste ed \`e pari all'estremo inferiore.
Mostriamo che $r_A=\lim_{n}\norm{A^n}^{1/n}$: 
\setlength{\leftmargini}{0cm}
\begin{itemize}
\item[$\boxed{\leq}$] Se $\la\in \sigma(A)$, cio\`e $\la-A$ non invertibile, allora anche $\la^n-A^n$ non \`e invertibile:
\[\la^n-A^n=(\la-A)B=B(\la-A),\quad B=\sum_{i=0}^{n-1}\la^iA^{n-1-i}\]
quindi $\la-A$ non invertibile implica per il teorema della mappa aperta (\ref{ThMappaAperta}) che $\la-A$ non \`e bigettiva, quindi non \`e iniettiva o non \`e surgettiva, e quindi neanche $\la^n-A^n$ lo \`e.


Dunque $\la^n\in \sigma(A^n)$ e quindi $\abs{\la^n}\leq \norm{A^n}$ da cui $\abs{\la}\leq \norm{A^n}^{1/n}$.
Questo mostra la disuguaglianza $r_A\leq \inf_{n\geq 1}\norm{A^n}^{1/n}=\lim_{n}\norm{A^n}^{1/n}$.
\item[$\boxed{\geq}$] Per ogni $z\in B_\C(0,\frac1{r_A})$ \`e ben definito\footnote{se $z=0$ ok, se $z\neq 0$ allora l'espressione vale $(\frac1z-A)\ii$ che \`e ben definita perch\'e abbiamo supposto $z<1/r_A\coimplies 1/z>r_A\implies 1/z\in \rho(A)$.} $z(1-zA)\ii$. La mappa
\[\funcDef{B_\C(0,\frac1{r_A})}{L(X)}{z}{z(1-zA)\ii}\]
\`e ``analitica": ha uno sviluppo locale in $0$ dato da
\[\sum_{n=0}^\infty z^{n+1}A^n\]
che per\`o si estende a tutto il disco. 

Siano $x\in X$ e $x^\ast\in X^\ast$ e consideriamo la funzione olomorfa
\[\funcDef{B_\C(0,\frac1{r_A})}{\C}{z}{\ps{x^\ast z(1-zA)\ii x}}\]
la quale ha sviluppo locale in $0$ dato da
\[\ps{x^\ast,\sum_{n=0}^\infty z^{n+1}A^nx}=\sum_{n=0}^\infty z^{n+1}\ps{x^\ast,A^nx}\]
che converge assolutamente per $\abs{z}=\frac1r<\frac1{r_A}\coimplies r>r_A$ grazie alla convergenza di $z(1-zA)\ii$. In particolare i termini della serie sono limitati
\[\abs{\ps{x^\ast,\pa{\frac A r}^{n+1}}x}\leq C_{x,x^\ast}.\]
Per Banach-Steinhaus (\ref{ThBanachSteinhausUniformeLimitatezza}) si ha che $\pa{\frac A r}^{n+1}$ \`e limitato: per $x$ fissato la disuguaglianza dice che $\cpa{\pa{\frac A r}^{n+1} x}_{n\in \N}$ \`e $w$-limitata in $X$, quindi limitata in norma (Banach-Steinhaus) e applicando di nuovo Banach-Steinhaus si ha che $\cpa{\pa{\frac A r}^{n+1} }_{n\in \N}$ sono operatori puntualmente limitati, quindi sono limitati in norma. Scriviamo
\[\norm{\pa{\frac A r}^{n+1}}\leq C'\]
cio\`e $\norm{A^n}^{1/n}\leq C'^{1/n}r$ da cui
\[\lim \norm{A^n}^{1/n}\leq r\lim C'^{1/n}=r\]
e questo per ogni $r>r_A$, quindi anche per $r_A$ stesso passando all'estremo inferiore.
\end{itemize}
\setlength{\leftmargini}{0.5cm}
\end{proof}

\begin{remark}
La stessa formula, con la stessa dimostrazione, funziona anche per il raggio spettrale di algebre di Banach.
\end{remark}

\begin{exercise}
Calcolare il raggio spettrale dell'\textbf{operatore di Volterra}
\[V:\funcDef{C^0([a,b])}{C^0([a,b])}{f}{\int_a^x f(t)dt}\]
con la formula del raggio spettrale e provando che per ogni $\la\in \C\nz,\ \la-V\in \GL(V)$.
\end{exercise}



\section{Teoria spettrale su spazi di Hilbert}

\begin{definition}[Operatore simmetrico]
Sia $A$ un operatore limitato su $H$ spazio di Hilbert. $A$ \`e \textbf{simmetrico} (scritto $A\in L^{sim}(H)$) se per ogni $x,y\in H$ si ha
\[\ps{Ax,y}=\ps{x,Ay}.\]
\end{definition}

\begin{proposition}[]\label{PrOperatoriSimmetriciSuHilbertOrtogonaliNucleiImmaginiChiusure}
Sia $A\in L^{sim}(H)$, allora
\begin{enumerate}
    \item $\ker A=(\imm A)^\perp$ e $\ol{\imm A}=(\ker A)^{\perp}$
    \item Se $H_0\subseteq H$ \`e un sottospazio $A$-invariante allora anche $H_0^\perp$ e $\ol{H_0}$ lo sono.
\end{enumerate}
\end{proposition}
\begin{proof}
Mostriamo le due affermazioni
\begin{enumerate}
    \item Segue dalle catena di equivalenze
    \begin{gather*}
        x\in \ker A\\
        Ax=0\\
        0=\ps{Ax,y}=\ps{x,Ay}\ \forall y\in H\\
        x\in (\imm A)^\perp
    \end{gather*}
    l'altra affermazione segue notando che $\ol V=(V^\perp)^\perp$.
    \item Se $x\in H_0^\perp$ allora per ogni $y\in H_0$ si ha $0=\ps{x,Ay}=\ps{Ax,y}$, cio\`e $Ax\in H_0^\perp$. Segue l'invarianza della chiusura prendendo l'ortogonale di nuovo.
\end{enumerate}
\end{proof}

\begin{definition}[Operatori simmetrici positivi]
Se $A\in L^{sim}(H)$ esso si dice \textbf{positivo} se $\ps{Ax,x}\geq 0$ per ogni $x$.
\end{definition}
\begin{remark}
La positivit\`a induce una relazione d'ordine parziale su $L^{sim}(H)$: 
\begin{center}
    $A\geq B\coimplies A-B$ positivo.
\end{center}
\end{remark}

\begin{fact}[]\label{FCTAsimmetricoPositivoAlloraAggiungereIdentitaRendeInvertibile}
Se $A$ \`e simmetrico positivo allora $I+A\in \GL(H)$
\end{fact}
\begin{proof}
$I+A$ \`e fortemente iniettiva in quanto per ogni $x\in H$
\[\norm{(I+A)x}^2=(x+Ax)(x+Ax)=\norm x^2+2\ps{Ax,x}+\norm{Ax}^2\geq \norm x^2\]
quindi in particolare \`e iniettivo con immagine chiusa.

Per il punto 1. di (\ref{PrOperatoriSimmetriciSuHilbertOrtogonaliNucleiImmaginiChiusure}) un operatore simmetrico e iniettivo ha immagine densa, dunque $I+A$ \`e anche surgettivo e quindi invertibile.
\end{proof}


\begin{proposition}[]\label{PrSpettroOperatoreSimmetricoEReale}
$\sigma(A)\subseteq \R$.
\end{proposition}
\begin{proof}
Per ogni $a,b\in\R$ con $b\neq 0$ si ha che $(a+ib-A)$ \`e invertibile perch\'e fattore di
\[(a+ib-A)(a-ib -A)=(a-A)^2+b^2=b^2\pa{I+\pa{\frac{a-A}b}^2}\]
e questo \`e invertibile per (\ref{FCTAsimmetricoPositivoAlloraAggiungereIdentitaRendeInvertibile}).
\end{proof}








\chapter{Spazi di Sobolev}

\section{Derivata debole}
\begin{definition}[Derivata debole]
Sia $I\subseteq \R$ un intervallo aperto. Una funzione $f:I\to \R$ in $L^1_{loc}(I)$ ha \textbf{derivata debole} (o \textbf{distribuzionale}) $g\in L^1_{loc}(I)$ se
\[\forall \vp\in C^\infty_C(I),\qquad \int_I\dot\vp fdt=-\int_I\vp gdt.\]
\end{definition}

\begin{remark}
Ricordiamo che vale una inclusione
\[T:\funcDef{L^1_{loc}(I)}{\Dc'(I)}{f}{T_f:\vp\mapsto \int_I\vp f}\]
dunque in questi termini $g$ \`e derivata debole per $f$ se $(T_f)'=T_g$.
\end{remark}


\begin{remark}
Se $f\in C^1(I)$ allora ha derivata debole e questa coincide con la sua derivata classica $f'\in C^0(I)$.
\end{remark}

\begin{exercise}
Se $u\in \Dc'(I)$ \`e tale che $u'=0$ allora $u$ \`e costante.
\end{exercise}

\begin{remark}
Se $f\in C^0(I)$ ed esiste $f'=g$ quasi ovunque NON \`e detto che $g$ sia la derivata distribuzionale.
\end{remark}

\begin{example}
Sia $f$ la funzione di Cantor, cio\`e $f:[0,1]\to \R$ \`e l'unica funzione che sia un punto fisso dell'endofunzione su $\cpa{f:[0,1]\to\R}$ che manda una funzione $f$ in
\[\begin{cases}
\frac12f\pa{3x} &x\in \spa{0,\frac13}\\
\frac12 &x\in \spa{\frac13,\frac23}\\
\frac12f(3x-2)+\frac12  &x\in \spa{\frac13,\frac23}
\end{cases}\] 
che esiste in quanto questa associazione \`e una contrazione.

La funzione di Cantor ha $f'=0$ per ogni $x\in I\bs C$, in particolare quasi ovunque. Eppure la derivata debole di $f$ non \`e $0$, segue dall'esercizio.
\end{example}

\section{Ripassino di analisi 3}
\begin{definition}[Funzioni assolutamente continue]
Una funzione $f:[a,b]\to \R$ \`e \textbf{assolutamente continua} se per ogni $\e>0$ esiste $\delta>0$ tale che per ogni $I_i=[a_i,b_i]\subseteq [a,b]$, $1\leq i\leq n$ intervalli disgiunti tali che $\sum_{i=1}^n\abs{b_i-a_i}<\delta$ si ha
\[\sum_{i=1}^n\abs{f(b_i)-f(a_i)}<\e.\]
\end{definition} 

\begin{fact}
$f$ \`e assolutamente continua se e solo se $f=\int_a^xgdt$ per qualche $g\in L^1([a,b])$. Inoltre la derivata $f'(x)$ esiste per quasi ogni $x\in [a,b]$ e vale $g(x)$.
\end{fact}

\begin{remark}
La funzione di Cantor non \`e assolutamente continua. 
\end{remark}

\begin{remark}
Le funzioni assolutamente continue sono la classe pi\`u ampia per cui vale il teorema fondamentale del calcolo integrale.
\end{remark}



\begin{fact}
Per assolutamente continue vale la formula di integrazione per parti, in particolare per ogni $\vp\in C^1_C(I)$ si ha
\[\int_I f\vp'=-\int_I f'\vp\]
quindi la derivata quasi ovunque di una assolutamente continua coincide con la derivata distribuzionale.
\end{fact}

\begin{theorem}[Radon-Nikodym]\label{ThRadonNikodym}
Sia $(X,\Sigma)$ uno spazio di misura e siano $\mu,\nu$ due misure $\sigma$-finite su esso. Se $\nu$ \`e assolutamente continua rispetto a $\mu$ (scritto $\nu\ll\mu$) allora esiste una funzione $f:X\to[0,\infty)$ misurabile per $\Sigma$ tale che per ogni $A\in \Sigma$ misurabile si ha
\[\nu(A)=\int_A fd\mu.\]
La funzione $f$ \`e detta funzione \`e detta \textbf{derivata di Radon-Nikodym} e si indica $\dd \mu\nu$.
\end{theorem}

\begin{remark}
Se $m$ \`e la misura $\sigma$-finita definita sugli intervalli da
\[m([\al,\beta))=f(\beta)-f(\al)\]
allora $m\ll \Lc$ dove $\Lc$ \`e la misura di Lebesgue.In questo caso $f'$ \`e anche la derivata di Radon-Nikodym di $m$.
\end{remark}

\begin{remark}
In generale vale una bigezione
\[\correspDef{\cpa{\text{misure $\geq 0$ finite su $[a,b)$}}}{\cpa{f:[a,b)\to\R\text{ cresc., cont. a sx, }f(a)=0}}{\mu}{f_\mu(x)=\mu([a,x))}{\mu_f([\al,\beta))=f(\beta)-f(\al)}{f}\]
dove in realt\`a $\mu_f$ \`e la misura determinata da quel comportamento sugli intervalli semiaperti\footnote{la $\sigma$-additivit\`a non \`e ovvia perch\'e una successione crescente di intervalli semiaperti potrebbe accumularsi in diversi punti, ma notiamo che \`e un insieme ben ordinato quindi possiamo indicizzare gli insiemi con un ordinale numerabile e concludiamo per induzione transfinita}.
\end{remark}


\section{Spazi di Sobolev}
\begin{definition}[Spazio di Sobolev]
Per $1 \leq p\leq\infty$ e $I\subseteq \R$ intervallo definiamo il relativo \textbf{spazio di Sobolev} come
\[W^{1,p}(I)=\cpa{f\in L^p(I)\mid f'_{distr.}\in L^p(I)}\]
\end{definition}
\begin{remark}
Uno spazio di Sobolev \`e uno spazio vettoriale.
\end{remark}

\begin{definition}[Norma del grafico]
Siano $X,Y$ Banach, $A:D\to Y$ con $D\subseteq X$ e $A$ operatore chiuso. La \textbf{norma del grafico} su $D$ \`e 
\[\norm{x}_A=\norm x_X+\norm{Ax}_Y.\]
Questa norma rende $D$ isomorfo al sottospazio chiuso $\Gamma A\subseteq X\times Y$.
\end{definition}

\begin{proposition}[]\label{PrOperatoreDiDerivazioneSuSobolevHaGraficoChiuso}
L'operatore di derivazione
\[D:\funcDef{W^{1,p}}{L^1}{f}{f'}\]
\`e lineare e chiuso, cio\`e il suo grafico
\[\Gamma D=\cpa{(f,f')\in L^p\times L^p\mid f\in W^{1,p}}\]
\`e chiuso in $L^p\times L^p$.
\end{proposition}
\begin{proof}
Se $(f_n,f_n')$ \`e una successione su $\Gamma D$ convergente in $L^p\times L^p$ a qualche coppia $(f,g)$, cio\`e $f_n\to f$ e $f'_n\to g$, allora segue che $g=f'$ e quindi $(f,g)\in \Gamma D$. Infatti
\[\int_I f_n\vp'=-\int_I f_n'\vp\]
passa al limite perch\'e $\vp\in C^\infty_C$, quindi
\[\int_I f\vp'=-\int_I g\vp.\]
\end{proof}

\begin{corollary}
La norma del grafico\footnote{$\norm f_{1,p}=\norm f_p+\norm{f'}_p$} sul dominio di $D$ rende $W^{1,p}$ uno spazio di Banach. 
\end{corollary}
\begin{proof}
Questa norma rende l'immersione ovvia
\[W^{1,p}\inj \Gamma D\]
una funzione continua e isometrica a valori in un sottospazio chiuso di $L^p\times L^p$, che \`e un Banach.
\end{proof}

\begin{remark}
$W^{1,p}$ per $1\leq p\leq\infty$ \`e uno spazio di Banach, separabile per $p<\infty$ in quanto lo sono gli $L^p(I)$, e riflessivo per $1<p<\infty$ (perch\'e sottospazi chiusi e prodotti di Banach riflessivi sono riflessivi).
\end{remark}

\begin{remark}
Dall'inclusione $W^{1,p}\subseteq L^p\times L^p$ segue una rappresentazione delle forme lineari continue su $W^{1,p}$.
\end{remark}
\begin{proof}
Se $L\in W^{1,p}(I)$ esistono $u,v\in (L^p)^\ast$ tali che per ogni $f\in W^{1,p}$ si ha
\[\ps{L,f}=\ps{u,f}=\ps{v,f'}\]
per esempio, se $u\in L^p$ e $v\in L^q$, poich\'e $(L^p)^\ast\cong L^q$
\[\ps{L,f}=\int_I(uf+vf')dt\]


?????????????[Questa parte me la sono persa a lezione]
\end{proof}


\begin{proposition}[Caratterizzazioni spazio di Sobolev]\label{PrCaratterizzazioniSpazioDiSobolev}
Per $f\in L^p(I)$ e $1< p\leq\infty$ le seguenti sono equivalenti
\begin{enumerate}
    \item $f\in W^{1,p}(I)$
    \item esiste $C\geq 0$ tale che per ogni $\vp\in C^\infty_C(I)$
    \[\abs{\int_I f\vp'}\leq C\norm\vp_{q}\]
    \item Esiste $C\geq 0$ tale che per ogni $J\subseteq I$ e per ogni $h\in \R$ con $J+h\subseteq I$ si ha
    \[\norm{\tau_h f-f}_{p,J}\leq C\abs h\]
\end{enumerate}
\end{proposition}
\begin{proof}
Diamo le implicazioni
\setlength{\leftmargini}{0cm}
\begin{itemize}
\item[$\boxed{1.\implies2.}$] Se $f\in W^{1,p}(I)$ allora
\[\abs{\int f\vp'}=\abs{\int f'\vp}\overset{\text{H\"older}}\leq \norm{f'}_p\norm\vp_q.\]
\item[$\boxed{2.\implies1.}$] Se vale $2.$ allora \`e ben definito il funzionale lineare continuo su $C^\infty_C(I)$ dato da
\[\vp\mapsto \int_I f\vp'\]
Per continuit\`a esso si estende per densit\`a ad un funzionale lineare continuo su $L^q$, ma allora \`e della forma $\vp\mapsto -\int g\vp$ con $g\in L^p$, dunque
\[\int f\dot\vp=-\int g\vp\]
ovvero esiste $f'_{dist}\in L^p$.
\item[$\boxed{1.\implies3.}$] SOON...

\end{itemize}
\setlength{\leftmargini}{0.5cm}
\end{proof}


\begin{proposition}[]\label{PrElementiDiSpazioDiSobolevSonoHolderiani}
Le $f\in W^{1,p}(I)$ per $1<p\leq \infty$ sono H\"older. 
\end{proposition}
\begin{proof}
Per $1<p<\infty$ si ha
\[\abs{f(x)-f(y)}=\abs{\int_x^y1f'(t)dt}\leq \abs{\int_x^y1}^{1/q}\norm{f'}_p=\norm{f'}_p\abs{x-y}^{1/q},\]
mentre per $p=\infty$
\[\abs{f(x)-f(y)}=\abs{\int_x^y1f'(t)dt}\leq\abs{x-y}\norm{f'}_\infty.\]
\end{proof}

\section{Parentesi Esercizi}
\begin{exercise}
Per $1<p\leq \infty$ e $u\in L^p(\R)$ sono equivalenti
\begin{enumerate}
    \item $u\in W^{1,p}(\R)$
    \item esiste $C\geq 0$ tale che $\norm{\tau_h(u)-u}_p\leq C\abs h$ per ogni $h\in \R$
\end{enumerate}
dove $\tau_h(u)(x)=u(x+h)$, quindi la seconda consizione significa che i rapporti incrementali di $u$
\[\delta_hu(x)=\frac{u(x+h)-u(x)}h\]
sono limitati in $L^p$ per ogni $h\in \R\nz$.
\end{exercise}
\begin{solution}
Mostriamo le due implicazioni
\setlength{\leftmargini}{0cm}
\begin{itemize}
\item[$\boxed{1.\implies2.}$] Se $u\in W^{1,p}$, $\chi_1=\chi_{[-1,0]}$ e $\chi_h(x)=\frac1h\chi_1(x/h)$ (notiamo $\chi_1\in L^1(\R)$, $\chi_1\geq 0$, $\int \chi_1=1$) allora si ha
\[\delta_hu(x)=\frac1h\int_x^{x+h}\dot u(t)dt=\frac1h\int_\R\chi_{[x,x+h]}(t)\dot u(t)dt=\int_\R\chi_{h}(x-t)\dot u(t)dt=\dot u\ast\chi_h.\]
Poich\'e $\dot u\in L^0$, dalla disuguaglianza di Young ($p\leq \infty$) si ha
\[\norm{\delta_h u}_p=\norm{\chi_h\ast\dot u}_p\leq \norm{\chi_h}_1\norm{\dot u}_p=\norm{\dot u}_p\]
Il ragionamento \`e analogo per $h<0$.
\item[$\boxed{2.\implies1.}$] Poich\'e $p>1$ si ha $L^p=(L^q)^\ast$, quindi la successione $\cpa{\delta_{h_j}u}_{j\geq 0}$ con $h_j=\frac1j$, che per ipotesi \`e limitata in $L^p$, ha una sottosuccessione che \`e $w^\ast$-convergente in $L^p$ e converge $w^\ast$ (uniformemente sui compatti (\ref{ThBoundedWeakStarELaTopologiaDiConvergenzaUniformeSuCompatti})) ad una $u\in L^p$. In particolare per ogni $\vp\in C^\infty_C(\R)$ si ha
\[\ps{\delta_{h_j}u,\vp}\to \ps{v,\vp}\]
ma $\ps{\delta_hu,\vp}=\ps{u,\delta_{-h}\vp}$, infatti
\begin{align*}
    \ps{\delta_hu,\vp}=&\frac1h\int_\R(\tau_h u-u)\vp dt=\frac1h\pa{\int_\R u(x+h)\vp(x)dx-\int_\R u(x)\vp(x)dx}=\\
    =&\frac1h\pa{\int_\R u(x)\vp(x-h)dx-\int_\R u(x)\vp(x)dx}=\\
    =&\int_\R u(x)\delta_{-h}\vp(x) dx
\end{align*}
Poich\'e $\vp\in C^\infty_C$, $\delta_{h_j}\vp\xrightarrow{L^1}\dot \vp$ (addirittura converge in $\Dc(\Omega)$) si conclude che
\[\ps{v,\vp}=\lim_j\ps{\delta_{h_j}u,\vp}=\lim_j\ps{u,\delta_{-h_j}\vp}=\ps{u,\dot \vp}\]
cio\`e $u$ ha una derivata debole $v\in L^p$.
\end{itemize}
\setlength{\leftmargini}{0.5cm}
\end{solution}
\begin{remark}
Cosa succede per $p=1$? 
Si ha $2.$ \`e equivalente a $u\in BV(\R)$, cio\`e $u\in L^1(\R)$ e $\dot u$ \`e una misura di Radon.


L'idea \`e che $L^1$ si immerge isometricamente in uno spazio che \`e un duale e poi ragiono in modo analogo ma la convergenza debole star in questo spazio non restituisce una funzione ma una misura.
\end{remark}


\begin{exercise}
Per $I=[a,b]$ e $1<p\leq \infty$, l'inclusione 
\[W^{1,p}(I)\inj C^0(I)\]
\`e compatta.
\end{exercise}
\begin{proof}
Abbiamo gi\`a visto che $u\in W^{1,p}(I)$ \`e H\"older di esponente $1/q$ (nota che $1\leq q<\infty$), infatti
\[\abs{u(y)-u(x)}=\abs{\int_x^y\dot u(t)dt}\overset{\text{H\"older}}\leq \norm{\dot u}_p\pa{\int_x^y 1dt}^{1/q}=\norm{\dot u}_p\abs{x-y}^{1/q}.\]
Dunque, per Ascoli-Arzel\'a si ha che la palla unitaria di $W^{1,p}$ (rispetto alla topologia della norma $\normd_{1,p}$ sullo spazio di Sobolev) \`e un insieme relativamente compatto in $C^0([a,b])$.
\end{proof}

\begin{remark}
\`E vero che $W^{1,p}(\R)\subseteq L^\infty(\R)$? \`E vero che \`e compatta?
\end{remark}
\begin{solution}
Considerando $u\in C^\infty_C\nz$ e la successione di traslate $\tau_nu$ si ha che questa successione converge puntualmente a $0$ ma la successione non tende a $0$ n\'e in $W^{1,p}$ n\'e in $L^\infty$.
\end{solution}

\begin{exercise}
\`E vero che \`e compatta l'inclusione $W^{1,1}([0,1])\subseteq C^0([0,1])$?
\end{exercise}
\begin{solution}
No, basta trovare una successione in $W^{1,1}$ limitata che non ha sottosuccessioni convergenti in $(C^0([0,1]),\normd_\infty)$.
\end{solution}













\appendix
\chapter{Topologia}


\begin{proposition}[Topologia iniziale]\label{PrTopologiaInizialeEsiste}
    Sia $X$ un insieme e $\Fc$ una famiglia di mappe a valori in uno spazio topologici. Notazione:
    \[\Fc=\cpa{f_j:X\to (Y_j,\tau_j)}_{j\in I}.\]
    Allora esiste la topologia meno fine su $X$ che rende continue le mappe $f_j$. Una prebase di questa topologia \`e data da
    \[\cpa{f_j\ii(A)\mid j\in I, A\in \tau_j}.\]
    In realt\`a basterebbe prendere una prebase per $\tau_j$ al posto di tutta la topologia.

    Questa topologia \`e detta \textbf{topologia iniziale della famiglia $\Fc$} e si denota $\tau_\Fc$.
\end{proposition}

\begin{remark}[Propriet\`a universale della topologia iniziale]\label{PrProprietaUniversaleTopologiaIniziale}
Data una mappa $\vp:(Z,\tau_Z)\to (X,\tau_\Fc)$ essa \`e continua se e solo se $f\circ \vp$ \`e continua per ogni $f\in \Fc$.
\end{remark}
\begin{proof}
Se $\vp$ \`e continua allora $f\circ \vp$ \`e composizione di continue. Se sappiamo che $f\circ \vp$ \`e continua per ogni $f\in\Fc$ allora, se $A$ \`e un aperto di $X$ per continuit\`a di $f\circ \vp$ abbiamo
\[\tau_Z\ni(f\circ \vp)\ii(A)=\vp\ii(f\ii(A))\]
cio\`e le preimmagini tramite $\vp$ di aperti di prebase sono aperti di $Z$, quindi $\vp$ \`e continua.
\end{proof}


\begin{proposition}[Transitivit\`a della topologia iniziale]\label{PrTransitivitaTopologiaIniziale}
Supponiamo di avere una famiglia di mappe $\Fc'=\cpa{f_i:X\to Y_i}_{i\in I}$ e per ogni $i\in I$ sia $\Gc_i=\cpa{g_{ij}:Y_i\to Z_{ij}}_{j\in J_i}$ una famiglia di mappe. Su ogni $Y_i$ consideriamo la topologia iniziale determinata da $\Gc_i$. Allora la topologia iniziale data da $\Fc'$ su $X$ coincide con la topologia iniziale su $X$ definita da $\Fc=\cpa{g_{ij}\circ f_i\mid i\in I,\ j\in J_i}$.
\end{proposition}
\begin{proof}
Entrambe le topologie in esame sono generate dagli insiemi $(g_{ij}\circ f_i)\ii(A)$ al variare di $i\in I$, $j\in J_i$ e $A\in \tau_{Z_{ij}}$, infatti
\[\text{prebase per $\Fc\to$ }(g_{ij}\circ f_i)\ii(A)=f_j\ii(g_{ij}\ii(A))\text{ $\leftarrow$ prebase per $\Fc'$}.\]
\end{proof}
\chapter{Duali di \texorpdfstring{$\ell_p$}{lp}}
\section{Norme estese}
\begin{definition}[Norma estesa]
Sia $\sigma:\K^\N\to [0,\infty]$ una \textbf{norma estesa}, cio\`e
\begin{enumerate}
    \item $\sigma(x+y)\leq \sigma(x)+\sigma(y)$
    \item $\sigma(\la x)=\abs\la\sigma(x)$
    \item $\sigma(x)=0\coimplies x=0$
\end{enumerate}
\end{definition}
\noindent
Inoltre supponiamo che 
\begin{enumerate}
    \item[4.] per ogni $n\in\N$ esista $C_n$ tale che per ogni $x\in\K^\N$ si abbia $\abs{x(n)}\leq C_n\sigma(x)$
    \item[5.] $\sigma$ \`e LSC\footnote{semicontinua inferiormente} rispetto alla convergenza puntale, cio\`e
    \[x^\nu\in \K^\N,\ \forall i\ x^\nu(i)\to x(i)\implies \sigma(x)\leq \liminf_{\nu\to+\infty}\sigma(x^\nu)\]
\end{enumerate}
\begin{example}
La funzione $\sigma(x)=(\sum\abs{x_i}^p)^{1/p}$ \`e una norma estesa su $\K^\N$ che ha propriet\`a indicate. Anche $\sigma(x)=\norm x_\infty$ ha queste propriet\`a.
\end{example}

\begin{remark}
Le propriet\`a 4. e 5. sono equivalenti a dire che $\cpa{\sigma\leq 1}$ \`e compatto in $\K^\N$, infatti
\[4.\coimplies \cpa{\sigma\leq 1}\subseteq \prod_n\ol{B(0,C_n)}\subseteq \K^\N\]
e $5.$ equivale a chiedere $\cpa{\sigma\leq 1}$ chiuso, quindi insieme dicono che $\cpa{\sigma\leq 1}$ \`e un chiuso in un compatto ($\prod_n\ol{B(0,C_n)}$ \`e prodotto di compatti).
\end{remark}

\begin{definition}[Dominio di finitezza]
Definiamo il \textbf{dominio di finitezza} della norma estesa $\sigma$ come
\[\ell_\sigma=\cpa{x\in\K^\N\mid \sigma<+\infty}\]
\end{definition}

\begin{exercise}
Il dominio di finitezza $\ell_\sigma$ \`e uno spazio di Banach e $\sigma$ induce la norma.
\end{exercise}
\begin{proof}
Traccia:
\begin{itemize}
    \item Verificare che $\ell_\sigma$ \`e uno spazio vettoriale
    \item $\sigma$ \`e una norma
    \item Verificare la completezza: 
    \begin{itemize}
        \item Sia $(x^\nu)_\nu\subseteq \ell_\sigma$ di Cauchy per $\sigma$. Allora per ogni $n\in\N$
        \[\pa{x^\nu(n)}_\nu\subseteq \K\]
        \`e una successione di Cauchy in $\K$ (propriet\`a $4.$ insieme al fatto che $(x^\nu)$ \`e Cauchy), quindi esiste $x\in\K^\N$ tale che $x^\nu\to x$ puntualmente.
        \item Verificare che $x\in\ell_\sigma$: essendo di Cauchy, $x^\nu$ \`e limitata, cio\`e $\sigma(x^\nu)\leq R$ per qualche $R\in \R$, dunque $\sigma(x)\leq R$ perch\'e $\cpa{\sigma\leq R}$ \`e chiuso.
        \item Verficare che $\sigma(x^\nu-x)\to 0$: Per ogni $\e>0$ esiste $n\in\N$ tale che per ogni $p,q\geq n$ vale $\sigma(x^p-x^q)\leq \e$. Notiamo che $x^p-x^n\to x-x^n$ puntualmente, quindi per la semicontinuit\`a si ha che per ogni $\e>0$ esiste $n\in\N$ tale che per ogni $q\geq n$ vale
        \[\sigma(x-x^q)\leq\liminf_{p\to +\infty}\sigma(x^p-x^q)\leq  \e\]
        cio\`e $\sigma(x-x^q)\to 0$ in norma $\sigma$.
    \end{itemize}
\end{itemize}
\end{proof}

\begin{remark}
Questa \`e una seconda dimostrazione della completezza di $\ell_p$ per $1\leq p\leq \infty$.
\end{remark}

\begin{remark}
Funziona anche l'analogo per paranorme, quindi in realt\`a segue anche la completezza di $\ell_p$ per $0<p\leq 1$.
\end{remark}




\section{Duali di \texorpdfstring{$\ell_p$}{lp}}

\begin{proposition}[Duali di $\ell_p$]\label{PrDualilp}
Se $p$ e $q$ sono esponenti coniugati ($\frac1p+\frac1q=1$, $\frac1\infty\doteqdot 0$) allora vale l'isometria $(\ell_p)^\ast\cong \ell_q$.
\end{proposition}
\begin{proof}
Esiste una inclusione lineare isometrica data da
\[\Phi:\funcDef{\ell_q}{(\ell_p)^\ast}{y}{\Phi_y: x\mapsto \sum_{i=0}^\infty y_ix_i},\]
dove la serie in esame converge assolutamente per la disuguaglianza di H\"older:
\[\sum\abs{x_iy_i}\leq \norm x_p\norm y_q.\]
Effettivamente $\Phi_y:\ell_p\to \K$ \`e lineare e continua per $\normd_{(\ell_p)^\ast}$, infatti
\[\norm{\Phi_y}_{(\ell_p^\ast)}=\sup_{\norm x_p\leq 1}\abs{\sum_{i=0}^\infty x_iy_i}\leq \sup_{\norm x_p\leq 1}\norm x_p\norm y_q=\norm y_q.\]
La stessa disuguaglianza mostra che $\Phi$ stessa \`e un elemento di $L(\ell_q,(\ell_p)^\ast)$ di norma minore o uguale a $1$.

Resta da mostrare che $\Phi$ \`e isometrica e surgettiva.

\setlength{\leftmargini}{0cm}
\begin{itemize}
\item[$\boxed{1< p<\infty}$] Per mostrare che $\norm{\Phi_y}_{\ell_p^\ast}=\norm y_q$ per ogni $y\in \ell_q$ consideriamo $x\in\K^\N$ dato da $x_i=\ol{\sgn y_i}\abs{y_i}^{q-1}$. Con questa scelta si ha che
\[x_iy_i=\ol{\sgn y_i}{\sgn y_i}\abs{y_i}^q=\abs{y_i}^q,\]
inoltre
\[\norm x_p^p=\sum_{i=0}^\infty \abs{x(i)}^p=\sum_{i=0}^\infty \abs{y_i}^{(q-1)p}=\sum_{i=0}^\infty \abs{y_i}^{q}=\norm y_q^q,\]
cio\`e $x\in \ell_p$ e 
\[\norm{\Phi_y}_{\ell_p^\ast}\geq \frac{\Phi_y(x)}{\norm x_p}=\frac{\sum_{i=0}^\infty \abs{y_i}^q}{(\norm y_q)^{q/p}}=\norm{y}_q^{q-q/p}=\norm{y}_q,\]
d'altronde sappiamo che vale anche l'altra disuguaglianza in generale, quindi abbiamo $\norm{\Phi_y}_{\ell_p^\ast}=\norm y_q$ come voluto.
\item[$\boxed{p=\infty,\ q=1}$] Sia $x_i=\ol{\sgn y_i}$. Segue che $\norm x_\infty\leq 1$ quindi \`e un elemento valido e
\[\Phi_y(x)=\norm y_1,\]
da cui segue $\norm{\Phi_y}_{\ell_\infty^\ast}\geq \norm y_1$ come voluto.
\item[$\boxed{p=1,\ q=\infty}$] In generale $\norm{\Phi_y}_{\ell_1^\ast}$ non \`e raggiunto come $\Phi_y(x)$ per qualche $x$ \footnote{per esempio $y_i=1-2^{-i}$ perch\'e in tal caso $\Phi_y(x)=\sum (1-2^{-i})x_i<\sum \abs{x_i}=\norm x_1$}. La conclusione per\`o vale comunque.
\end{itemize}
Mostriamo ora che l'inclusione \`e surgettiva per $1\leq p<\infty$. Per ogni $\vp\in \ell_p^\ast$ cerchiamo $y\in\ell_q$ tale che $\vp=\Phi_y$. C'\`e un solo $y$ possibile, basta valutare $\vp$ negli $e_i=(\delta_{ij})_j$. Per ogni $m\in\N$ sia $P_m:\K^\N\to \K^m$ il proiettore sulle prime $m$-entrate. Considerando $P_m$ come operatore $P_m:\ell_p\to\K^m\subseteq \ell_p$ restringendo il dominio, definiamo $\vp_m=\vp\circ P_m=P_m^\ast\vp$. Infine, sia 
\[y_m=P_m y=(y(0),y(1),\cdots, y(m-1), 0,0,\cdots)=\sum_{i=0}^{m-1}y_ie_i,\] 
e notiamo che
\[\vp_m=\Phi_{y_m}\]
infatti sono entrambi elementi di $\ell_p^\ast$ e 
\begin{align*}
    \vp_m(e_k)=\vp(P_m(e_k))=&\begin{cases}
        \vp(e_k) &\text{se }k<m\\
        0 &\text{se }k\geq m
        \end{cases}\\
    \Phi_{y_m}(e_k)=\sum_{i=0}^\infty y_m(i)e_k(i)=y_m(i)=&\begin{cases}
        \vp(e_k) &\text{se }k<m\\
        0 &\text{se }k\geq m
        \end{cases}
\end{align*}
quindi $\vp_m$ e $\Phi_{y_m}$ coincidono su $(e_k)$, quindi sullo span di questi e quindi sulla chiusura di questo span, che \`e $\ell_p$ se $p<\infty$.

Essendo $\Phi$ isometrica
\[\norm{y_m}_q=\norm{\Phi_{y_m}}_{\ell_p^\ast}=\norm{\vp_m}_{\ell_p^\ast}\leq \norm\vp_{\ell_p^\ast}\]
quindi $\sum_{i=0}^{m-1}\abs{y(i)}^q\leq \norm\vp_{\ell_p^\ast}^q$ per ogni $m$, dunque passando al $\sup$ in $m$ 
\[\norm y_q\leq \norm\vp_{\ell_p^\ast}\]
e quindi $y$ era un elemento valido di $\ell_q$.
\setlength{\leftmargini}{0.5cm}
\end{proof}


\begin{proposition}\label{PrDualec0El1}
Si ha che $\ell_1\cong c_0^\ast$
\end{proposition}
\begin{proof}
Consideriamo
\[\Phi:\funcDef{\ell_1}{c_0^\ast}{y}{x\mapsto \sum_{i=0}^\infty x_i y_i}\]
Allora $\Phi$ \`e lineare e $\abs{\Phi_y(x)}\leq \norm x_\infty\norm y_1$. $\Phi$ \`e isometrica
\[\norm{\Phi_y}_{c_0^\ast}=\sup_{\norm x_\infty\leq 1, x\in c_0}\sum x_i y_i=\norm y_1,\]
infatti l'estremo superiore si realizza con la successione $x^n=\ol{\sgn y}\chi_{[0,n]}$ ($\Phi_y(x^n)=\sum_{i=0}^n\abs{y_i}$ e passando al limite in $n$ troviamo proprio $\norm y_1$). Inoltre $\Phi$ \`e surgettiva infatti $(e_k)_k\in\N\subseteq c_0$ genera un sottospazio denso.
\end{proof}

\subsection{\texorpdfstring{$\ell_1,\ c_0$}{l1, c0} e \texorpdfstring{$\ell_\infty$}{linf}}
\begin{definition}[Finita additivit\`a]
Una funzione $\mu:\Ps(S)\to\K$ \`e \textbf{finitamente additiva} se per ogni $A,B\subseteq S$ disgiunti, $\mu(A\cup B)=\mu(A)+\mu(B)$.
\end{definition}

\begin{remark}
Domanda: $c_0$ \`e un duale? Cio\`e, esiste $X$ Banach tale che $X^\ast$ \`e linearmente omeomorfo a $c_0$?

NO! Perch\'e $c_0$ non \`e complementato in $\ell_\infty$ (difficile da mostrare). Questo basta per (\ref{LmDualeComplementatoNelTriduale}).
\end{remark}

\begin{lemma}\label{LmSuccessioneInl1}
Se $X$ \`e un sottospazio $\infty$-dimensionale di $\ell_1$ allora esiste una successione $(x_k)\subseteq X$ e una successione di naturali $(T_k)\subseteq \N$ strettamente crescente tali che 
\[\begin{cases}
\norm {x_k}_1=1\\
\norm{x_k}_{1,[0,T_k]}=\sum_{i=0}^{T_k}\abs{x_k(i)}\geq 3/4\\
x_{k+1}\res{[0,T_k]}=0
\end{cases}\]
\end{lemma}
\begin{proof}
Scegliamo $x_0\in X$ di norma 1 e $T_0\in \N$ che abbia la seconda propriet\`a. Supponiamo ora di aver definito $x_0,\cdots, x_k$ e di avere $T_0<\cdots, T_k$, allora
\[X\cap \cpa{x\in\ell_1\mid x\res{[0,T_k]}=0}\neq \emptyset\]
in qunato intersezione fra un sottospazo di dimensione infinita e dei sottospazi di codimensione finita, infatti quell'intersezione si pu\`o scrivere come
\[\bigcap_{0\leq i\leq T_k}\cpa{x\in X\mid x(i)=0}.\]
Prendendo un elemento $x_{k+1}$ normalizzato in questa intersezione abbiamo esteso la successione. Per scegliere $T_{k+1}$ basta prenderlo maggiore di $T_k$ e tale che 
\[\norm{x_{k+1}}_{1,[0,T_{k+1}]}\geq 3/4.\]
\end{proof}

\begin{proposition}[]\label{PrSottospazioChiusoDil1Contienel1}
Se $Y\subseteq \ell_1$ \`e un sottospazio chiuso di dimensione infinita allora $Y$ contiene una copia di $\ell_1$.

\filosofia{\textsl{Se guardi a lungo dentro $\ell_1$, $\ell_1$ guarda dentro di te.}}
\end{proposition}
\begin{proof}
Sia $X$ sottospazio chiuso di dimensione infinita di $\ell_1$ e siano $(x_k)\subseteq \ell_1$ e $(T_k)\subseteq \N$ come nel lemma (\ref{LmSuccessioneInl1}) Definiamo l'operatore lineare
\[L:\funcDef{\ell_1}{X}{\la}{\sum_{k=0}^\infty\la_k x_k}\]
$L$ \`e ben definita perch\'e la serie \`e assolutamente convergente rispetto a $\normd_1$
\[\norm{\sum_{k=0}^\infty\la_k x_k}_1\leq \sum_{k=0}^\infty\abs{\la_k} \norm{x_k}_1\leq \norm \la_1.\]
Notiamo anche che $X$ chiuso e quindi $L$ continuo di norma $\norm L\leq 1$.

Sia $I_k=[0,T_k]$ e notiamo che
\begin{align*}
\norm{L\la}=&\norm{\sum_{k=0}^\infty\la_k x_k}\geq\norm{\sum_{k=0}^\infty\la_k x_k\res{I_k}}-\sum_{k=0}^\infty\norm{\la_k x_k\res{I_k^c}}=\\
=&\sum_{k=0}^\infty\abs{\la_k} \norm{x_k\res{I_k}}_1-\sum_{k=0}^\infty\abs{\la_k} \norm{x_k\res{I_k^c}}\geq\\
\geq&\frac34\norm\la_1-\frac14\norm{\la}_1=\frac12\norm\la_1
\end{align*}
dunque $L:\ell_1\to X$ \`e fortemente iniettivo e quindi \`e un isomorfismo con l'immagine in quanto questa \`e chiusa.
\end{proof}


\begin{exercise}
$c_0$ non \`e un duale.
\end{exercise}
\begin{proof}
Segue dalla proposizione (\ref{PrSottospazioChiusoDil1Contienel1}): se esistesse $X$ tale che $X^\ast\cong c_0$ allora $\iota_X:X\inj X^{\ast\ast}\cong \ell_1$ e quindi per la proposizione $X$ contiene un sottospazio $Y$ isomorfo a $\ell_1$, ma allora da $Y\subseteq X$ segue 
\[\ell_\infty\cong \ell_1^\ast\cong Y^\ast\pasgnl\cong{(\ref{PrDualeDiSottospaziEDualeQuoziente})} X^\ast/\Ann(Y)\cong c_0/\Ann Y\]
ma $\ell_\infty$ non \`e separabile mentre $c_0$ \`e separabile e ogni quoziente di un separabile deve essere separabile.
\end{proof}


\begin{proposition}\label{PrDualelinftyContienel1}
Si ha che $\ell_1\inj \ell_\infty^\ast$ \`e una immersione isometrica NON surgettiva.
\end{proposition}
\begin{proof}
L'iniezione \`e chiara. Consideriamo le funzioni che hanno limite (le costanti a meno di una infinitesima)
\[c=\cpa{x\in\ell_\infty\mid \exists \lim_{i\to\infty}x_i}\cong c_0\oplus \R\]
Esiste un funzionale su $c$ dato da
\[\la:\funcDef{c}{\K}{y}{\lim y_i}.\]
Questo \`e continuo perch\'e $\norm \la\leq 1$ (perch\'e $\abs{\lim y_i}\leq \norm y_\infty$). Per il teorema di Hahn-Banach (\ref{CorHahnBanachPerSpaziNormati}) si estende ad un funzionale continuo in $\ell_\infty$.

Consideriamo
\[ba=\cpa{\mu:\Ps(\N)\to\K\mid \text{limitate e finitamente additive.}}\subseteq (\Bs(\Ps(\N),\K),\normd_\infty)\]
e la mappa
\[\Psi:\funcDef{\ell_\infty^\ast}{ba}{y}{A\mapsto y(\chi_A)}\]
Notiamo che $\Psi$ \`e surgettiva: se $\mu\in ba$ e definiamo una funzione lineare su $\ell_\infty$ come segue
\begin{itemize}
    \item Se $x\in\ell_\infty$ \`e della forma $x=\sum c_i\chi_{A_i}$, cio\`e $x(\N)\subseteq \cpa{\sum_{i\in J}c_i\mid J\subseteq \cpa{0,\cdots, n}}$ \`e finito, quindi
    \[x=\sum_{c\in\K}c\chi_{\cpa{x=0}}\text{ \`e una somma finita}\]
    Sia $S$ il sottoinsieme di $\ell_\infty$ delle successioni di questa forma.
    Mostriamo che $\ol{S}=\ell_\infty$ dove la chiusura \`e presa rispetto a $\normd_\infty$. Per ogni $x\in \ell_\infty$ con $x:\N\to\R$ si ha che
    \[x-2^{-n}\leq x^n=\frac{\floor{2^n x}}{2^n}\leq x\]
    \item Per $x\in S$ dato da $x=\sum_{i=1}^n c_i\chi_{A_i}$ poniamo $\ps{\mu,x}=\sum_{i=1}^n c_i\mu(A_i)$. Chiaramente abbiamo linearit\`a e la buona definizione segue dal fatto che questa espressione coincide con $\sum_{c\in\K}c\mu\pa{\cpa{x=0}}$ per finita additivit\`a, ma questa forma \`e univocamente determinata da $x$.
    \item Vale che $\abs{\ps{\mu,x}}\leq \norm x_\infty\pa{\sum_{c\in \K} \abs{\mu(\cpa{x=c})}}$ infatti per ogni $c$, se $\mu(\cpa{x=c})\neq 0$ allora $\abs c\leq \norm x_\infty$ per definizione.
    \item Per ogni $\mu$ esiste $C$ tale che per ogni $x\in S$ si ha 
    \[\abs{\ps{\mu,x}}\leq C\norm x_\infty\]
    (VERIFICARE!)
    \item Quindi $\mu$ si estende alla chiusura di $S$, che \`e tutto $\ell_\infty$. 
\end{itemize}
Notiamo che $ba=\ell_\infty^\ast=\ell_1^{\ast\ast}=c_0^{\ast\ast\ast}$, quindi $\ell_1\inj \ell_1^{\ast\ast}=ba$ \`e complementato per il lemma (\ref{LmDualeComplementatoNelTriduale}).
\end{proof}



\begin{proposition}[Convergenza forte e debole coincidono su $\ell_1$]\label{PrConvergenzaForteEDeboleCoincidonoSul1}
La convergenza debole e la convergenza in norma per $\ell_1$ sono la stessa cosa.
\end{proposition}
\begin{proof}
Poich\'e la topologia debole \`e meno fine della topologia forte basta mostrare che convergenza debole implica convergenza in $\normd_1$.

Sia $f_n\to f$ in $w$-$\ell_1$, cio\`e per ogni funzionale $\phi$ lineare continuo su $\ell_1$ si ha che $\ps{\phi,f_n}\to\ps{\phi,f}$. Dunque, ricordando (\ref{PrDualilp}) che $\ell_\infty=\ell_1^\ast$, per ogni $\vp\in\ell_\infty$ si ha
\[\sum_i\vp(i)f_n(i)\to \sum_i\vp(i)f(i).\]
Notiamo che, portando $f$ al primo membro possiamo supporre senza perdita di generalit\`a $f=0$. Per la propriet\`a di Uhrison (\ref{PrProprietaUhrisohn}) basta provare che esiste una sottosuccessione di $f_n$ che converge a $0$ infatti $f_n\to 0$ se e solo se per ogni sottosuccessione $f_{n_k}$ esiste una sotto-sottosuccessione $f_{h_{k_j}}\to 0$.

Nel caso particolare di successioni $f_n$ a supporto disgiunto la tesi \`e facile: Se siamo in questo caso scegliamo $\vp\in\ell_\infty$ data da 
\[\vp=\sum \ol{\sgn f_i}\text{ dove }(\sgn f_i)(x)=\sgn(f_i(x))=\begin{cases}
    f_i(x)/\abs{f_i(x)} & f_i(x)\neq 0\\
    0 & f_i(x)=0
\end{cases}\] 
in modo tale che $\ps{\vp,f_n}=\ps{\ol{\sgn(f_n)},f_n}=\norm{f_n}_1$ e stesso per $f$, quindi in questo caso \`e chiaro che convergenza debole implica convergenza in $\normd_1$.

Assumiamo dunque che $f_n\to 0$ debolmente (e quindi puntualmente guardando i funzionali che estraggono la $n$-esima entrata). Basta provare che esiste una successione $(g_j)_{j\geq 0}\subseteq \ell_1$ e una sottosuccessione $f_{n_j}$ tale che $\norm{f_{n_j}-g_j}_1\to 0$ e $g_j$ hanno supporto disgiunto. 

Costruiamo le $g_j$ per induzione. Notiamo che 
\begin{itemize}
    \item per ogni $f_n$ si ha che $\norm{f_n\chi_{\N\bs[0,T]}}_1\to 0$ per $T\to\infty$
    \item per ogni $T$ si ha $\norm{f_n\chi_{[0,T]}}_1\to 0$ per $n\to \infty$
\end{itemize}
quindi per costruire le $g_j$ basta alternare questi fatti prendendo opportuni limiti (credo?).
\end{proof}

\begin{remark}
Questo \`e un esempio dove due topologie diverse hanno ``le stesse successioni convergenti".
\end{remark}







\begin{fact}\label{FcQuasipartizioneDiNConInfinitiElementiMaFiniteIntersezioni}
Esiste $\As\subseteq \Ps(\N)$ di cardinalit\`a del continuo tale che per ogni $A,B\in \As$ dove $A\neq B$ allora $\abs{A\cap B}<\aleph_0$ e $\abs{A}=\abs{B}=\aleph_0$.
\end{fact}
\begin{proof}
Consideriamo $\Q$ al posto di $\N$, tanto per le cardinalit\`a non cambia nulla. Per ogni irrazionale $\la$ consideriamo $A_\la=\cpa{\floor{n\la}/n\mid n\in\N}\subseteq \Q$. Poniamo $\As=\cpa{A_\la}_{\la\in\R\bs\Q}$. Ogni $A_\la$ \`e inifinito ma se $\la\neq \mu$ allora $A_\la\cap A_\mu$ \`e finito perch\'e ogni successione che converge a $\la$ cade definitivamente in un intorno di $\la$ e similmente per $\mu$, allora scelgo intorni disgiunti.
\end{proof}



\begin{lemma}
Ogni sottospazio $Y$ di $\ell_\infty$ ha duale $w^\ast$-separabile.
\end{lemma}
\begin{proof}[Dimostrazione (modo diretto).]
Per ogni $k\in\N$ sia $e_k:Y\to \K$ la valutazione $f\mapsto f(k)$. Allora\footnote{$\wt \Q=\Q$ se $\K=\R$ e $\wt \Q=\Q+i\Q$ se $\K=\C$.} $\Span_{\wt \Q}(\cpa{e_k}_{k\in\N})$ \`e $w^\ast$-denso in $Y^\ast$ e quindi $Y^\ast$ \`e $w^\ast$-separabile: stiamo usando il criterio $\Span S\subseteq Y^\ast$ \`e $w^\ast$-denso se e solo se $S_\perp=(0)$ (proposizione (\ref{PrNucleiImmaginiTrasposteAnnullatoriEPreannullatori}) e corollario (\ref{CorIniettivitaEAggiunti})) e 
\[(\cpa{e_k}_{k\in\N})_\perp=\cpa{y\in Y\mid \ps{e_k,y}=0\ \forall k\in\N}=\cpa{0}\subseteq \ell_\infty.\]
\end{proof}
\begin{proof}[Dimostrazione (concettuale).]
Se $(X,\normd)$ \`e uno spazio normato separabile, $X^{\ast\ast}$ \`e sempre $w^\ast$-separabile, quindi consideriamo $X=c_0,\ X^{\ast\ast}=\ell_\infty$. Questo \`e vero perch\'e se $S\subseteq X$ \`e numerabile e $\normd$-denso allora la sua chiusura debole$^\ast$ in $X\subseteq X^{\ast\ast}$ contiene almeno $X$ (su $X$ la $\sigma(X^{\ast\ast},X)$ induce $\sigma(X,X^\ast)$ che \`e meno fine della topologia forte e quindi $\ol X^{\sigma(X^{\ast\ast},X^\ast)}=X^{\ast\ast}$ per Goldstine (\ref{ThGoldstine})).

Se poi $Y\subseteq \ell_\infty$ sappiamo che $Y^\ast$ \`e un quoziente di $\ell^\ast$ (dato da $\ell_\infty^\ast/Y^\perp$) in quanto la mappa $\pi:\ell^\ast_\infty\to \ell^\ast_\infty/Y^\perp$ \`e $w^\ast$-continua e surgettiva\footnote{se $j:Y\to X$ \`e l'inclusione, $j^\ast=\pi:X^\ast\to Y^\ast$ \`e surgettiva e $w^\ast$-continua (\ref{PrDualeOperatoreContinuoEDeboleStarContinua}).} (e quindi l'immagine di $S\subseteq \ell_\infty^\ast$ densa resta densa). Se $\ol S=\ell_\infty^\ast$ allora
\[\pi(\ell^\ast_\infty)=\pi(\ol{S})\subseteq \ol{\pi(S)}\]
\end{proof}

\begin{proposition}
$c_0$ non \`e complementato in $\ell_\infty$.
\end{proposition}
\begin{proof}
Supponiamo che esista $Y$ sottospazio chiuso di $\ell_\infty$ tale che $\ell_\infty=Y\oplus c_0$, allora $Y$ sarebbe omeomorfo al quoziente $\ell_\infty/c_0$, ma ogni sottospazio di $\ell_\infty$ \`e $w^\ast$-separabile per il lemma, ma $\ell_\infty/c_0$ non lo \`e:

Fissiamo $\As$ come nel fatto (\ref{FcQuasipartizioneDiNConInfinitiElementiMaFiniteIntersezioni}). Per ogni $A\in \As$ notiamo che $\chi_A\in\ell_\infty$. Sia $\xi_A$ l'immagine di questa caratteristica in $\ell_\infty/c_0$. Sia $g\in (\ell_\infty/c_0)^\ast$, affermo che
\[S=\cpa{A\in\As\mid \ps{g,\xi_A}\neq 0}\text{ \`e al pi\`u numerabile.}\]
Basta mostrare che per ogni $\e>0$ l'insieme $S_\e=\cpa{A\in\As\mid \abs{\ps{g,\xi_A}}\geq \e}$ \`e finito (infatti $S=\bigcup_{\Q\ni\e>0}S_\e$).

Siano $A_1,\cdots, A_m\in S_\e$ e definiamo $\xi=\sum_{i=1}^m \ol{\sgn(\ps{g,\xi_{A_i}})}\xi_{A_i}$. Per linearit\`a
\[\xi=\pi\pa{\sum_{i=1}^m\ol{\sgn(\ps{g,\xi_{A_i}})\chi_{A_i}}},\]
dunque
\[\ps{g,\xi}=\sum_{i=1}^m\ol{\sgn(\ps{g,\xi_{A_i}})}\ps{g,\xi_{A_i}}=\sum_{i=1}^m\abs{\ps{g,\xi_{A_i}}}\geq m\e.\]
Inoltre $\norm{\xi}_{\ell_\infty/c_0}=1$ perch\'e qualunque combinazione lineare $\vp$ delle $\chi_{A_i}$ ha valore superiore a 1 solo sulle intersezioni delle $A_i$ e questo \`e un insieme finito, quindi $\vp=1+h$ per $h\in c_0$, quindi $[\vp]=1$.

Mettendo tutto insieme $\norm{g}\geq \norm{\ps{g,\xi}}\geq m\e$ e quindi $m\leq \norm g/\e$, dunque \[\abs{S_\e}\leq \frac{\norm g}\e.\]

Se $G\subseteq (\ell_\infty/c_0)^\ast$ \`e un sottoinsieme numerabile allora \`e numerabile anche
\[\bigcup_{g\in G}\cpa{A\in\As\mid \ps{g,\xi_A}\neq 0}\]
perch\'e unione numerabile di insiemi al pi\`u numerabili. Poich\'e $\As$ non \`e numerabile esiste $\wt A\in \As$ tale che
\[\ps{g,\xi_{\wt A}}=0\quad \forall g\in G\]
e $\xi_{\wt A}\neq 0$ perch\'e ha norma pari a $1$, cio\`e 
\[\xi_{\wt A}\in G_\perp\neq (0)\]
ma ricordiamo che $\ol{\Span(G)}^{w^\ast}=(G_\perp)^\perp$ e che $\Span(G)$ \`e $w^\ast$-denso se e solo se $G_\perp=(0)$ (\ref{PrPolarePrepolareIteratiDannoChiusuraAssolutamenteConvessa}).
\end{proof}
\chapter{Domande degli orali}

Segue una lista (immagino non esaustiva) delle domande che sono state fatte durante gli orali di quest'anno. Il numero di asterischi indica quante volte la stessa domanda \`e stata fatta a diverse persone. Le indentature indicano domande che sono discese da quella con indentatura minore precedente. Indicherò queste domande anche al livello base in modo da poter segnalare la molteplicit\`a.
\begin{itemize}
\item Duale di $(X,\sigma(X,\Fc))$ e relativi lemmi (\ref{PrDualePerTopologiaDebole})
\begin{itemize}
    \item Che ruolo giocano le seminorme? Cosa vuol dire topologizzare con seminorme?
    \item Cosa vuol dire continuit\`a rispetto ad una famiglia di seminorme?
    \item Quando uno spazio topologizzato da seminorme \`e $T_2$? Quando separato?
\end{itemize}
\item Dimostra che la chiusura di $\assco(A)$ \`e $(A^0)_0$ (\ref{PrPolarePrepolareIteratiDannoChiusuraAssolutamenteConvessa})
\item Teorema del grafico chiuso (\ref{ThGraficoChiuso}) (sketch)
\item Teorema dell'immagine chiusa (\ref{ThImmagineChiusa})
\item Dimostra il lemma di iterazione
\begin{itemize}
    \item $\Sc\Uc$ \`e aperto
    \item $T^\ast$ fortemente iniettiva implica $T$ surgettiva
    \item Teorema della mappa aperta
    \item Estensione di Tietze
\end{itemize}
\item $\Sc\Uc$ \`e aperto (\ref{PrOperatoriSurgettiviSonoUnAperto})
\item $T^\ast$ fortemente iniettiva implica $T$ surgettiva (\ref{ThSurgettivitaEAggiunti})
\item Teorema della mappa aperta (\ref{ThMappaAperta})
\item[$\bullet\bullet$] Dimostra il teorema di estensione di Tietze (\ref{ThEstensioneTietze})
\begin{itemize}
    \item Esercizio inerente
\end{itemize}
\item Dimostra Banach-Steinhaus (\ref{ThBanachSteinhausUniformeLimitatezza})
\begin{itemize}
    \item limitato debole = limitato forte
    \item Raggio spettrale con formula di Cauchy Hadamard
\end{itemize}
\item Limitato debole = limitato forte (\ref{PrLimitatoDeboleUgualeLimitatoForte})
\item[$\bullet\bullet$] Dimostra Banach-Alaoglu (\ref{ThBanachAlaogluBourbaki})
\begin{itemize}
    \item Conseguenze di Banach-Alaoglu? Kakutani per esempio.
\end{itemize}
\item Dimostra il teorema di Goldstine (\ref{ThGoldstine})
\begin{itemize}
    \item Kakutani
    \item Milman-Pettis
\end{itemize}
\item Cosa sappiamo sui riflessivi?
\begin{itemize}
    \item Kakutani
    \item gli $\ell_p$ sono riflessivi
    \item $X$ \`e riflessivo se e solo se $Y\subseteq X$ chiuso e $X/Y$ lo sono (sketch)
\end{itemize}
\item[$\bullet\bullet\bullet\bullet$] Dimostra Kakutani (\ref{ThKakutani})
\item[$\bullet$] Dimostra Milman-Pettis (\ref{ThMilmanPettis})
\item Separabilit\`a di $X$ e $X^\ast$ in termini di metrizzabilit\`a (\ref{ThSeparabilitaInTerminiDiMetrizzabilitaDiPalle})
\item Se $Y\subseteq X$ sottospazio chiuso, $X$ \`e riflessivo se e solo se $Y$ e $X/Y$ lo sono (\ref{PrCriterioRiflessivoConSottospazioChiuso})
\item Parla del calcolo funzionare per operatori autoaggiunti
\begin{itemize}
    \item come si costruisce $f(T)$
    \item Caso complesso vs caso reale
    \item fatti generali e cosa segue da cosa
\end{itemize}
\item[$\bullet\bullet$] Costruzione di $f(T)$ per $T$ simmetrico
\item Dimostra la formula di Cauchy-Hadamard-Gelfand (\ref{FormulaCauchyHadamardGelfand})
\item Parla di Distribuzioni e supporto
\item Caratterizzazione di distribuzioni a supporto in un punto
\item Le distribuzioni a supporto compatto sono il duale delle funzioni $C^\infty(\Omega)$ 
\item Esistenza e unicit\`a della radice quadrata
\item Se $T$ \`e compatto, $\dim\ker(I-T)$ \`e finita
\item Perch\'e $c_0$ non \`e addendo diretto di $\ell_\infty$? (solo citando i lemmi coinvolti)
\item[$\bullet\bullet$] Fatti e fatterelli su $\ell_1$, per esempio: \`e il duale di $c_0$, non \`e riflessivo (pi\`u modi), convergenza debole equivale convergenza forte
\item Duali di $c_0$ e $\ell_1$
\begin{itemize}
    \item Dove si usa la claratterizzazione del duale di $c_0$? (Dieudonne)**********
\end{itemize}
\item $\ell_p$ \`e riflessivo per $1<p<\infty$
\end{itemize}





\chapter{Ringraziamenti}
Ringrazio in generale i colleghi che hanno seguito il corso e sopratutto coloro che hanno scritto sul gruppo telegram per aver indirettamente sciolto diversi dei miei dubbi.


Ringranzio in particolare i seguenti per aver contribuito correzioni e suggerimenti per queste note:
\begin{itemize}
\item Toccotelli Tommaso
\end{itemize}





\end{document}
