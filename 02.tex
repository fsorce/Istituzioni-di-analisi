\chapter{Spazi vettoriali topologici}

\begin{definition}[Spazio vettoriale topologico]
Uno \textbf{spazio vettoriale topologico} \`e uno spazio vettoriale $X$ su $\K\in\cpa{\R,\C}$ munito di una topologia che rende continue le mappe 
\[+:X\times X\to X\quad \text{ e }\quad\cdot:\K\times X\to X.\]
\end{definition}
\begin{example}
Esempi di SVT sono
\begin{itemize}
    \item Ogni spazio normato
    \item $C(\R,\R)$ con la topologia della convergenza uniforme sui compatti.
    \item Se $X$ \`e uno spazio topologico qualunque considero $C(X,\R)$ con topologia di convergenza uniforme su compatti.
\end{itemize}
\end{example}

\begin{exercise}
La topologia della convergenza uniforme su compatti su $C(\R,\R)$ non \`e indotta da una norma.
\end{exercise}
\begin{proof}
TRACCIA
\begin{itemize}
    \item Su uno spazio normato, se $U$ e $V$ sono intorni di 0 allora esiste $\la\in\R$ tale che $\la U\supseteq V$.
    \item Mostrare che la topologia della convergenza uniforme su compatti non ha questa propriet\`a.
\end{itemize}
\end{proof}

\begin{exercise}
Ogni SVT che \`e $T_0$ \`e anche\footnote{In questo corso con $T_3$ intendiamo $T_3$ e Hausdorff} $T_3$ e\footnote{$T_{3\frac12}$ \`e $T_3$ pi\`u esiste una funzione continua che vale $1$ sul punto e $0$ sul chiuso che sto separando} $T_{3\frac12}$
\end{exercise}
\begin{exercise}[Spazi non $T_0$ non sono troppi interessanti]
Ogni SVT $X$ si decompone in somma diretta topologica $X=Y\oplus \ol{\cpa0}$ con $Y$ qualunque addendo algebrico di $\ol{\cpa{0}}$. Segue che $Y\cong X/\ol{\cpa0}$, $Y$ risulta essere $T_0$ e $\ol{\cpa0}$ ha la topologia indiscreta.
\end{exercise}

\section{Intorni dell'origine in SVT}

\begin{definition}[Filtro]
Un \textbf{filtro} $\Fc$ su un insieme $X$ \`e una famiglia non vuota di sottoinsiemi di $X$ tale che
\begin{itemize}
    \item per ogni $F\in\Fc$, $F\neq \emptyset$
    \item Se $F\in \Fc$ e $F\subseteq F'$ allora $F'\in\Fc$
    \item Se $F,F'\in\Fc$ allora $F\cap F'\in\Fc$
\end{itemize}
\end{definition}

\begin{definition}[Sottoinsieme bilanciato]
Sia $X$ un $\K$-spazio vettoriale e $A\subseteq X$. $A$ \`e \textbf{bilanciato} se per ogni $\la\in\K$ tale che $\abs\la\leq1$ si ha $a\in A\implies \la a\in A$, cio\`e 
\[B_\K(0,1)\cdot A\subseteq A.\]
\end{definition}

\begin{remark}
Se $V$ \`e bilanciato allora $0\in V$ perch\'e $0\in B_\K(0,1)$.
\end{remark}

\begin{definition}[Sottoinsieme assorbente]
Sia $X$ un $\K$-spazio vettoriale e $B\subseteq X$. $B$ \`e \textbf{assorbente} se per ogni $x\in X$ esiste $n_x\in\N$ tale che per ogni $t\geq n_x$ si ha $x\in tB$.
\end{definition}


\begin{remark}
Poich\'e in uno SVT le traslazioni $X\to X$ con $x\mapsto x+x_0$ sono omeomorfismi, per descrivere la topologia basta descrivere il filtro degli intorni di $0$.
\end{remark}



Come notazione sia $\Uc=\Uc_X$ l'insieme degli intorni di $0\in X$.

\begin{proposition}[Propriet\`a intorni di 0]\label{PrProprietaIntorni0}
$\Uc$ ha le seguenti propriet\`a
\begin{enumerate}
    \item $\Uc$ \`e un filtro
    \item Per ogni $U\in\Uc$ esiste $V\in\Uc$ tale che $V+V\subseteq U$
    \item Per ogni $U\in\Uc$ esiste $V\in\Uc$ con $V\subseteq U$ e $V$ bilanciato
    \item Ogni elemento di $\Uc$ \`e assorbente
\end{enumerate}
\end{proposition}
\begin{proof}
Dimostriamo le varie propriet\`a
\begin{enumerate}
    \item La propriet\`a $1.$ \`e vera per ogni insieme definito come ``gli intorni di $x$" per $x$ fissato in spazio topologico $X$.
    \item Segue dalla continuit\`a di $+$ in $(0,0)\in X\times X$. Basta definire $V$ in modo tale che $V\times V\subseteq +\ii(U)$.
    \item Segue dalla continuit\`a di $\cdot$ in $(0,0)$. Se $U$ intorno di $0$ in $X$, siano $\e>0$ e $V\in \Uc$ tali che $B_\K(0,\e)\times V\subseteq \cdot\ii(U)$. Allora $B_\K(0,\e)\cdot V$ \`e bilanciato e contenuto in $U$ per costruzione. Questo insieme \`e anche un intorno perch\'e si pu\`o scrivere come
    \[\bigcup_{\abs{\la}\leq \e}\la V \]
    e poich\'e $V$ \`e un intorno di $0$, ogni $\la V$ \`e un intorno di $0$, quindi anche questa unione.
    \item Segue dalla continuit\`a della mappa $\R_+\to X$ che per fissato $x_0\in X$ assegna $s\mapsto sx_0$. Infatti per ogni $U\in\Uc$ esiste $\e>0$ tale che per ogni $0\leq s\leq \e$, $sx_0\in U$ e riscrivendo questo in termini di $t=1/s$ abbiamo $x_0\in tU$ per ogni $t\geq 1/\e$. Come $n_{x_0}$ basta scegliere $\floor{\e\ii}$.
\end{enumerate}
\end{proof}

\begin{exercise}\label{ExTopologiaIndottaDaIntorniDi0}
Sia $X$ spazio vettoriale su $\K$ e $\Uc$ una famiglia si sottoinsiemi di $X$ tali che valgano le quattro propriet\`a della proposizione precedente (\ref{PrProprietaIntorni0}). Allora esiste un'unica topologia su $X$ che rende $X$ uno SVT e tale che $\Uc$ \`e il filtro degli intorni di $0$. In questa topologia $\Uc$ \`e un sistema fondamentale di intorni per $0$.
\end{exercise}
\begin{proof}
L'idea \`e che definiamo $A\subseteq X$ aperto se e solo se per ogni $a\in A$, $A-a\in\Uc$ (sto traducendo ``aperto $\coimplies$ intorno di ogni suo punto"). Si pu\`o mostrare che questa scelta definisce una topologia che rende $X$ uno SVT.
\end{proof}

\begin{exercise}
Definire analogamente una topologia di SVT su $X$ tramite degli assiomi che si basano una una base di intorni di $0$ (al posto di tutti gli intorni). Per esempio la famiglia degli intorni bilanciati di $0$.
\end{exercise}

\begin{remark}
Se uno SVT \`e $T_0$ allora \`e automaticamente $T_1$ e $T_2$, basta sfruttare propriet\`a di simmetria.
\end{remark}

\begin{remark}
Ogni SVT \`e uno spazio topologico regolare, cio\`e ogni punto ha una base di intorni chiusi. Se $X$ \`e anche $T_0$ allora $X$ \`e $T_3$.
\end{remark}
\begin{proof}
Sia $C$ un chiuso di $X$ e $x\in X$ con $x\notin C$. Sia $U\in\Uc_X$ tale che $x+U\cap C=\emptyset$, che esiste perch\'e $C$ \`e chiuso. Sia $V\in \Uc_X$ tale che $V-V\subseteq U$, allora\footnote{Un insieme come $C+V$ \`e detto intorno uniforme di $C$} $(x+V)\cap (C+V)=\emptyset$ dove $C+V$ \`e un intorno di $c$ per ogni $c\in C$ per definizione.
\end{proof}

\begin{remark}\label{PrSeparoCompattoEChiusoDisgiunti}
Se $K$ \`e compatto, $C$ chiuso con $K\cap C=\emptyset$ allora esiste $V$ tale che $(K+V)\cap (C+V)=\emptyset$.
\end{remark}
\begin{proof}
Per ogni $x\in K$ sia $V_x\in \Uc_X$ tale che $x+(V_x+V_x-V_x)$ \`e disgiunto da $C$. Abbiamo dunque un ricoprimento $\cpa{x+V_x}_{x\in K}$ di $K$, che \`e compatto, quindi estraggo un sottoricoprimento finito $\cpa{x_i+V_{x_i}}$ e definisco $V$ come l'intersezione di questi. Allora
\[(K+V)\cap (C+V)=\emptyset,\]
infatti se $x\in K+V$ allora $x=k+v$ con $k\in K$ e $v\in V$ ma $k\in x_i+V_{x_i}$ per qualche $i$, quindi $x=x_i+v_i+v$, e avendo supposto che $x_i+(V_{x_i}+V_{x_i}-V_{x_i})\cap C=\emptyset$ abbiamo che $x=x_i+v_i+v\notin C+V$.
\end{proof}


\section{SVT localmente convessi}
\begin{definition}[SVT localmente convesso]
Uno \textbf{spazio vettoriale topologico localmente convesso} (\textbf{SVTLC}) \`e uno SVT tale che $0$ ha una base di intorni convessi.
\end{definition}
\begin{example}
Diamo alcuni esempi
\begin{itemize}
    \item Ogni spazio normato
    \item $C(X)$ con $X$ spazio topologico con la topologia della convergenza uniforme sui compatti
    \item $C^\infty(\Omega)$ con $\Omega\subseteq \R^n$ aperto e topologia della convergenza uniforme sui compatti di tutte le derivate in ogni ordine
\end{itemize}
\end{example}

\begin{exercise}
Sia $\Ms=\cpa{f:[0,1]\to \R\mid \text{misurabili}}$, allora esiste una metrica su $\Ms$ che lo rende uno SVT e tale che $f_n\to f$ se e solo se $f_n\to f$ in misura, cio\`e per ogni 
\[\forall \e>0,\quad \lim_{n\to\infty}\abs{\cpa{\abs{f_n}>\e}}= 0\]
Mostrare che l'unico intorno convesso di $0$ \`e $\Ms$ stesso, da cui segue $\Ms^\ast=\cpa{0}$.
\end{exercise}

\begin{remark}
Per ci\`o che sappiamo sugli intorni di $0$ in uno SVT, se $X$ \`e SVTLC allora esiste una base $\Bc$ data dagli intorni di $0$ assorbenti, bilanciati e convessi.
\end{remark}

\begin{definition}[Disco]
Un insieme $B$ \`e detto \textbf{disco} se \`e assorbente, bilanciato e convesso.
\end{definition}

\begin{proposition}[]\label{PrTopologiaConvessaIndotta}
Sia $X$ un $\R$-SV e $\Bc$ una famiglia di sottoinsiemi di $X$ tale che
\begin{itemize}
    \item Per ogni $B\in \Bc$, $B$ \`e Assorbente, Bilanciato e Convesso
    \item Per ogni $B_1,B_2\in\Bc$ si ha $B_1\cap B_2\in \Bc$
\end{itemize}
allora $\Uc=\cpa{U\subseteq X\mid \exists r>0,\ \exists B\in\Bc\mid rB\subseteq U}$ \`e un filtro di insiemi che induce una topologia che rende $X$ uno SVT come da esercizio (\ref{ExTopologiaIndottaDaIntorniDi0}). La topologia indotta \`e anche localmente convessa.
\end{proposition}
\begin{proof}
Mostriamo le quattro propriet\`a:
\begin{itemize}
    \item Chiaramente $\Uc$ \`e un filtro.
    \item Ogni $U\in \Uc$ \`e assorbente perch\'e lo sono gli elementi di $\Bc$
    \item Per ogni $U\in\Uc$ esiste $V\in\Uc$ tale che $V+V\subseteq U$, basta scegliere $V=\frac12 B$ con $B\subseteq U$ convesso in quanto se $B$ \`e convesso $B+B=2B$
    \item Ogni $U\in\Uc$ contiene un bilanciato perch\'e contiene una versione scalata di un elemento di $\Bc$.
\end{itemize}
\end{proof}



\begin{remark}
Se $\Bc$ \`e una famiglia di dischi allora definendo $\wt \Bc=\cpa{B_1\cap B_2\mid B_1,B_2\in\Bc}$ si ha che $\wt \Bc$ rispetta gli assiomi della proposizione (\ref{PrTopologiaConvessaIndotta}) e quindi induce una topologia su $X$ che lo rende uno SVT. Questa \`e la meno fine tale che $\Bc\subseteq \Uc_X$. In particolare $\Uc_X$ ha una base data da $\cpa{rB\mid B\in\wt \Bc}$.
\end{remark}






\subsection{Funzionali di Minkowski}
\begin{definition}[Funzionale di Minkowski]
Sia $X$ un $\R$-spazio vettoriale, $C\subseteq X$ convesso, $0\in C$. Il \textbf{funzionale di Minkowski} associato a $C$ \`e dato da:
\[p_C:\funcDef{X}{[0,+\infty]}{x}{\inf\cpa{t\geq 0\mid x\in tC}}\]
dove $\inf\emptyset=\infty$ in questo formalismo.
\end{definition}

\begin{remark}
Se $B(0,1)\subseteq C\subseteq \ol{B(0,1)}$ per $X$ normato allora $p_C(x)=\norm x$.
\end{remark}

\begin{proposition}[Propriet\`a funzionali di Minkowski]\label{PrProprietaFunzionaliMinkowski}
Valgono le seguenti propriet\`a
\begin{itemize}
    \item $C$ \`e assorbente se e solo se $p_C(x)<\infty$ per ogni $x\in X$.
    \item Si ha $\cpa{p_C<1}\subseteq C\subseteq\cpa{p_C\leq 1}$
\end{itemize}
\end{proposition}
\begin{proof}
Mostriamo le varie propriet\`a
\begin{itemize}
    \item Evidente dalla definizione di assorbente.
    \item Se $p_C(x)<1$ allora esiste $0\leq t\leq 1$ tale che $x\in tC$, cio\`e $x=tc$. Poich\'e $(1-t)0=0$ si ha $x=tc+(1-t)0$ e per convessit\`a questo \`e un elemento di $C$, cio\`e $x\in C$.
    
    Se $x\in C$ allora $1\in \cpa{t\geq 0\mid x\in tC}$, quindi $p_C(x)\leq 1$.
\end{itemize}
\end{proof}


\begin{remark}[Famiglia di seminorme induce SVTLC]
    Se $\Pc$ \`e una famiglia di seminorme su $X$, possiamo definire
    \[\Bc=\cpa{B_p(0,r)\mid p\in\Pc,\ r\in \R_+},\quad B_p(0,r)=\cpa{y\in X\mid p(x-y)<r}\]
    Si pu\`o mostrare che $\Bc$ \`e un insieme di dischi e quindi induce una struttura di SVTLC su $X$.
\end{remark}

\begin{remark}
Se $\Pc$ \`e una famiglia di seminorme su $X$ e definiamo
\[\wt\Pc=\cpa{\max (p_1,\cdots, p_n)\mid p_i\in \Pc}\]
allora $\Uc=\cpa{B_p(0,r)\mid p\in\wt \Pc,r>0}$ \`e una base di intorni di $0$ che induce la topologia dell'osservazione precedente.
\end{remark}

\begin{remark}[Ogni SVTLC \`e indotto da seminorme]\label{RmOgniSVTLCDerivaDaSeminorme}
Poich\'e se $B$ \`e assorbente, bilanciato e convesso, esso produce una seminorma $p_B$ data dal funzionale di Minkowski tale che $\cpa{p_B<1}\subseteq B\subseteq \cpa{p_B\leq 1}$, ogni topologia di $X$ come SVTLC si pu\`o ottenere a partire da famiglie di seminorme.
\end{remark}

\begin{proposition}
La topologia di SVTLC indotta da $\Pc$ insieme di seminorme \`e $T_0$ se e solo se $\Pc$ \`e separante, cio\`e per ogni $x\in X\nz$ esiste $p\in\Pc$ tale che $p(x)\neq 0$.
\end{proposition}
\begin{proof}
Se $p(x)=0$ per ogni $p\in\Pc$ allora $x\in B(0,r)$ per ogni $p\in\wt \Pc$ e per ogni $r>0$, quindi $x\in U$ per ogni $U\in\Uc_X$, ovvero
\[x\in \bigcap_{U\in\Uc_X}U=\ol{\cpa 0}.\]
\end{proof}


\section{Continuit\`a di operatori lineari in SVT}
\begin{proposition}[Continuit\`a mappe lineari]\label{PrContinuitaLineariSVT}
Sia $T:X\to Y$ lineare tra SVT. Valgono le seguenti affermazioni
\begin{enumerate}
    \item $T$ \`e continua se e solo se \`e continua in $0$
    \item $T$ \`e continua se e solo se per ogni $U\in \Uc_Y$ esiste $V\in \Uc_X$ tale che $T(V)\subseteq U$
    \item Se $X$ e $Y$ sono SVTLC con topologia indotta dalle famiglie di seminorme $\Pc$ e $\Qc$ rispettivamente, $T$ \`e continua se e solo se
    \[\forall q\in \Qc,\ \exists p_1,\cdots, p_n\in \Pc,\ \exists M\geq 0\quad\text{tali che}\]
    \[\forall x\in X,\ q(Tx)\leq M\max\cpa{p_1(x),\cdots,p_n(x)}\]
    \item Se $X$ e $Y$ sono SVTLC con topologia indotta dalle famiglie di seminorme $\Pc$ e $\Qc$ rispettivamente con $\Pc$ e $\Qc$ stabili per $\max$ allora $T$ \`e continua se e solo se $\forall q\in\Qc$ esistono $p\in\Pc$ e $M\geq 0$ tali che
    \[q(Tx)\leq Mp(x)\]
\end{enumerate}
\end{proposition}
\begin{proof}
Dimostriamo le affermazioni
\begin{enumerate}
    \item Basta traslare dato che traslare \`e un omeomorfismo.
    \item Ovvio.
    \item La condizione significa che la palla di centro $0$ e raggio 1 rispettivamente alla seminorma $\max(p_1,\cdots, p_n)$ di $X$ ha immagine tramite $T$ contenuta nella palla di raggio $M$ rispetto a $q$, concludendo per il punto 2. a meno di omotetia.
    \item Caso sopra.
\end{enumerate}
\end{proof}

\begin{proposition}[Caratterizzazione funzionali continui]\label{PrCaratterizzazioneFunzionaliContinui}
Sia $f\in X'_{alg}\nz$ con $X$ un $\K$-spazio vettoriale. Le seguenti affermazioni sono equivalenti
\begin{enumerate}
    \item $f$ \`e continua
    \item $\ker f$ \`e chiuso
    \item $\ker f$ non \`e denso
    \item $f$ non \`e surgettiva su un aperto non vuoto
    \item $f$ \`e limitata su un intorno di $0$
\end{enumerate}
\end{proposition}
\begin{proof}
    Diamo le implicazioni
\setlength{\leftmargini}{0cm}
\begin{itemize}
\item[$\boxed{1.\implies2.}$] Ovvio perch\'e $\cpa{0}$ \`e chiuso in $\K$.
\item[$\boxed{2.\implies3.}$] Se $\ker f$ \`e denso allora $\ol{\ker f}=X$ e quindi ha codimensione 0, ma $\ker f$ ha codimensione 1 in quanto $f\neq0$, quindi $\ker f\neq \ol{\ker f}$, cio\`e non \`e chiuso.
\item[$\boxed{3.\implies4.}$] Se $\ker f$ non \`e denso esiste un aperto non vuoto $A$ disgiunto da $\ker f$, cio\`e $0\notin f(A)$ e in particolare $f$ non \`e surgettiva su $A$.
\item[$\boxed{4.\implies5.}$] Se $f$ non \`e surgettiva su aperto non vuoto allora non lo \`e su un intorno bilanciato $U$ di $0$ e quindi $f(U)$ \`e un insieme bilanciato di $\K$ diverso da $\K$ in quanto $f\neq 0$, dunque $f(U)$ \`e un disco e in particolare \`e limitato.
\item[$\boxed{5.\implies1.}$] Se $\abs{fx}\leq M$ per ogni $x\in U\in\Uc_X$ allora per omogeneit\`a
\[\abs{f(x)}\leq \e\quad \forall x\in \frac\e M U\in \Uc_X\]
per un qualsiasi $\e>0$, quindi $f$ \`e continua in $0$. Questo conclude perch\'e
\[f(x)=f(x_0)+f(x-x_0).\]
\end{itemize}
\setlength{\leftmargini}{0.5cm}
\end{proof}

\section{SVT I-numerabili e paranorme}
\begin{definition}[Paranorma]
Una \textbf{paranorma} sull $\K$-spazio vettoriale $X$ \`e una funzione $q:X\to[0,\infty)$ tale che
\begin{enumerate}
    \item $q(x+y)\leq q(x)+q(y)$
    \item $q(\la x)\leq q(x)$ per ogni $x\in X$ e $\la\in\K$ tale che $\abs{\la}\leq 1$
    \item Se $\la_k\to 0$ in $\K$ allora $q(\la_k x)\to 0$
\end{enumerate}
Inoltre $q$ \`e \textbf{definita} se vale
\[q(x)=0\coimplies x=0.\]
\end{definition}

\begin{remark}
Dalla propriet\`a $2.$ segue che $q(\la x)=q(x)$ se $\abs\la=1$ e che $q(\la x)\leq q(\mu x)$ se $\abs\la\leq\abs\mu$. In particolare $q(x)=q(-x)$.

Quindi $d(x,y)=q(x-y)$ \`e una (semi)distanza su $X$ (distanza se $q$ definita).
\end{remark}

\begin{exercise}
Dimostrare che $(X,d)$ \`e uno SVT per $d$ indotta da paranorma $q$.
\end{exercise}

\begin{exercise}
Sia $X$ un $\K$-SVT I-numerabile. Allora la sua topologia proviene da una paranorma (la quale \`e definita sse $X$ \`e $T_0$).
\end{exercise}
\begin{proof}
TRACCIA
\begin{itemize}
    \item Sia $\cpa{U_n}_{n\geq 0}$ base numerabile di intorni bilanciati di $0$ tali che $U_{n+1}+U_{n+1}\subseteq U_n$.
    \item Estendiamo la successione per $n<0$ ponendo $U_k=U_{k+1}+U_{k+1}$ per ogni $k<0$.
    
    Nota che $U_{k+1}+U_{k+1}\subseteq U_k$ per ogni $k\in \Z$ e gli $\cpa{U_k}_{k\in\Z}$ sono intorni bilanciati. 
    \item Poniamo
    \[q(x)=\inf\cpa{\sum_{i=1}^r2^{-ki}\mid r\in\N,(k_1,\cdots, k_r)\in\Z^r\ t.c.\ x\in U_{k_1}+U_{k_2}+\cdots+U_{k_r}}\]
    Mostra che $q$ \`e una paranorma su $X$.
    \item Nota che $\cpa{q<2^{-n-1}}\subseteq U_n\subseteq \cpa{q\leq 2^{-n}}$ e quindi $q$ induce la topologia di $X$.
\end{itemize}
\end{proof}



\section{Topologie deboli}

\begin{proposition}[Topologia iniziale nel caso SVT]\label{PrTopologiaInizialeCasoSVT}
Sia $X$ uno spazio vettoriale su $\K$ e sia $\Fc:\cpa{T_i:X\to Y_i}$ dove ogni $Y_i$ \`e SVT e $T_i$ \`e lineare, allora la topologia iniziale su $X$ indotta\footnote{vedi (\ref{PrTopologiaInizialeEsiste})} da $\Fc$ rende $X$ uno SVT.
\end{proposition}
\begin{proof}
Voglio verificare che $+$ e $\cdot$ sono mappe continue per la topologia iniziale.
% https://q.uiver.app/#q=WzAsNCxbMCwwLCJYXFx0aW1lcyBYIl0sWzEsMCwiWCJdLFswLDEsIllfaVxcdGltZXMgWV9pIl0sWzEsMSwiWV9pIl0sWzEsMywiVF9pIl0sWzAsMiwiVF9pXFx0aW1lcyBUX2kiLDJdLFswLDEsIisiXSxbMiwzLCIrX2kiLDJdXQ==
\[\begin{tikzcd}
	{X\times X} & X \\
	{Y_i\times Y_i} & {Y_i}
	\arrow["{+}", from=1-1, to=1-2]
	\arrow["{T_i\times T_i}"', from=1-1, to=2-1]
	\arrow["{T_i}", from=1-2, to=2-2]
	\arrow["{+_i}"', from=2-1, to=2-2]
\end{tikzcd}\]
% https://q.uiver.app/#q=WzAsNCxbMCwwLCJcXEtcXHRpbWVzIFgiXSxbMSwwLCJYIl0sWzAsMSwiXFxLXFx0aW1lcyBZX2kiXSxbMSwxLCJZX2kiXSxbMSwzLCJUX2kiXSxbMCwyLCJpZF9cXEtcXHRpbWVzIFRfaSIsMl0sWzAsMSwiXFxjZG90Il0sWzIsMywiXFxjZG90X2kiLDJdXQ==
\[\begin{tikzcd}
	{\K\times X} & X \\
	{\K\times Y_i} & {Y_i}
	\arrow["\cdot", from=1-1, to=1-2]
	\arrow["{id_\K\times T_i}"', from=1-1, to=2-1]
	\arrow["{T_i}", from=1-2, to=2-2]
	\arrow["{\cdot_i}"', from=2-1, to=2-2]
\end{tikzcd}\]
Per la propriet\`a universale della topologia iniziale (\ref{PrProprietaUniversaleTopologiaIniziale}), vogliamo verificare che $T_i\circ +=+_i\circ (T_i\times T_i)$ \`e continua per ogni $i$ e similmente per $T_i\circ \cdot$. Questo \`e vero perch\'e la topologia iniziale \`e rende $T_i$ continua per ogni $i$.
\end{proof}

\begin{remark}
Se ogni $Y_i$ inoltre \`e SVTLC allora anche $X$ lo \`e.
\end{remark}


\begin{definition}[Topologie deboli]
Sia $X$ un $\K$-spazio vettoriale e $\Fc\subseteq X'$ (duale algebrico). La topologia iniziale indotta da $\Fc$ viene detta la \textbf{topologia debole di $\Fc$} e si indica $\sigma(X,\Fc)$.
\end{definition}

\begin{remark}
$\sigma(X,\Fc)=\sigma(X,\Span_\K(\Fc))$ quindi senza perdita di generalit\`a possiamo sempre supporre $\Fc$ sottospazio vettoriale di $X'$.
\end{remark}

\begin{remark}
La famiglia di seminorme associata a $\Fc$ (quella che induce la stessa topologia di $SVTLC$) \`e data da
\[\Pc=\cpa{\abs{f}\mid f\in\Fc}\]
\end{remark}

\begin{remark}
La topologia debole $\sigma(X,\Fc)$ \`e $T_0$ (e quindi Hausdorff perch\'e SVT) se e solo se la famiglia $\Fc$ \`e separante ($\forall x\in X\nz,\ \exists f\in\Fc$ tale che $f(x)\neq0$).
\end{remark}


\begin{lemma}[]\label{LmDualeAlgebricoIndipendenzaEContinuita}
Siano $f_0,\cdots,f_n\in X'_{alg}$ per $X$ un $\K$-spazio vettoriale, allora sono equivalenti
\begin{enumerate}
    \item $f_0=\sum_{i=1}^n\la_if_i$
    \item $\abs{f_0}\leq M\max_{i\in\cpa{1,\cdots, n}}\abs{f_i}$ per qualche $M\geq 0$
    \item $\ker f_0\supseteq \bigcap_{i=1}^n\ker f_i$
\end{enumerate}
\end{lemma}
\begin{proof}
    Diamo le tre implicazioni
    \setlength{\leftmargini}{0cm}
    \begin{itemize}
    \item[$\boxed{1.\implies2.}$] Da $1.$ segue $\abs{f_0}\leq\sum_{i=1}^n\abs{\la_i}\abs{f_i}\leq M\max\abs{f_i}$ per $M=\sum\abs{\la_i}$.
    \item[$\boxed{2.\implies3.}$] Se $x\in \bigcap\ker f_i$, cio\`e $\ps{f_i,x}=0$ per ogni $i$, allora $\ps{f_0,x}\leq M0=0$, cio\`e $f_0(x)=0$ e abbiamo l'inclusione voluta.
    \item[$\boxed{3.\implies1.}$] Sia $F:X\to\K^n$ data da $F=(f_1,\cdots, f_n)$, allora
    \[\ker F=\bigcap\ker f_i\subseteq \ker f_0\]
    quindi abbiamo una fattorizzazione
    % https://q.uiver.app/#q=WzAsMyxbMCwwLCJYIl0sWzAsMSwiXFxLXm4iXSxbMSwwLCJcXEsiXSxbMSwyLCJMIiwyLHsic3R5bGUiOnsiYm9keSI6eyJuYW1lIjoiZGFzaGVkIn19fV0sWzAsMSwiRiIsMl0sWzAsMiwiZl8wIl1d
\[\begin{tikzcd}
	X & \K \\
	{\K^n}
	\arrow["{f_0}", from=1-1, to=1-2]
	\arrow["F"', from=1-1, to=2-1]
	\arrow["L"', dashed, from=2-1, to=1-2]
\end{tikzcd}\]
    dove $L(x_1,\cdots, x_n)=\sum \la_i x_i$ per dei $\la_i$ (in quanto \`e una forma lineare). Ma allora $f_0=L\circ F=\sum \la_i f_i$ come voluto.
    \end{itemize}
    \setlength{\leftmargini}{0.5cm}
\end{proof}

\begin{proposition}[Duale per topologia debole]\label{PrDualePerTopologiaDebole}
Dato $X$ $\K$-spazio vettoriale e $\Fc$ sottospazio di $X'_{alg}$ allora
\[(X,\sigma(X,\Fc))^\ast=\Fc\]
\end{proposition}
\begin{proof}
Sia $f_0\in (X,\sigma(X,\Fc))^\ast$, allora per la proposizione (\ref{PrContinuitaLineariSVT}) esistono $f_1,\cdots, f_n\in\Fc$ e $M\geq0$ tali che per ogni $x\in X$
\[\abs{f_0(x)}\leq M\max_{i}\abs{f_i(x)}.\]
Dunque per il lemma (\ref{LmDualeAlgebricoIndipendenzaEContinuita}) $f_0$ si scrive come combinazione lineare delle $f_i$ e quindi in particolare $f_0\in\Fc$.

L'altra inclusione \`e ovvia per definizione di topologia debole.
\end{proof}

\begin{remark}
Se $X$ ha dimensione infinita, $\sigma(X,\Fc)$ non \`e mai localmente limitata. In particolare ogni intorno di $0$ contiene uno spazio vettoriale di codimensione finita.
\end{remark}
\begin{proof}
Se $U$ intorno di $0$ per $\sigma(X,\Fc)$ allora esistono $f_1,\cdots,f_n\in\Fc$ tali che\footnote{vedi lemma (\ref{LmDualeAlgebricoIndipendenzaEContinuita})}
\[U\supseteq \bigcap_{i=1}^n\cpa{\abs{f_i}<1}\supseteq \bigcap_{i=1}^n \ker f_i\]
e l'intersezione di questi nuclei ha codimensione al massimo $n$.
\end{proof}

\begin{proposition}[Duale di lineare continua \`e debole$^\ast$-continua]\label{PrDualeOperatoreContinuoEDeboleStarContinua}
Se $T:E\to F$ \`e un operatore lineare e continuo allora $T^\ast:F^\ast\to E^\ast$ \`e debole$^\ast$-continua.
\end{proposition}
\begin{proof}
Considera le opportune composizione e la definizione di topologia debole.
\end{proof}

\subsection{Caso degli spazi normati}

\begin{definition}[Topologia debole]
Se $X$ \`e normato, la \textbf{topologia debole} su $X$ \`e la topologia debole associata a $X^\ast$, cio\`e $\sigma(X,X^\ast)$.
\end{definition}

\begin{proposition}
La topologia debole \`e localmente convessa e Hausdorff.
\end{proposition}
\begin{proof}
Per Hahn-Banach (\ref{ThHahnBanach}), il duale $X^\ast$ separa i punti
\end{proof}


\begin{definition}[Topologia debole$^\ast$]
Su $X^\ast$ possiamo considerare la topologia debole associata alle valutazioni $X\subseteq X^{\ast\ast}$, cio\`e scegliendo
\[\Fc=\cpa{val_x\in (X^\ast)'\mid x\in X}.\]
Questa \`e la \textbf{topologia debole$^\ast$} su $X^\ast$ e la indichiamo $\sigma(X^\ast,X)$.
\end{definition}

\begin{remark}
La topologia debole$^\ast$ rende $X^\ast$ uno SVTLC $T_0$ (e quindi Hausdorff), infatti se $f\in X^\ast\nz$ allora esiste $x\in X$ tale che $f(x)\neq 0$.
\end{remark}

\begin{remark}
In generale $\sigma(X^\ast,X)$ \`e meno fine di $\sigma(X^\ast,X^{\ast\ast})$. Abbiamo uguaglianza solo quando $X=X^{\ast\ast}$ in quanto se $X\neq X^{\ast\ast}$ allora dalla proposizione (\ref{PrDualePerTopologiaDebole}) ricaviamo
\[(X^\ast,\sigma(X^\ast,X))^\ast=X\neq X^{\ast\ast}=(X^\ast,\sigma(X^\ast,X^{\ast\ast}))^\ast\]
e quindi in partenza $\sigma(X^\ast,X^{\ast\ast})\neq \sigma(X^\ast,X)$
\end{remark}

\begin{remark}
Poich\'e $(X,\normd)\inj (X^{\ast\ast},\normd)$ isometricamente allora $(X,\sigma(X,X^\ast))$ ha la topologia indotta come sottospazio da\footnote{nota che $X^\ast$ lo si pu\`o pensare come immerso in $X^{\ast\ast\ast}=(X^{\ast\ast})^\ast$, quindi stiamo considerando la topologia debole$^\ast$ su $(X^{\ast})^\ast$} $(X^{\ast\ast},\sigma(X^{\ast\ast},X^\ast))$.
\end{remark}
\begin{proof}
Questo deriva dalla transitivit\`a della topologia iniziale (\ref{PrTransitivitaTopologiaIniziale}) dove la prima famiglia \`e la mappa $X\inj X^{\ast\ast}$ e l'unica altra famiglia sono gli elementi di $X^\ast$ che vanno verso $\K$.
\end{proof}


\section{Teorema di Riesz}
\begin{theorem}[Riesz]\label{ThRiesz}
Per $X$ SVT $T_0$ su $\K$ sono equivalenti
\begin{enumerate}
    \item $X$ ha dimensione finita
    \item $X\cong \K^n$ per qualche $n\in\N$
    \item $X$ \`e localmente compatto
\end{enumerate}
\end{theorem}
\begin{proof}
    Diamo le implicazioni
\setlength{\leftmargini}{0cm}
\begin{itemize}
\item[$\boxed{1.\implies2.}$] Sia $X$ SVT $T_0$ di dimensione $n$ e sia $x_1,\cdots, x_n$ una sua base di Hamel. Allora
\[\vp:\funcDef{\K^n}{X}{\la=(\la_1,\cdots,\la_n)}{\sum_{i=1}^n}\la_i x_i\]
\`e lineare, bigettiva e continua. 

Dimostriamo che \`e aperta: L'insieme $\del B(0,1)\subseteq \K^n$ visto con la norma euclidea \`e compatto, quindi $\vp(\del B(0,1))$ \`e compatto, e quindi chiuso perch\'e $X$ \`e Hausdorff. Per bigettivit\`a $0\notin \vp(\del B(0,1))$, quindi esiste un intorno $V$ di $0$ in $X$ disgiunto da $\vp(\del B(0,1))$. Senza perdita di generalit\`a $V$ bilanciato, allora $\vp\ii(V)$ \`e un insieme bilanciato di $\K^n$ disgiunto da $\del B(0,1)$, dunque $\vp\ii(V)\subseteq B(0,1)$ (se avesse un punto di modulo maggiore a $1$ allora in quanto bilanciato conterrebbe tutti i punti tra esso e $0$, intersecando il bordo). 

Questo mostra che $B(0,1)$ \`e un intorno di $0$ e quindi $\vp$ \`e aperta (per traslazione e omotetia $\vp(B(\la,r))$ \`e intorno di $\vp(\la)$ per ogni $\la\in\K^n$ e $r>0$ e concludo notando che aperti di $\K^n$ sono dati da unioni di palle).
\item[$\boxed{2.\implies3.}$] $\K^n$ \`e localmente compatto perch\'e conosciamo la topologia euclidea, quindi anche $X$ lo \`e.
\item[$\boxed{3.\implies1.}$] Sia $X$ SVT localmente compatto e $T_0$. Mostriamo che $X$ \`e I-numerabile:

Sia $V$ intorno compatto di $0$. Mostriamo che $\cpa{\frac1n V}$ \`e una base di intorni di $0$. Sia $U$ un intorno (senza perdita di generalit\`a $U$ bilanciato). Poich\'e $V$ \`e compatto e\footnote{$U$ assorbente} $V\subseteq \bigcup_{n\geq 1}nU=X$ possiamo estrarre un sottoricoprimento finito
\[V\subseteq \bigcup_{1\leq i\leq k}n_i U\pasgnl={$U$ bilanciato}\pa{\max_{1\leq i\leq k}n_i}U\]
infatti $\dfrac{n_i}{\max n_i}U\subseteq U$. Questo mostra che $\cpa{\frac1n V}$ \`e una base numerabile di intorni di $0\in X$.


Notiamo che $V$ si pu\`o coprire con un numero finito di traslati di $\frac12 V$ in quanto $V\subseteq V+\frac12V$ e applico compattezza al variare di $v+\frac12V$ per $v\in V$. Sia allora $F$ tale che $V\subseteq\bigcup_{v\in F}v+\frac12V$ con $F$ finito e poniamo $Y=\Span_\K F$. Notiamo che $Y$ ha dimensione finita.

Procedendo per induzione, per ogni $n\in\N$ si ha $V\subseteq Y+2^{-n}V$, ma $\cpa{2^{-n}V}_{n\geq0}$ \`e una base di intorni, quindi 
\[\ol Y=\bigcap_{n\geq 0}Y+2^{-n}V\supseteq V\]
e dato che $V$ \`e un intorno assorbente, $X=\bigcup_{n\geq 0}nV\subseteq \ol Y$, cio\`e $Y$ \`e denso in $X$.

Poich\'e $Y$ ha dimensione finita, per l'implicazione precendente $Y\cong \K^n$, in particolare $Y$ \`e completo. Se $x\in X=\ol Y$, poich\'e $X$ \`e I-numerabile, si ha che esiste $y_k\to x$ in $X$ con $y_k\in Y$ con $(y_k)$ di Cauchy in $X$ e quindi anche in $Y$, che per\`o \`e completo, quindi $y_k\to y$ per $y\in Y$, ma $X$ \`e Hausdorff, quindi $y=x$.
\end{itemize}
\setlength{\leftmargini}{0.5cm}
\end{proof}

\begin{remark}
Se non avessimo supposto $T_0$ potremmo considerare $\quot X{\ol{\cpa0}}$ e troveremmo $X\cong \K^n\oplus \ol{\cpa0}$.
\end{remark}


\section{Successioni generalizzate (nets)}
\begin{definition}[Net]
Un \textbf{net} su un insieme $X$ \`e una funzione $f:D\to X$ su $(D,\geq)$ poset diretto\footnote{diretto nel senso che per ogni $i,j\in D$ esiste $k\in D$ tale che $i\leq k$ e $j\leq k$.}.
\end{definition}

\begin{example}[Somme di Riemann]
Sia $u:[a,b]\to X$ una funzione con $X$ SVT. La \textbf{somma di Riemann} per $u$ relativa ad una suddivisione $P=\cpa{a=t_0<t_1<\cdots<t_n=b}$ e una scelta di punti $\Xi=\cpa{\xi_1,\cdots, \xi_n}$ con $\xi_i\in [t_{i-1},t_i]$ \`e
\[S(u;P,\Xi)=\sum_{i=1}^nu(\xi_i)(t_i-t_{i-1}).\]
Possiamo prendere $D=\cpa{(P,\Xi)}$ l'insieme delle possibili partizioni e scelte di punti. $D$ \`e un poset: $(P,\Xi)\geq (P',\Xi')$ se $P\supseteq P'$.

In questo contesto l'integrale di Riemann sarebbe il limite rispetto al net $D\to X$ dato da $(P,\Xi)\mapsto S(u;P,\Xi)$.
\end{example}

\begin{example}[Somme infinite]
Data $\cpa{x_i}_{i\in I}\subseteq X$ con $X$ SVT consideriamo
\[S:\funcDef{\Ps_{fin}(I)}{X}{F}{\sum_{i\in F}x_i}\]
$\Ps_{fin}(I)$ \`e parzialmente ordinato per inclusione e la somma sarebbe il limite.
\end{example}

\begin{definition}[Definitivamente e frequentemente]
Diciamo che se $\cpa{P_\al}_{\al\in D}$ sono propriet\`a indicizzate su $D$ insieme diretto allora \textbf{$P_\al$ vale definivamente} (risp. \textbf{frequentemente}) se esiste $\al\in D$ tale che per ogni $\beta\geq \al$ in $D$ vale $P_\beta$ (risp. per ogni $\al\in D$ esiste $\beta\in D$ tale che vale $P_\beta$).
\end{definition}
\begin{remark}
Se $D\neq \N$ allora pu\`o succedere che ``frequentemente"$\neq$``infinite volte".
\end{remark}

\begin{definition}[Convergenza per net]
Se $f:D\to X$ \`e un net su $X$ spazio topologico si ha che $f$ \textbf{converge a $x\in X$} se per ogni $U$ intorno di $x$ si ha che $f(i)\in U$ definitivamente.
\end{definition}

\begin{definition}[Punti di accumulazione per net]
Se $f:D\to X$ \`e un net su $X$ spazio topologico si ha che $x$ \`e un \textbf{punto di accumulazione} di $f$ se per ogni $U$ intorno di $x$, $f(i)\in U$ frequentemente.
\end{definition}

\begin{definition}[Sottonet]
Una $\vp:D'\to D$ con $D,D'$ insiemi diretti tale che per ogni $i\in D$ esiste $i'\in D'$ tale che $\vp(j)\geq i$ per ogni $j\geq i'$ \`e detta \textbf{cofinale}.

Sia $f:D\to X$ un net, allora $f\circ\vp:D'\to X$ per $\vp$ cofinale \`e un \textbf{sottonet} di $f$.
\end{definition}
\begin{remark}
Una successione \`e un net su $\N$, una sottosuccessione \`e quindi in particolare un sottonet, ma non tutti i sottonet di una successione sono sottosuccessioni.
\end{remark}

\begin{exercise}
Se $f:D\to X$ spazio topologico e $x\in X$ allora $x$ \`e aderente a $f$ se e solo se $x$ \`e limite di qualche sottonet di $f$.
\end{exercise}

\begin{remark}
Dato $f:D\to X$ net, l'insieme $A$ dei punti aderenti a $f$ \`e
\[A=\bigcap_{j\in D} \ol{\cpa{f(i)\mid i\geq j}}\]
infatti $x$ \`e aderente se e sono se per ogni intorno $U$ e ogni $j\in D$ esiste $i\geq j$ tale che $f(i)\in U$, cio\`e per ogni $j\in D$ $U\cap\cpa{f(i)\mid i\geq j}\neq \emptyset$, ovvero per ogni $j\in D$ si ha $x\in \ol{f(i)\mid i\geq j}$.
\end{remark}

\begin{exercise}
$X$ spazio topologico \`e compatto per ricoprimenti se e solo se ogni net in $X$ ha punti aderenti, cio\`e se e solo se per ogni net su $X$ esiste un sottonet convergente.
\end{exercise}

\begin{exercise}
Usare l'esercizio sopra per dimostrare Tychonoff.
\end{exercise}
\begin{proof}
IDEA:
\begin{itemize}
	\item Sia $f:D\to \prod_{\la\in \Lambda}X_\la$ un net, vogliamo trovare dei punti aderenti.
	\item Consideriamo l'insieme
	\[S=\cpa{(N,x)\mid x\in \prod_{\la\in N}X_\la,\ N\subseteq \Lambda,\ x\text{ aderente per }P_N\circ f:D\to \prod_{\la\in N}X_\la}\]
	esso \`e non vuoto perch\'e se $N$ \`e un singoletto allora $P_N\circ f$ \`e un net verso uno spazio compatto, quindi ha un punto aderente. Ordiniamo $S$ ponendo $(N,x)\leq (N',x')$ se $N\subseteq N'$ e $P_N(x')=x$.

	Vale la condizione della catena, infatti se $\cpa{(N_\al,x_\al)}$ \`e una catena ascendente in $S$ allora basta considerare $N=\bigcup N_\al$ e $x\in \prod_{\la\in N} X_\la$ dato da $x(\la)=x_\al(\la)$ per un qualche $\al$ tale che $\la\in N_\al$. Notiamo che $x$ cos\`i definito \`e aderente a $P_N\circ f$ perch\'e gli $x_\al$ sono aderenti e questo basta per la definizione di topologia prodotto.

	Dunque per il lemma di Zorn esiste un dominio massimale $(N,x)$
	\item Se per assurdo $N\neq \Lambda$ allora esiste $\la\in \Lambda\bs N$, ma allora possiamo estendere $(N,x)$ a $(N\cup \cpa\la,\wt x)$ per $\wt x=x$ fuori $\la$ e uguale a un qualche aderente a $P_{\cpa{\la}}\circ f$ in $\la$. Questo nega la massimalit\`a.
\end{itemize}
\end{proof}

\begin{exercise}
Per $X$ spazio topologico e $S\subseteq X$ si ha $x\in \ol S$ se e solo se esiste $f:D\to S$ net convergente a $x$.
\end{exercise}

\begin{definition}[Net di Cauchy]
Sia $X$ SVT. Un net $f:D\to X$ \`e \textbf{di Cauchy} se per ogni $U\in\Uc_X$ esiste $i\in D$ tale che per ogni $p\geq i$ e $q\geq i$ vale $f(p)-f(q)\in U$.

Equivalentemente il net $\wt f:D\times D\to X$ definito da $\wt f(i,j)=f(i)-f(j)$ con $(i,j)\geq(i',j')\coimplies i\geq i'\wedge j\geq j'$ converge a $0$.
\end{definition}

\begin{definition}[Completo per nets]
Uno SVT \`e \textbf{completo per nets} se ogni net di Cauchy converge.
\end{definition}

\begin{exercise}
Uno SVT I-numerabile \`e completo per nets se e solo se \`e completo per successioni.
\end{exercise}