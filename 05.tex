\chapter{Lemma di iterazione e Iniettivit\`a / Surgettivit\`a di mappe lineari}



\section{Lemma di iterazione}

\begin{lemma}[di iterazione]\label{LmDiIterazione}
Siano $X$ e $Y$ spazi di Banach, $B$ palla unitaria chiusa di $X$, $T\in L(X,Y)$, $U$ limitato, $U\subseteq Y$ tali che se $0<t<1$ allora
\[U\subseteq TB+tU.\]
Allora si ha $(1-t)U\subseteq TB$.
\end{lemma}
\begin{proof}
Sia $u_0\in U$, allora esistono $x_0\in B$ e $u_1\in U$ tali che 
\[u_0=T(x_0)+tu_1\]
Iterando troviamo $u_2\in U$ e $x_1\in B$ tali che $u_1=T(x_1)+tu_2$ e cos\`i via. Questo definisce quindi due successioni $(u_n)\subseteq U$ e $(x_n)\subseteq B$. Notiamo che per ogni $n\in\N$
\[u_0=t^{n+1}y_{n+1}+\sum_{i=0}^nt^iT(x_i)=T\pa{\sum_{i=0}^nt^ix_i}+t^{n+1}y_{n+1}.\]
Poich\'e $X$ \`e completo, la serie $\sum_{i=0}^\infty t^i x_i$ converge ad un punto $x^\ast\in \frac1{1-t}B$ in quanto $\sum_{i=0}^\infty t^i=\frac1{1-t}$.

Poich\'e $U$ \`e limitato esiste $M>0$ tale che $U\subseteq B(0,M)$, quindi $\norm{t^{n+1} y_{n+1}}\leq t^{n+1}M$ e questa successione converge a $0$ quindi $t^{n+1} y_{n+1}$ converge a $0$. Segue che
\[y_0=\lim_{n\to\infty} T\pa{\sum_{i=0}^nt^i x_i}+\under{=o(1)}{t^{n+1}y_{n+1}}\pasgnl={$T$ continua}T\pa{\lim_{n\to\infty}\sum_{i=0}^nt^ix_i}=T(x^\ast)\]
quindi $y_0\in \frac1{1-t}T B$, cio\`e
\[U\subseteq \frac1{1-t}TB\coimplies (1-t)U\subseteq TB.\]
\end{proof}

\begin{remark}
Se $U$ \`e un intorno di $0$ limitato in $Y$, o anche $U$ assorbente, allora $T$ \`e surgettivo.
\end{remark}

\begin{theorem}[Lemma di Urysohn]\label{ThLemmaUrysohn}
Se $X$ \`e normale e $F_0$,$F_1$ sono chiusi disgiunti di $X$ allora esiste $f$ tale che $F_0=\cpa{f=0}$ e $F_1=\cpa{f=1}$.
\end{theorem}

\begin{theorem}[Teorema di estensione di Tietze]\label{ThEstensioneTietze}
Se $X$ \`e T4, $Y\subseteq X$ chiuso, $f\in C^0(Y,\R)$, allora $f$ si estende ad una continua su $X$.
\end{theorem}
\begin{proof}
Basta il caso di $f$ limitata tanto la continuit\`a \`e una condizione che \`e invariante componendo per un omeomorfismo e $\R\cong (0,1)$.

La tesi \`e che l'operatore di restrizione (il quale \`e lineare e continuo)
\[R:C_b^0(X)\to C_b^0(Y)\]
\`e surgettivo. Basta applicare il lemma (\ref{LmDiIterazione}) come segue:
\[3B_{C_b(Y)}\subseteq R\pa{B_{C_b(X)}}+2B_{C_b(Y)}\]
e chiamiamo $U=3B_{C_b(Y)}$, $T=(2/3)\cdot$. Sia $f\in 3B_{C_b(Y)}$. Per il lemma di Urysohn esiste $g:X\to[-1,1]$ continua tale che $g=-1$ su $\cpa{x\in Y\mid -3\leq f\leq -1}$ e $g=1$ su $\cpa{x\in Y\mid 3\geq f\geq 1}$ (i due insiemi sono chiusi perch\'e $Y$ \`e chiuso e $f$ \`e continua).
\[f=g\res Y+(f-g\res Y)\]
ma notiamo allora che $g\in B_{C_b(X)}$ e quindi $g\res Y\in R(B_{C_b(X)})$, mentre $f-g\res Y\in 2B_{C_b(Y)}$, infatti su $\cpa{x\in Y\mid -3\leq f\leq -1}$ abbiamo $g=-1$ e quindi $\norm{f-g}_{\infty,Y}\leq 2$, su $\cpa{x\in Y\mid 3\geq f\geq 1}$ abbiamo $g=1$ e quindi di nuovo $\norm{f-g}_{\infty,Y}\leq 2$, e infine sui punti rimanenti, siccome $f\in 3B_{C_b(Y)}$, si ha $\norm f\leq 1$ e stesso per $g$, quindi $\norm{f-g\res Y}_{\infty,Y}\leq 2$ di nuovo.

Questo verifica le ipotesi del lemma di iterazione (\ref{LmDiIterazione}), quindi 
\[(1-2/3)B_{C_b(Y)}\subseteq R(B_{C_b(X)}).\]
\end{proof}

\begin{theorem}[Dugundji]\label{ThEstensioneDugundji}
Sia $(M,d)$ spazio metrico, $A\subseteq M$ chiuso, $E$ banach e $f:A\to E$ continua (basta limitata) allora esiste una estensione di $f$ continua a tutto $M$ con la stessa norma.
\end{theorem}

\begin{remark}
In realt\`a l'estensione di $f$ a $M$ si pu\`o dare come un operatore di estensione
\[\Ec:C_b(A,E)\to C_b(M,E).\]
Questo operatore \`e inverso destro dell'operatore di restrizione $R:C_b(M,E)\to C_b(A,E)$ che abbiamo usato nel teorema di Tietze (\ref{ThEstensioneTietze}).
\end{remark}


\begin{theorem}[Sollevamento per operatori lineari / Bartles-Groves]\label{ThSollevamentoOperatoriLineariBartles-Groves}
Sia $L:E\to F$ lineare continuo surgettivo con $E,F$ banach. $M$ spazio metrico e $f:M\to F$ continua, allora $f$ si pu\`o sollevare a $E$, cio\`e esiste $\wt f:M\to E$ continua tale che $f=L\circ \wt f$.
% https://q.uiver.app/#q=WzAsMyxbMSwwLCJFIl0sWzEsMSwiRiJdLFswLDEsIk0iXSxbMCwxLCJMIl0sWzIsMSwiZiIsMl0sWzIsMCwiXFx3dCBmIiwwLHsic3R5bGUiOnsiYm9keSI6eyJuYW1lIjoiZGFzaGVkIn19fV1d
\[\begin{tikzcd}
	& E \\
	M & F
	\arrow["L", from=1-2, to=2-2]
	\arrow["{\wt f}", dashed, from=2-1, to=1-2]
	\arrow["f"', from=2-1, to=2-2]
\end{tikzcd}\]
In altre parole, \`e surgettivo l'operatore (lineare e continuo)
\[L_\ast:\funcDef{C_b(M,E)}{C_b(M,F)}{g}{L\circ g}.\]
\end{theorem}
\begin{proof}
Applichiamo il lemma di surgettivit\`a lineare come segue: sia $f\in C_b(M,E)$ e consideriamo un sollevamento approssimato $g$ costruito con partizioni dell'unit\`a a partire da sollevamenti approssimati locali che sono costanti. 
\end{proof}
\begin{remark}
Se $M=F$ e $f=id_F$ allora questo restituisce una inversa destra continua  (ma possibilmente non lineare) $\sigma$ di $L$. Quindi ogni operatore lineare surgettivo ammette una inversa destra continua. Inoltre se $L$ non ammette inversa destra lineare allora $\sigma$ non \`e neanche differenziabile in alcun punto (se fosse differenziabile $L\circ \sigma=id\implies L\circ \Dc\sigma=id$)
\end{remark}

\begin{remark}
Se $X,Y$ banach, l'insieme degli operatori surgettivi $\Sc\Uc=\cpa{L\in L(X,Y,L\text{ surg})}$ \`e aperto in $L(X,Y)$. 
\end{remark}
\begin{proof}
Se $T\in \Sc\Uc(X,Y)$ allora esso induce
% https://q.uiver.app/#q=WzAsMyxbMCwwLCJYIl0sWzEsMCwiWSJdLFswLDEsIlgvXFxrZXIgVCJdLFswLDEsIlQiXSxbMCwyXSxbMiwxLCJcXHd0IFQiLDJdXQ==
\[\begin{tikzcd}
	X & Y \\
	{X/\ker T}
	\arrow["T", from=1-1, to=1-2]
	\arrow[from=1-1, to=2-1]
	\arrow["{\wt T}"', from=2-1, to=1-2]
\end{tikzcd}\]
Sia $k=(\norm{\wt T\ii})\ii\in \R$, allora per ogni $H\in L(X,Y)$ con $\norm H<k$ abbiamo $T+H\in \Sc\Uc(X,Y)$: per definizione di $k$ vale $\wt T\ii(B_Y)\subseteq \frac1k B_X$ perch\'e $\frac1k=\norm{\wt T\ii}$ e quindi
\[kB_Y\subseteq TB_X=\wt T(\pi B_X)=\wt T(B_{X/\ker T})\]
Dunque 
\[kB_Y\subseteq TB_X\subseteq (T+H)B_X+HB_X\subseteq (T+H)B_X+\frac{\norm H}k(kB_Y)\]
quindi $T+H$ verifica le ipotesi del lemma di iterazione (\ref{LmDiIterazione}) con $t=\frac{\norm H}k<1$.
\end{proof}


\subsection{Teorema della mappa aperta}

\begin{theorem}[Mappa aperta]\label{ThMappaAperta}
Siano $X,Y$ Banach e $T:X\to Y$ lineare continuo e tale che $T(X)$ \`e di II-categoria in $Y$ (per esempio $T$ surgettivo). Allora $T$ \`e una mappa aperta.
\end{theorem}
\begin{proof}
Sia $B$ la palla unitaria chiusa di $X$. Basta mostrare che $T(B)$ \`e un intorno di $0$ in $Y$ (per omotetia e traslazione seguir\`a che $T$ manda intorni di $x$ in intorni di $T(x)$, cio\`e \`e aperta). Notiamo che 
\[X=\bigcup_n nB\implies T(X)=\bigcup_n nT(B)\]
Per ipotesi $T(X)$ \`e di II-categoria in $Y$, quindi per qualche $n$ si ha che $\ol{n T(B)}$ ha parte interna non vuota e quindi $\ol{T(B)}$ stesso ha parte interna non vuota. Poich\'e\footnote{ricorda che in generale se $f$ \`e continua allora $f(\ol A)\subseteq \ol{f(A)}$. 

In questo caso la mappa \`e $(x,y)\mapsto x-y$ e usiamo il fatto che $T$ \`e lineare e \[\ol{T(B)\times T(B)}=\ol{T(B)}\times \ol{T(B)}.\]} 
\[\ol{T(B)}-\ol{T(B)}\subseteq \ol{T(B-B)}=\ol{T(2B)}=2\ol{T(B)}\]
si ha che $\ol{T(B)}$ \`e un intorno di $0\in Y$.


Mostriamo ora che $T(B)$ stesso \`e un intorno di $0$. Poich\'e la chiusura \`e l'intersezione degli aperti che contengono $T(B)$ si ha in particolare che
\[\ol{T(B)}=T(B)+\frac12\ol{T(B)}.\]
Siccome $T$ \`e continua, $T(B)$ \`e limitato e quindi $\ol{T(B)}$ \`e limitato, quindi per il lemma di iterazione (\ref{LmDiIterazione}) di ha
\[\pa{1-\frac12}\ol{T(B)}\subseteq T(B)\coimplies \ol{T(B)}\subseteq 2T(B),\]
in particolare $T(B)$ \`e un intorno di $0$ per omotetia.
\end{proof}

\begin{remark}[Lineare continuo allora omeo se e solo se bigettivo]
Un operatore lineare continuo \`e un omeomorfismo se e solo se \`e bigettivo. Questo \`e immediato da mappa aperta (\ref{ThMappaAperta}).
\end{remark}

\begin{remark}
Se $T:X\to Y$ lineare continuo allora induce
% https://q.uiver.app/#q=WzAsMyxbMCwwLCJYIl0sWzEsMCwiWSJdLFswLDEsIlgvXFxrZXIgVCJdLFswLDIsIlxccGkiLDJdLFswLDEsIlQiXSxbMiwxLCJcXHd0IFQiLDJdXQ==
\[\begin{tikzcd}
	X & Y \\
	{X/\ker T}
	\arrow["T", from=1-1, to=1-2]
	\arrow["\pi"', from=1-1, to=2-1]
	\arrow["{\wt T}"', from=2-1, to=1-2]
\end{tikzcd}\]
con $\wt T$ lineare continua iniettiva. Se $T$ \`e surgettiva allora per il teorema della mappa aperta (\ref{ThMappaAperta}) $\wt T$ \`e un omeomorfismo lineare.
\end{remark}

\begin{remark}
Se $T:X\to Y$ \`e lineare e continua allora
\[\text{aperta $\coimplies$ surgettiva $\coimplies$ identificazione.}\]
\end{remark}

\begin{theorem}[Grafico chiuso]\label{ThGraficoChiuso}
Siano $X,Y$ Banach, $T:X\to Y$ lineare. Allora $T$ \`e continua se e solo se
\[\Gamma=\cpa{(x,T(x))\in X\times Y\mid x\in X}\]
\`e chiuso.
\end{theorem}
\begin{proof}
Data una qualsiasi mappa continua $f:X\to Y$ con $Y$ Hausdorff si ha che $\Gamma$ \`e la preimmagine della diagonale di $Y\times Y$ rispetto alla mappa $id_Y\times f$. Poich\'e $Y$ \`e un Banach (e quindi metrico e quindi $T_2$) effettivamente abbiamo la prima implicazione.
\smallskip

Supponiamo ora che $\Gamma$ sia chiuso. Poich\'e $X\times Y$ \`e prodotto di Banach esso stesso \`e banach e quindi $\Gamma$ \`e banach perch\'e chiuso di un Banach. Osserviamo ora che
\[T(x)=P_Y((x,T(x)))=P_Y((P_X\res{\Gamma})\ii(x))\implies T=P_Y\circ(P_X\res{\Gamma})\ii.\]
Poich\'e $P_X\res{\Gamma}$ \`e bigettiva, continua e lineare, per il teorema della mappa aperta (\ref{ThMappaAperta}) essa \`e un omeomorfismo, quindi $T$ \`e continuo in quanto composizione di $P_Y$ e $P_X\res{\Gamma}\ii$ continue.
\end{proof}

\begin{exercise}
Sia $T:X\to Y$ lineare fra Banach. Controntare la continuit\`a di $T$ con le topologie forti e deboli di $X$ e $Y$
\begin{align*}
(X,w)&\to (Y,w)\\
(X,w)&\to (Y,s)\\
(X,s)&\to (Y,w)\\
(X,s)&\to (Y,s)
\end{align*}
\end{exercise}
\begin{proof}
Hint: usare grafico chiuso (\ref{ThGraficoChiuso}) ricordando che sottospazi vettoriali di Banach sono chiusi forti se e solo se sono chiusi deboli e osservando chi \`e la topologia debole di $X\times Y$ (topologia prodotto)

Tre di queste nozioni sono equivalenti e una no. Quella diversa \`e pi\`u forte? Pi\`u debole?
\end{proof}


\subsubsection{Norme confrontabili}

\begin{proposition}[Norme confrontabili su Banach sono equivalenti]\label{PrNormeConfrontabiliSuBanachSonoEquivalenti}
Due norme su Banach confrontabili sullo stesso $\K$-spazio vettoriale sono equivalenti.
\end{proposition}
\begin{proof}
Se le norme sono confrontabili, $id_X$ \`e continua se sul dominio consideriamo la topologia pi\`u fine. Chiaramente $id_X$ \`e lineare, quindi per il teorema della mappa aperta (\ref{ThMappaAperta}) si ha che $id_X$ \`e aperta. Poich\'e $id_X$ \`e bigettiva questo mostra che $id_X$ \`e un omeomorfismo.
\end{proof}

\begin{exercise}
Su uno spazio normato $X$ di dimensione infinita esistono sempre forme lineari non continue. 
\end{exercise}


\begin{remark}
Esistono $L:X\to X$ lineari bigettive non continue
\end{remark}
\begin{proof}
Fisso $f$ forma discontinua e fisso $u\in X$, definiamo
\[L(x)=x+f(x)u\]
e notiamo che
\begin{align*}
L^2(x)=&L(x+f(x)u)=L(x)+f(x)L(u)=\\
=&x+f(x)u +f(x)(u+f(u)u)=\\
=&x+(2f(x)+f(x)f(u))u.
\end{align*}
Se $u$ \`e tale che $f(u)=-2$ allora $L^2=id_X$, cio\`e $L$ involuzione. In particolare $L$ \`e bigettiva ma continua se e solo se $f$ lo \`e, e non lo \`e quindi $L$ non continua su $(X,\normd_1)$ Banach.

Poniamo $\norm x_2=\norm{L(x)}_2$. Notiamo che $\normd_2$ rende $X$ Banach in quanto $L:(X,\normd_1)\to (X,\normd_2)$ \`e una isometria. Notiamo dunque che $\normd_1$ e $\normd_2$ sono norme che rendono $X$ banach e che non sono equivalenti ($L$ \`e discontinua per $\normd_1$ ma continua per $\normd_2$).
\end{proof}

\begin{exercise}
Siano $\normd_1,\ \normd_2$ norme sullo stesso $X$ e sia $\normd_3=\normd_1+\normd_2$. Allora
\begin{enumerate}
    \item Una successione $(x_n)$ converge a $x\in X$ in $\normd_3$ se e solo se converge a $x$ in $\normd_1$ e $\normd_2$.
    \item $(x_n)$ \`e di Cauchy in $X$ se e solo se \`e di Cauchy sia per $\normd_1$ che per $\normd_2$.
\end{enumerate}
\end{exercise}

\begin{exercise}
TROVA L'IMBROGLIO:

\noindent
``\textbf{Proposizione.}" Tutte le norme di Banach sullo stesso $X$ sono equivalenti.
\begin{proof}[``Dimostrazione"]
    Siano $\normd_1$ e $\normd_2$ di Banach. Notiamo che $\normd_3$ \`e pi\`u fine della altre due e che $(x_n)$ \`e di Cauchy per $\normd_3$ se e solo se lo \`e per le altre due, quindi per il punto 1. della proposizione precedente la successione converge in $\normd_3$. Segue dunque che, poich\'e $\normd_3$ \`e pi\`u fine allora \`e confrontabile con le altre due, quindi le tre norme sono equivalenti.
\end{proof}
\end{exercise}



\section{Iniettivit\`a e surgettivit\`a di mappe lineari}

Cerchiamo di capire che relazione c'\`e tra iniettivit\`a e surgettivit\`a delle mappe $T$ e $T^\ast$ per $T:X\to Y$ lineare continua.

\subsection{Forte iniettivit\`a}

\begin{definition}[Forte iniettivit\`a]
Una mappa $T:X\to Y$ lineare continua \`e \textbf{fortemente iniettiva} se esiste $c>0$ tale che 
\[\forall x\in X\qquad \norm{T(x)}\geq c\norm x.\]
\end{definition}

\begin{proposition}
Se $X$ e $Y$ sono banach e $T:X\to Y$ lineare continua, $T$ \`e fortemente iniettiva se e solo se $T$ \`e iniettiva e $\imm T$ \`e chiuso.
\end{proposition}
\begin{proof}
Diamo le implicazioni
\setlength{\leftmargini}{0cm}
\begin{itemize}
\item[$\boxed{\implies}$] Iniettiva ok. Sia $T':X\to \imm T\subseteq Y$ la stessa mappa di $T$ ma con codominio ristretto. Notiamo che $T'$ \`e invertibile perch\'e iniettiva e surgettiva per costruzione e che ha inversa continua per la disuguaglianza in ipotesi, quindi $\imm(T)$ \`e Banach perch\'e $X$ \`e Banach e quindi $\imm T$ \`e chiuso in $Y$.
\item[$\boxed{\impliedby}$] Se $T$ \`e iniettiva con immagine chiusa allora $T':X\to \imm T$ \`e invertibile. Inoltre, poich\'e $\imm T$ \`e Banach perch\'e chiuso di $Y$, si ha che per mappa aperta (\ref{ThMappaAperta}) vale $(T')\ii$ continua, cio\`e $T$ fortemente iniettiva.
\end{itemize}
\setlength{\leftmargini}{0.5cm}
\end{proof}

\begin{proposition}[Retrazioni e sezioni per lineari continue]\label{PrRetrazioniSezioniPerOperatoriLineariContinui}
Sia $T\in L(X,Y)$ con $X,Y$ Banach. Allora $T$ \`e una\footnote{cio\`e esiste $S:Y\to X$ tale che $T$ \`e l'inversa destra / sinistra di $S$.}
\begin{itemize}
    \item inversa destra $\coimplies$ iniettiva e $\imm T$ \`e complementato\footnote{cio\`e esiste $V\subseteq Y$ tale che $Y=\imm T\oplus V$.}
    \item inversa sinistra $\coimplies$ surgettivo e $\ker T$ \`e complementato.
\end{itemize}
\end{proposition}
\begin{proof}
Se $T:X\to Y$ e $S:Y\to X$ sono una coppia tale che $S\circ T=id_X$ allora $T\circ S=P$ \`e un proiettore lineare continuo, infatti
\[P^2=(T\circ S)\circ (T\circ S)=T\circ id_X \circ S=T\circ S.\]
Quindi $Y=\ker P\oplus \imm P$ e $\ker P=\ker S$, $\imm P=\imm T$, ovvero
\[Y=\ker S\oplus \imm T\]
come volevamo.
\medskip

Viceversa, se $T$ \`e iniettivo e $\imm T$ \`e complementata (rispettivamente $S$ \`e surgettivo e $\ker S$ complementato) allora considero un proiettore $P_{\imm T}$ (ok per la decomposizione in somma diretta) e definisco $S=(T')\ii\circ P_{\imm T}$ che \`e inversa sinistra di $T$ (rispettivamente definisco un proiettore $Q$ su $\ker S$ con $id_Y-Q$ proiettore sul supplementare $V$ fissato di $\ker S$, a questo punto considero $S\res V\ii$, che diventa inversa destra).
\end{proof}

\begin{theorem}[Surgettivit\`a e aggiunti]\label{ThSurgettivitaEAggiunti}
Sia $T\in L(X,Y)$ con $X,Y$ banach e tale che $T^\ast$ fortemente iniettivo (iniettivo pi\`u immagine chiusa). Allora $T$ \`e surgettivo.
\end{theorem}
\begin{proof}
Senza perdita di generalit\`a supponiamo $\norm{T^\ast y}\geq \norm y$ (ricordiamo che fortemente iniettivo significa $\norm{T^\ast y}\geq k\norm y$ per qualche $k<1$, ma a meno di riscalare $T$ supponiamo $k=1$). Sia $0<t<1$ e mostriamo che $B_Y\subseteq TB_X+tB_Y$, cos\`i facendo possiamo invocare il lemma di iterazione (\ref{LmDiIterazione}) e mostrare $B_Y\subseteq T(B_X)$, in particolare $T$ \`e surgettivo.

Supponiamo per assurdo che non valga $B_Y\subseteq TB_X+tB_Y$, allora esiste $y_0\in B_Y$ tale che $y_0\notin TB_X+tB_Y$ (convesso aperto). Per il teorema di Hahn-Banach in forma di separazione di aperti convessi (\ref{ThSeparazioneDiConvessi}) esiste $y^\ast_0\in Y^\ast\nz$ tale che $\forall x\in B_X$ e $\forall y\in B_y$ si ha
\[\ps{y^\ast_0,Tx+ty}\leq \ps{y^\ast_0,y_0}\leq \norm{y^\ast_0}\]
Allora
\[\ps{T^\ast y_0^\ast,x}+t\ps{y_0^\ast,y}\leq \norm{y_0^\ast}\]
passando all'estremo superiore per $x\in B_X$ e $y\in B_Y$ troviamo
\[\norm{T^\ast y_0}+t\norm{y^\ast_0}\leq\norm{y_0^\ast}\]
cio\`e $\norm{T^\ast y_0}\leq (1-t)\norm{y^\ast_0}$, contraddicendo l'ipotesi di forte iniettivit\`a ($\norm{T^\ast y}\geq \norm y$ per ogni $y$)
\end{proof}

\begin{remark}
In realt\`a vale anche $T^\ast$ surgettivo se e solo se $T$ fortemente iniettivo.
\end{remark}


\subsection{Polare, prepolare, annullatore, preannullatore}
\begin{definition}[Assolutamente convesso]
Un insieme bilanciato e convesso si dice \textbf{assolutamente convesso}. Per un insieme $S$ ha senso l'\textbf{inviluppo assolutamente convesso}
\begin{align*}
    \assco(S)=&\bigcap_{\smat{C\text{ ass.conv.}\\ C\supseteq S}}C=\cpa{\sum_{i=1}^n\la_i a_i\mid a_1,\cdots, a_n\in S,\ \la_i\in \K,\ \sum_{i=1}^n\abs{\la_i}\leq1}=\\
    =&\ol{B_\K(0,1)}\co(S)
\end{align*}
\end{definition}

\begin{definition}[Polare e prepolare]
Sia $X$ SVT, $A\subseteq X$, $B\subseteq X^\ast$. Definiamo la \textbf{polare di $A$} come
\[A^0=\cpa{x^\ast\in X^\ast\mid \abs{\ps{x^\ast,x}}\leq 1,\ \forall x\in A}=\bigcap_{x\in A}\cpa{x}^0\]
Definiamo $x^0=\cpa{x^\ast\in X^\ast\mid \abs{\ps{x^\ast,x}}\leq 1}\supseteq \ker(\iota_{X}(x))$.

Definiamo il \textbf{prepolare} di $B$ come
\[B_0=\cpa{x\in X\mid \abs{\ps{x^\ast,x}}\leq 1,\forall x^\ast\in B}=\bigcap_{x^\ast\in B}\cpa{x^\ast}_0\]
\end{definition}

\begin{remark}
La polare di un qualche insieme \`e assolutamente convessa e w$^\ast$-chiusa. La prepolare \`e assolutamente convessa e chiusa in $X$ (anche forte).
\end{remark}

\begin{definition}[Annullatore e preannullatore]
Sia $X$ SVT, $A\subseteq X$ e $B\subseteq X^\ast$. Definiamo l'\textbf{annullatore} di $A$ come
\[A^\perp=\cpa{x^\ast\in X^\ast\mid \ps{x^\ast,x}=0,\ \forall x\in A}=\bigcap_{x\in A}\Ann(x)=\bigcap_{x\in A}\cpa{x}^\perp\]
e il \textbf{preannullatore} di $B$ come
\[B_\perp=\cpa{x\in X\mid \forall \ps{x^\ast,x}=0,\ \forall x^\ast\in B}=\bigcap_{x^\ast\in B}\ker x^\ast=\bigcap_{x^\ast\in B}(x^\ast)_\perp.\]
\end{definition}

\begin{remark}
Se $A$ e $B$ sono sottospazi vettoriali o coni in generale allora
\[A^0=A^\perp,\quad B_0=B_\perp.\]
\end{remark}

Da ora in poi supponiamo $(X,\normd)$ normato e sia $i_X:X\inj X^{\ast\ast}$. 

\begin{proposition}[Polare e prepolare in normato]\label{PrPolareEPrepolareInNormato}
Della definizione si ha
\begin{itemize}
    \item $A^0=(i_X(A))_0$
    \item $B_0=i_X\ii(B^0)=B^0\cap X$
\end{itemize}
\end{proposition}
\begin{proof}
Segue dal fatto che $\ps{x^\ast,x}=\ps{i_X(x),x^\ast}$. Per esempio
\begin{align*}
    A^0=&\cpa{x^\ast\in X^\ast\mid \abs{\ps{x^\ast,x}}\leq 1, \forall x\in A}=\\
    =&\cpa{x^\ast\in X^\ast\mid \abs{\ps{i_X(x),x^\ast}}\leq 1, \forall x\in A}=\\
    =&\cpa{x^\ast\in X^\ast\mid \abs{\ps{y,x^\ast}}\leq 1, \forall y\in i_X(A)\subseteq X^{\ast\ast}}=(i_X(A))_0.
\end{align*}
\end{proof}

\begin{remark}
Per le palle unitarie chiuse vale
\[(B_X)^0=B_{X^\ast},\qquad (B_{X^\ast})_0=B_X\]
dove per la seconda uguaglianza usiamo Hahn-Banach per dire $(B_{X^\ast})^0\cap X=B_X$.
\end{remark}

\begin{proposition}\label{PrPolarePrepolareIteratiDannoChiusuraAssolutamenteConvessa}
Siano $A\subseteq X$ e $B\subseteq X^\ast$, allora
\[(A^0)_0=\ol{\assco(A)},\qquad (B_0)^0=\ol{\assco(B)}^{w^\ast}.\]
\end{proposition}
\begin{proof}
Dalla definizione \`e chiaro che $A\subseteq (A^0)_0$ e $B\subseteq (B_0)^0$. Poich\'e $(A^0)_0$ \`e assolutamente convesso e chiuso vale
\[(A_0)^0\supseteq\ol{\assco(A)}\]
e per lo stesso motivo $(B_0)^0\supseteq \ol{\assco(B)}^{w^\ast}$.

Sia $a\notin \ol{\assco(A)}$. Per Hahn-Banach (\ref{ThSeparazioneDiConvessi}) esiste\footnote{$X_\R$ \`e $X$ visto come $\R$-spazio vettoriale.} $f_0\in X_\R^\ast$ tale che $\ps{f_0,a}>\gamma\geq \ps{f_0,x}$ per ogni $x\in \ol{\assco(A)}$. A meno di riscalare $f_0$ supponiamo $\gamma=1$. Allora $\abs{\ps{f_0,x}}\leq 1$ per ogni $x\in \ol{\assco(A)}$. 

Se $\K=\R$ poniamo $f=f_0$, se $\K=\C$ allora poniamo $\ps{f,x}=\ps{f_0,x}-i\ps{f_0,ix}$ e notiamo che
\[\sup_{x\in\ol{\assco(A)}}\abs{\ps{f,x}}=\sup_{x\in\ol{\assco(A)}}\abs{\ps{f_0,x}}\]

Dunque $f\in A^0$, ma $\abs{\ps{f,a}}\geq \ps{f_0,a}>1$, quindi $a\notin (A^0)_0$. Questo mostra l'inclusione $(A^0)_0\subseteq \ol{\assco(A)}$.
\end{proof}

\begin{remark}
$(A^\perp)_\perp=\ol{\Span(A)}$ e $(B_\perp)^\perp=\ol{\Span(B)}^{w^\ast}$, infatti polare e prepolare coincidono con annullatore e preannullatore per coni e chiaramente 
\[(A^\perp)_\perp=(\Span(A)^\perp)_\perp,\qquad (B_\perp)^\perp=(\Span(B)_\perp)^\perp.\]
\end{remark}

\begin{remark}
Se\footnote{stiamo usando il fatto che la chiusura di $\Span_\R Y$ \`e uguale alla chiusura di $\Span_\Q Y$ per topologie meno fini della forte.} $A\subseteq X$ allora $A$ \`e denso se e solo se $A^\perp=(0)$ e $B\subseteq X^\ast$ \`e $w^\ast$-denso se e solo se $B_\perp=(0)$.
\end{remark}

\begin{proposition}[Relazione tra nucleo e immagine tra $T$ e $T^\ast$]\label{PrNucleiImmaginiTrasposteAnnullatoriEPreannullatori}
Se $T\in L(X,Y)$ allora
\begin{itemize}
    \item $\ker T=(\imm T^\ast)_\perp$
    \item $\ker T^\ast=(\imm T)^\perp$
    \item $(\ker T)^\perp =\ol{\imm T^\ast}^{w^\ast}$
    \item $(\ker T^\ast)_\perp=\ol{\imm T}$.
\end{itemize}
\end{proposition}
\begin{proof}
Abbiamo una catena di equivalenze
\begin{gather*}
    x\in \ker T\\
    Tx=0\\
    \ps{y^\ast,Tx}=0\quad \forall y^\ast\in Y^\ast\\
    \ps{T^\ast y^\ast,x}=0\quad \forall y^\ast\in Y^\ast\\
    x\in (\imm T^\ast)_\perp.
\end{gather*}
dove la seconda equivalenza \`e data da Hahn-Banach (\ref{CorHahnBanachPerSpaziNormati}).
Segue che $(\ker T)^\perp=((\imm T^\ast)_\perp)^\perp=\ol{\imm T^\ast}^{w^\ast}$.

L'altro caso si fa allo stesso modo.
\end{proof}

\begin{corollary}[Iniettivit\`a e aggiunti]\label{CorIniettivitaEAggiunti}
Sia $T\in L(X,Y)$, allora
\begin{align*}
    T\text{ iniettivo }\coimplies&\imm T^\ast\text{ \`e w$^{\ast}$-denso in }X^\ast\\
    T^\ast\text{ iniettivo }\coimplies&\imm T\text{ \`e denso in }Y
\end{align*}
\end{corollary}
\begin{proof}
Segue da (\ref{PrNucleiImmaginiTrasposteAnnullatoriEPreannullatori}), dove per\`o per dire che $(\ker T)^\perp=X^\ast\implies \ker T=(0)$ stiamo usando Hahn-Banach (\ref{CorHahnBanachPerSpaziNormati}) (se $\ker T$ contiene un vettore non nullo allora possiamo costruire un elemento di $X^\ast$ che non si annulla su quel vettore, e quindi che non si annulla su $\ker T$).
\end{proof}

\begin{exercise}
Scrivere un criterio per ``essere inverso sinistro lineare" per $T\in L(X,Y)$ deducendolo dal lemma di iterazione.
\end{exercise}


\begin{theorem}[Goldstine]\label{ThGoldstine}
Sia $(X,\normd)$ normato e $B_X=\ol{B_X(0,1)}$, allora
\[\ol{i_X(B_X)}^{\sigma(X^{\ast\ast},X^\ast)}=B_{X^{\ast\ast}}\]
e quindi $\ol{X}^{w^\ast}=X^{\ast\ast}$.
\end{theorem}
\begin{proof}
Calcoliamo (la topologia debole$^\ast$ su $X^{\ast\ast}$ \`e $\sigma(X^{\ast\ast},X^\ast)$):
\begin{align*}
    \ol{i_X(B_X)}^{\sigma(X^{\ast\ast},X^\ast)}=(i_X(B_X)_0)^0\pasgnl={(\ref{PrPolareEPrepolareInNormato})}(B_X^0)^0=(B_{X^\ast})^0=B_{X^{\ast\ast}}
\end{align*}
\end{proof}


\subsection{Caso dei Banach}
\begin{proposition}[Duale di sottospazi e di un quoziente]\label{PrDualeDiSottospaziEDualeQuoziente}
Dato $Y$ sottospazio chiuso di $X$ Banach abbiamo le seguenti isometrie lineari:
\begin{enumerate}
    \item $Y^\ast\cong X^\ast/Y^\perp$
    \item $(X/Y)^\ast\cong Y^\perp\subseteq X^\ast$
\end{enumerate}
\end{proposition}
\begin{proof}
Data l'inclusione $j_Y:Y\to X$ otteniamo $j_Y^\ast:X^\ast\to Y^\ast$. Il nucleo di $j_Y^\ast$ sono i funzionali in $X^\ast$ che si restringono al funzionale nullo su $Y^\ast$, cio\`e gli $f\in X^\ast$ tali che \[j_Y^\ast(f)=f\circ j_Y=f\res Y=0\] e quindi $\ker j_Y^\ast=Y^\perp$. Per il teorema di Hahn-Banach (\ref{ThHahnBanach}), $j_Y^\ast$ \`e surgettiva in quanto ogni funzionale su $Y$ si estende ad uno su $X$ perch\'e $X$ Banach e $Y$ chiuso. Per il teorema di isomorfismo esiste un'unica mappa $\phi$ che fa commutare
% https://q.uiver.app/#q=WzAsMyxbMCwwLCJYXlxcYXN0Il0sWzAsMSwiWF5cXGFzdC9cXEFubihZKSJdLFsxLDAsIlleXFxhc3QiXSxbMCwyLCJqX1leXFxhc3QiXSxbMCwxLCJcXHBpIiwyXSxbMSwyLCJcXHBoaSIsMix7InN0eWxlIjp7ImJvZHkiOnsibmFtZSI6ImRhc2hlZCJ9fX1dXQ==
\[\begin{tikzcd}
	{X^\ast} & {Y^\ast} \\
	{X^\ast/Y^\perp}
	\arrow["{j_Y^\ast}", from=1-1, to=1-2]
	\arrow["\pi"', from=1-1, to=2-1]
	\arrow["\phi"', dashed, from=2-1, to=1-2]
\end{tikzcd}\]
Per questioni di algebra $\phi$ \`e lineare e poich\'e $j_Y^\ast$ \`e continua e $\pi$ induce la topologia quoziente, $\phi$ \`e continua. Verifichiamo che \`e una isometria.
\begin{gather*}
    B_{X^\ast/Y^\perp}(0,1)=\pi(B_{X^\ast}(0,1))\\
    \phi(B_{X^\ast/Y^\perp}(0,1))=\phi(\pi(B_{X^\ast}(0,1)))=j_Y^\ast(B_{X^\ast}(0,1))\pasgnl={(\ref{ThHahnBanach})}B_{Y^\ast}(0,1).
\end{gather*}
dove nell'ultimo passaggio abbiamo usato il fatto che l'estensione data da Hahn-Banach mantiene la norma.

\bigskip
\noindent
Data la proiezione $\pi:X\to X/Y$ otteniamo $\pi^\ast:(X/Y)^\ast\to X^\ast$.
Sia $\vp\in(X/Y)^\ast$ e $f=\pi^\ast(\vp)=\vp\circ \pi$. Si ha che
\[\vp(B_{X/Y})=\vp(\pi(B_X))=f(B_X),\]
quindi $\norm\vp_{(X/Y)^\ast}=\norm f_{X^\ast}$, cio\`e $\pi^\ast$ \`e una immersione isometrica.

Sia $f\in X^\ast$, si ha che $f\in Y^\perp$ se e solo se $Y\subseteq \ker f$ che succede se e solo se $f$ si fattorizza tramite $\pi$ per propriet\`a universale. Quindi $f\in Y^\perp$ se e solo se $f=\vp\circ \pi=\pi^\ast(\vp)$ per qualche $\vp:X/Y\to \R$, cio\`e se e solo se $f\in \pi^\ast((X/Y)^\ast)$. Quindi $\imm \pi^\ast=Y^\perp$.

Restringendo il codominio all'immagine troviamo quanto voluto.
\end{proof}


\begin{proposition}[Banach riflessivi]\label{PrBanachERiflessivoSSEIlDualeERiflessivo}
Sia $X$ banach\footnote{banach serve perch\'e $X^\ast$ \`e isometrico a $\wh X^\ast$ dove $\wh X$ \`e il completamento di $X$.}, allora $X$ \`e riflessivo se e solo se $X^\ast$ \`e riflessivo.
\end{proposition}
\begin{proof}
Ricordiamo che
% https://q.uiver.app/#q=WzAsMyxbMCwxLCJYXlxcYXN0Il0sWzEsMCwiWF57XFxhc3RcXGFzdFxcYXN0fSJdLFsyLDEsIlheXFxhc3QiXSxbMCwxLCJpX3tYXlxcYXN0fSJdLFsxLDIsIihpX1gpXlxcYXN0Il0sWzAsMiwiaWRfe1heXFxhc3R9IiwyXV0=
\[\begin{tikzcd}
	& {X^{\ast\ast\ast}} \\
	{X^\ast} && {X^\ast}
	\arrow["{(i_X)^\ast}", from=1-2, to=2-3]
	\arrow["{i_{X^\ast}}", from=2-1, to=1-2]
	\arrow["{id_{X^\ast}}"', from=2-1, to=2-3]
\end{tikzcd}\]
Se $X$ \`e riflessivo, cio\`e $i_X$ \`e isomorfismo, allora $(i_X)^\ast$ \`e un isomorfismo per funtorialit\`a. Dal diagramma allora segue che $i_{X^\ast}$ \`e un isomorfismo, infatti
\[(i_X)^\ast\circ i_{X^\ast}=id_{X^\ast}\implies i_{X^\ast}=((i_X)^\ast)\ii\circ (i_X)^\ast\circ i_{X^\ast} = ((i_X)^\ast)\ii.\]
Quindi $i_{X^\ast}$ \`e un isomorfismo, cio\`e $X^\ast$ \`e riflessivo.


Viceversa, se $i_{X^\ast}$ \`e un isomorfismo allora $(i_X)^\ast$ \`e un isomorfismo per motivi analoghi a prima, quindi $i_X$ ha immagine densa (iniettivit\`a di $(i_X)^\ast$ e (\ref{PrNucleiImmaginiTrasposteAnnullatoriEPreannullatori})), ma l'immagine di $i_X$ \`e sempre chiusa, quindi $X$ \`e riflessivo (iniettivit\`a di $i_X$ vale sempre perch\'e immersione isometrica). 
\end{proof}

\begin{remark}
Se $X$ non \`e riflessivo allora nessun duale successivo pu\`o essere riflessivo.
\end{remark}



\begin{theorem}[Immagine chiusa]\label{ThImmagineChiusa}
Siano $X,Y$ banach, $T\in L(X,Y)$, allora sono equivalenti
\begin{enumerate}
    \item $\imm T$ \`e $\normd$-chiuso
    \item $\imm T$ \`e $w$-chiuso
    \item $\imm T^\ast=(\ker T^\ast)_\perp$
    \item $\imm T^\ast$ \`e $\normd$-chiuso
    \item $\imm T^\ast$ \`e $w^\ast$-chiuso
    \item $\imm T^\ast=(\ker T)^\perp$
\end{enumerate}
\end{theorem}
\begin{proof}
1. e 2. sono sempre equivalenti per sottospazi vettoriali.
\[\ol{\imm T}\pasgnl={(\ref{PrNucleiImmaginiTrasposteAnnullatoriEPreannullatori})}((\ker T^\ast)_\perp)\]
quindi 2. \`e equivalente a 3. Similmente 5. e 6. sono equivalenti per $\ol{\imm T^\ast}^{w^\ast}=(\ker T)^\perp$. Poich\'e la topologia debole$^\ast$ \`e meno fine della topologia forte, 5. implica 4.

Resta da mostrare solo 4.$\implies$1. e 1.$\implies$6.
\setlength{\leftmargini}{0cm}
\begin{itemize}
\item[$\boxed{4.\implies1.}$] Supponiamo $\imm T^\ast$ chiuso forte. Siano $Z=\ol{\imm T}$ e $S:X\to Z$ la mappa ottenuta da $T$ restringendo il codominio, che possiamo fare perch\'e $\imm T\subseteq Z$.

Per costruzione $\imm S$ \`e densa in $Z$ e la tesi \`e $S$ surgettiva. Dualizzando la successione
% https://q.uiver.app/#q=WzAsMyxbMCwxLCJYIl0sWzEsMCwiWiJdLFsyLDEsIlkiXSxbMCwxLCJTIl0sWzEsMiwiIiwwLHsic3R5bGUiOnsidGFpbCI6eyJuYW1lIjoiaG9vayIsInNpZGUiOiJ0b3AifX19XSxbMCwyLCJUIiwyXV0=
\[\begin{tikzcd}
	& Z \\
	X && Y
	\arrow[hook, from=1-2, to=2-3]
	\arrow["S", from=2-1, to=1-2]
	\arrow["T"', from=2-1, to=2-3]
\end{tikzcd}\]
troviamo
% https://q.uiver.app/#q=WzAsMyxbMiwxLCJYXlxcYXN0Il0sWzEsMCwiWl5cXGFzdCJdLFswLDEsIlleXFxhc3QiXSxbMiwwLCJUXlxcYXN0IiwyXSxbMSwwLCJTXlxcYXN0Il0sWzIsMV1d
\[\begin{tikzcd}
	& {Z^\ast} \\
	{Y^\ast} && {X^\ast}
	\arrow["{S^\ast}", from=1-2, to=2-3]
	\arrow[from=2-1, to=1-2]
	\arrow["{T^\ast}"', from=2-1, to=2-3]
\end{tikzcd}\]
dove la mappa $Y^\ast\to Z^\ast$ \`e la restrizione del dominio, che \`e surgettiva per il teorema di Hahn-Banach (\ref{CorHahnBanachPerSpaziNormati}), dunque $S^\ast(Z)=T^\ast(Y)$. Notiamo che $T^\ast(Y)$ \`e chiuso in norma, quindi anche $S^\ast(Z)$ lo \`e. Poich\'e $\imm S$ \`e densa, $S^\ast$ \`e iniettiva, quindi per la caratterizzazione (\ref{ThSurgettivitaEAggiunti}) $S^\ast$ \`e fortemente iniettivo e perci\`o $S$ \`e surgettivo.
\item[$\boxed{1.\implies6.}$] \`E sempre vero che $\imm T^\ast\subseteq (\ker T)^\perp$ in quanto $(\ker T)^\perp$ \`e la chiusura di $\imm T$ per la topologia debole$^\ast$ (\ref{PrNucleiImmaginiTrasposteAnnullatoriEPreannullatori}).

Sia $x^\ast\in (\ker T)^\perp$, cio\`e $\ker T\subseteq \ker x^\ast$. Consideriamo la mappa lineare (a priori non continua)
\[\xi:\funcDef{\imm T}{\K}{T(x)}{x^\ast(x)}\]
che \`e ben definita perch\'e se $T(x)=T(x')$ allora $x-x'\in \ker T\subseteq \ker x^\ast$. Notiamo che $x^\ast=\xi\circ T$.

Poich\'e $\imm T$ \`e chiuso (stiamo assumendo 1.) esso \`e banach, quindi si ha che $T$ \`e aperta come mappa $X\to \imm T$ per il teorema della mappa aperta (\ref{ThMappaAperta}), quindi induce la topologia quoziente, il che significa che $\xi$ era un funzionale lineare CONTINUO.

Per il teorema di Hahn-Banach (\ref{CorHahnBanachPerSpaziNormati}) $\xi$ si estende a $y^\ast\in Y^\ast$ e poich\'e $x^\ast=\xi\circ T$ si ha $x^\ast=y^\ast\circ T=T^\ast(y^\ast)$, cio\`e $x^\ast\in \imm T^\ast$.
\end{itemize}
\setlength{\leftmargini}{0.5cm}
\end{proof}

Abbiamo dunque
\[T\text{surg.}\coimplies \begin{cases}
    \imm T&\text{chiusa}\\
    \imm T&\text{densa}
\end{cases} \coimplies \begin{cases}
    \imm T^\ast&\text{chiusa}\\
    T^\ast&\text{iniettiva}
\end{cases}\coimplies T^\ast\text{ fortemente iniettiva}\]


\begin{exercise}
Per $X,Y$ banach, i seguenti sottoinsiemi di $L(X,Y)$ sono aperti
\begin{itemize}
    \item Surgettive
    \item Inverse sinistre
    \item Inverse destre
    \item Fortemente iniettive
    \item Invertibili
\end{itemize}
Per $T$ che appartiene ad uno di questi trovare $r>0$ tale che $B(T,r)$ sia contenuto nell'aperto.
\end{exercise}
\begin{solution}
Esempio, per $T$ invertibile posso prendere $B(T,1/\norm{T\ii})$.
\end{solution}

\begin{proposition}[Duale \`e endofuntore su Banach]\label{PrDualeEEndofuntoreSuBanachEsatto}
La corrispondenza
\[\functorDef{\Ban\op}{\Ban}{X}{X^\ast}{T:X\to Y}{T^\ast:Y^\ast\to X^\ast}\]
\`e un endofuntore controvariante esatto\footnote{Ricordiamo che un funtore \`e esatto se per ogni successione esatta corta $0\to X\overset{\al}\to Y\overset{\beta}\to Z\to 0$ (cio\`e $\al$ iniettiva, $\beta$ surgettiva e $\imm \al=\ker \beta$) allora $0\ot X^\ast\overset{\alpha^\ast}\ot Y^\ast\overset{\beta^\ast}\ot Z^\ast\ot 0$ \`e ancora esatta.}.
\end{proposition}
\begin{proof}
Se $\ker \al=(0)$ e $\imm \al=\ker\beta$ \`e chiusa (cio\`e $\al$ \`e fortemente iniettiva) allora $\al^\ast$ ha immagine chiusa e $\imm \al^\ast=(\ker \al)^\perp=X^\ast$, quindi $\al^\ast$ \`e surgettiva.

Se $\imm \al=\ker \beta$ allora
\[\ker \al^\ast=(\imm\al)^\perp=(\ker\beta)^\perp=\ol{\imm \beta^\ast}\pasgnl={$\imm \beta$ chiusa}\imm \beta^\ast\]
Infine $\beta$ surgettiva implica $\beta^\ast$ iniettiva (\ref{CorIniettivitaEAggiunti}).
\end{proof}

\begin{corollary}\label{CorBidualeEndofuntoreEsattoConTrasformazioneNaturaleDaIdentita}
Il funtore biduale
\[\functorDef{\Ban}{\Ban}{X}{X^{\ast\ast}}{T:X\to Y}{T^{\ast\ast}:X^{\ast\ast}\to Y^{\ast\ast}}\]
\`e un funtore covariante esatto. L'inclusione $i_X:X\to X^{\ast\ast}$ induce una trasformazione naturale tra il funtore $id_{\Ban}$ e $\cdot^{\ast\ast}$.
\end{corollary}
\begin{proof}
% https://q.uiver.app/#q=WzAsOSxbMSwxLCJYIl0sWzIsMSwiWF57XFxhc3RcXGFzdH0iXSxbMSwyLCJZIl0sWzIsMiwiWV57XFxhc3RcXGFzdH0iXSxbMCwwLCJ4Il0sWzAsMywiVCh4KSJdLFszLDAsInZhbF94Il0sWzIsMywidmFsX3tUKHgpfSJdLFszLDIsIlRee1xcYXN0XFxhc3R9KHZhbF94KSJdLFswLDIsIlQiLDJdLFsxLDMsIlRee1xcYXN0XFxhc3R9Il0sWzAsMSwiaV9YIl0sWzIsMywiaV9ZIiwyXSxbNCw1LCIiLDIseyJzdHlsZSI6eyJ0YWlsIjp7Im5hbWUiOiJtYXBzIHRvIn19fV0sWzQsNiwiIiwwLHsic3R5bGUiOnsidGFpbCI6eyJuYW1lIjoibWFwcyB0byJ9fX1dLFs1LDcsIiIsMix7InN0eWxlIjp7InRhaWwiOnsibmFtZSI6Im1hcHMgdG8ifX19XSxbNiw4LCIiLDAseyJzdHlsZSI6eyJ0YWlsIjp7Im5hbWUiOiJtYXBzIHRvIn19fV0sWzcsOCwiPSIsMyx7InN0eWxlIjp7ImJvZHkiOnsibmFtZSI6Im5vbmUifSwiaGVhZCI6eyJuYW1lIjoibm9uZSJ9fX1dLFs1LDIsIlxcaW4iLDMseyJzdHlsZSI6eyJib2R5Ijp7Im5hbWUiOiJub25lIn0sImhlYWQiOnsibmFtZSI6Im5vbmUifX19XSxbNywzLCJcXGluIiwzLHsic3R5bGUiOnsiYm9keSI6eyJuYW1lIjoibm9uZSJ9LCJoZWFkIjp7Im5hbWUiOiJub25lIn19fV0sWzgsMywiXFxpbiIsMyx7InN0eWxlIjp7ImJvZHkiOnsibmFtZSI6Im5vbmUifSwiaGVhZCI6eyJuYW1lIjoibm9uZSJ9fX1dLFs2LDEsIlxcaW4iLDMseyJzdHlsZSI6eyJib2R5Ijp7Im5hbWUiOiJub25lIn0sImhlYWQiOnsibmFtZSI6Im5vbmUifX19XSxbNCwwLCJcXGluIiwzLHsic3R5bGUiOnsiYm9keSI6eyJuYW1lIjoibm9uZSJ9LCJoZWFkIjp7Im5hbWUiOiJub25lIn19fV1d
\[\begin{tikzcd}
	x &&& {val_x} \\
	& X & {X^{\ast\ast}} \\
	& Y & {Y^{\ast\ast}} & {T^{\ast\ast}(val_x)} \\
	{T(x)} && {val_{T(x)}}
	\arrow[maps to, from=1-1, to=1-4]
	\arrow["\in"{marking, allow upside down}, draw=none, from=1-1, to=2-2]
	\arrow[maps to, from=1-1, to=4-1]
	\arrow["\in"{marking, allow upside down}, draw=none, from=1-4, to=2-3]
	\arrow[maps to, from=1-4, to=3-4]
	\arrow["{i_X}", from=2-2, to=2-3]
	\arrow["T"', from=2-2, to=3-2]
	\arrow["{T^{\ast\ast}}", from=2-3, to=3-3]
	\arrow["{i_Y}"', from=3-2, to=3-3]
	\arrow["\in"{marking, allow upside down}, draw=none, from=3-4, to=3-3]
	\arrow["\in"{marking, allow upside down}, draw=none, from=4-1, to=3-2]
	\arrow[maps to, from=4-1, to=4-3]
	\arrow["\in"{marking, allow upside down}, draw=none, from=4-3, to=3-3]
	\arrow["{=}"{marking, allow upside down}, draw=none, from=4-3, to=3-4]
\end{tikzcd}\]
dove $T^{\ast\ast}(val_x)(f)$ per $f:Y\to \K$ lineare continua \`e data da
\[(T^{\ast\ast}(val_x)(f))=((val_x\circ T^\ast)(f))=(T^\ast(f))(x)=f(T(x))=val_{T(x)}(f).\]
\end{proof}

\begin{remark}
Se $j:Y\to X$ \`e inclusione di sottospazio chiuso (in generale per $j$ fortemente iniettiva) allora $j^{\ast\ast}$ \`e fortemente iniettiva, infatti il biduale \`e un funtore esatto e
\[\imm j^{\ast\ast}=(\ker j^\ast)^\perp=(Y^\perp)^\perp=((i_X(Y))_\perp)^\perp=\ol{i_X(Y)}^{\sigma(X^{\ast\ast},X^\ast)}\]
cio\`e $Y^{\ast\ast}$ \`e la chiusura $w^\ast$ di $Y$ visto in $X^{\ast\ast}$.
\end{remark}

\begin{remark}
Se $Y\subseteq X$ chiuso e $X$ \`e riflessivo allora anche $Y$ \`e riflessivo. Infatti se $i_X$ \`e surgettivo allora $(X,\sigma(X,X^\ast))\cong (X^{\ast\ast},\sigma(X^{\ast\ast},X^\ast))$, quindi $i_X(Y)$ deve essere $w^\ast$-chiusa in $X^{\ast\ast}$ (perch\'e era $w$-chiuso in $X$). Perci\`o $i_X(Y)=\imm j^{\ast\ast}$ e quindi $i_Y:Y\to Y^{\ast\ast}$ \`e surgettiva.
% https://q.uiver.app/#q=WzAsNCxbMCwwLCJZIl0sWzEsMCwiWCJdLFsxLDEsIlhee1xcYXN0XFxhc3R9Il0sWzAsMSwiWV57XFxhc3RcXGFzdH0iXSxbMCwzLCJpX1kiLDJdLFsxLDIsImlfWCJdLFszLDIsImpee1xcYXN0XFxhc3R9IiwyXSxbMCwxLCJqIl1d
\[\begin{tikzcd}
	Y & X \\
	{Y^{\ast\ast}} & {X^{\ast\ast}}
	\arrow["j", from=1-1, to=1-2]
	\arrow["{i_Y}"', from=1-1, to=2-1]
	\arrow["{i_X}", from=1-2, to=2-2]
	\arrow["{j^{\ast\ast}}"', from=2-1, to=2-2]
\end{tikzcd}\] 
\end{remark}

\begin{proposition}[Criterio riflessivo con sottospazio chiuso]\label{PrCriterioRiflessivoConSottospazioChiuso}
Se $Y\subseteq X$ \`e chiuso. $X$ \`e riflessivo se e solo se $Y$ e $X/Y$ sono riflessivi.
\end{proposition}
\begin{proof}
Consideriamo il diagramma
% https://q.uiver.app/#q=WzAsMTAsWzEsMCwiWSJdLFsyLDAsIlgiXSxbMiwxLCJYXntcXGFzdFxcYXN0fSJdLFsxLDEsIllee1xcYXN0XFxhc3R9Il0sWzMsMSwiKFgvWSlee1xcYXN0XFxhc3R9Il0sWzMsMCwiWC9ZIl0sWzQsMCwiMCJdLFs0LDEsIjAiXSxbMCwxLCIwIl0sWzAsMCwiMCJdLFswLDMsImlfWSIsMl0sWzEsMiwiaV9YIl0sWzMsMiwial57XFxhc3RcXGFzdH0iLDJdLFswLDEsImoiXSxbOSwwXSxbMSw1XSxbNSw2XSxbMiw0XSxbNCw3XSxbNSw0LCJpX3tYL1l9Il0sWzgsM11d
\[\begin{tikzcd}
	0 & Y & X & {X/Y} & 0 \\
	0 & {Y^{\ast\ast}} & {X^{\ast\ast}} & {(X/Y)^{\ast\ast}} & 0
	\arrow[from=1-1, to=1-2]
	\arrow["j", from=1-2, to=1-3]
	\arrow["{i_Y}"', from=1-2, to=2-2]
	\arrow[from=1-3, to=1-4]
	\arrow["{i_X}", from=1-3, to=2-3]
	\arrow[from=1-4, to=1-5]
	\arrow["{i_{X/Y}}", from=1-4, to=2-4]
	\arrow[from=2-1, to=2-2]
	\arrow["{j^{\ast\ast}}"', from=2-2, to=2-3]
	\arrow[from=2-3, to=2-4]
	\arrow[from=2-4, to=2-5]
\end{tikzcd}\]
Se $X$ \`e riflessivo abbiamo gi\`a detto che anche $Y$ lo \`e. Se $Y$ e $X/Y$ sono riflessivi, due frecce verticali su tre sono isomorfismi, quindi anche la terza lo \`e per il lemma dei 5.

Per un motivo analogo se $X$ \`e riflessivo allora anche $Y$ lo \`e e quindi di nuovo per il lemma dei 5 anche $X/Y$ riflessivo.
\end{proof}