\chapter{Completezza e duali di qualche spazio}

\section{Elenco di spazi completi}

\begin{proposition}
Sia $S$ insieme e $E$ Banach, allora lo spazio normato $(\Bs(S,E),\normd_{\infty,S})$ \`e completo
\end{proposition}
\begin{proof}
~[PERSO, RIGUARDA POI]

tale che $\norm{f(s)}=\norm{\sum_kf_k(s)}\leq \sum_k\norm{f_k(s)}\leq \sum \norm f_{\infty,S}$

quindi $\norm f_{\infty,S}$
\end{proof}

\filosofia{Uno degli strumenti dell'analista: aggiungere e togliere, cio\`e
\[\pi\rho o\sigma\tau\al\vp\al\acute{\iota}\rho\e\sigma\iota\varsigma\]}

\begin{lemma}
Se $(f_k)_{k\in\N}\subseteq \Bs(S,E)$ con $f_k$ continua in $s_0$ per ogni $k$ e $f_k\to f$ uniformemente allora anche $f$ \`e continua in $s_0$
\end{lemma}
\begin{proof}
Consideriamo
\begin{align*}
\norm{f(s)-f(s_0)}\leq&\norm{f(s)-f_k(s)}+\norm{f_k(s)-f_k(s_0)}+\norm{f_k(s_0)-f(s_0)}\leq\\
\leq&2\norm{f-f_k}_{\infty,S}+\norm{f_k(s)-f_k(s_0)}
\end{align*}
Per la convergenza uniforme di $f_k\to f$ si ha che per ogni $\e>0$ esiste $n\in\N$ tale che $\norm{f-f_n}_{\infty,S}\leq \e/3$.

Per la continuit\`a in $s_0$ di $f_n$ esiste un intorno $U$ di $s_0$ rale che $\norm{f_n(s)-f_n(s_0)}\leq \e/3$ per ogni $s\in U$. Allora per ogni $s\in U$ si ha
\[\norm{f(s)-f(s_0)}\leq2\e/3+\e/3=\e.\]
\end{proof}

\begin{proposition}
Sia $S$ spazio topologico, $E$ banach, allora $\Bs C(S,E)$ \`e completo.
\end{proposition}
\begin{proof}
Basta mostrare che $\Bs C(S,E)$ \`e chiuso in $\Bs(S,E)$. Questo segue dal fatto che la continuit\`a in un punto $s_0\in S$ si conserva per convergenza uniforme, che \`e il lemma precedente.
\end{proof}

\begin{example}
Sia $S=\N\cup\cpa{\infty}$ la compattificazione di Alexandrov di $\N$ e $E$ un banach, allora
\[c(E)\doteqdot\cpa{x:\N\to E,\text{ convergente}}\cong \Bs C(S,E)\]
Questo mostra che $c(E)$ \`e chiuso (e quindi completo) in $\ell_\infty(E)=\Bs(\N,E)$.
\end{example}


Conseguenze:
\begin{proposition}
Lo spazio $(L(X,Y),\normd)$ \`e completo
\end{proposition}
\begin{proof}
Considerando l'inclusione isometrica 
\[R:\funcDef{L(X,Y)}{\Bs(B_X(0,1),Y)}{T}{T\res{B_X(0,1)}}\]
basta vedere che $R(L(X,Y))$ \`e chiuso.

Se $(T_n)_{n\in\N}\subseteq L(X,Y)$ \`e tale che $R(T_n)\to f$ uniformemente in $\Bs(B_X(0,1),Y)$ allora mostriamo che $f$ \`e la restrizione a $B_X(0,1)$ di una qualche lineare $T$.

Mostriamo che le $T_n$ convergono puntualmente per ogni $x\in X$: se $x=0$ ok, se $x\neq 0$
\[T_n(x)=\norm x T_n(x/\norm x)=\norm x R(T_n)(x/\norm x)\to \norm x f(x/\norm x)\]

Sia $T:X\to Y$ definita da $T(x)=\norm x f(x/\norm x)$

[MOSTRARE CHE LA CONVERGENZA \`E UNIFORME, ME LO SONO PERSO]
\end{proof}

\begin{corollary}[Duale di spazio normato \`e banach]\label{CorDualeNormatoEBanach}
Il duale di uno spazio normato \`e sempre banch.
\end{corollary}

\begin{theorem}[Integrazione per serie]\label{ThIntegrazionePerSerie}
    Sia $(X,\Qs,\mu)$ \`e uno spazio di misura e sia $(f_k)_{k\in\N}\subseteq \Lc^1(X,\Qs,\mu)$ tali che
    \[\sum_{k\in \N}\norm{f_k}_1<\infty\]
Allora $\sum_{k\in\N}f_k$ converge q.o. e in norma 1.
\end{theorem}
\begin{proof}
Per ogni $n\in\N$ sia $g_n:X\to \R$ data da
\[g_n(x)=\sum_{k=0}^n\abs{f_k(x)}.\]
Notiamo che $(g_n)$ \`e una successione di funzioni misurabili non negative crescente. Inoltre $g_n\to \sum_{k\in\N}\abs{f_k(x)}$ per definizione di serie.

Per convergenza monotona
\[\sum_{k\in\N}\norm f_1\leftarrow\sum_{k=0}^n\norm{f_k}_1=\inf_X g_nd\mu\to \int_X gd\mu\]
cio\`e $\inf_X gd\mu=\sum{k\in\N}\norm f_1<\infty$, cio\`e $g\in\Lc^1$.

Inoltre $s_n=\sum_{k=0}^nf_k$ \`e una successione dominata da $g$:
\[\abs{s_n(x)}\leq \sum_{k=0}^n\abs{f_k(x)}\leq g(x).\]
Quindi la serie $\sum f_k(x)$ \`e una serie assolutamente convergente per ogni $x$ dove $g<\infty$. Poich\'e $\int g<\infty$ le eccezioni sono trascurabili, quindi quasi ovunque $\sum f_k(x)$ \`e assolutamente convergente.

Sia $f(x)=\sum f_k(x)$ dove la serie converge. Notiamo che
\[\abs{f(x)}\leq \sum_{k\in\N}\abs{f_k(x)}=g(x),\]
quindi $\norm f_1\leq \int gd\mu=\sum_{k\in\N}\norm{f_k}_1$.

Applicando come prima la stima alle code
\[\norm{f-s_n}_1=\norm{\sum_{k=n+1}^\infty f_k}_1\leq \sum_{k>n}\norm{f_k}_1=o(1)\]
dove l'ultimo termine va a 0 perch\'e $\sum\norm {f_k}_1$ \`e convergente.
\end{proof}

\begin{corollary}[Weil]\label{CorTeoremaWeil}
Siano $f_n\in \Lc^1(X,\Qs,\mu)$ convergenti in $\normd_1$. Allora esiste $n_k$ successione strettamente crescente di indici tali che $f_{n_k}$ converge quasi ovunque ed \`e dominata in $\Lc^1$.
\end{corollary}
\begin{proof}
Sia $f$ il limite in $\normd_1$. Data questa convergenza consideriamo una sottosuccessione $n_k$ tale che $\norm{f-f_{n_k}}_1<2^{-k}$. Scrivendo la successione in termini di una somma telescopica
\[f_{n_k}=f_{n_0}+\sum_{j=1}^k(f_{n_j}-f_{n_{j-1}})\]
si ha per il teorema di integrazione per serie\footnote{$\norm{f_{n_0}}_1+\sum_{j=1}^\infty \norm{f_{n_j}-f_{n_{j-1}}}_1\leq \norm{f_{n_0}}_1+\sum_{j=1}^\infty \norm{f_{n_j}-f}_1+ \sum_{j=1}^\infty \norm{f_{n_{j-1}}-f}_1<\infty$} (\ref{ThIntegrazionePerSerie}) $f_{n_k}$ converge quasi ovunque e in $Lc^1$, inoltre \`e dominata da
\[g(x)=\abs{f_{n_0}(x)}+\sum_{j=0}^\infty\abs{f_{n_j}-f_{n_{j-1}}}\geq \abs{f_{n_k}(x)}\]
con $g(x)\in \Lc^1$.
\end{proof}

\begin{proposition}[$L^1$ \`e completo]\label{PrL1Completo}
Se $(X,\Qs,\mu)$ \`e uno spazio di misura, $L^1(X,\Qs,\mu)$ \`e completo.
\end{proposition}
\begin{proof}
Segue immediatamente dal teorema di integrazione per serie (\ref{ThIntegrazionePerSerie}).
\end{proof}

\begin{remark}
La convergenza quasi ovunque di funzioni $\Lc^1(\R,dx)$ \`e \textbf{NON} \`e la convergenza rispetto a una topologia opportuna su $\Lc^1(\R,dx)$.

Ogni convergenza topologica in $X$ insieme ha la seguente propriet\`a \textbf{di Urisohn}: $x_n\to x$ rispetto alla topologia se e solo se per ogni sottosuccessione $x_{n_k}$ esiste una sotto-sottosuccessione $x_{n_{k_j}}\to x$.
\begin{proof}
Se $x_n\to x$ converge ok. Se non converge allora esiste un intorno $U$ di $x$ tale che $x_n\notin U$ frequentemente, quindi troviamo una sottosuccessione $x_{n_k}$ che sta sempre fuori da $U$, quindi nessuna sua sotto-sottosuccessione pu\`o convergere a $x$.
\end{proof} 

La convergenza q.o. per successioni in $\Lc^1(\R)$ non ha la propriet\`a di Urisohn.
\end{remark}

\begin{definition}[Operatore di composizione]
Se $E$ \`e uno spazio di funzioni con codominio $\R$ e $f:\R\to\R$, definiamo l'operatore di composizione per $f$ come $E\ni u\mapsto f\circ u$.
\end{definition}

\begin{lemma}
Sia $u_k$ una successione che converge a $u$ in $\normd_p$. A meno di sottosuccessione $u_k\to u$ quasi ovunque e dominata in $\Lc^p$.
\end{lemma}
\begin{proof}
Teorema di Weil (\ref{CorTeoremaWeil}) in $\Lc^p$.
\end{proof}

\begin{proposition}
Lo spazio $L^p(X,\Qs,\mu)$ per $0\leq p<\infty$ \`e completo.
\end{proposition}
\begin{proof}
$L^p$ ed $L^1$ NON sono isomorfi come spazi di Banach in generale\footnote{cursiosit\`a non banale da vedere}, ma esiste un omeomorfismo localmente Lipschitz e questo basta a mostrare la completezza: se $u_k$ \`e una successione di Cauchy in $L^p$, se $\Phi$ \`e Lipschitz allora $\Phi(u_k)$ \`e ancora di Cauchy in $L^1$ e quindi converge, poi torno indietro con $\Phi\ii$, che mantiene il limite per continuit\`a.


Consideriamo
\[\Phi:\funcDef{\Lc^p}{\Lc^1}{u}{\abs{u}^p\sgn(u)}\]
Chiaramente \`e invertibile mandando $v\in L^1$ in $\abs{v}^{1/p}\sgn v$. La mappa $\Phi$ \`e l'operatore di composizione con la funzione $f(t)=\abs{t}^p\sgn t$. La continuit\`a degli operatori di composizione \`e un fatto generale. Se $u_k\to u$ converge in $\normd_p$ allora per il lemma a meno di sottosuccessione converge q.o. e dominata, quindi componendo con $f$ abbiamo ancora convergenza quasi ovunque per continuit\`a ($f(u_k)\to f(u)$ q.o.). Se $\abs{u_k}\leq g$ in $\Lc^p$ allora $\abs{u_k}^p\leq g^p$ in $\Lc^1$, similmente per $\Phi\ii$, quindi effettivamente $\Phi$ \`e un omeomorfismo.


Mostriamo ora che $\Phi$ \`e localmente lipschitz: siano $u,v\in \Lc^p(X)$
\[\abs{\Phi(u)-\Phi(v)}_1=\int_X\abs{f(u(x))-f(v(x))}d\mu(x)\]
ma se $t<s$ allora $\abs{f(t)-f(s)}\leq \sup_{t\leq \xi\leq s}\abs{f'(\xi)}\abs{t-s}$ e $\abs{f'(xi)}=p\abs{xi}^{p-1}\leq p(\max\cpa{\abs t,\abs s})^p$, quindi
\begin{align*}
    \abs{\Phi(u)-\Phi(v)}_1\leq& p\int_X\max\cpa{\abs{u(x)}^{p-1},\abs{v(x)}^{p-1}}\abs{u(x)-v(x)}d\mu\leq\\
    \leq &p\int_X\pa{\abs{u(x)}^{p-1}+\abs{v(x)}^{p-1}}\abs{u(x)-v(x)}d\mu\pasgnl\leq{H\"older}\\
    \leq&p\pa{\pa{\int_X\abs{u}^{(p-1)q}}^{1/q}+\pa{\int_X\abs{v}^{(p-1)q}}^{1/q}}\pa{\int_X\abs{u-v}^p}^{1/p}=\\
    \pasgnlmath={p-1=p/q}&p(\norm u_p^{p-1}+\norm v_p^{p-1})\norm{u-v}_p
\end{align*}
quindi $\Phi$ \`e Lipschitz di costante $2pR^{p-1}$ sulla palla $B_{L^p}(0,R)\subseteq L^p$
\end{proof}



\begin{proposition}
Lo spazio $L^\infty(X,\Qs,\mu)$ \`e completo
\end{proposition}
\begin{proof}
~[NON HO VISTO, RIGUARDA I PDF]
\end{proof}

$\norm f_{C^1}=\norm f_{\infty,\Omega}+\sum_{i=1}^n\norm{\del_i f}_{\infty,\Omega}$. Questa norma rende continua l'immaersione $C^1_b\to (C_b^0)^{n+1}$ data da $f\mapsto(f,\del_1f,\cdots,\del_n f)$

\begin{proposition}
Sia $\Omega\subseteq \R^n$ aperto. Lo spazio 
\[C^k_b(\Omega)=\cpa{f:\Omega\to\R\mid \text{di classe $C^k$ con derivate limitate su $\Omega$ fino all'ordine $k$}}\]
\`e completo.
\end{proposition}
\begin{proof}
Il caso $k=1$ \`e una conseguenza del teorema di limite sotto il segno di derivata, infatti se $f_k:\Omega\to \R$, $\del_i f_k:\Omega\to\R$ \`e tale che $\del_i f_k\to g_i$ uniformemente in $\Omega$ e $f_k\to f$ puntualmente in $\Omega$ allora esiste $\del_i f$ e vale $g_i$. Se poi $f_k\in C^1(\Omega)$ allora la $g_i$ \`e continua perch\'e limite uniforme di $\del_i f_k$ continue, quindi per il teorema del differenziale totale la $f$ \`e anche $C^1$.

Per il teorema di limite sotto il segno di derivata, l'immersione $C^1_b\to (C_b^0)^{n+1}$ ha immagine chiusa, infatti una successione $(f_k,\del_1f_k,\cdots,\del_nf_k)$ nell'immagine convergente a $(f,g_1,\cdots, g_n)$ \`e proprio una delle ipotesi del teorema di convergenza sotto segno di derivata, quindi $f_k\to f$ in $C^1$
\end{proof}

\section{Duali di spazi concreti}
