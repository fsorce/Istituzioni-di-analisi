\chapter{Operatori compatti fra Banach}

\section{Definizioni}
\begin{definition}[Mappa compatta]
Una mappa $T:X\to Y$ con $X,Y$ spazi di Banach \`e \textbf{compatta} se \`e continua e per ogni $S\subseteq X$ limitato, $T(S)$ \`e relativamente compatto in $Y$, cio\`e $\ol{T(S)}$ \`e compatto.
\end{definition}
\begin{remark}
Siccome $Y$ \`e completo basta chiedere che $T$ mandi limitati in totalmente limitati.
\end{remark}

\begin{remark}
Se $T$ \`e lineare allora non serve imporre continuit\`a in quanto un insieme totalmente limitato \`e in particolare limitato. Inoltre basta controllare solo $S=B(0,1)$ palla chiusa.
\end{remark}

\begin{remark}
$T\in L(X,Y)$ \`e compatto se e solo se per ogni $(x_n)$ successione limitata in $X$, $(Tx_n)$ ha una sottosuccessione convergente.
\end{remark}

\begin{proposition}\label{PrOperatoreCompattoDominioRiflessivo}
Se $X$ \`e riflessivo allora $T$ \`e compatto se e solo se per ogni successione $(x_n)$ debolmente convergente a $0$ vale $\norm{Tx_n}\to 0$ in $Y$, cio\`e $T$ \`e sequenzialmente continuo per le topologie $(X,w)\to (Y,s)$.
\end{proposition}
\begin{proof}
Consideriamo prima il caso di $X$ riflessivo e separabile. 

Se $T$ \`e compatto, $(x_n)\xrightarrow{w}0\implies (Tx_n)\xrightarrow{w}0$ e $(Tx_n)$ ha sottosuccessione convergente a $0$ in quanto separabile (\ref{ThSeparabilitaInTerminiDiMetrizzabilitaDiPalle}), ma allora $(Tx_n)$ stessa converge a $0$ per la propriet\`a di Uhrisohn (\ref{PrProprietaUhrisohn}).

Viceversa se $T$ \`e sequenzialmente continuo da debole a forte e $(x_n)$ \`e una successione limitata. Per il teorema di Kakutani (\ref{ThKakutani}) $X$ riflessivo implica $B_X$ $w$-compatta e per Eberlein-\v Smulian (\ref{ThEberleinSmulian}) questo \`e equivalente a $B_X$ $w$-sequenzialmente compatta. Poich\'e $(x_n)$ \`e limitata essa \`e contenuta in qualche $nB_X$ e per quanto detto questo insieme \`e $w$-sequenzialmente compatto, quindi $(x_n)$ ammette una estratta $w$-convergente. Per ipotesi su $T$, l'immagine di questa sottosuccessione \`e una sottosuccessione di $(Tx_n)$ convergente.

\medskip

Se $X$ \`e riflessivo (potenzialmente non separabile) allora posso considerare il sottospazio chiuso generato dalla successione $(x_n)$ e questo \`e riflessivo separabile quindi la tesi passa.
\end{proof}


\begin{exercise}
Se $X$ e $Y$ sono entrambi riflessivi, $T$ \`e compatto se e solo se per ogni $(x_n)\subseteq X$ con $x_n\xrightarrow{w}0$ e per ogni $(y_n^\ast)\subseteq Y^\ast$ con $y_n^\ast\xrightarrow{w^\ast}0$ vale $\ps{y_n^\ast,T x_n}\to 0$.
\end{exercise}
\begin{proof}
ESERCIZIO, caso particolare di quella sopra.
\end{proof}

\begin{remark}
Nota che le ipotesi di riflessivit\`a sono necessarie, per $Y=\ell_\infty$ e la successione data da $(e_n)$ la tesi fallisce.
\end{remark}


\begin{proposition}
Se $H$ \`e spazio di Hilbert e $(x_n)\subseteq H$ allora essa converge $\normd$ a $x\in H$ se e solo se $x_n\xrightarrow{w} x$ e $\norm{x_n}\to \norm x$.
\end{proposition}
\begin{proof}
Sviluppiamo
\[\norm{x_n-x}^2=\norm{x_n}^2-2\Real(\ps{x,x_n})+\norm x^2\to \norm x^2-2\under{=\ps{x,x}}{\Real(\ps{x,x})}+\norm x^2=0.\]
\end{proof}

\begin{exercise}
Esprimere la compattezza di $T\in L(X,Y)$ tra $X,Y$ Hilbert usando la propriet\`a sopra.
\end{exercise}


\begin{remark}\label{ReImmagineOperatoreCompattoESeparabile}
L'immagine di un operatore compatto \`e separabile.
\end{remark}
\begin{proof}
$\imm(T)=\bigcup_{n\geq 0}nT(B)$ e $T(B)$ separabile perch\'e relativamente compatto in metrico\footnote{Per ogni $n\in\N\nz$ possiamo costruire il ricoprimento $\cpa{B(x,n\ii)}_{x\in \ol{T(B)}}$ ed estrarre un numero finito di centri di queste palle. Unendo questi insiemi di centri abbiamo una unione numerabile di insiemi finiti, quindi numerabile, e la chiusura di questo insieme \`e tutto $\ol{T(B)}$ perch\'e se una palla ha raggio $\e>n\ii$ deve contenere uno dei punti definiti al livello $n$.}.
\end{proof}



\section{Propriet\`a di \texorpdfstring{$L_C(X,Y)$}{LC(X,Y)}}

\begin{definition}[]
Sia $L_C(X,Y)$ lo \textbf{spazio degli operatori compatti} tra $X,Y$ Banach.
\end{definition}
\begin{remark}
$L_C(X,Y)$ \`e un sottospazio vettoriale chiuso di $L(X,Y)$.
\end{remark}
\begin{proposition}
Se $X,Y,Z$ Banach e $T\in L(X,Y),\ S\in L(Y,Z)$ allora $ST\in L_C(X,Z)$ se almeno uno tra $T$ e $S$ \`e compatto.

In particolare se $X=Y=Z$ allora $L_C(X)=L_C(X,X)$ \`e un ideale bilatero chiuso dell'algebra di Banach\footnote{Spazio di Banach che \`e un'algebra tale che $\norm{xy}\leq\norm x\norm y$ e $\norm 1=1$.} $L(X)$ degli operatori limitati su $X$.
\end{proposition}
\begin{proof}
Mostriamo che $L_C(X,Y)$ \`e uno spazio vettoriale chiuso con la propriet\`a di assorbimento data:
\setlength{\leftmargini}{0cm}
\begin{itemize}
\item[$\boxed{\text{sp.vett.}}$] Siano $T,S\in L_C(X,Y)$. Allora 
\[(T+S)(B_X)\subseteq T(B_X)+S(B_X)\subseteq +(\ol{T B_X}\times\ol{S B_X})\]
poich\'e $\ol{T B_X}$ e $\ol{S B_X}$ sono compatti anche il loro prodotto lo \`e, e quindi anche l'immagine sotto $+:Y\times Y\to Y$. Dunque $(T+S)(B_X)$ \`e relativamente compatto in $Y$.

$\la T$ \`e compatto perch\'e $\la T(B_X)=T(\la B_X)$.
\item[$\boxed{\text{chiuso}}$] Sia $T\in \ol{L_C(X,Y)}$. Per $S\in L_C(X,Y)$ si ha
\[T B_X=(S+(T-S))B_X\subseteq S B_X+(T-S)B_X\subseteq S(B_X)+\norm{T-S}_{L(X,Y)} B_Y\]
dunque $T(B_X)$ \`e totalmente limitato in quanto per ogni $\e>0$ scegliamo $S\in L_C(X,Y)$ con $\norm{S-T}_{L(X,Y)}<\e/2$. Poich\'e $S(B_X)$ \`e totalmente limitato esiste $F\subseteq S(B_X)$ finito tale che $S(B_X)\subseteq F+\e B_Y$, allora
\[T(B_X)\subseteq F+\frac\e2 B_Y+\frac\e2 B_Y=F+\e B_Y\]
cio\`e \`e totalmente limitato per arbitrareit\`a di $\e$.
\item[$\boxed{\text{assorb.}}$] Notiamo che
\[B_X\xrightarrow{T}T(B_X)\xrightarrow{S}S(T(B_X))\]
e $ST(B_X)$ \`e compatto perch\'e ogni opertatore limitato manda limitati in limitati e (relativamente) compatti in (relativamente) compatti.
\end{itemize}
\setlength{\leftmargini}{0.5cm}

\end{proof}

\begin{definition}[Algebra di Calkin]
L'\textbf{algebra di Calkin} di $X$ spazio di Banach \`e l'algebra quoziente
\[c(X)=\quot{L(X)}{L_C(X)}\]
\end{definition}


\section{Operatori compatti di rango finito}

\begin{definition}[]
Dati $X,Y$ Banach definiamo
\[L_f(X,Y)=\cpa{T\in L(X,Y)\mid \rnk(T)\in\N}\]
\end{definition}
\begin{remark}
Notiamo che $L_f(X,Y)$ \`e un sottospazio vettoriale di $L_C(X,Y)$.
\end{remark}
\begin{proof}
Se $T\in L_f(X,Y)$ allora manda limitati di $X$ in limitati di $\imm(T)\cong \R^n$ e i limitati di $\R^n$ sono anche totalmente limitati.
\end{proof}

\begin{remark}
In generale
\[L_f(X,Y)\subsetneq\ol{L_f(X,Y)}\subsetneq L_C(X,Y)\]
\end{remark}

\begin{exercise}
$L_f(X,Y)=\ol{L_f(X,Y)}$ se e solo se almeno uno tra $X$ e $Y$ ha dimensione finita.
\end{exercise}


\begin{proposition}\label{PrFormaOperatoriRangoFinito}
    Gli operatori $T\in L_f(X,Y)$ sono quelli della forma
    \[Tx=\sum_{k=1}^n\ps{\al_k,x}y_k\]
    con $\al_1,\cdots,\al_n\in X^\ast$, $y_1,\cdots, y_n\in Y$.
\end{proposition}

\begin{fact}
Pu\`o accadere che
\[L(X)=\R id_X+\ol{L_f(X)}\]
\end{fact}


\begin{proposition}[]\label{PrInHilbertOperatoriRangoFinitoDenseInOperatoriCompatti}
Se $H$ \`e di Hilbert allora $\ol{L_f(H)}=L_C(H)$.
\end{proposition}
\begin{proof}
Sia $T\in L_C(H)$. Sia $(P_n)$ una successione di proiettori ortogonali di rango finito tale che
\[\ol{\bigcup_{n\geq 0}P_n(H)}\pasgnl={(\ref{ReImmagineOperatoreCompattoESeparabile})}\ol{T(H)}\]
Per costruzione $T_n=P_nT\in L_f(H)$, mostriamo che $T_n\xrightarrow{\normd}T$. Poich\'e
\[\norm{T_n-T}=\sup_{\norm x\leq 1}\norm{P_nT x-Tx}=\sup_{y\in T(B_H)}\norm{P_n y-y}=\norm{P_n-id_H\res{\ol{T(B_H)}}}_{\infty,\ol{T(B)H}}\]
si ha che questa convergenza in norma significa che $P_n\res{\ol{T(B_H)}}:\ol{TB_H}\to H$ convege a $I:=id_H\res{\ol{T(B_H)}}$ uniformemente su $\ol{T(B_H)}$.

La successione $\pa{P_n T\res{\ol{T(B_H)}}}$ \`e una successione di mappe equilipschitz che converge puntualmente all'identit\`a:
\[\norm{P_nT}\leq \norm{P_n}\norm T\leq \under{\text{indip. da }n}{\norm T},\]
quindi per Ascoli-Arzel\'a abbiamo convergenza uniforme sui compatti e quindi in particolare sul compatto $\ol{T(B_H)}$.
\end{proof}

\begin{lemma}[]\label{LmOperatoreNonChiusoSiRestringeAInvertibileSuSottospazioDiDimensioneInfinita}
Se $T\in L(X)$ per $X$ Banach NON \`e compatto allora esiste $Y\subseteq X$ sottospazio chiuso di dimensione infinita tale che $T:Y\to TY$ \`e invertibile.
\end{lemma}

\begin{exercise}
Se $H$ \`e di Hilbert
\begin{enumerate}
    \item $L_C(H)$ \`e il pi\`u piccolo ideale bilatero chiuso non nullo di $L(H)$
    \item Se $H$ \`e separabile e infinito dimesionale (isomorfo $\ell_2$) allora $L_C(H)$ \`e l'unico ideale bilatero chiuso proprio
    \[(0)\subsetneq L_C(H)\subsetneq L(H).\]
\end{enumerate}
\end{exercise}
\begin{proof}
Mostriamo i due fatti
\begin{enumerate}
    \item Sia $I$ un ideale bilatero chiuso non nullo di $L(H)$. Allora, scegliendo opportuni elementi di $L(H)$ con cui comporre un funzionale non nullo di $I$, $I$ contiene ogni operatore di rango 1
    \[x\mapsto \ps{\al,x}y\]
    e quindi (\ref{PrFormaOperatoriRangoFinito}) anche ogni operatore di rango finito, ma allora in quanto chiuso contiene $\ol{L_f(H)}=L_C(H)$.
    \item Poich\'e $H$ \`e Hilbert separabile, ogni sottospazio di $H$ chiuso di dimensione infinita \`e isomorfo a $H$. Se $T\in I\bs L_C(H)$ allora per il lemma (\ref{LmOperatoreNonChiusoSiRestringeAInvertibileSuSottospazioDiDimensioneInfinita}) esiste $Y$ di dimensione infinita tale che $T\res Y$ invertibile, cio\`e abbiamo isometrie $U,V$ tali che
    \[H\xrightarrow{U} Y\xrightarrow{T}  TY \xrightarrow{V} H\]
    restituendo $VTU\in \GL(H)$ e quindi $I=(1)$.
\end{enumerate}
\end{proof}

\begin{example}[Operatore integrale con nucleo $k(x,y)$]
Sia $(M,d)$ metrico compatto e $\mu$ misura di borel finita su $M$. Sia $k\in C^0(M\times M)$ e definiamo
\[T_k:\funcDef{C^0(M)}{C^0(M)}{u}{x\mapsto\int_M k(x,y)u(y)d\mu(y)}\]
Allora $T_k$ \`e lineare e continua: per ogni $u\in C^0(M)$ e per ogni $x\in M$
\begin{align*}
\abs{T_ku(x)}\leq&\int_M\abs{k(x,y)}\abs{u(y)}d\mu(y)\leq \mu(M)\norm{k}_{\infty,M\times M}\norm{u}_{\infty,M}
\end{align*}
quindi $\norm{T_k u}_{\infty,M}\leq \mu(M)\norm{k}_{\infty,M\times M}\norm u_{\infty,M}$ da cui
\[\norm{T_k}\leq \mu(M)\norm k_{\infty,M\times M}.\]
Mostriamo ora che $T_k$ \`e compatto. Sia $\omega$ un modulo di continuit\`a\footnote{cio\`e $\abs{k(x,y)-k(x',y')}\leq \omega(\abs{x-x'}+\abs{y-y'})$. Esiste perch\'e $M$ \`e compatto e quindi $k$ continua implica $k$ uniformemente continua per Heine-Cantor.} per $k$, allora per $u\in C^0(M)$
\[\abs{T_ku(x)-T_ku(x')}\leq \int_M\abs{k(x,y)-k(x',y)}\abs{u(y)}d\mu(y)\leq \omega(\abs{x-x'})\mu(M)\norm u_{\infty,M}\]
dunque $T_k(B_M(0,1))$ \`e una famiglia di funzioni equicontinue (con modulo di continuit\`a $\mu(M)\omega$) ed equilimitate (da $\mu(M)\norm k_{\infty,M\times M}$), quindi \`e compatto per Ascoli-Arzel\'a.
\end{example}

\begin{exercise}
Sia $(M,d)$ metrico compatto e $\mu$ misura di borel finita su $M$. Sia $k\in L^2(M\times M,\mu)$ e definiamo
\[T_k:\funcDef{L^2(M)}{L^2(M)}{u}{x\mapsto\int_M k(x,y)u(y)d\mu(y)}\]
che \`e ben definito per Fubini. $T_k$ \`e un operatore compatto.
\end{exercise}
\begin{proof}
TRACCIA:
Vediamo $T_k$ come limite di operatori di rango finito $T_{k_n}$. Precisamente se $\cpa{E_j}_{1\leq j\leq n}$ \`e una partizione misurabile di $M$ e $k_n$ \`e della forma 
\[k_n=\sum_{1\leq i,j\leq n}c_{i,j}\chi_{E_i\times E_j}\]
allora $k_n\in L_f(L^2(M))$ e per scelte opportune delle partizioni e dei coefficienti delle $k_n$ si ha che $k_n\to k$ in $L^2$. Conseguentemente $T_{k_n}\to T_k$ in norma degli operatori.

La scelta ottimale per $c_{i,j}$ fissato $\cpa{E_j}$ \`e la proiezione ortogonale di $k$ sullo spazio generato dalle $\chi_{E_i\times E_j}$, cio\`e
\[c_{i,j}=\frac1{\mu(E_i)\mu(E_j)}\int_{E_i\times E_j}k(x,y)d(\mu\otimes\mu).\]
Per ogni $u\in L^2(M)$ si ha
\begin{align*}
    \norm{T_k u}_2^2=&\int_M\abs{\int_M k(x,y)u(y)d\mu(y)}^2d\mu(x)\overset{\text{Cauchy-Schwarz}}\leq\\
    \leq&\int_M\pa{\int_M \abs{k(x,y)}^2d\mu(y)}\pa{\int_M \abs{u(y)}^2d\mu(y)}d\mu(x)=\\
    =&\norm k_{2,M\times M}^2\norm u_{2,M}^2
\end{align*}
dunque $T_k$ ha norma degli operatori limitata da $\norm k_{2,M\times M}$, quindi anche la corrispondenza
\[\funcDef{L^2(M\times M)}{L(L^2(M))}{k}{T_k}\]
\`e lineare e continua.
\end{proof}


\begin{example}[Operatori diagonali su $\ell_2$]
Sia $u\in \ell_\infty$ e definiamo un operatore ``{diagonale}" $T_u$ su $\ell_2$ ponendo per ogni $x\in \ell_2$
\[T_u(x)=(u(i)x(i))_{i}\]
(cio\`e moltipliplichiamo le entrate corrispondenti tra $x$ e $u$). Notiamo che
\[\norm{T_u x}_2\leq \norm{u}_\infty\norm x_2\]
dunque $T_u$ ha la norma degli operatori maggiorata da $\norm u_\infty$. In realt\`a $\norm{T_u}=\norm u_\infty$ prendendo opportune approssimazioni.

Quindi abbiamo una inclusione isometrica
\[\ell_\infty\inj L(\ell_2)\]
\ul{Quali $u\in \ell_\infty$ danno luogo a $T_u\in L_C(\ell_2)$? }\smallskip

Se $u$ ha supporto finito allora $T_u$ ha rango finito e quindi in particolare \`e un operatore compatto. Essendo $u\to T_u$ isometrica, considerando le chiusure si ha che le $u\in c_0$ producono $T_u\in L_C(\ell_2)$. Questi sono tutti perch\'e $\ell_2$ \`e uno spazio di Hilbert:
\[\cpa{T_u\mid u\in c_0}=L_C(\ell_2)\cap\ell_\infty.\]
\end{example}



\begin{theorem}[Shauder]\label{ThShauder}
Se $T\in L(X,Y)$ allora $T$ \`e compatto se e solo se $T^\ast$ \`e compatto.
\end{theorem}
\begin{proof}
Diamo le due implicazioni
\setlength{\leftmargini}{0cm}
\begin{itemize}
\item[$\boxed{\implies}$] Sia $T\in L_C(X,Y)$ e sia $(x_n^\ast)\subseteq T^\ast(B_{Y^\ast})$. Vogliamo mostrare che questa successione ha estratte convergenti. Si ha che $x^\ast_n=T^\ast y_n^\ast=y_n^\ast\circ T$ per qualche successione $(y_n^\ast)\subseteq B_{Y^\ast}$. Come funzioni $Y\to \K$ si ha che le $y_n^\ast$ sono $1$-Lipschitz quindi per Ascoli-Arzel\'a hanno una sottosuccessione uniformemente convergente sul compatto $\ol{T(B_X)}$ (e quindi di Cauchy), cio\`e $x_n^\ast =y_n^\ast\circ T$ \`e di Cauchy in $X^\ast$ e dunque converge.
\item[$\boxed{\impliedby}$] Sia $T^\ast:Y^\ast\to X^\ast$ compatto. Allora per la prima parte $T^{\ast\ast}:X^{\ast\ast}\to Y^{\ast\ast}$ \`e compatto e quindi anche $T=T^{\ast\ast}\res{X}$ lo \`e ($X\inj X^{\ast\ast}$ \`e una immersione isometrica).
% https://q.uiver.app/#q=WzAsNCxbMCwwLCJYIl0sWzEsMCwiWSJdLFswLDEsIlhee1xcYXN0XFxhc3R9Il0sWzEsMSwiWV57XFxhc3RcXGFzdH0iXSxbMCwxLCJUIl0sWzIsMywiVF57XFxhc3RcXGFzdH0iLDJdLFswLDIsIiIsMix7InN0eWxlIjp7InRhaWwiOnsibmFtZSI6Imhvb2siLCJzaWRlIjoidG9wIn19fV0sWzEsMywiIiwwLHsic3R5bGUiOnsidGFpbCI6eyJuYW1lIjoiaG9vayIsInNpZGUiOiJ0b3AifX19XV0=
\[\begin{tikzcd}
	X & Y \\
	{X^{\ast\ast}} & {Y^{\ast\ast}}
	\arrow["T", from=1-1, to=1-2]
	\arrow[hook, from=1-1, to=2-1]
	\arrow[hook, from=1-2, to=2-2]
	\arrow["{T^{\ast\ast}}"', from=2-1, to=2-2]
\end{tikzcd}\]
\end{itemize}
\setlength{\leftmargini}{0.5cm}
\end{proof}







