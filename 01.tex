\chapter{Norme e Seminorme}
Il corso si concentra sulla relazione che si crea tra la struttura lineare e la struttura topologia degli spazi normati. 

Per $\K$ intendiamo un campo tra $\R$ o $\C$.

\section{Norme e seminorme}

\begin{definition}[Seminorma]
Se $X$ \`e uno spazio vettoriale su $\K$, una \textbf{seminorma} \`e una funzione $\normd:X\to [0,+\infty)$ tale che
\begin{enumerate}
    \item $\norm{x+y}\leq \norm x+\norm y$ (\textit{Disuguaglianza triangolare})
    \item $\norm{\la x}=\la\norm x$ se $\la\in\R,\ \la>0$ (\textit{Positivamente omogenea})
    \item[2'.] $\norm{\la x}=\norm x$ se $|\la|=1$ (\textit{Isotropa})
\end{enumerate}
Se inoltre vale $\norm x=0\coimplies x=0$ allora $\norm\cdot$ \`e detta \textbf{norma}.

La coppia $(X,\norm \cdot)$ si dice \textbf{spazio (semi)normato}.
\end{definition}
\begin{remark}
Su uno spazio (semi)normato possiamo definire una (semi)distanza indotta ponendo
\[d(x,y)=\norm{x-y}.\]
\end{remark}

Diamo alcuni esempi di spazi normati e seminormati:
\begin{example}
\begin{enumerate}
    \item $X=\R^n$, $\displaystyle \norm x_\infty=\max_{i\in\cpa{1,\cdots,n}} \abs{x_i}$
    \item Per $1\leq p<\infty$, $\ell_p=\cpa{x\in\K^\N\mid \sum_{i\geq 0}\abs{x_i}^p<\infty}$ con $\norm x_p=\sum_{i\geq 0}\abs{x_i}^p$
    \item $\ell_\infty=\cpa{x\in\K^\N\mid \sup\abs{x_i}<\infty}$ con $\norm x_\infty=\sup\abs{x_i}$
    \item $\Lc^p(X,\mu)=\cpa{f:X\to\K,\text{ misurabile, }\norm f_p<\infty}$ con
    \[\norm f_p=\begin{cases}
        \pa{\int_X\abs{f(x)}^pd\mu}^{1/p} &\text{se }1\leq p<\infty\\\\
        \displaystyle \supess_{x\in X}\abs{f(x)}=\inf_{\smat{N\subseteq X,\\ \mu(N)=0}}\sup_{x\in X\bs N}\abs{f(x)} &\text{se }p=\infty
    \end{cases}\]
    \`e uno spazio seminormato ma non normato. 
    \item Spazi di Hilbert.
\end{enumerate}
\end{example}

\begin{definition}[Funzioni continue, limitate e lineari]
Siano $E,F$ spazi normati e $S$ un insieme, definiamo i seguenti spazi normati:
\begin{align*}
    \Bs(S,E)=&\cpa{f:S\to E,\text{ limitate}},\quad &&\norm{f}_{\infty,S}=\sup_{s\in S}\norm{f(s)}_E\\
    \Bs C(S,E)=&\cpa{f:S\to E,\text{ continue e limitate}},\quad &&\norm{f}_{\infty,S}=\sup_{s\in S}\norm{f(s)}_E\\
    L(E,F)=&\cpa{T:E\to F\text{ lineare, }\norm T<\infty},\quad &&\norm T=\sup_{x\in B_E(0,1)}\norm{T(x)}_F
\end{align*}
\end{definition}

\begin{definition}[Spazio duale]
Sia $V$ uno spazio vettoriale. Denotiamo con $V'$ \textbf{il duale algebrico}, cio\`e l'insieme delle mappe lineari $V\to \K$. 

Definiamo lo \textbf{spazio duale} a $V$ come $V^\ast=L(V,\K)$, cio\`e come il sottoinsieme di $V'$ dato dalle mappe continue. La norma su $V^\ast$ \`e quindi data da
\[\norm f_{V^\ast}=\sup_{\norm x\leq1}\abs{f(x)}\pasgnl={Lineare}\sup_{\norm x=1}\abs{f(x)}.\]
\end{definition}
\begin{proposition}[Per funzionale limitato equivale continuo]\label{PrPerFunzionaleLineareLimitatoEquivaleContinuo}
Per un funzionale lineare in $V^\ast$, essere limitato \`e equivalente ad essere continuo.
\end{proposition}
\begin{proof}
Se $\norm f=M\in\R_+$ allora
\[\norm{f(x)-f(y)}=\norm{f(x-y)}=\norm{f\pa{\frac{x-y}{\norm{x-y}}}}\norm{x-y}\leq\norm f\norm{x-y}=M\norm{x-y},\]
cio\`e $f$ \`e $M$-lipschitz, e quindi continua.

Sia ora $f$ lineare e continua. Per definizione di continuit\`a in $0$ esiste $\delta>0$ tale che $\norm {f(x)}=\norm{f(x)-f(0)}\leq 1$ per ogni $x\in B_V(0,\delta)$. Segue che
\[\norm{f(x)}=\norm{\frac{\norm x}\delta f\pa{\delta\frac x{\norm x}}}\leq \frac{\norm x}\delta,\]
cio\`e $\norm f_{V^\ast}\leq 1/\delta$ e quindi $f$ limitato.
\end{proof}


\begin{remark}
Se $(X,\norm\cdot)$ \`e uno spazio seminormato e $N=\ker\norm\cdot=\cpa{x\in X\mid \norm x=0}$ allora $\norm\cdot$ passa al quoziente e lo rende uno spazio normato.
\end{remark}
\begin{example}
    Considerando lo spazio seminormato $(\Lc^p(X,\mu),\normd_p)$, la costruzione sopra corrisponde a definire lo spazio normato $(L^p(X,\mu),\normd_p)$, infatti $\ker \normd_p$ sono le funzioni con supporto in un insieme trascurabile.
\end{example}

\begin{remark}
$L(E,F)\inj \Bs(B_E(0,1),F)$ mandando $T\mapsto T\res{B_E(0,1)}$. Infatti per definizione questa mappa \`e isometrica\footnote{$\norm T=\norm{T\res{B_E(0,1)}}_{\infty,B_E(0,1)}$}. Questo identifica il primo spazio con un chiuso del secondo.
\end{remark}


\subsection{Teoremini filosofici}
\begin{theorem}[Banach Mazur]\label{ThBanachMazur}
Sia $(E,\norm\cdot)$ normato, $f:E\to E$ isometria\footnote{con questo termine intendiamo che la mappa, oltre a rispettare le distanze, \`e anche bigettiva. Se non vale bigettivit\`a diremo ``inclusione isometrica"}. Allora $f$ \`e affine.
\end{theorem}
\begin{proof}[Dimostrazione. (ESERCIZIO)]
TRACCIA:
\begin{itemize}
    \item Basta provare che $\forall a,b\in E$ vale
    \[f\pa{\frac{a+b}2}=\frac{f(a)+f(b)}2\]
    (conservando questa conserva i razionali $2$-adici e quindi per continuit\`a ogni combinazione convessa)
    \item Fissati $a,b\in E$, definiamo la \emph{deficienza affine} di $f$ (rispetto ad $a$ e $b$)
    \[def(f)=\norm{\cpa{f\pa{\frac{a+b}2}-\frac{f(a)+f(b)}2}}\]
    La tesi \`e $def(f)=0$.
    \item Notiamo che 
    \[def(f)\leq \norm{f\pa{\frac{a+b}2}}+\norm{\frac{f(a)}2}+\norm{\frac{f(b)}2}=\frac12\pa{\norm{a+b}+\norm a+\norm b}\]
    \item Consideriamo l'applicazione affine che scambia $f(a)$ e $f(b)$ data da
    \[\rho(y)=f(a)+f(b)-y\]
    Poniamo $\wt f=f\ii\circ \rho\circ f$.
    \item Mostrare $def(\wt f)=2def(f)$.
    \item Se $def(f)\neq 0$, iterando otteniamo che esiste $g$ tale che $def(g)$ \`e arbitrariamente grande (raddoppio $def(f)$ tante volte), ma questo \`e assurdo perch\'e abbiamo il limite trovato prima che non dipende dalla funzione.
\end{itemize}
\end{proof}

\begin{center}
\textbf{Filosoficamente questo vuol dire che la struttura metrica in un qualche modo determina la struttura vettoriale.}
\end{center}

\begin{theorem}[Inclusione isometrica / Fr\'echet-Kuratowski]\label{ThInclusioneIsometricaFrechetKuratowski}
Sia $(M,d)$ spazio metrico. Allora esso si immerge isometricamente in uno spazio normato\footnote{addirittura di Banach.}. In particolare si immerge in $(\Bs C(M,\R),\norm\cdot_\infty)$ via l'assegnazione seguente:

Fissiamo un punto base $x_0\in M$.\footnote{saremmo tentati da $x\mapsto d(\cdot,x)$, ma la funzione in arrivo non \`e limitata e quindi non esiste una norma ben definita}
\[\funcDef{M}{\Bs C(M,\R)}{x}{d(\cdot,x)-d(\cdot,x_0)}\]
\end{theorem}
\begin{proof}
ESERCIZIO
\end{proof}

\begin{center}
\textbf{Filosoficamente questo vuol dire che studiando mappe tra spazi metrici, possiamo pensare al codomino come spazi normati.\\
Se consideriamo l'immersione di uno spazio metrico in un Banach, possiamo ``incicciottirlo" e trovare uno spazio metrico ``vicino" che \`e localmente contraibile. Queste idee a volte possono aiutare.}
\end{center}


\section{Completezza}
\begin{definition}[Successione di Cauchy]
Una successione $(x_n)$ \`e \textbf{di Cauchy} o \textbf{fondamentale} se $\forall \e>0\ \exists n\in\N$ tale che per ogni $p,\ q>n$ si ha $d(x_p,x_q)<\e$.
\end{definition}
\begin{fact}[Propriet\`a delle successioni di Cauchy]
    ~
\begin{enumerate}
    \item Ogni successione convergente \`e di Cauchy.
    \item Se $(x_n)$ \`e di Cauchy e $\wt x\in X$ \`e un punto ad essa aderente allora $\wt x$ \`e il limite.
    \item Se $(x_n)$ come sopra ha una sottosuccessione convergente, la successione converge allo stesso limite.
    \item Ogni successione di Cauchy\footnote{questa propriet\`a \`e comoda perch\'e implica $d(x_{n_k},x_{n_p})<2^{-k+1}$ per ogni $p>k$} $(x_n)$ ha una sottosuccessione $(x_{n_k})$ tale che 
    \[d(x_{n_{k+1}},x_{n_{k}})<2^{-k}.\]
\end{enumerate}
\end{fact}

\begin{definition}[Spazio completo]
Uno spazio metrico $(X,d)$ \`e \textbf{completo} se ogni successione di Cauchy in $X$ converge.

Se $(X,\norm \cdot)$ spazio normato \`e completo rispetto alla distanza indotta da $\norm\cdot$ allora si dice \textbf{di Banach}.
\end{definition}

\begin{remark}
Uno spazio normato $(X,\norm \cdot)$ \`e di Banach se e solo se ogni serie $\sum x_k$ definita a partire da una successione tale che $\norm{x_k}<2^{-k}$ \`e convergente.

Equivalentemente $X$ di Banach se ogni serie $\sum x_k$ assolutamente convergente\footnote{cio\`e $\sum \norm{x_k}$ convergente} \`e convergente.
\end{remark}
\begin{proof}
Ogni successione si pu\`o scrivere come serie, infatti $y_n=\sum_{i=0}^n x_i$ per $x_i=y_i-y_{i-1}$. Il resto segue pensando sulle definizioni.
\end{proof}

\begin{remark}
Sia $Y\subseteq X$ con $(X,d)$ metrico. 
\begin{itemize}
    \item Se $X$ \`e completo e $Y$ \`e chiuso allora $Y$ \`e completo. 
    \item Se $Y$ \`e completo allora \`e anche chiuso.
\end{itemize}
\end{remark}

\begin{proposition}[Completamento]\label{PrCompletamento}
Sia $(X,d)$ uno spazio metrico, allora
\begin{enumerate}
    \item esiste una inclusione isometrica densa di $X$ in uno spazio metrico completo
    \[j:(X,d)\inj (\wt X,\wt d)\]
    \item il completamento \`e universale, cio\`e se $j':(X,d)\to (\wt X',\wt d')$ \`e un'altra mappa come sopra allora esiste un'unica isometria $\phi:\wt X\to \wt X'$ che fa commutare il diagramma
    % https://q.uiver.app/#q=WzAsMyxbMCwwLCJYIl0sWzEsMCwiXFx3dCBYIl0sWzEsMSwiXFx3dCBYJyJdLFswLDEsImoiLDAseyJzdHlsZSI6eyJ0YWlsIjp7Im5hbWUiOiJob29rIiwic2lkZSI6InRvcCJ9fX1dLFswLDIsImonIiwyLHsic3R5bGUiOnsidGFpbCI6eyJuYW1lIjoiaG9vayIsInNpZGUiOiJ0b3AifX19XSxbMSwyLCJcXHBoaSIsMCx7InN0eWxlIjp7ImJvZHkiOnsibmFtZSI6ImRhc2hlZCJ9fX1dXQ==
    \[\begin{tikzcd}
        X & {\wt X} \\
        & {\wt X'}
        \arrow["j", hook, from=1-1, to=1-2]
        \arrow["{j'}"', hook, from=1-1, to=2-2]
        \arrow["\phi", dashed, from=1-2, to=2-2]
    \end{tikzcd}\]
\end{enumerate}
\end{proposition}
\begin{proof}
Consideriamo un paio di costruzioni
\setlength{\leftmargini}{0cm}
\begin{itemize}
\item[$\boxed{Costruzione\ 1}$] Consideriamo
\[C_X=\cpa{\xi=(x_n)_{n\in\N}\in X^\N\mid \xi\text{ di Cauchy}}\]
con una semidistanza\footnote{VERIFICARE CHE LO \`E}
\[d(\xi,\eta)=\lim_{n\to\infty}d(\xi_n,\eta_n).\]
Questo limite esiste perch\'e la successione di queste distanze \`e di Cauchy in $\R$, che \`e completo. Notiamo che
\[d(\xi,\eta)=0\coimplies d(\xi_n,\eta_n)=o(1).\]
Notiamo che $X$ ha una inclusione isometrica in $(C_X,d)$ data associando a $x$ la successione costante al valore $x$.

Consideriamo
\[\wt X=\quot{C_X}\Rs,\qquad \xi\Rs\eta\coimplies d(\xi,\eta)=0.\]
L'inclusione isometrica di prima definisce $X\inj \wt X$, ma stavolta $\wt X$ \`e uno spazio metrico per costruzione.

ESERCIZIO: VERIFICA PROPRIET\`A DI NORMA E DENSIT\`A
\item[$\boxed{Costruzione\ 2}$] Definiamo $\wt X$ come la chiusura in $(\Bs C(X),\norm\cdot_\infty)$ dell'immagine di $X$ tramite l'inclusione di Fr\'echet Kuratowski (\ref{ThInclusioneIsometricaFrechetKuratowski}).
\item[$\boxed{Costruzione\ 3}$] (Solo per $X$ spazio normato, ma per il teorema di inclusione isometrica (\ref{ThInclusioneIsometricaFrechetKuratowski}) questo \`e sufficiente) Vedremo che esiste una inclusione isometrica di $X$ nel suo biduale ($x\mapsto val_x$) e che il biduale stesso \`e completo, quindi un completamento di $X$ \`e fornito dalla chiusura di $val_\cdot(X)\subseteq X^{\ast\ast}$
\end{itemize}
\setlength{\leftmargini}{0.5cm}

\end{proof}


\begin{proposition}[Estensione per densit\`a di uniformemente continue]\label{PrEstensioneUniformementeContinue}
Siano $X$ e $Y$ spazi metrici, $Y$ completo, $D\subseteq X$ denso e $f:D\to Y$ uniformemente continua, allora esiste un'unica estensione continua $\wt f$ di $f$ a tutto $X$, inoltre $\wt f$ \`e essa stessa uniformemente continua con lo stesso modulo di continuit\`a.

% https://q.uiver.app/#q=WzAsMyxbMCwwLCJEIl0sWzEsMCwiWSJdLFswLDEsIlgiXSxbMCwyLCJcXHN1YnNldGVxIiwzLHsic3R5bGUiOnsiYm9keSI6eyJuYW1lIjoibm9uZSJ9LCJoZWFkIjp7Im5hbWUiOiJub25lIn19fV0sWzAsMSwiZiJdLFsyLDEsIlxcd3QgZiIsMix7InN0eWxlIjp7ImJvZHkiOnsibmFtZSI6ImRhc2hlZCJ9fX1dXQ==
\[\begin{tikzcd}
	D & Y \\
	X
	\arrow["f", from=1-1, to=1-2]
	\arrow["\subseteq"{marking, allow upside down}, draw=none, from=1-1, to=2-1]
	\arrow["{\wt f}"', dashed, from=2-1, to=1-2]
\end{tikzcd}\]
\end{proposition}

\begin{definition}[Categorie di spazi metrici]
Sia $\Met$ la categoria degli spazi metrici con mappe date da applicazioni uniformemente continue e $\CMet$ la sottocategoria piena dove gli oggetti sono spazi metrici completi
\end{definition}

\begin{remark}
L'operazione di completamento \`e un funtore\footnote{preserva composizione per l'unicit\`a della mappa tra estensioni} $\wt\cdot:\Met\to \CMet$. Questo funtore \`e aggiunto al funtore dimenticante / di inclusione $j:\CMet\to \Met$, infatti
\[\Hom_{\CMet}(\wt X,Y)=UC(\wt X,Y)\pasgnl\cong{(\ref{PrEstensioneUniformementeContinue})} UC(X,j(Y))=\Hom_{\Met}(X,j(Y)).\]
\end{remark}

\begin{exercise}
Verificare l'aggiunzione.
\end{exercise}


\section{Prodotto di spazi (semi)normati}
\begin{remark}
Se $Y\subseteq X$ \`e un sottospazio vettoriale e $(X,\normd)$ \`e normato allora $Y$ \`e (semi)normato con la norma indotta. La topologia indotta \`e quella di sottospazio
\end{remark}

\begin{definition}[Prodotto di spazi (semi)normati]
    Se $(X,\normd_X)$ e $(Y,\normd_Y)$ sono spazi (semi)normati, la (semi)norma prodotto \`e data da
    \[\norm{(x,y)}_{X\times Y}=\max\cpa{\norm x_X,\norm y_Y}.\]
    Questa rende $X\times Y$ uno spazio (semi)normato e
    \[B_{X\times Y}((0,0),1)=B_X(0,1)\times B_Y(0,1),\]
    cio\`e la topologia indotta \`e la topologia prodotto.
\end{definition}

\begin{definition}[Somma diretta topologica]
Due sottospazi di $(X,\normd)$ $Y$ e $Z$ sono in \textbf{somma diretta algebrica} se $+\res{Y\times Z}:Y\times Z\to X$ \`e bigettiva. Se $+\res{Y\times Z}$ \`e anche un omeomorfismo diciamo che $X$ \`e la \textbf{somma diretta topologica} di $Y$ e $Z$.
\end{definition}

\begin{remark}
$X$ \`e la somma diretta topologica di $Y$ e $Z$ se $X$ \`e isomorfo come spazio normato a $(Y\times Z,\normd_{Y\times Z})$.
\end{remark}

\begin{remark}
La mappa $+\res{Y\times Z}$ \`e sempre continua, ma in generale non \`e un omeomorfismo.
\end{remark}

\begin{definition}[Proiettore]
Un endomorfismo lineare $P:X\to X$ si dice \textbf{proiettore} se \`e idempotente, cio\`e $P^2=P$.
\end{definition}

\begin{remark}
Un proiettore definisce una decomposizione in somma diretta algebrica $X=\ker P\oplus \imm P$. Viceversa, ad ogni decomposizione in somma diretta algebrica possiamo associare un proiettore
\end{remark}

\begin{remark}
I proiettori $P_Y:X\to Y$ e $P_Z=id-P_Y:X\to Z$ sono continui se e solo se la somma \`e topologica, infatti
\[(+\res{Y\times Z})\ii=P_Y\times P_Z.\]
\end{remark}

\begin{definition}[Spazio (semi)normato quoziente]
Se $(X,\normd)$ \`e (semi)normato e $Y$ \`e un suo sottospazio allora come spazio vettoriale
\[X/Y=\cpa{x+Y\mid x\in X}.\]
Su essa definiamo la seguente norma: se $\xi\in X/Y$ allora\footnote{pensando a $\xi$ come un traslato di $Y$, la norma che stiamo definendo \`e la distanza di questo spazio affine dall'origine.}
\[\norm{\xi}_{X/Y}=\inf_{x\in\xi}\norm x.\]
\end{definition}

\begin{exercise}
$\normd_{X/Y}$ \`e una seminorma su $X/Y$ e rende la proiezione $\pi:X\to X/Y$ una applicazione aperta e continua. Pi\`u precisamente
\[\pi(B_X(0,1))=B_{X/Y}(0,1)\]
\end{exercise}
\begin{proof}
Continua perch\'e $\norm{\pi(x)}_{X/Y}\leq \norm x$ per definizione di estremo inferiore, quindi $\pi$ ha norma come operatore $\leq 1$, e quindi \`e continua.
\end{proof}

\begin{remark}
Notiamo che $X/Y$ ha effettivamente la topologia quoziente indotta da $\pi$
\end{remark}

\begin{exercise}
La (semi)norma quoziente \`e una norma se e solo se $Y$ \`e chiuso (a prescindere dal fatto che $\normd_X$ sia una norma o seminorma).
\end{exercise}

\begin{remark}
Se $Y$ e $Z$ sono seminormati allora $Y\cong \frac{Y\times Z}Z$ come spazi seminormati.
\end{remark}

\begin{remark}
Se $Y\subseteq X$ ed esiste\footnote{ci sono casi in cui non esite, come $c_0\subseteq \ell_\infty$} $Z$ tale che $X=Y\oplus Z$ allora $Z\cong X/Y$.
\end{remark}

\begin{remark}
In generale $X$ non \`e isomorfo a $Y\times X/Y$.
\end{remark}

\begin{remark}
Per quanto riguarda la completezza in queste costruzioni:
\begin{itemize}
    \item $Y$ sottospazio di $X$ con $X$ di Banach \`e un Banach se e solo se \`e chiuso
    \item $(Y\times Z,\normd_{Y\times Z})$ \`e Banach se e solo se lo sono sia $Y$ che $Z$
    \item Se $(X,\normd)$ \`e normato e $Y\subseteq X$ \`e un sottospazio chiuso allora $(X,\normd)$ \`e completo se e solo se sia $Y$ che $X/Y$ sono completi.
\end{itemize}
Notiamo che l'ultima propriet\`a implica la seconda, infatti $Y\cong \frac{Y\times Z}Z$
\end{remark}


\begin{proposition}[Duale del prodotto]\label{PrDualeProdottoEProdottoDuali}
Dati $X$ e $Y$ spazi di Banach, il duale di $X\times Y$ \`e isometricamente isomorfo a
\[(X^\ast\times Y^\ast,\normd)\]
dove $\norm{(\xi,\eta)}=\norm\xi_{X^\ast}+\norm{\eta}_{Y^\ast}$ (che \`e topologicamente equivalente a $\normd_{X^\ast\times Y^\ast}$).
\[(X^\ast\times Y^\ast,\norm{P_{X^\ast}(\cdot)}_{X^\ast}+\norm{P_{Y^\ast}(\cdot)}_{Y^\ast})\cong ((X\times Y)^\ast,\normd_{(X\times Y)^\ast}).\]
\end{proposition}


\section{Elenco di spazi completi}

\begin{proposition}
Sia $S$ insieme e $E$ Banach, allora lo spazio normato $(\Bs(S,E),\normd_{\infty,S})$ \`e completo.
\end{proposition}
\begin{proof}
~[PERSO, RIGUARDA POI]

tale che $\norm{f(s)}=\norm{\sum_kf_k(s)}\leq \sum_k\norm{f_k(s)}\leq \sum \norm f_{\infty,S}$

quindi $\norm f_{\infty,S}$
\end{proof}

\filosofia{Uno degli strumenti dell'analista: aggiungere e togliere, cio\`e
\[\pi\rho o\sigma\tau\al\vp\al\acute{\iota}\rho\e\sigma\iota\varsigma\]}

\begin{lemma}
Se $(f_k)_{k\in\N}\subseteq \Bs(S,E)$ con $f_k$ continua in $s_0$ per ogni $k$ e $f_k\to f$ uniformemente allora anche $f$ \`e continua in $s_0$.
\end{lemma}
\begin{proof}
Consideriamo
\begin{align*}
\norm{f(s)-f(s_0)}\leq&\norm{f(s)-f_k(s)}+\norm{f_k(s)-f_k(s_0)}+\norm{f_k(s_0)-f(s_0)}\leq\\
\leq&2\norm{f-f_k}_{\infty,S}+\norm{f_k(s)-f_k(s_0)}
\end{align*}
Per la convergenza uniforme di $f_k\to f$ si ha che per ogni $\e>0$ esiste $n\in\N$ tale che $\norm{f-f_n}_{\infty,S}\leq \e/3$.

Per la continuit\`a in $s_0$ di $f_n$ esiste un intorno $U$ di $s_0$ rale che $\norm{f_n(s)-f_n(s_0)}\leq \e/3$ per ogni $s\in U$. Allora per ogni $s\in U$ si ha
\[\norm{f(s)-f(s_0)}\leq2\e/3+\e/3=\e.\]
\end{proof}

\begin{proposition}
Sia $S$ spazio topologico, $E$ banach, allora $\Bs C(S,E)$ \`e completo.
\end{proposition}
\begin{proof}
Basta mostrare che $\Bs C(S,E)$ \`e chiuso in $\Bs(S,E)$. Questo segue dal fatto che la continuit\`a in un punto $s_0\in S$ si conserva per convergenza uniforme, che \`e il lemma precedente.
\end{proof}

\begin{example}
Sia $S=\N\cup\cpa{\infty}$ la compattificazione di Alexandrov di $\N$ e $E$ un banach, allora
\[c(E)\doteqdot\cpa{x:\N\to E,\text{ convergente}}\cong \Bs C(S,E)\]
Questo mostra che $c(E)$ \`e chiuso (e quindi completo) in $\ell_\infty(E)=\Bs(\N,E)$.
\end{example}


Conseguenze:
\begin{proposition}
Lo spazio $(L(X,Y),\normd)$ \`e completo
\end{proposition}
\begin{proof}
Considerando l'inclusione isometrica 
\[R:\funcDef{L(X,Y)}{\Bs(B_X(0,1),Y)}{T}{T\res{B_X(0,1)}}\]
basta vedere che $R(L(X,Y))$ \`e chiuso.

Se $(T_n)_{n\in\N}\subseteq L(X,Y)$ \`e tale che $R(T_n)\to f$ uniformemente in $\Bs(B_X(0,1),Y)$ allora mostriamo che $f$ \`e la restrizione a $B_X(0,1)$ di una qualche lineare $T$.

Mostriamo che le $T_n$ convergono puntualmente per ogni $x\in X$: se $x=0$ ok, se $x\neq 0$
\[T_n(x)=\norm x T_n(x/\norm x)=\norm x R(T_n)(x/\norm x)\to \norm x f(x/\norm x)\]

Sia $T:X\to Y$ definita da $T(x)=\norm x f(x/\norm x)$

[MOSTRARE CHE LA CONVERGENZA \`E UNIFORME, ME LO SONO PERSO]
\end{proof}

\begin{corollary}[Duale di spazio normato \`e banach]\label{CorDualeNormatoEBanach}
Il duale di uno spazio normato \`e sempre banch.
\end{corollary}

\begin{theorem}[Integrazione per serie]\label{ThIntegrazionePerSerie}
    Sia $(X,\Qs,\mu)$ \`e uno spazio di misura e sia $(f_k)_{k\in\N}\subseteq \Lc^1(X,\Qs,\mu)$ tali che
    \[\sum_{k\in \N}\norm{f_k}_1<\infty\]
Allora $\sum_{k\in\N}f_k$ converge q.o. e in norma 1.
\end{theorem}
\begin{proof}
Per ogni $n\in\N$ sia $g_n:X\to \R$ data da
\[g_n(x)=\sum_{k=0}^n\abs{f_k(x)}.\]
Notiamo che $(g_n)$ \`e una successione di funzioni misurabili non negative crescente. Inoltre $g_n\to \sum_{k\in\N}\abs{f_k(x)}$ per definizione di serie.

Per convergenza monotona
\[\sum_{k\in\N}\norm f_1\leftarrow\sum_{k=0}^n\norm{f_k}_1=\inf_X g_nd\mu\to \int_X gd\mu\]
cio\`e $\inf_X gd\mu=\sum{k\in\N}\norm f_1<\infty$, cio\`e $g\in\Lc^1$.

Inoltre $s_n=\sum_{k=0}^nf_k$ \`e una successione dominata da $g$:
\[\abs{s_n(x)}\leq \sum_{k=0}^n\abs{f_k(x)}\leq g(x).\]
Quindi la serie $\sum f_k(x)$ \`e una serie assolutamente convergente per ogni $x$ dove $g<\infty$. Poich\'e $\int g<\infty$ le eccezioni sono trascurabili, quindi quasi ovunque $\sum f_k(x)$ \`e assolutamente convergente.

Sia $f(x)=\sum f_k(x)$ dove la serie converge. Notiamo che
\[\abs{f(x)}\leq \sum_{k\in\N}\abs{f_k(x)}=g(x),\]
quindi $\norm f_1\leq \int gd\mu=\sum_{k\in\N}\norm{f_k}_1$.

Applicando come prima la stima alle code
\[\norm{f-s_n}_1=\norm{\sum_{k=n+1}^\infty f_k}_1\leq \sum_{k>n}\norm{f_k}_1=o(1)\]
dove l'ultimo termine va a 0 perch\'e $\sum\norm {f_k}_1$ \`e convergente.
\end{proof}

\begin{corollary}[Weil]\label{CorTeoremaWeil}
Siano $f_n\in \Lc^1(X,\Qs,\mu)$ convergenti in $\normd_1$. Allora esiste $n_k$ successione strettamente crescente di indici tali che $f_{n_k}$ converge quasi ovunque ed \`e dominata in $\Lc^1$.
\end{corollary}
\begin{proof}
Sia $f$ il limite in $\normd_1$. Data questa convergenza consideriamo una sottosuccessione $n_k$ tale che $\norm{f-f_{n_k}}_1<2^{-k}$. Scrivendo la successione in termini di una somma telescopica
\[f_{n_k}=f_{n_0}+\sum_{j=1}^k(f_{n_j}-f_{n_{j-1}})\]
si ha per il teorema di integrazione per serie\footnote{$\norm{f_{n_0}}_1+\sum_{j=1}^\infty \norm{f_{n_j}-f_{n_{j-1}}}_1\leq \norm{f_{n_0}}_1+\sum_{j=1}^\infty \norm{f_{n_j}-f}_1+ \sum_{j=1}^\infty \norm{f_{n_{j-1}}-f}_1<\infty$} (\ref{ThIntegrazionePerSerie}) $f_{n_k}$ converge quasi ovunque e in $Lc^1$, inoltre \`e dominata da
\[g(x)=\abs{f_{n_0}(x)}+\sum_{j=0}^\infty\abs{f_{n_j}-f_{n_{j-1}}}\geq \abs{f_{n_k}(x)}\]
con $g(x)\in \Lc^1$.
\end{proof}

\begin{proposition}[$L^1$ \`e completo]\label{PrL1Completo}
Se $(X,\Qs,\mu)$ \`e uno spazio di misura, $L^1(X,\Qs,\mu)$ \`e completo.
\end{proposition}
\begin{proof}
Segue immediatamente dal teorema di integrazione per serie (\ref{ThIntegrazionePerSerie}).
\end{proof}

\begin{remark}
La convergenza quasi ovunque di funzioni $\Lc^1(\R,dx)$ \`e \textbf{NON} \`e la convergenza rispetto a una topologia opportuna su $\Lc^1(\R,dx)$.

\begin{proposition}[Propriet\`a di Urysohn]\label{PrProprietaUrysohn}
Ogni convergenza topologica in $X$ insieme ha la seguente propriet\`a \textbf{di Urysohn}: $x_n\to x$ rispetto alla topologia se e solo se per ogni sottosuccessione $x_{n_k}$ esiste una sotto-sottosuccessione $x_{n_{k_j}}\to x$.
\end{proposition}
\begin{proof}
Se $x_n\to x$ converge ok. Se non converge allora esiste un intorno $U$ di $x$ tale che $x_n\notin U$ frequentemente, quindi troviamo una sottosuccessione $x_{n_k}$ che sta sempre fuori da $U$, quindi nessuna sua sotto-sottosuccessione pu\`o convergere a $x$.
\end{proof} 

La convergenza q.o. per successioni in $\Lc^1(\R)$ non ha la propriet\`a di Uhrisohn.
\end{remark}

\begin{definition}[Operatore di composizione]
Se $E$ \`e uno spazio di funzioni con codominio $\R$ e $f:\R\to\R$, definiamo l'operatore di composizione per $f$ come $E\ni u\mapsto f\circ u$.
\end{definition}

\begin{lemma}
Sia $u_k$ una successione che converge a $u$ in $\normd_p$. A meno di sottosuccessione $u_k\to u$ quasi ovunque e dominata in $\Lc^p$.
\end{lemma}
\begin{proof}
Teorema di Weil (\ref{CorTeoremaWeil}) in $\Lc^p$.
\end{proof}

\begin{proposition}
Lo spazio $L^p(X,\Qs,\mu)$ per $0\leq p<\infty$ \`e completo.
\end{proposition}
\begin{proof}
$L^p$ ed $L^1$ NON sono isomorfi come spazi di Banach in generale\footnote{cursiosit\`a non banale da vedere}, ma esiste un omeomorfismo localmente Lipschitz e questo basta a mostrare la completezza: se $u_k$ \`e una successione di Cauchy in $L^p$, se $\Phi$ \`e Lipschitz allora $\Phi(u_k)$ \`e ancora di Cauchy in $L^1$ e quindi converge, poi torno indietro con $\Phi\ii$, che mantiene il limite per continuit\`a.


Consideriamo
\[\Phi:\funcDef{\Lc^p}{\Lc^1}{u}{\abs{u}^p\sgn(u)}\]
Chiaramente \`e invertibile mandando $v\in L^1$ in $\abs{v}^{1/p}\sgn v$. La mappa $\Phi$ \`e l'operatore di composizione con la funzione $f(t)=\abs{t}^p\sgn t$. La continuit\`a degli operatori di composizione \`e un fatto generale. Se $u_k\to u$ converge in $\normd_p$ allora per il lemma a meno di sottosuccessione converge q.o. e dominata, quindi componendo con $f$ abbiamo ancora convergenza quasi ovunque per continuit\`a ($f(u_k)\to f(u)$ q.o.). Se $\abs{u_k}\leq g$ in $\Lc^p$ allora $\abs{u_k}^p\leq g^p$ in $\Lc^1$, similmente per $\Phi\ii$, quindi effettivamente $\Phi$ \`e un omeomorfismo.


Mostriamo ora che $\Phi$ \`e localmente lipschitz: siano $u,v\in \Lc^p(X)$
\[\abs{\Phi(u)-\Phi(v)}_1=\int_X\abs{f(u(x))-f(v(x))}d\mu(x)\]
ma se $t<s$ allora $\abs{f(t)-f(s)}\leq \sup_{t\leq \xi\leq s}\abs{f'(\xi)}\abs{t-s}$ e $\abs{f'(xi)}=p\abs{xi}^{p-1}\leq p(\max\cpa{\abs t,\abs s})^p$, quindi
\begin{align*}
    \abs{\Phi(u)-\Phi(v)}_1\leq& p\int_X\max\cpa{\abs{u(x)}^{p-1},\abs{v(x)}^{p-1}}\abs{u(x)-v(x)}d\mu\leq\\
    \leq &p\int_X\pa{\abs{u(x)}^{p-1}+\abs{v(x)}^{p-1}}\abs{u(x)-v(x)}d\mu\pasgnl\leq{H\"older}\\
    \leq&p\pa{\pa{\int_X\abs{u}^{(p-1)q}}^{1/q}+\pa{\int_X\abs{v}^{(p-1)q}}^{1/q}}\pa{\int_X\abs{u-v}^p}^{1/p}=\\
    \pasgnlmath={p-1=p/q}&p(\norm u_p^{p-1}+\norm v_p^{p-1})\norm{u-v}_p
\end{align*}
quindi $\Phi$ \`e Lipschitz di costante $2pR^{p-1}$ sulla palla $B_{L^p}(0,R)\subseteq L^p$
\end{proof}



\begin{proposition}
Lo spazio $L^\infty(X,\Qs,\mu)$ \`e completo
\end{proposition}
\begin{proof}
~[NON HO VISTO, RIGUARDA I PDF]
\end{proof}

$\norm f_{C^1}=\norm f_{\infty,\Omega}+\sum_{i=1}^n\norm{\del_i f}_{\infty,\Omega}$. Questa norma rende continua l'immaersione $C^1_b\to (C_b^0)^{n+1}$ data da $f\mapsto(f,\del_1f,\cdots,\del_n f)$

\begin{proposition}
Sia $\Omega\subseteq \R^n$ aperto. Lo spazio 
\[C^k_b(\Omega)=\cpa{f:\Omega\to\R\mid \text{di classe $C^k$ con derivate limitate su $\Omega$ fino all'ordine $k$}}\]
\`e completo.
\end{proposition}
\begin{proof}
Il caso $k=1$ \`e una conseguenza del teorema di limite sotto il segno di derivata, infatti se $f_k:\Omega\to \R$, $\del_i f_k:\Omega\to\R$ \`e tale che $\del_i f_k\to g_i$ uniformemente in $\Omega$ e $f_k\to f$ puntualmente in $\Omega$ allora esiste $\del_i f$ e vale $g_i$. Se poi $f_k\in C^1(\Omega)$ allora la $g_i$ \`e continua perch\'e limite uniforme di $\del_i f_k$ continue, quindi per il teorema del differenziale totale la $f$ \`e anche $C^1$.

Per il teorema di limite sotto il segno di derivata, l'immersione $C^1_b\to (C_b^0)^{n+1}$ ha immagine chiusa, infatti una successione $(f_k,\del_1f_k,\cdots,\del_nf_k)$ nell'immagine convergente a $(f,g_1,\cdots, g_n)$ \`e proprio una delle ipotesi del teorema di convergenza sotto segno di derivata, quindi $f_k\to f$ in $C^1$
\end{proof}