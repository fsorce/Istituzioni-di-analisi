\chapter{Spazi di Sobolev}

\section{Derivata debole}
\begin{definition}[Derivata debole]
Sia $I\subseteq \R$ un intervallo aperto. Una funzione $f:I\to \R$ in $L^1_{loc}(I)$ ha \textbf{derivata debole} (o \textbf{distribuzionale}) $g\in L^1_{loc}(I)$ se
\[\forall \vp\in C^\infty_C(I),\qquad \int_I\dot\vp fdt=-\int_I\vp gdt.\]
\end{definition}

\begin{remark}
Ricordiamo che vale una inclusione
\[T:\funcDef{L^1_{loc}(I)}{\Dc'(I)}{f}{T_f:\vp\mapsto \int_I\vp f}\]
dunque in questi termini $g$ \`e derivata debole per $f$ se $(T_f)'=T_g$.
\end{remark}


\begin{remark}
Se $f\in C^1(I)$ allora ha derivata debole e questa coincide con la sua derivata classica $f'\in C^0(I)$.
\end{remark}

\begin{exercise}
Se $u\in \Dc(I)$ \`e tale che $u'=0$ allora $u$ \`e costante.
\end{exercise}

\begin{remark}
Se $f\in C^0(I)$ ed esiste $f'=g$ quasi ovunque NON \`e detto che $g$ sia la derivata distribuzionale.
\end{remark}

\begin{example}
Sia $f$ la funzione di Cantor, cio\`e $f:[0,1]\to \R$ \`e l'unica funzione che sia un punto fisso dell'endofunzione su $\cpa{f:[0,1]\to\R}$ che manda una funzione $f$ in
\[\begin{cases}
\frac12f\pa{3x} &x\in \spa{0,\frac13}\\
\frac12 &x\in \spa{\frac13,\frac23}\\
\frac12f(3x-2)+\frac12  &x\in \spa{\frac13,\frac23}
\end{cases}\] 
che esiste in quanto questa associazione \`e una contrazione.

La funzione di Cantor ha $f'=0$ per ogni $x\in I\bs C$, in particolare quasi ovunque. Eppure la derivata debole di $f$ non \`e $0$, segue dall'esercizio.
\end{example}

\section{Ripassino di analisi 3}
\begin{definition}[Funzioni assolutamente continue]
Una funzione $f:[a,b]\to \R$ \`e \textbf{assolutamente continua} se per ogni $\e>0$ esiste $\delta>0$ tale che per ogni $I_i=[a_i,b_i]\subseteq [a,b]$, $1\leq i\leq n$ intervalli disgiunti tali che $\sum_{i=1}^n\abs{b_i-a_i}<\delta$ si ha
\[\sum_{i=1}^n\abs{f(b_i)-f(a_i)}<\e.\]
\end{definition} 

\begin{fact}
$f$ \`e assolutamente continua se e solo se $f=\int_a^xgdt$ per qualche $g\in L^1([a,b])$. Inoltre la derivata $f'(x)$ esiste per quasi ogni $x\in [a,b]$ e vale $g(x)$.
\end{fact}

\begin{remark}
La funzione di Cantor non \`e assolutamente continua. 
\end{remark}

\begin{remark}
Le funzioni assolutamente continue sono la classe pi\`u ampia per cui vale il teorema fondamentale del calcolo integrale.
\end{remark}



\begin{fact}
Per assolutamente continue vale la formula di integrazione per parti, in particolare per ogni $\vp\in C^1_C(I)$ si ha
\[\int_I f\vp'=-\int_I f'\vp\]
quindi la derivata quasi ovunque di una assolutamente continua coincide con la derivata distribuzionale.
\end{fact}

\begin{theorem}[Radon-Nikodym]\label{ThRadonNikodym}
Sia $(X,\Sigma)$ uno spazio di misura e siano $\mu,\nu$ due misure $\sigma$-finite su esso. Se $\nu$ \`e assolutamente continua rispetto a $\mu$ (scritto $\nu\ll\mu$) allora esiste una funzione $f:X\to[0,\infty)$ misurabile per $\Sigma$ tale che per ogni $A\in \Sigma$ misurabile si ha
\[\nu(A)=\int_A fd\mu.\]
La funzione $f$ \`e detta funzione \`e detta \textbf{derivata di Radon-Nikodym} e si indica $\dd \mu\nu$.
\end{theorem}

\begin{remark}
Se $m$ \`e la misura $\sigma$-finita definita sugli intervalli da
\[m([\al,\beta))=f(\beta)-f(\al)\]
allora $m\ll \Lc$ dove $\Lc$ \`e la misura di Lebesgue.In questo caso $f'$ \`e anche la derivata di Radon-Nikodym di $m$.
\end{remark}

\begin{remark}
In generale vale una bigezione
\[\correspDef{\cpa{\text{misure $\geq 0$ finite su $[a,b)$}}}{\cpa{f:[a,b)\to\R\text{ cresc., cont. a sx, }f(a)=0}}{\mu}{f_\mu(x)=\mu([a,x))}{\mu_f([\al,\beta))=f(\beta)-f(\al)}{f}\]
dove in realt\`a $\mu_f$ \`e la misura determinata da quel comportamento sugli intervalli semiaperti\footnote{la $\sigma$-additivit\`a non \`e ovvia perch\'e una successione crescente di intervalli semiaperti potrebbe accumularsi in diversi punti, ma notiamo che \`e un insieme ben ordinato quindi possiamo indicizzare gli insiemi con un ordinale numerabile e concludiamo per induzione transfinita}.
\end{remark}


\section{Spazi di Sobolev}
\begin{definition}[Spazio di Sobolev]
Per $1 \leq p\leq\infty$ e $I\subseteq \R$ intervallo definiamo il relativo \textbf{spazio di Sobolev} come
\[W^{1,p}(I)=\cpa{f\in L^p(I)\mid f'_{distr.}\in L^p(I)}\]
\end{definition}
\begin{remark}
Uno spazio di Sobolev \`e uno spazio vettoriale.
\end{remark}

\begin{definition}[Norma del grafico]
Siano $X,Y$ Banach, $A:D\to Y$ con $D\subseteq X$ e $A$ operatore chiuso. La \textbf{norma del grafico} su $D$ \`e 
\[\norm{x}_A=\norm x_X+\norm{Ax}_Y.\]
Questa norma rende $D$ isomorfo al sottospazio chiuso $\Gamma A\subseteq X\times Y$.
\end{definition}

\begin{proposition}[]\label{PrOperatoreDiDerivazioneSuSobolevHaGraficoChiuso}
L'operatore di derivazione
\[D:\funcDef{W^{1,p}}{L^1}{f}{f'}\]
\`e lineare e chiuso, cio\`e il suo grafico
\[\Gamma D=\cpa{(f,f')\in L^p\times L^p\mid f\in W^{1,p}}\]
\`e chiuso in $L^p\times L^p$.
\end{proposition}
\begin{proof}
Se $(f_n,f_n')$ \`e una successione su $\Gamma D$ convergente in $L^p\times L^p$ a qualche coppia $(f,g)$, cio\`e $f_n\to f$ e $f'_n\to g$, allora segue che $g=f'$ e quindi $(f,g)\in \Gamma D$. Infatti
\[\int_I f_n\vp'=-\int_I f_n'\vp\]
passa al limite perch\'e $\vp\in C^\infty_C$, quindi
\[\int_I f\vp'=-\int_I g\vp.\]
\end{proof}

\begin{corollary}
La norma del grafico\footnote{$\norm f_{1,p}=\norm f_p+\norm{f'}_p$} sul dominio di $D$ rende $W^{1,p}$ uno spazio di Banach. 
\end{corollary}
\begin{proof}
Questa norma rende l'immersione ovvia
\[W^{1,p}\inj \Gamma D\]
una funzione continua e isometrica a valori in un sottospazio chiuso di $L^p\times L^p$, che \`e un Banach.
\end{proof}

\begin{remark}
$W^{1,p}$ per $1\leq p\leq\infty$ \`e uno spazio di Banach, separabile per $p<\infty$ in quanto lo sono gli $L^p(I)$, e riflessivo per $1<p<\infty$ (perch\'e sottospazi chiusi e prodotti di Banach riflessivi sono riflessivi).
\end{remark}

\begin{remark}
Dall'inclusione $W^{1,p}\subseteq L^p\times L^p$ segue una rappresentazione delle forme lineari continue su $W^{1,p}$.
\end{remark}
\begin{proof}
Se $L\in W^{1,p}(I)$ esistono $u,v\in (L^p)^\ast$ tali che per ogni $f\in W^{1,p}$ si ha
\[\ps{L,f}=\ps{u,f}=\ps{v,f'}\]
per esempio, se $u\in L^p$ e $v\in L^q$, poich\'e $(L^p)^\ast\cong L^q$
\[\ps{L,f}=\int_I(uf+vf')dt\]


?????????????[Questa parte me la sono persa a lezione]
\end{proof}


\begin{proposition}[Caratterizzazioni spazio di Sobolev]\label{PrCaratterizzazioniSpazioDiSobolev}
Per $f\in L^p(I)$ e $1< p\leq\infty$ le seguenti sono equivalenti
\begin{enumerate}
    \item $f\in W^{1,p}(I)$
    \item esiste $C\geq 0$ tale che per ogni $\vp\in C^\infty_C(I)$
    \[\abs{\int_I f\vp'}\leq C\norm\vp_{q}\]
    \item Esiste $C\geq 0$ tale che per ogni $J\subseteq I$ e per ogni $h\in \R$ con $J+h\subseteq I$ si ha
    \[\norm{\tau_h f-f}_{p,J}\leq C\abs h\]
\end{enumerate}
\end{proposition}
\begin{proof}
Diamo le implicazioni
\setlength{\leftmargini}{0cm}
\begin{itemize}
\item[$\boxed{1.\implies2.}$] Se $f\in W^{1,p}(I)$ allora
\[\abs{\int f\vp'}=\abs{\int f'\vp}\overset{\text{H\"older}}\leq \norm{f'}_p\norm\vp_q.\]
\item[$\boxed{2.\implies1.}$] Se vale $2.$ allora \`e ben definito il funzionale lineare continuo su $C^\infty_C(I)$ dato da
\[\vp\mapsto \int_I f\vp'\]
Per continuit\`a esso si estende per densit\`a ad un funzionale lineare continuo su $L^q$, ma allora \`e della forma $\vp\mapsto -\int g\vp$ con $g\in L^p$, dunque
\[\int f\dot\vp=-\int g\vp\]
ovvero esiste $f'_{dist}\in L^p$.
\item[$\boxed{1.\implies3.}$] SOON...

\end{itemize}
\setlength{\leftmargini}{0.5cm}
\end{proof}


\begin{proposition}[]\label{PrElementiDiSpazioDiSobolevSonoHolderiani}
Le $f\in W^{1,p}(I)$ per $1<p\leq \infty$ sono H\"older. 
\end{proposition}
\begin{proof}
Per $1<p<\infty$ si ha
\[\abs{f(x)-f(y)}=\abs{\int_x^y1f'(t)dt}\leq \abs{\int_x^y1}^{1/q}\norm{f'}_p=\norm{f'}_p\abs{x-y}^{1/q},\]
mentre per $p=\infty$
\[\abs{f(x)-f(y)}=\abs{\int_x^y1f'(t)dt}\leq\abs{x-y}\norm{f'}_\infty.\]
\end{proof}

\section{Parentesi Esercizi}
\begin{exercise}
Per $1<p\leq \infty$ e $u\in L^p(\R)$ sono equivalenti
\begin{enumerate}
    \item $u\in W^{1,p}(\R)$
    \item esiste $C\geq 0$ tale che $\norm{\tau_h(u)-u}_p\leq C\abs h$ per ogni $h\in \R$
\end{enumerate}
dove $\tau_h(u)(x)=u(x+h)$, quindi la seconda consizione significa che i rapporti incrementali di $u$
\[\delta_hu(x)=\frac{u(x+h)-u(x)}h\]
sono limitati in $L^p$ per ogni $h\in \R\nz$.
\end{exercise}
\begin{solution}
Mostriamo le due implicazioni
\setlength{\leftmargini}{0cm}
\begin{itemize}
\item[$\boxed{1.\implies2.}$] Se $u\in W^{1,p}$, $\chi_1=\chi_{[-1,0]}$ e $\chi_h(x)=\frac1h\chi_1(x/h)$ (notiamo $\chi_1\in L^1(\R)$, $\chi_1\geq 0$, $\int \chi_1=1$) allora si ha
\[\delta_hu(x)=\frac1h\int_x^{x+h}\dot u(t)dt=\frac1h\int_\R\chi_{[x,x+h]}(t)\dot u(t)dt=\int_\R\chi_{h}(x-t)\dot u(t)dt=\dot u\ast\chi_h.\]
Poich\'e $\dot u\in L^0$, dalla disuguaglianza di Young ($p\leq \infty$) si ha
\[\norm{\delta_h u}_p=\norm{\chi_h\ast\dot u}_p\leq \norm{\chi_h}_1\norm{\dot u}_p=\norm{\dot u}_p\]
Il ragionamento \`e analogo per $h<0$.
\item[$\boxed{2.\implies1.}$] Poich\'e $p>1$ si ha $L^p=(L^q)^\ast$, quindi la successione $\cpa{\delta_{h_j}u}_{j\geq 0}$ con $h_j=\frac1j$, che per ipotesi \`e limitata in $L^p$, ha una sottosuccessione che \`e $w^\ast$-convergente in $L^p$ e converge $w^\ast$ (uniformemente sui compatti (\ref{ThBoundedWeakStarELaTopologiaDiConvergenzaUniformeSuCompatti})) ad una $u\in L^p$. In particolare per ogni $\vp\in C^\infty_C(\R)$ si ha
\[\ps{\delta_{h_j}u,\vp}\to \ps{v,\vp}\]
ma $\ps{\delta_hu,\vp}=\ps{u,\delta_{-h}\vp}$, infatti
\begin{align*}
    \ps{\delta_hu,\vp}=&\frac1h\int_\R(\tau_h u-u)\vp dt=\frac1h\pa{\int_\R u(x+h)\vp(x)dx-\int_\R u(x)\vp(x)dx}=\\
    =&\frac1h\pa{\int_\R u(x)\vp(x-h)dx-\int_\R u(x)\vp(x)dx}=\\
    =&\int_\R u(x)\delta_{-h}\vp(x) dx
\end{align*}
Poich\'e $\vp\in C^\infty_C$, $\delta_{h_j}\vp\xrightarrow{L^1}\dot \vp$ (addirittura converge in $\Dc(\Omega)$) si conclude che
\[\ps{v,\vp}=\lim_j\ps{\delta_{h_j}u,\vp}=\lim_j\ps{u,\delta_{-h_j}\vp}=\ps{u,\dot \vp}\]
cio\`e $u$ ha una derivata debole $v\in L^p$.
\end{itemize}
\setlength{\leftmargini}{0.5cm}
\end{solution}
\begin{remark}
Cosa succede per $p=1$? 
Si ha $2.$ \`e equivalente a $u\in BV(\R)$, cio\`e $u\in L^1(\R)$ e $\dot u$ \`e una misura di Radon.


L'idea \`e che $L^1$ si immerge isometricamente in uno spazio che \`e un duale e poi ragiono in modo analogo ma la convergenza debole star in questo spazio non restituisce una funzione ma una misura.
\end{remark}


\begin{exercise}
Per $I=[a,b]$ e $1<p\leq \infty$, l'inclusione 
\[W^{1,p}(I)\inj C^0(I)\]
\`e compatta.
\end{exercise}
\begin{proof}
Abbiamo gi\`a visto che $u\in W^{1,p}(I)$ \`e H\"older di esponente $1/q$ (nota che $1\leq q<\infty$), infatti
\[\abs{u(y)-u(x)}=\abs{\int_x^y\dot u(t)dt}\overset{\text{H\"older}}\leq \norm{\dot u}_p\pa{\int_x^y 1dt}^{1/q}=\norm{\dot u}_p\abs{x-y}^{1/q}.\]
Dunque, per Ascoli-Arzel\'a si ha che la palla unitaria di $W^{1,p}$ (rispetto alla topologia della norma $\normd_{1,p}$ sullo spazio di Sobolev) \`e un insieme relativamente compatto in $C^0([a,b])$.
\end{proof}

\begin{remark}
\`E vero che $W^{1,p}(\R)\subseteq L^\infty(\R)$? \`E vero che \`e compatta?
\end{remark}
\begin{solution}
Considerando $u\in C^\infty_C\nz$ e la successione di traslate $\tau_nu$ si ha che questa successione converge puntualmente a $0$ ma la successione non tende a $0$ n\'e in $W^{1,p}$ n\'e in $L^\infty$.
\end{solution}

\begin{exercise}
\`E vero che \`e compatta l'inclusione $W^{1,1}([0,1])\subseteq C^0([0,1])$?
\end{exercise}
\begin{solution}
No, basta trovare una successione in $W^{1,1}$ limitata che non ha sottosuccessioni convergenti in $(C^0([0,1]),\normd_\infty)$.
\end{solution}











