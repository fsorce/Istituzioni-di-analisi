\chapter{Teoria spettrale per operatori limitati su Banach}
\section{Spettro e operatori risolventi}

\begin{definition}[Spettro di un operatore]
Per $X$ spazio di Banach e $A\in L(X)$, lo \textbf{spettro} di $A$ \`e
\[\sigma(A)=\cpa{\la\in\C\mid \la-A\notin \GL(X)}\]
L'insieme $\rho(A)=\C\bs \sigma(A)$ \`e detto \textbf{insieme risolvente}.
\end{definition}

\begin{remark}[Spettro \`e chiuso]
Notiamo che $\la\mapsto \la-A$ \`e continua e $\GL(X)$ \`e un aperto di $L(X)$, quindi $\sigma(A)=\rho(A)^c=(\la\mapsto\la-A)\ii(\GL(X))^c$ \`e chiuso.
\end{remark}

\begin{remark}
Nel caso di $X$ Banach su $\R$ si considera la sua complessificazione $X_\C=\C\otimes_\R X=X\times X$ dove il prodotto \`e inteso munito della struttura complessa indotta da
\[J=\mat{0&-id_X\\id_X&0}\]
cio\`e $(a+bi)\ulx=a\ulx +bJ\ulx$ per ogni $\ulx\in X\times X$.
\end{remark}

\begin{remark}
Anche se $\la\in \sigma(A)$ non necessariamente $\la-A$ non \`e iniettiva. 
\end{remark}

\begin{definition}[Spettro puntuale]
Definiamo lo \textbf{spettro puntuale} come
\[\sigma_{pt}(A)=\cpa{\la\in \C\mid \ker(\la-A)\neq 0}.\]
Gli elementi dello spettro puntuale sono detti \textbf{autovalori}.
\end{definition}

\begin{proposition}[]\label{PrProprietaSpettro}
Sia $A\in L(X)$, allora
\begin{itemize}
    \item $\sigma(A)$ \`e contenuto in $\ol B(0,\norm A)$ e quindi compatto\footnote{abbiamo gi\`a visto che \`e chiuso}.
    \item L'applicazione
    \[\funcDef{\rho(A)}{L(X)}{\la}{(\la-A)\ii}\]
    \`e analitica e infinitesima per $\la\to \infty$. Pi\`u precisamente: Per ogni $\la$ tale che $\abs\la\geq \norm A$ si ha
    \[(\la-A)\ii=\sum_{n=0}^\infty \la^{-n-1}A^n\]
    e per ogni $\la_0\in \rho(A)$ e ogni $\la\in B(\la_0,\norm{(\la_0-A)\ii})$ si ha
    \[(\la-A)\ii=\sum_{n=0}^\infty (\la_0-\la)^n((\la_0-A)\ii)^{n+1}.\]
\end{itemize}
\end{proposition}
\begin{proof}
Se mostriamo il secondo punto abbiamo il primo in quanto se per ogni $\abs{\la}\geq \norm A$ abbiamo questo sviluppo, in particolare $(\la-A)\ii$ \`e ben definita per $\abs{\la}\geq \norm A$, quindi l'inversa pu\`o venire a mancare solo per $\la$ contenuti in $\ol{B}(0,\norm A)$.

Se vale il primo sviluppo in serie allora
\[\norm{(\la-A)\ii}\leq \sum_{n=0}^\infty \norm{\la^{-n-1}A^n}=\frac1{\abs\la-\norm A}\]
e quindi la mappa in esame \`e infinitesima per $\la\to \infty$.
\medskip

Tutto segue se mostriamo che per un operatore $H$ di norma $\norm H<1$ in uno spazio di Banach vale lo sviluppo in serie di $(1-H)\ii$, infatti la seconda espansione in serie \`e una caso particolare dello sviluppo
\[(K-H)\ii=K\ii+K\ii H K\ii+K\ii HK\ii H K\ii+\cdots\]
per $K\in \GL(X)$ e $\norm H<\norm{K\ii}\ii$, che segue dal caso $K=1$ notando
\[(K-H)\ii=K\ii(I-HK\ii)\ii,\qquad \norm{H K\ii}<1.\]
Notiamo che se $\norm H<1$ allora la serie $\sum_{n=0}^\infty H^n$ converge assolutamente ad un operatore lineare continuo e per questioni algebriche questa espansione \`e l'inversa di $(1-H)$.
\end{proof}


\begin{definition}[Operatore risolvete]
Se $\la\in \rho(A)$, l'\textbf{operatore risolvente relativo a $\la$} \`e $(\la-A)\ii$.
\end{definition}

\begin{remark}
Vale l'\textbf{identit\`a risolvente}
\[(\la-A)\ii-(\mu-A)\ii=(\mu-\la)(\la-A)\ii(\mu-A)\ii.\]
\end{remark}
\begin{proof}
Basta calcolare:
\begin{align*}
    (\mu-\la)(\la-A)\ii(\mu-A)\ii+(\mu-A)\ii=&((\mu-\la)(\la-A)\ii+1)(\mu-A)\ii=\\
    =&(\la-A)\ii(\mu-\la+(\la-A))(\mu-A)\ii=\\
    =&(\la-A)\ii(\mu-A)(\mu-A)\ii=\\
    =&(\la-A)\ii.
\end{align*}
\end{proof}

\begin{proposition}[]\label{PrSpettroENonVuoto}
Se $A\in L(X)$ e $X\neq 0$ allora $\sigma(A)$ \`e non vuoto.
\end{proposition}
\begin{proof}
Segue dal teorema di Liouville applicato a
\[\funcDef{\rho(A)}{\C}{\la}{\ps{x^\ast,(\la-A)\ii x}}\]
con $x\in X$ e $x^\ast\in X^\ast$ variabili. Infatti queste funzioni sono olomorfe su $\rho(A)$ e infinitesime per $\la\to\infty$ (\ref{PrProprietaSpettro}), quindi se avessimo $\sigma(A)=\emptyset$ allora le mappe sarebbero olomorfe definite su tutto $\C$ e infinitesime all'infinito (in particolare limitate), quindi costanti (al valore $0$ in quanto infinitesime) per ogni $x$ e $x^\ast$, quindi $(\la-A)\ii=0$ come mappa, che \`e assurdo perch\'e per definizione di $\rho(A)$ \`e invertibile ma $X\neq 0$.
\end{proof}

\subsection{Raggio spettrale e Cauchy-Hadamard-Gelfand}
\begin{definition}[Raggio spettrale]
Sia $A\in L(X)$. Il suo \textbf{raggio spettrale} \`e
\[r_A=\max_{\la\in \sigma(A)}\abs{\la}\in [0,\norm A]\]
Notiamo che questo massimo esiste perch\'e $\sigma(A)$ \`e compatto (\ref{PrProprietaSpettro}).
\end{definition}




\begin{lemma}[]\label{LmSuccessioneSubadditivaConvergeSeDivisaPerIndice}
Sia $(a_n)\subseteq\R$ una successione subadditiva\footnote{$a_{n+m}\leq a_n+a_m$} allora esiste il limite
\[\lim_{n\to\infty}\frac{a_n}n=\inf_n\frac{a_n}n.\] 
\end{lemma}
\begin{proof}
Dato $d\geq 1$, ogni $n\in \N$ si scrive $n=p_nd+k_n$ con $0\leq k_n<d$ e $p_n=\floor{\frac nd}$. Allora per ogni $n\geq 1$
\begin{align*}
    \inf_{m\geq 1}\frac{a_m}m\leq& \frac{a_n}n=\frac{a_{p_n d+k_n}}{n}\leq \frac1n\pa{p_n a_d+a_{k_n}}\leq\\
    \leq&\frac{dp_n}n \frac{a_d}d+\frac1n\max_{1\leq k<d}a_k=\\
    =&(1+o(1))\frac{a_d}d+o(1)
\end{align*}
quindi, prendendo il $\limsup_{n}$ e poi $\inf_{d\geq 1}$
\[\inf_{m\geq 1}\frac{a_m}m\leq\liminf_{n}\frac{a_n}n\leq \limsup_{n}\frac{a_n}n\leq \inf_{d\geq 1}\frac{a_d}d\]
dunque esiste il limite ed \`e pari all'estremo inferiore.
\end{proof}


\begin{proposition}[Formula di Cauchy-Hadamard-Gelfand]\label{FormulaCauchyHadamardGelfand}
Vale la seguente identit\`a
\[r_A=\lim_{n\to\infty}\norm{A^n}^{1/n}=\inf_{n\geq 1}\norm{A^n}^{1/n}.\]
\end{proposition}
\begin{proof}
Applichiamo il lemma (\ref{LmSuccessioneSubadditivaConvergeSeDivisaPerIndice}) alla successione $a_n=\log\norm{A^n}$, che \`e subadditiva perch\'e $\norm{A^{n+m}}\leq \norm{A^n}\norm{A^m}$. Per continuit\`a dell'esponenziale questo mostra che il limite nel testo esiste ed \`e pari all'estremo inferiore.
Mostriamo che $r_A=\lim_{n}\norm{A^n}^{1/n}$: 
\setlength{\leftmargini}{0cm}
\begin{itemize}
\item[$\boxed{\leq}$] Se $\la\in \sigma(A)$, cio\`e $\la-A$ non invertibile, allora anche $\la^n-A^n$ non \`e invertibile:
\[\la^n-A^n=(\la-A)B=B(\la-A),\quad B=\sum_{i=0}^{n-1}\la^iA^{n-1-i}\]
quindi $\la-A$ non invertibile implica per il teorema della mappa aperta (\ref{ThMappaAperta}) che $\la-A$ non \`e bigettiva, quindi non \`e iniettiva o non \`e surgettiva, e quindi neanche $\la^n-A^n$ lo \`e.


Dunque $\la^n\in \sigma(A^n)$ e quindi $\abs{\la^n}\leq \norm{A^n}$ da cui $\abs{\la}\leq \norm{A^n}^{1/n}$.
Questo mostra la disuguaglianza $r_A\leq \inf_{n\geq 1}\norm{A^n}^{1/n}=\lim_{n}\norm{A^n}^{1/n}$.
\item[$\boxed{\geq}$] Per ogni $z\in B_\C(0,\frac1{r_A})$ \`e ben definito\footnote{se $z=0$ ok, se $z\neq 0$ allora l'espressione vale $(\frac1z-A)\ii$ che \`e ben definita perch\'e abbiamo supposto $z<1/r_A\coimplies 1/z>r_A\implies 1/z\in \rho(A)$.} $z(1-zA)\ii$. La mappa
\[\funcDef{B_\C(0,\frac1{r_A})}{L(X)}{z}{z(1-zA)\ii}\]
\`e ``analitica": ha uno sviluppo locale in $0$ dato da
\[\sum_{n=0}^\infty z^{n+1}A^n\]
che per\`o si estende a tutto il disco. 

Siano $x\in X$ e $x^\ast\in X^\ast$ e consideriamo la funzione olomorfa
\[\funcDef{B_\C(0,\frac1{r_A})}{\C}{z}{\ps{x^\ast z(1-zA)\ii x}}\]
la quale ha sviluppo locale in $0$ dato da
\[\ps{x^\ast,\sum_{n=0}^\infty z^{n+1}A^nx}=\sum_{n=0}^\infty z^{n+1}\ps{x^\ast,A^nx}\]
che converge assolutamente per $\abs{z}=\frac1r<\frac1{r_A}\coimplies r>r_A$ grazie alla convergenza di $z(1-zA)\ii$. In particolare i termini della serie sono limitati
\[\abs{\ps{x^\ast,\pa{\frac A r}^{n+1}}x}\leq C_{x,x^\ast}.\]
Per Banach-Steinhaus (\ref{ThBanachSteinhausUniformeLimitatezza}) si ha che $\pa{\frac A r}^{n+1}$ \`e limitato: per $x$ fissato la disuguaglianza dice che $\cpa{\pa{\frac A r}^{n+1} x}_{n\in \N}$ \`e $w$-limitata in $X$, quindi limitata in norma (Banach-Steinhaus) e applicando di nuovo Banach-Steinhaus si ha che $\cpa{\pa{\frac A r}^{n+1} }_{n\in \N}$ sono operatori puntualmente limitati, quindi sono limitati in norma. Scriviamo
\[\norm{\pa{\frac A r}^{n+1}}\leq C'\]
cio\`e $\norm{A^n}^{1/n}\leq C'^{1/n}r$ da cui
\[\lim \norm{A^n}^{1/n}\leq r\lim C'^{1/n}=r\]
e questo per ogni $r>r_A$, quindi anche per $r_A$ stesso passando all'estremo inferiore.
\end{itemize}
\setlength{\leftmargini}{0.5cm}
\end{proof}

\begin{remark}
La stessa formula, con la stessa dimostrazione, funziona anche per il raggio spettrale di algebre di Banach.
\end{remark}

\begin{exercise}
Calcolare il raggio spettrale dell'\textbf{operatore di Volterra}
\[V:\funcDef{C^0([a,b])}{C^0([a,b])}{f}{\int_a^x f(t)dt}\]
con la formula del raggio spettrale e provando che per ogni $\la\in \C\nz,\ \la-V\in \GL(V)$.
\end{exercise}



\section{Teoria spettrale su spazi di Hilbert}

\begin{definition}[Operatore simmetrico]
Sia $A$ un operatore limitato su $H$ spazio di Hilbert. $A$ \`e \textbf{simmetrico} (scritto $A\in L^{sim}(H)$) se per ogni $x,y\in H$ si ha
\[\ps{Ax,y}=\ps{x,Ay}.\]
\end{definition}

\begin{proposition}[]\label{PrOperatoriSimmetriciSuHilbertOrtogonaliNucleiImmaginiChiusure}
Sia $A\in L^{sim}(H)$, allora
\begin{enumerate}
    \item $\ker A=(\imm A)^\perp$ e $\ol{\imm A}=(\ker A)^{\perp}$
    \item Se $H_0\subseteq H$ \`e un sottospazio $A$-invariante allora anche $H_0^\perp$ e $\ol{H_0}$ lo sono.
\end{enumerate}
\end{proposition}
\begin{proof}
Mostriamo le due affermazioni
\begin{enumerate}
    \item Segue dalle catena di equivalenze
    \begin{gather*}
        x\in \ker A\\
        Ax=0\\
        0=\ps{Ax,y}=\ps{x,Ay}\ \forall y\in H\\
        x\in (\imm A)^\perp
    \end{gather*}
    l'altra affermazione segue notando che $\ol V=(V^\perp)^\perp$.
    \item Se $x\in H_0^\perp$ allora per ogni $y\in H_0$ si ha $0=\ps{x,Ay}=\ps{Ax,y}$, cio\`e $Ax\in H_0^\perp$. Segue l'invarianza della chiusura prendendo l'ortogonale di nuovo.
\end{enumerate}
\end{proof}

\begin{definition}[Operatori simmetrici positivi]
Se $A\in L^{sim}(H)$ esso si dice \textbf{positivo} se $\ps{Ax,x}\geq 0$ per ogni $x$.
\end{definition}
\begin{remark}
La positivit\`a induce una relazione d'ordine parziale su $L^{sim}(H)$: 
\begin{center}
    $A\geq B\coimplies A-B$ positivo.
\end{center}
\end{remark}

\begin{fact}[]\label{FCTSimmetricoPositivoAlloraAggiungereIdentitaRendeInvertibile}
Se $A$ \`e simmetrico positivo allora $I+A\in \GL(H)$
\end{fact}
\begin{proof}
$I+A$ \`e fortemente iniettiva in quanto per ogni $x\in H$
\[\norm{(I+A)x}^2=(x+Ax)(x+Ax)=\norm x^2+2\ps{Ax,x}+\norm{Ax}^2\geq \norm x^2\]
quindi in particolare \`e iniettivo con immagine chiusa.

Per il punto 1. di (\ref{PrOperatoriSimmetriciSuHilbertOrtogonaliNucleiImmaginiChiusure}) un operatore simmetrico e iniettivo ha immagine densa, dunque $I+A$ \`e anche surgettivo e quindi invertibile.
\end{proof}


\begin{proposition}[]\label{PrSpettroOperatoreSimmetricoEReale}
Se $A\in L^{sim}(H)$ allora $\sigma(A)\subseteq \R$.
\end{proposition}
\begin{proof}
Per ogni $a,b\in\R$ con $b\neq 0$ si ha che $(a+ib-A)$ \`e invertibile perch\'e fattore di
\[(a+ib-A)(a-ib -A)=(a-A)^2+b^2=b^2\pa{I+\pa{\frac{a-A}b}^2}\]
e questo \`e invertibile per (\ref{FCTSimmetricoPositivoAlloraAggiungereIdentitaRendeInvertibile}).
\end{proof}

\begin{proposition}[]\label{PrNormaInTerminiDiFormaQuadraticaAssociata}
Per $A\in L^{sim}(H)$ si ha $\norm A=\sup_{\norm x<1}\abs{Ax\cdot x}=\norm{q_A}_{\infty,B}$
\end{proposition}
\begin{proof}
Siano $x,y\in H$ e consideriamo l'identit\`a di polarizzazione:
\[Ax\cdot y=\frac14\pa{q_A(x+y)-q_A(x-y)}.\]
Segue che
\begin{align*}
Ax\cdot y\leq& \frac14\pa{\norm{q_A}_{\infty}\norm{x+y}^2+\norm{q_A}_{\infty}\norm{x-y}^2}=\\
=&\frac{\norm{q_A}_{\infty}}4\pa{2\norm x^2+2\norm y^2}\overset{\smat{\norm x<1\\\norm y<1}}\leq\\
\leq&\norm{q_A}_{\infty}
\end{align*}
quindi vale $\norm A\leq \norm{q_A}_{\infty}$. L'altra disuguaglianza segue dal fatto che per $\norm x<1$
\[\abs{Ax\cdot x}\leq \sup_{\norm y<1}\abs{Ax\cdot y}=\norm{Ax}\leq \norm A.\]
\end{proof}

\begin{notation}
\[m_A=\inf_{\norm x=1}q_A(x),\qquad M_A=\sup_{\norm x=1}q_A(x).\]
\end{notation}

\begin{proposition}[]\label{PrIntervalloOttimaleCheLimitaSpettroDiOperatoreSimmetrico}
$\sigma(A)\subseteq[m_A,M_A]$. Inoltre $m_A=\min \sigma(A)$ e $M_A=\max\sigma(A)$.
\end{proposition}
\begin{proof}
Sia $t<m_A$, allora, poich\'e $q_A(x)\geq m_A\norm x^2$, si ha che come operatore simmetrico
\[\frac{A-m_A}{m_A-t}\geq 0\]
quindi $-1\notin \sigma(\frac{A-m_A}{m_A-t})$ in quanto $I+\frac{A-m_A}{m_A-t}$ \`e invertibile (\ref{FCTSimmetricoPositivoAlloraAggiungereIdentitaRendeInvertibile}).
Quindi 
\[t-m_A\notin \sigma(A-m_A)=\sigma(A)-m_A\coimplies t\notin \sigma(A).\]
Un conto analogo mostra che se $t>M_A$ allora $t\notin \sigma(A)$, quindi $\sigma(A)\subseteq [m_A,M_A]$.
\bigskip

Mostriamo ora che $M_A=\max\sigma(A)$ e $m_A=\min\sigma(A)$: si ha che
\[\norm{A^2}\pasgnl={(\ref{PrNormaInTerminiDiFormaQuadraticaAssociata})}\sup_{\norm x\leq 1}q_{A^2}(x)=\sup_{\norm x\leq 1}(Ax\cdot Ax)=\sup_{\norm x\leq 1}\norm{Ax}^2=\norm A^2.\]
Analogamente 
\[\norm A=\norm{A^2}^{1/2}=\norm{A^4}^{1/4}=\cdots=\norm{A^{2^n}}^{1/2^n},\]
dunque per la formula di Cauchy-Hadamard-Gelfand (\ref{FormulaCauchyHadamardGelfand}) si ha $\norm A=r_A$. Poich\'e $\norm A=\norm{q_A}_{\infty}$ (\ref{PrNormaInTerminiDiFormaQuadraticaAssociata}) si ha che $r_A=\norm A$ vale $M_A$ oppure $-m_A$ perch\'e questi sono maggiorati da $\norm{q_A}_\infty$ per definizione.

Se $A\geq 0$ allora $\sigma(A)\subseteq [0,\infty)$ e quindi $0\leq m_A\leq M_A=r_A$, cio\`e $M_A=\max\sigma(A)$.
In generale per $t>\norm A$ si ha $A+t\geq 0$ e quindi 
\[M_A+t=M_{A+t}=\max\sigma(A+t)=\sigma(A)+t\]
cio\`e di nuovo $M_A=\sigma(A)$ come voluto.
Analogamente si mostra $\min\sigma(A)=m_A$.
\end{proof}

\begin{remark}
Un'altra dimostrazione usa la caratterizzazione variazionale di $\sigma(A)$ e il principio variazionale di Ekeland.
\end{remark}


\subsection{Autovalori di operatori simmetrici}
\begin{remark}
Autovettori di autovalori differenti sono ortogonali, cio\`e se $\la\neq \mu$ autovalori per $u$ e $v$ autovettori allora
\[\la(u\cdot v)=A u\cdot v=u\cdot A v=\mu u\cdot v.\]
\end{remark}


\begin{remark}
La molteplicit\`a geometrica di $\la$ autovalore di $A$ (cio\`e $\dim \ker(\la-A)$) \`e uguale alla molteplicit\`a algebrica (cio\`e $\sup_{n\geq 0}\dim\ker\pa{\la-A}^n$).
\end{remark}
\begin{proof}
Senza perdita di generalit\`a $\la=0$. Basta notare che $\ker A=\ker A^2$:
\[A^2x=0\implies \norm{Ax}^2=A^2x\cdot x=0\implies Ax=0.\]
\end{proof}

\subsubsection{Caratterizzazione variazionale degli autovalori di operatori simmetrici}
Ricordiamo che se $f:\Omega\subseteq H\to \R$ differenziabile allora il gradiente di $f$ in $x_0\in \Omega$ \`e l'elemento $\nabla f(x_0)$ di $H$ che rappresenta via prodotto scalare il differerenziale $df(x_0)\in H^\ast$, cio\`e
\[f(x_0+h)=f(x_0)+\nabla(f)(x_0)\cdot h+o(h)\]
Inoltre $x_0$ \`e un punto critico di $f$ se $\nabla f(x_0)=0$. La $f(x_0)$ \textbf{corrispondente} \`e detto \textbf{valore critico}.

Ricordiamo che se $u,v:\Omega\subseteq H\to \R$ sono differenziabili in $x_0$ e $v(x_0)\neq 0$ il quoziente $u/v$ \`e differenziabile in $x_0$ e abbiamo
\[\nabla\pa{\frac uv}=\frac{v\nabla u-u\nabla v}{v^2}.\]



\begin{proposition}[Caratterizzzione variazionale]\label{PrCaratterizzazioneVariazionaleAutovalori}
Una coppia $(x,\la)\in H\times \R$ \`e una coppia autovettore-autovalore di $A$ se \`e una coppia punto critico-valore critico della funzione
\[f_A:\funcDef{H\nz}{\R}{x}{\dfrac{Ax\cdot x}{x\cdot x}}\]
detta \textbf{quoziente di Rayleigh}. 
\end{proposition}
\begin{proof}
Calcolando troviamo
\[\nabla(q_A(x))=2Ax,\qquad \nabla{q_I(x)}=2x.\]
Allora
\[\nabla f_A(x)=2\frac{\norm x^2Ax-(Ax\cdot x)x}{\norm x^4}=\frac2{\norm x^2}\pa{Ax-f_A(x)x}\]
dunque $\nabla f_A(x)=0$ se e solo se $x$ \`e un autovettore di $A$ corrispondente all'autovalore $f_A(x)$.
\end{proof}


\subsection{Spettro di operatori simmetrici compatti}
\begin{proposition}[]\label{PrMassimoRaggiuntoComeAutovaloreSeNonNulloOperatoreSimmetricoCompatto}
Sia $A\in L^{sim}_C(H)$, allora se $M_A> 0$ si ha che $M_A$ \`e un autovalore di $A$.
\end{proposition}
\begin{proof}
Per compattezza, $q_A$ ha massimo sulla palla $\ol B(0,1)$, infatti se $(x_k)\subseteq \ol B(0,1)$ \`e una successione massimizzante per $q_A$ su $\ol B$ allora estraendo una sottosuccessione si pu\`o assumere\footnote{siamo in uno spazio di Hilbert e quindi uno spazio riflessivo, dunque per Kakutani (\ref{ThKakutani}) la palla unitaria chiusa \`e $w$-compatta, quindi per Eberlein-\v Smulian (\ref{ThEberleinSmulian}) anche $w$-sequenzialmente compatta.} $x_k\xrightarrow{w}x$ per $x\in \ol B(0,1)$, ma essendo $A$ compatto si ha $Ax_k\to Ax$ in norma e quindi\footnote{stiamo usando il fatto che $y_k\xrightarrow{s}y$ e $x_k\xrightarrow{w}x$ implica $y_k\cdot x_k=(y\cdot x_k)+(y_k-y)\cdot x_k\to y\cdot x +o(1)$ perch\'e $y x_k\to y x$ e $\abs{(y-y_k)\cdot x_k}\leq \norm{y-y_k}\norm{x_k}=O(\norm{y-y_k})$ in quanto convergenza debole implica limitato e quindi $\norm{x_k}$ \`e limitato da una costante fissata per ogni $k$.} $Ax_k\cdot x_k\to Ax\cdot x$.

Siccome $M_A\neq 0$ si ha che $x\neq 0$ e quindi $\norm x=1$ (in quanto $q_A(x/\norm x)\geq q_A(x)$ se $x\in \ol B(0,1)$). Allora $x$ \`e anche il massimo di $f_A$ si $H\nz$ perch\'e $f_A$ \`e $0$-omogenea, in particolare \`e un punto critico e quindi un autovettore per (\ref{PrCaratterizzazioneVariazionaleAutovalori}).
\end{proof}


\begin{remark}
Analogamente $m_A$ \`e il minimo autovalore se $A$ simmetrico compatto.
\end{remark}

\begin{remark}
In particolare $\norm A$ oppure $-\norm A$ \`e un autovalore.
\end{remark}

\begin{corollary}[]
Se $H_0\subseteq H$ \`e un sottospazio non banale chiuso $A$-invariante allora $A$ ha un autovettore in $H_0$.
\end{corollary}

\begin{corollary}[]\label{CorCompattoSimmetricoEOrtogonalmenteDiagonalizzabile}
Un operatore $A\in L^{sim}_C(H)$ ammette una base ortonormale di autovettori.
\end{corollary}
\begin{proof}
Si considera un sistema $\cpa{u_i}_{i\in I}$ ortonormale di autovettori di $A$ che sia massimale per inclusione (esiste per lemma di Zorn). Esso deve essere una base Hilbertiana, cio\`e 
\[\ol{\Span(\cpa{u_i}_{i\in I})}=H.\]
Se non fosse cos\`i allora $H_0:=\ol{\Span(\cpa{u_i}_{i\in I})}^\perp\neq 0$ \`e un sottospazio chiuso $A$-invariante di $H$, quindi esisterebbe $u\in H_0$ autovettore che per costruzione \`e ortogonale al sistema considerato, negando la massimalit\`a.
\end{proof}

\begin{corollary}[Operatore simmetrico compatto ammette una forma diagonale]
Un operatore $A\in L^{sim}_C(H)$ \`e unitariamente coniugato ad un operatore di moltiplicazione per elementi di $c_0(I)$ su uno spazio $\ell_2(I)$ dove $I$ \`e un insieme che indicizza una base di autovettori di $A$.
% https://q.uiver.app/#q=WzAsOCxbMCwwLCJcXGVsbF8yKEkpIl0sWzMsMCwiSCJdLFsxLDEsIih4X2kpIl0sWzIsMSwiXFxzdW1fe2lcXGluIEl9eF9pIHVfaSJdLFszLDMsIkgiXSxbMCwzLCJcXGVsbF8yKEkpIl0sWzIsMiwiXFxzdW1fe2lcXGluIEl9eF9pIFxcbGFfaSB1X2kiXSxbMSwyLCIoXFxsYV9pIHhfaSkiXSxbMCwxLCJ1Il0sWzIsMywiIiwzLHsic3R5bGUiOnsidGFpbCI6eyJuYW1lIjoibWFwcyB0byJ9fX1dLFsxLDQsIkEiXSxbMCw1LCJcXHdoIEEiLDJdLFs1LDQsInUiLDJdLFsyLDcsIiIsMyx7InN0eWxlIjp7InRhaWwiOnsibmFtZSI6Im1hcHMgdG8ifX19XSxbMyw2LCIiLDMseyJzdHlsZSI6eyJ0YWlsIjp7Im5hbWUiOiJtYXBzIHRvIn19fV0sWzcsNiwiIiwzLHsic3R5bGUiOnsidGFpbCI6eyJuYW1lIjoibWFwcyB0byJ9fX1dLFsyLDAsIlxcaW4iLDMseyJzdHlsZSI6eyJib2R5Ijp7Im5hbWUiOiJub25lIn0sImhlYWQiOnsibmFtZSI6Im5vbmUifX19XSxbNyw1LCJcXGluIiwzLHsic3R5bGUiOnsiYm9keSI6eyJuYW1lIjoibm9uZSJ9LCJoZWFkIjp7Im5hbWUiOiJub25lIn19fV0sWzYsNCwiXFxpbiIsMyx7InN0eWxlIjp7ImJvZHkiOnsibmFtZSI6Im5vbmUifSwiaGVhZCI6eyJuYW1lIjoibm9uZSJ9fX1dLFszLDEsIlxcaW4iLDMseyJzdHlsZSI6eyJib2R5Ijp7Im5hbWUiOiJub25lIn0sImhlYWQiOnsibmFtZSI6Im5vbmUifX19XV0=
\[\begin{tikzcd}
	{\ell_2(I)} &&& H \\
	& {(x_i)} & {\sum_{i\in I}x_i u_i} \\
	& {(\la_i x_i)} & {\sum_{i\in I}x_i \la_i u_i} \\
	{\ell_2(I)} &&& H
	\arrow["u", from=1-1, to=1-4]
	\arrow["{\wh A}"', from=1-1, to=4-1]
	\arrow["A", from=1-4, to=4-4]
	\arrow["\in"{marking, allow upside down}, draw=none, from=2-2, to=1-1]
	\arrow[maps to, from=2-2, to=2-3]
	\arrow[maps to, from=2-2, to=3-2]
	\arrow["\in"{marking, allow upside down}, draw=none, from=2-3, to=1-4]
	\arrow[maps to, from=2-3, to=3-3]
	\arrow[maps to, from=3-2, to=3-3]
	\arrow["\in"{marking, allow upside down}, draw=none, from=3-2, to=4-1]
	\arrow["\in"{marking, allow upside down}, draw=none, from=3-3, to=4-4]
	\arrow["u"', from=4-1, to=4-4]
\end{tikzcd}\]
\end{corollary}

\subsubsection{Indicizzazione degli autovalori}
\begin{remark}
Se gli autovalori di $A\in L^{sim}_C(H)$ hanno un punto di accumulazione come sottoinsieme di $\R$ allora questo punto \`e $0$.
\end{remark}
\begin{proof}
Supponiamo ci siano infiniti autovalori e che per assurdo tendano ad un elemento diverso da $0$.


Possiamo definire una successione autovettori ortogonali. Questa ha un limite debole per Kakutani (\ref{ThKakutani}) e Eberlein-\v Smulian (\ref{ThEberleinSmulian}). Poich\'e $A$ \`e compatto avremmo una successione di vettori ortogonali (gli scalati dei vettori originali) che convergono in norma, assurdo se il limite degli autovalori non \`e $0$.
\end{proof}

Grazie a questa osservazione possiamo indicizzare gli autovalori di $A\in L^{sim}_C(H)$ in modo monotono:
\begin{itemize}
    \item Se ci sono infiniti autovalori positivi essi sono una successione che converge a $0$, che possiamo indicizzare in modo decrescente verso $0$. Se essi sono in numero finito ci fermiamo (o volendo da quell'indice in poi ripetiamo $0$).
    \item Procediamo in modo simile con gli autovalori negativi.
    \item Se un autovalore $\la$ ha molteplicit\`a $r$ allora nella successione lo ripetiamo $r$ volte nelle posizioni opportune.
\end{itemize}
\[\la_{-1}<\la_{-2}<\cdots<0<\cdots<\la_2<\la_1\]
Sia $I$ l'insieme di indici cos\`i creato.


\begin{theorem}[Curant-Fischer-Weil / Minimax]\label{ThCurantFisherWeilMinimax}
Sia $H$ spazio di Hilbert separabile.
Indicizzando gli autovalori di $A\in L^{sim}_C(H)$ come sopra si ha che per ogni $n\in I$ tale che $n>0$ allora
\[\la_n(A)=\inf_{\smat{E\subseteq H\\ \codim E<n}}\sup_{\smat{\norm x=1\\x\in E}}q_A(x)=\sup_{\smat{F\subseteq H\\ \dim F\geq n}}\inf_{\smat{\norm x=1\\ x\in F}}q_A(x)\]
\[\la_{-n}(A)=\sup_{\smat{E\subseteq H\\ \codim E<n}}\inf_{\smat{\norm x=1\\x\in E}}q_A(x)=\inf_{\smat{F\subseteq H\\ \dim F\geq n}}\sup_{\smat{\norm x=1\\ x\in F}}q_A(x)\]
\end{theorem}
\begin{proof}
Evidentemente basta mostrare il caso di $n>0$.
\medskip

\noindent
Sia $E_n=\Span(e_1,\cdots, e_n)$ dove $e_i$ \`e un fissato autovettore di $\la_i$ e gli $e_i$ sono ortogonali tra loro. Notiamo che (poich\'e $H$ separabile)
\[E_n^\perp=\ol{\Span(e_i\mid i\in I\bs \cpa{1,\cdots, n})}\]
e 
\begin{align*}
    A\res{E_n}=&diag(\la_1,\cdots, \la_n):E_n\to E_n\\
    A\res{E_n^\perp}=&diag(\la_i\mid i\in I\bs \cpa{1,\cdots, n}):E_n^\perp\to E_n^\perp.
\end{align*}
Notiamo che $\la_n$ \`e il minimo autovalore di $A\res{E_n}$ e anche il massimo autovalore di $E_{n-1}^\perp$.
Dunque per la caratterizzazione variazionale (\ref{PrCaratterizzazioneVariazionaleAutovalori}) si ha che
\[\la_n=\min_{\smat{\norm x=1\\ x\in E_n}}q_A(x)=\max_{\smat{\norm x=1\\ x\in E_n^\perp}}q_A(x)\]
Siano ora $E\subseteq H$ sottospazio di codimensione $\codim E<n$ e $F\subseteq H$ un sottospazio di dimensione $\dim F\geq n$. Per questione di dimensione
\[E_n\cap E\neq (0),\qquad E_n^\perp \cap F\neq (0)\]
quindi esistono elementi $x_0\in E_n\cap E$ e $y_0\in E_{n-1}^\perp\cap F$ non nulli e quindi senza perdita di generalit\`a di norma $1$.
\[\sup_{\smat{\norm x=1\\ x\in E}}q_A(x)\geq q_A(x_0)\geq \min_{\smat{\norm x=1\\ x\in E_n}}q_A(x)=\la_n=\max_{\smat{\norm x=1\\ x\in E_n^\perp}}q_A(x)\geq q_A(y_0)\geq \inf_{\smat{\norm x=1\\ x\in F}}q_A(x)\]
in particolare, poich\'e $\la_n$ \`e raggiunto da $E=E_n$ e $F=E_n^\perp$ abbiamo mostrato che
\[\inf_{\smat{E\subseteq H\\\codim E<n}}\sup_{\smat{\norm x=1\\ x\in E}}q_A(x)=\la_n=\sup_{\smat{F\subseteq H\\ \dim F\geq n}}\inf_{\smat{\norm x=1\\ x\in F}}q_A(x).\]
\end{proof}

\begin{corollary}[Principio autovalori intervallati]\label{CorPrincipioAutovaloriIntervallati}
Sia $A\in L^{sim}_C(H)$ e sia $H_0\subseteq H$ un iperpiano chiuso con incusione $j_0:H_0\to H$ e proiettore ortogonale\footnote{$P_0=j_0^\ast$ perch\'e siamo su uno spazio di Hilbert} $P_0:H\to H_0$. Sia 
\[A_0=P_0 A j_0:H_0\to H_0.\]
Notiamo che\footnote{Eredita compattezza di $A$ e $A_0^\ast=(P_0 A j_0)^\ast=j_0^\ast A^\ast P_0^\ast=P_0 A j_0$.} $A_0\in L^{sim}_C(H_0)$. Notiamo che $q_{A_0}=q_A\res{H_0}$.\medskip

\noindent
Allora per ogni $n>0$ si ha
\[\la_{n+1}(A)\leq \la_n(A_0)\leq \la_n(A)\]
e analogamente 
\[\la_{-n}(A)\leq \la_{-n}(A_0)\leq \la_{-n-1}(A).\]
\end{corollary}
\begin{proof}
Usando il principio di minimax (\ref{ThCurantFisherWeilMinimax}) si ha
\begin{align*}
\la_{n+1}(A)=&\inf_{\smat{E\subseteq H\\ \codim E<n+1}}\sup_{\smat{\norm x=1\\x\in E}}q_A(x)
\overset{\codim_{H_0} E<n\implies \codim_HE<n+1}\leq\\
\leq& \inf_{\smat{E\subseteq H_0\\ \codim E<n}}\sup_{\smat{\norm x=1\\x\in E}}q_A(x)=\la_n(A_0)=\\
=&\sup_{\smat{F\subseteq H_0\\ \dim F\geq n}}\inf_{\smat{\norm x=1\\ x\in F}}q_A(x)\leq\\
\leq&\sup_{\smat{F\subseteq H\\ \dim F\geq n}}\inf_{\smat{\norm x=1\\ x\in F}}q_A(x)=\la_n(A).
\end{align*}
\end{proof}

\begin{exercise}
Scrivere una stima di dipendenza Lipschitz e di dipendenza monotona di $\la_n(A)$ in funzione di $A\in L^{sim}_C(H)$ ($I=\Z\nz$).
\end{exercise}


\section{Calcolo funzionale \texorpdfstring{$C^0$}{C0} per operatori limitati autoaggiunti}
Sia $A=A^\ast\in L(H)$, $H$ di Hilbert e sia
\[\Sigma=\sigma(A)\subseteq[-\norm A,\norm A].\]
Dato $p\in \R[x]$, scrivendo $p(x)=\sum_{k=0}^nc_kx^k$ \`e definito
\[p(A)=\sum_{k=0}^n c_kA^k.\]
Vogliamo estendere questa costruzione definendo $f(A)$ per $f\in C^0(\Sigma,\R)$ o $f\in C^0(\Sigma,\C)$.

\begin{theorem}[della mappa spettrale]\label{ThTeoremaDellaMappaSpettrale}
Sia $p\in \C[z]$ e $T\in L(X)$ per $X$ Banach. Allora
\[\sigma(p(T))=p(\sigma(T)).\]
\end{theorem}
\begin{proof}
Sia $\la\in \sigma(T)$ e scriviamo $p(z)=\sum_{k=0}^n a_kz^k$ con $a_n\neq 0$. Fattorizzando $p(z)-\la$ si ha
\[p(z)-\la=a_n\prod_{j=1}^n(z-\mu_j)\]
dunque $p(\mu)=\la$ se e solo se $\mu\in\cpa{\mu_1,\cdots, \mu_n}$, cio\`e
\[p\ii(\la)=\cpa{\mu_1,\cdots, \mu_n}.\]
Dunque
\[p(T)-\la=a_n\prod_{j=1}^n(T-\mu_j)\]
e quindi $p(T)-\la$ \`e invertibile se e solo se lo sono tutti i fattori dato che commutano, cio\`e
\[\la\in \sigma(p(T))\coimplies \exists j\ t.c.\ \mu_j\in \sigma(T)\coimplies p\ii(\la)\cap \sigma(T)\neq \emptyset\coimplies\la\in p(\sigma(T))\]
\end{proof}
\begin{remark}
La stessa dimostrazione funziona per $T$ elemento di una algebra di Banach.
\end{remark}

\subsubsection{Hilbert reali}

\begin{proposition}[]\label{PrOmomorfismoIsometricoHilbertReali}
Se $A=A^\ast$ in $L(H)$ allora la mappa
\[\Phi:\funcDef{\Pi_\Sigma=\cpa{\text{funzioni polinomiali }\Sigma\to \R}}{L(H)}{p}{p(A)}\]
\`e un omomorfismo di algebre isometrico.
\end{proposition}
\begin{proof}
Per ogni $p\in \R[x]$ si ha che
\[\norm{p(A)}_{L(H)}\overset{p(A)=p(A)^\ast}=r_{p(A)}=\sup_{\la\in \sigma(p(A))}\abs\la\pasgnl={(\ref{ThTeoremaDellaMappaSpettrale})} \sup_{\la\in\Sigma}\abs{p(\la)}=\norm{p}_{\infty,\Sigma}.\]
In particolare $p(A)$ dipende solo dalla funzione $p\res{\Sigma}$.
Che $\Phi$ sia un omomorfismo \`e ovvio dalla definizione di $p(A)$.
\end{proof}

\begin{remark}
Se $f\in C^0(\Sigma,\R)$ e $f\geq 0$ su $\Sigma$ allora per Stone-Weierstrass esiste una successione di polinomi $p_n\xrightarrow{\normd_{\infty,\Sigma}} f$ e possiamo prendere $p_n\geq 0$ su $\Sigma$ (a meno di sotituire $p_n$ con $p_n+\norm{f-p_n}_{\infty}$).
In particolare
\[\ol{\Pi_\Sigma}^{\normd_{\infty,\Sigma}}=C^0(\Sigma,\R).\]
\end{remark}

\begin{remark}
Essendo $\Phi$ isometrico esso si estende alle chiusure, quindi troviamo
\[\ol{\Pi_\Sigma}^{\normd_{\infty,\Sigma}}=C^0(\Sigma,\R)\xrightarrow{\Phi}\ol{\Phi(\Pi_\Sigma)}^{\normd_{L(H)}}=\ol{\R[A]}\subseteq L(H)\]
dove $\ol{\R[A]}$ \`e l'algebra chiusa generata da $A$.
\end{remark}

\begin{remark}
Per continuit\`a, la $\Phi$ estesa \`e ancora un omomorfismo, inoltre se $p\geq 0$ su $\Sigma$ allora $\Phi(p)=p(A)$ \`e un operatore simmetrico $\geq 0$, infatti
\[\sigma(p(A))\overset{(\ref{ThTeoremaDellaMappaSpettrale})}=p(\Sigma)\subseteq[0,\infty)\coimplies p(A)\geq 0.\]
\end{remark}


\begin{notation}
Per $f\in C^0(\Sigma,\R)$ scriviamo $\Phi(f)=f(A)$ in analogia con il caso polinomiale.
\end{notation}

\begin{exercise}
Se $A\in L^{sim}(H)$ e $A\geq 0$ allora $A$ ammette una radice quadrata, cio\`e esiste $B\in L^{sim}(H)$ con $B\geq 0$ e $B^2=A$.
Pi\`u in generale, per ogni $n$, $A$ ammette una radice $n$-esima che sia un operatore simmetrico positivo.
\end{exercise}

\begin{exercise}
Dato $A\in L^{sim}(H)$ con $A\geq 0$ esiste un unico $B$ simmetrico e positivo tale che $B^2=A$. Similmente per le radici $n$-esime.

Hint: usare la radice costruita col calcolo funzionale.
\end{exercise}


\begin{exercise}
Per $A\in L^{sim}(H)$ e $f\in C^0(\Sigma,\R)$ si ha che $f(A)$ commuta con ogni $B$ che commuta con $A$.
\end{exercise}

\subsubsection{Hilbert complessi}

\begin{proposition}[]\label{PrOmomorfismoIsometricoHilbertComplessi}
Sia $A=A^\ast$ in $L(H)$, allora esiste un unico omomorfismo continuo
\[\Phi:C^0(\Sigma,\C)\to L(H)\]
tale che $\Phi(id_\Sigma)=A$. Risulta inoltre che
\begin{enumerate}
    \item $\Phi$ \`e isometrica
    \item $\Phi(\ol f)=\Phi(f)^\ast$ ($\Phi$ \`e uno $\ast$-omomorfismo).
    \item Se $f\in C^0(\Sigma,\R)$ e $f\geq 0$ allora $\Phi(f)=\Phi(f)^\ast\geq 0$.
    \item Se $[A,B]=0$ allora $[\Phi(f),B]$
\end{enumerate}
\end{proposition}
\begin{proof}
L'unicit\`a segue dal fatto che $z\mapsto A$ implica che $p\mapsto p(A)$ per ogni poliniomio $p\in \C[z]$ e questi sono densi in $C^0(\Sigma,\C)$ (stiamo usando il teorema di Stone-Weierstrass su intervalli, basta estendere $f:\Sigma\to\C$ a $\wt f:[-\norm A,\norm A]\to\C$ continua e poi approssimare questa con polinomi).

Mostriamo esistenza: tutto segue dal fatto che per $p\in\C[z]$ vale ancora che la corrispondenza $p\mapsto p(A)$ \`e isometrica
\begin{align*}
    \norm{p(A)}^2=&\sup_{\norm x\leq 1}\norm{p(A)x}^2=\sup_{\norm x\leq 1}\ps{p(A)^\ast p(A)x,x}=\\
    =&\norm{p(A)^\ast p(A)}\overset{(c_kA^k)^\ast=\ol{c_k}(A^\ast)^k=\ol{c_k}A^k}=\\
    =&\norm{\ol p(A)p(A)}=\norm{(\ol pp)(A)}\overset{(\ref{PrOmomorfismoIsometricoHilbertReali})}=\\
    =&\norm{\ol pp}_{\infty,\Sigma}=\norm{p}_{\infty,\Sigma}^2.
\end{align*}
\end{proof}
